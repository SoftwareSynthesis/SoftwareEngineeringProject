\section{Diagrammi dei casi d'uso}

\section{Requisiti}

% TODO:
% da cambiare nel sistema di stampa
% occorre usare longtable!
% dimensione della colonna con p
\begin{longtable}{lp{.55\textwidth}l}
\toprule Codice & Requisito & Fonte\\
\midrule
% TODO:
% typo: proipria
RSDD2.1.0 & L'utente dovrebbe avere una stima della complessità per la propria password. & Capitolato d'appalto \\
RSDD2.2.0 & Convalidare username (e-mail) dell'utente & Interno \\
RSDD2.3.0 & Inserimento dati anagrafici (nome, cognome), immagine e indirizzo email (utilizzato anche come username) & Interno \\
RSDO10.0.0 & L'intero sistema deve essere contenuto in un unica pagina Web & Capitolato d'appalto \\
RSDO10.1.0 & L'interfaccia grafica non deve subire refresh per ogni operazione dell'utente & Interno \\
RSDO22.0.0 & L'applicativo deve funzionare sotto l'ultima versione (23.0.1271.97m) del browser Google Chrome & Capitolato d'appalto \\
RSDO6.0.0 & Gestire le comunicazioni utente tramite WebRTC & Capitolato d'appalto \\
% TODO:
% typo: piu senza accento
RSFF11.2.0 & Utilizzo del protocollo webSocket per creare una connessione tra più di 2 utenti & Interno \\
RSFF12.1.0 & Estendere la connessione ad altri client & Interno \\
RSFO11.0.0 & Creare una connessione tra client, mediante l'utilizzo di un server & Capitolato d'appalto \\
RSFO11.1.0 & Utilizzo del protocollo webSocket per creare una connessione tra 2 utenti & Interno \\
% TODO:
% typo: apostrofo saltato, usare ' come carattere
RSFO12.0.0 & Gestire gli eventi dell'utente durante la connessione & Interno \\
RSQD21.0.0 & Gestione interfaccia grafica in più lingue & Interno \\
RSQF23.0.0 & Verificare che l'applicativo funzioni anche sotto gli altri browser del S.O. Windows & Interno \\
RSQF23.1.0 & Verificare che funzioni con Opera & Capitolato d'appalto \\
RSQF23.2.0 & Verificare che funzioni con Firefox & Capitolato d'appalto \\
RSQF23.3.0 & Verificare che funzioni con Internet Explorer (v9 e superiori) & Capitolato d'appalto \\
RSQF23.4.0 & Verificare che funzioni con Safari & Capitolato d'appalto \\
RSQF24.0.0 & Verificare che l'applicativo funzioni anche sotto gli altri browser del S.O. Linux & Interno \\
RSQF24.1.0 & Verificare che funzioni con Opera & Capitolato d'appalto \\
RSQF24.2.0 & Verificare che funzioni con Firefox & Capitolato d'appalto \\
RSQF24.3.0 & Verificare che funzioni con Chromium & Interno \\
RSQF25.0.0 & Verificare che l'applicativo funzioni anche sotto gli altri browser del S.O. Macintosh & Interno \\
RSQF25.1.0 & Verificare che funzioni con Opera & Capitolato d'appalto \\
RSQF25.2.0 & Verificare che funzioni con Firefox & Capitolato d'appalto \\
RSQF25.3.0 & Verificare che funzioni con Safari & Capitolato d'appalto \\
RUFD1.1.0 & Gestione password dimenticata & Interno \\
RUFD1.1.1 & Proporre la domanda segreta all'utente & Interno \\
RUFD1.1.2 & Invio di una mail all'utente contenete la password dimenticata & Interno \\
RUFD12.2.0 & Chi crea la connessione può eliminare i membri del gruppo & Interno \\
RUFD13.1.0 & Servizio chat tra 2 utenti & Interno \\
RUFF13.0.0 & Servizio chat testuale & Capitolato d'appalto \\
RUFF13.2.0 & Servizio chat tra più di 2 utenti & Interno \\
RUFF14.0.0 & Registrazione della chiamata & Capitolato d'appalto \\
RUFF14.1.0 & Necessità di autorizzazione dagli utenti della chiamata per poter avviare la registrazione & Interno \\
RUFF14.2.0 & Registrazione audio & Interno \\
RUFF14.3.0 & Possibilità di riascoltare la registrazione & Interno \\
% TODO:
% typo: disposizine
RUFF15.0.0 & L'utente avrà a disposizione una segreteria telefonica & Interno \\
RUFF15.1.0 & Possibilità di lasciare un audio messaggio in segreteria & Interno \\
RUFF15.2.0 & Possibilità di lasciare un audio/video messaggio in segreteria & Capitolato d'appalto \\
RUFF15.3.0 & Possibilità di ascoltare la propria segreteria & Interno \\
RUFF15.4.0 & Possibilità di cancellare messaggi della segretaria & Interno \\
RUFF15.5.0 & Possibilità di cambiare lo stato del messaggio tra ascoltato/non ascoltato & Interno \\
RUFF16.0.0 & Possibilità di impostare uno stato utente & Interno \\
RUFF17.0.0 & Dare la possibilità di vedere gli stati personali altrui. & Interno \\
RUFF18.0.0 & Creazione di una Blacklist & Interno \\
RUFF19.0.0 & Possibilità di tenere uno storico delle chiamate & Interno \\
RUFF20.0.0 & Condividere risorse & Interno \\
RUFF20.1.0 & Condividere monitor & Interno \\
RUFF20.2.0 & Condivisione pdf & Interno \\
% TODO:
% typo: Condivione
RUFF20.3.0 & Condivisione lavagna grafica & Interno \\
RUFF20.4.0 & Invio file & Interno \\
RUFF3.0.0 & Modifica dati utente (ad eccezione dell'indirizzo email) & Interno \\
RUFF4.0.0 & L'utente dovrebbe poter tenere tenere una rubrica personale dove raggruppare i propri contatti. & Interno \\
RUFF4.1.0 & Un utente deve poter inserire utenti dalla propria rubrica. & Capitolato d'appalto \\
RUFF4.2.0 & Eliminare un utente dalla propria rubrica & Interno \\
RUFF4.3.0 & Possibilità di ordinare la rubrica su alcuni parametri rilevanti & Interno \\
RUFF4.4.0 & Possibilità di suddividere la rubrica in gruppi & Interno \\
RUFF4.4.1 & Possibilità di togliere un elemento da un gruppo & Interno \\
RUFF4.4.2 & Possibilità di aggiungere un elemento in un gruppo & Interno \\
RUFF4.4.3 & Possibilità di creare un gruppo & Interno \\
RUFF4.4.4 & Possibilità di eliminare un gruppo & Interno \\
RUFF4.5.0 & Possibilità di esportare in xml la rubrica personale & Interno \\
RUFF4.6.0 & Possibilità di modificare la rubrica importando un file xml & Interno \\
RUFF4.7.0 & Ricerca di un utente nella propria rubrica & Interno \\
RUFF5.1.0 & Possibilità di cercare un utente dalla lista & Interno \\
RUFF6.1.2 & Stabilire una comunicazione audio con un utente registrato e presente nella rubrica & Capitolato d'appalto \\
RUFF6.1.4 & Promuovere una comunicazione audio avviata con un utente in una comunicazione audio/video & Interno \\
RUFF6.2.2 & Stabilire una comunicazione audio/video con un utente registrato e presente nella rubrica & Capitolato d'appalto \\
RUFF6.2.4 & Declassare una comunicazione audio/video avviata con un utente in una comunicazione solo audio & Interno \\
RUFF6.2.5 & Disattivare la webcam utente pur continuando a ricevere il segnale video proveniente dall'altro capo della comunicazione & Interno \\
RUFF6.3.0 & Disattivare il microfono utente pur continuando a ricevere il segnale video proveniente dall'altro capo della comunicazione & Interno \\
RUFF9.3.0 & Rilevazione frame per secondo & Interno \\
RUFO1.0.0 & L'utente deve potersi autentificare nel server, cosi da permettere a quest'ultimo di rilevare la sua presenza nel sistema. Il Login dovrà essere gestito con username (indirizzo mail) e password. & Interno \\
RUFO12.3.0 & Chi partecipa alla connessione può togliersi da essa & Interno \\
RUFO2.0.0 & Registrazione del nuovo utente & Capitolato d'appalto \\
RUFO5.0.0 & Possibilità di visualizzare la lista di utenti registrati nel sistema & Capitolato d'appalto \\
RUFO6.1.0 & Stabilire e gestire una comunicazione audio con un utente in linea & Capitolato d'appalto \\
RUFO6.1.1 & Stabilire una comunicazione audio mediante inserimento d'indirizzo IP & Capitolato d'appalto \\
RUFO6.1.3 & Stabilire una comunicazione audio con un utente registrato e NON presente nella rubrica & Capitolato d'appalto \\
RUFO6.2.0 & Stabilire e gestire  una comunicazione audio/video con un utente & Capitolato d'appalto \\
RUFO6.2.1 & Stabilire una comunicazione audio/video mediante inserimento d'indirizzo IP & Capitolato d'appalto \\
RUFO6.2.3 & Stabilire una comunicazione audio/video con un utente registrato e NON presente nella rubrica & Capitolato d'appalto \\
RUFO7.0.0 & Indicare il tempo di comunicazione & Capitolato d'appalto \\
RUFO8.0.0 & Valutare il numero di byte trasmessi & Capitolato d'appalto \\
RUFO8.1.0 & Valutare il numero di byte inviati & Interno \\
RUFO8.2.0 & Valutare il numero di byte ricevuti & Interno \\
RUFO9.0.0 & Indicare la qualità della linea di trasmissione & Capitolato d'appalto \\
RUFO9.1.0 & Rilevare latenza & Capitolato d'appalto \\
RUFO9.2.0 & Rilevare velocità di trasmissione & Capitolato d'appalto \\
\bottomrule
\end{longtable}

\section{Tracciamento dei requisiti}

\begin{tabularx}{\textwidth}{lXl}
\toprule Codice UC & Nome UC  & Requisito\\
\midrule
\bottomrule
\end{tabularx}
