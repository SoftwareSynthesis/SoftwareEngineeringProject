% analisi_dei_requisiti/analisi_dei_requisiti.tex

%TODO:
% sistemare date e autori sulla base del PdP!

% **************************************************
% Macro specifiche per il documento corrente
% **************************************************
% Nome
\newcommand{\docName}{Analisi dei requisiti}
% Nome file
\newcommand{\docFileName}{analisi\_dei\_requisiti.1.0.pdf}
% Versione
\newcommand{\docVers}{1.0}
% Data creazione
\newcommand{\creationDate}{2012-12-03}
% Data ultima modifica
\newcommand{\modificationDate}{2012-12-17}
% Stato in {Approvato, Non approvato}
\newcommand{\docState}{Approvato}
% Uso in {Interno, Esterno}
\newcommand{\docUsage}{Esterno}
% Redattori da specificare come nome1\\ &nome2\\ ecc.
\newcommand{\docAuthors}{Andrea Meneghinello \\& Andrea Rizzi\\& Diego Beraldin\\& Elena Zecchinato\\& Riccardo Tresoldi\\& Stefano Farronato}
% Approvato da
\newcommand{\approvedBy}{Elena Zecchinato}
% Verificatori
\newcommand{\verifiedBy}{Diego Beraldin\\& Marco Schivo}
% Perscorso (relativo o assoluto) che punta alla directory contenente shared/
% come sua sottodirectory (per comodità chiamiamola 'doc root').
\newcommand{\docRoot}{..}
% definire se si vuole l'indice delle tabelle
\def\INDICETABELLE{false}
% definire se si vuole l'indice delle figure
\def\INDICEFIGURE{false}


% importa il preambolo condiviso da tutti i documenti
% shared/preamble.tex
%
% Questo documento contiene la parte del preambolo condivisa e viene pertanto
% richiamato nel 'master' di tutti i documenti di progetto.  Al suo interno
% contiene le inclusioni (e le configurazioni) di tutti i package richiesti per
% la compilazione dei documenti, le macro di carattere generale e la definizione
% degli stili di pagina.

\documentclass[a4paper,10pt]{article}

% **************************************************
% Macro generiche
% **************************************************
\newcommand{\team}{Software Synthesis}                    % chi siamo
\newcommand{\email}{info@softwaresynthesis.org}           % e-mail
\newcommand{\caName}{MyTalk}                              % titolo capitolato
\newcommand{\manager}{SynthesisRequirementManager}        % nome del sistema di tracciamento
\newcommand{\memberdata}[1]{%
  \texttt{\textcolor{RedOrange}{#1}}}                     % attributi di una classe
\newcommand{\method}[1]{\texttt{\textcolor{Emerald}{#1}}} % metodi di una classe
\newcommand{\exception}[1]{%
  \texttt{\textcolor{RedViolet}{#1}}}                     % eccezione
% \newcommand{\handler}[1]{\texttt{\textcolor{Maroon}{#1}}} % per gli event handler
\newcommand{\inglese}[1]{%
  \foreignlanguage{english}{\textit{#1}}}                 % per i testi in lingua inglese
\newcommand{\purpose}{%                                     scopo del prodotto
Con il progetto ``\caName'' si intende un sistema software di comunicazione tra utenti mediante \underline{browser} senza la necessit{\`a} di installazione di \underline{plugin} e/o software esterni. L'utilizzatore avr{\`a} la possibilit{\`a} di interagire con un altro utente tramite una comunicazione audio - audio/video - testuale e, inoltre, ottenere delle statistiche sull'attivit{\`a} in tempo reale.%
}
\newcommand{\glossaryIntro}{%                               introduzione al glossario
Al fine di evitare incomprensioni dovute all'uso di termini tecnici nei documenti, viene redatto e allegato il documento \textit{glossario.4.0.pdf} dove vengono definiti e descritti tutti i termini marcati con una sottolineatura.%
}


% **************************************************
% Codifica e lingua dei documenti
% **************************************************
\usepackage[utf8x]{inputenc}                              % codifica caratteri dei documenti sorgenti
\usepackage[english,italian]{babel}                       % localizzazione ai fini di sillabazione e cross-references
\usepackage[T1]{fontenc}                                  % codifica font di output

% **************************************************
% Definizione geometria della pagina
% **************************************************
\usepackage[a4paper,head=4cm,top=4.5cm,bottom=3cm,left=3cm,right=3cm,bindingoffset=5mm]{geometry}

% *************************************************
% Intestazioni e piè di pagina personalizzati
% *************************************************
\usepackage{fancyhdr}
% stile normale
\fancypagestyle{normal}{
\fancyhead{}                                              % intestazione
\fancyhead[RE,RO]{
\begin{picture}(0,0)
  \put(-410,0){\includegraphics[width=1.02\textwidth]{header_logo}}
  \put(-410,10){\sffamily\large\leftmark}
\end{picture}
\vspace{-4pt}
}
\renewcommand{\headrulewidth}{0pt}                       % riga sotto l'intestazione
\cfoot{}                                                  % piè di pagina
\fancyfoot[RO,LE]{\sffamily
  pag.~\thepage{} di \pageref{LastPage}}                  % a dx nelle pag. dispari e a sx in quelle pari
\fancyfoot[RE,LO]{\sffamily\docFileName{}}
\renewcommand{\footrulewidth}{.4pt}                       % riga sopra il piè di pagina
}
% stile per gli indici
\fancypagestyle{toc}{
\fancyhead{}                                              % intestazione
\fancyhead[RE,RO]{
\begin{picture}(0,0)
  \put(-410,0){\includegraphics[width=1.02\textwidth]{header_logo}}
\end{picture}
}
\renewcommand{\headrule}{}                                % nessuna riga sotto l'intestazione
\cfoot{}                                                  % piè di pagina
\fancyfoot[RO,LE]{\sffamily\thepage{}}                    % a dx nelle pag. dispari e a sx in quelle pari
\fancyfoot[RE,LO]{\sffamily\docFileName{} -- v.\docVers}
\renewcommand{\footrulewidth}{.4pt}                       % riga sopra il piè di pagina
}

\pagestyle{fancy}                                         % premetto: non so usare bene le marche:
\renewcommand{\sectionmark}[1]{\markboth{#1}{#1}}         % se qualcuno ha idee migliori si faccia avanti!

% **************************************************
% Tabelle
% **************************************************
\usepackage{tabularx}                                     % tabelle di larghezza fissa con una o più colonne variabili
\usepackage{multirow}                                     % colonne con colonne che si estendono per più righe
\usepackage{booktabs}                                     % per inserire l'ambiente table e le righe orizz. nelle tabelle
\usepackage{longtable}			                              % tabelle oltre i limiti di pagina

% **************************************************
% Cross-references e collegamenti ipertestuali
% **************************************************
\usepackage[hidelinks]{hyperref}
\hypersetup{%
  colorlinks=false, linktocpage=false, pdfborder={0,0,0}, pdfstartpage=1, pdfstartview=FitV,%
  urlcolor=Cyan, linkcolor=Cyan, citecolor=Black, %pagecolor=Black,%
  pdftitle={\docName}, pdfauthor={\team}, pdfsubject={}, pdfkeywords={},%
  pdfcreator={pdflatex}, pdfproducer={pdflatex with hyperref package}%
}

% **************************************************
% Immagini e grafica
% **************************************************
\usepackage{graphicx}                                     % supporto ad aspetti avanzati delle immagini
\usepackage[table,usenames,dvipsnames]{xcolor}            % tabelle con righe colorate e alternate
\graphicspath{{\docRoot/pics/}}                           % percorso contenente tutti i file immagini
\usepackage{float}                                        % per rendere non flottanti gli ambienti flottanti
\usepackage[italian]{varioref}                            % testo completo riferimenti in italiano

% **************************************************
% Definizioni di colori
% **************************************************
\definecolor{myBlue}{RGB}{1,167,236}
\definecolor{lightblue}{RGB}{213,243,253}%{119,218,247}
\definecolor{llightblue}{RGB}{229,255,255}

% **************************************************
% Altri pacchetti opzionali
% **************************************************     
\usepackage{lastpage}                                     % per sapere il numero totale di pagine
\usepackage{eurosym}                                      % per il simbolo dell'euro usare \EUR{x} dove x è l'importo
\usepackage{ifthen}                                       % permette la scelta di rami condizionali nella compilazione
\usepackage{enumitem}                                     % permette di configurare gli elenchi puntati e numerati


% Fine del preambolo e inizio del documento
\begin{document}

% Inclusione della prima pagina
% shared/firstpage.tex
%
% Questo documento definisce il contenuto della prima pagina, che si suppone
% essere uguale in tutti i documenti.  Oltre al logo e al titolo, la prima
% pagina contiene i metadati relativi al documento in cui viene inclusa.


% rimuove intestazioni e piè di pagina
\pagestyle{empty}

\begin{center}

% logo del gruppo
\includegraphics[width=1.5\textwidth]{logo}

\vspace{1in}

% titolo del documento
{\Huge\bfseries \docName}

\vspace{1in}

% tabella riepilogativa
\begin{tabularx}{.7\textwidth}{>{\bfseries\sffamily}l>{\sffamily}l}
\toprule
\multicolumn{2}{>{\sffamily}c}{Informazioni sul documento}\\
\midrule
Nome file:            & \docFileName\\
Versione:             & \docVers\\
Data creazione:       & \creationDate\\
Data ultima modifica: & \modificationDate\\
Stato:                & \docState\\
Uso:                  & \docUsage\\
Redattori:            & \docAuthors\\
Approvato da:         & \approvedBy\\
Verificatori:         & \verifiedBy\\
\bottomrule
\end{tabularx}

\end{center}

\newpage


% Storico delle modifiche
\section*{Storia delle modifiche}
\begin{tabularx}{\textwidth}{lXll}
\toprule
Versione & Descrzione intervento & Redattore & Data\\
\midrule % inserire qui il contenuto della tabella
1.0 & Approvazione documento. & Elena Zecchinato & 2012-12-18\\\\
0.13 & Verifica dell'intero documento. & Diego Beraldin & 2012-12-18\\\\
0.12 & Correziona casi d'uso errati. Aggiunta requisiti mancanti. Fine stesura documento. & Andrea Rizzi & 2012-12-17\\\\
0.11 & Seconda stesura dei casi d'uso e loro inserimento nel sistema SynthesisRequirmentManager. & Riccardo Tresoldi & 2012-12-15\\\\
0.10 & Correzione requisiti errati. Correzione casi d'uso errati. & Stefano Farronato & 2012-12-14\\\\
0.9 & Seconda stesura dei requisiti e loro inserimento nel sistema SynthesisRequirmentManager. & Andrea Meneghinello & 2012-12-13\\\\
0.8 & Incontro con il proponenete. Stesura dei requisiti emersi. Correzione requisiti errati. & Riccardo Tresoldi & 2012-12-10\\\\
0.7 & Verifica documento fino alla versione 0.4. Correzione errori nel punto ``funzionalità''. & Marco Schivo & 2012-12-08\\\\
0.6 & Prima stesura dei casi d'uso e loro inserimento nel sistema SynthesisRequirmentManager. Stesura del punto ``Requisiti'' (introduzione) & Riccardo Tresoldi & 2012-12-07\\\\
0.5 & Continuazione stesura dei requisiti e loro inserimento nel sistema SynthesisRequirmentManager. Correzione requisiti errati. & Diego Beraldin & 2012-12-06\\\\
0.4 & Prima stesura dei requisiti e loro inserimento nel sistema SynthesisRequirmentManager. & Andrea Rizzi & 2012-12-05\\\\
0.3 & Stesura del punto ``Funzionalità'' e ``Assunzioni e prerequisiti''. & Diego Beraldin & 2012-12-04\\\\
0.2 & Stesura del punto ``Riferimenti'' e ``Descrizione generale''. & Riccardo Tresoldi & 2012-12-03\\\\
0.1 & Creazione del documento e stesura del punto ``Scopo del documento''. & Andrea Rizzi & 2012-12-03\\
\bottomrule
\end{tabularx}
\newpage

% inclusione dell'indice
% shared/toc.tex
%
% Questo file contiene le istruzioni che generano l'indice o gli indici del
% documento (utile nel caso in cui decidessimo di avere anche un indice delle
% tabelle e/o un indice delle figure).

% imposta lo stile di pagina per i titoli definito nel preambolo
\pagestyle{toc}
% imposta i numeri di pagina romani minuscoli
\pagenumbering{roman}

% genera automaticamente l'indice di LaTeX
\tableofcontents

% se è true \INDICETABELLE allora genera l'indice delle tabelle, altrimenti non fa nulla
\ifthenelse{\equal{\INDICETABELLE}{true}}{%
  \clearpage % l'indice delle tabelle, se c'è, deve andare a pagina nuova
  \listoftables
}{}

% se è true |INDICEFIGURE allora genera l'indice delle figure, altrimenti non fa nulla
\ifthenelse{\equal{\INDICEFIGURE}{true}}{%
  \clearpage % l'indice delle figure, se c'è, deve andare a pagina nuova
  \listoffigures
}{}

%in ogni caso occorre andare a pagina nuova dopo gli indici
\clearpage


% Alcuni aggiustamenti per le pagine
\pagenumbering{arabic}
\setcounter{page}{1}
\pagestyle{normal}

% Qui ha inizio il documento vero e proprio...

\section{Introduzione}
\subsection{Scopo del prodotto}
\purpose

\subsection{Scopo del documento}
Il presente documento riporta il risultato dell'attività di analisi dei requisiti svolta dal gruppo \team{} in fase preliminare, vale a dire un insieme di requisiti -- in parte esplicitati nel testo del capitolato C1, in parte emersi durante incontri con il committente Zucchetti Srl, in parte inferiti dal dominio e in parte auto-imposti dal gruppo -- che il prodotto software è tenuto a soddisfare in termini funzionali, prestazionali, qualitativi e dichiarativi.

Il comportamento del sistema osservabile dall'utente finale, in conformità con i requisiti sopra enunciati, è riportato secondo il formalismo noto come diagrammi dei casi d'uso. La corrispondenza fra casi d'uso e requisiti è illustrata mediante la tabella riportata nella sezione~\ref{sec:tracciamento}.
%TODO: aggiungere riferimenti incrociati e non basarsi su numeri hard-coded qui dentro

\subsection{Glossario}
\glossaryIntro

\clearpage
\section{Riferimenti}

\subsection{Normativi}
\begin{itemize}
\item[] verbale \textit{verbale\_incontro\_2012-12-11.pdf} allegato;
\item[] \textit{piano\_di\_qualifica.1.0.pdf} allegato.
\end{itemize}

\subsection{Informativi}
\begin{itemize}
\item[] Capitolato d'appalto: \caName{}, v1.0, redatto e rilasciato dal proponente Zucchetti s.r.l. reperibile all'indirizzo \url{http://www.math.unipd.it/~tullio/IS-1/2012/Progetto/C1.pdf};
\item[] testo di consultazione: \textit{Software Engineering (8th edition) Ian Sommerville, Pearson Education | Addison Wesley};
\item[] \textit{glossario.1.0.pdf} allegato.
\end{itemize}

\clearpage
\section{Descrizione generale}

\subsection{Contesto di utilizzo}
Il sistema software realizzato nell'ambito del progetto \caName{} si configura come una piattaforma di comunicazione fra utenti connessi alla rete. Oggetto della condivisione -- poiché `comunicare' letteralmente significa `mettere in comune' -- saranno non solo flussi audio/video, secondo le modalità tipiche della (video)chiamata via \inglese{softphone}, ma anche file e documenti presenti in locale, per i quali è previsto tanto il trasferimento quanto la condivisione delle modifiche in tempo reale.

Dal punto di vista dell'utente, \caName{} si configura inoltre come un applicativo fruibile attraverso la sola mediazione di un \underline{browser} per la navigazione web, senza necessità di installazione di alcun programma \inglese{stand-alone} sul proprio sistema né di \underline{plugin} per il \underline{browser} forniti da \team{} o da terze parti.

Al fine di gestire l'utenza è prevista la realizzazione di una parte di \inglese{back-end} da installare in un server sotto il controllo del fornitore del servizio agli utenti. L'installazione dell'applicativo lato server è prevista in un ambiente \underline{TomCat}.

\subsection{Utente finale}
Il modello di utente finale di riferimento prevede la disponibilità di una connessione a Internet funzionante e di un \underline{browser} di ultima generazione in grado di supportare lo standard \underline{WebRTC}.

Fra i prerequisiti non figura il possesso di conoscenze particolari estranee alla comune esperienza di navigazione web (in particolare, l'eventuale confidenza con sistemi esistenti di telefonia via internet non rientra nelle assunzioni).

\subsection{Funzionalità}
Le funzionalità offerte dal prodotto agli utenti finali possono essere suddivise in due categorie principali:
\begin{itemize}
  \item gestione di scambi di informazioni in tempo reale, in particolare
  \begin{itemize}
  \item[--] comunicazione audio e audio/video tra due (o più) utenti;
  \item[--] condivisione di risorse (ad esempio lavagna grafica, schermo e documenti PDF). 
  \end{itemize}
  \item scambio di informazioni asincrone 
  \begin{itemize}
  \item[--] segreteria (sia per messaggi vocali che videomessaggi);
  \item[--] chat testuale;
  \item[--] trasferimento file.
  \end{itemize}
\end{itemize}

A integrazione e supporto di tali funzionalità di base è previsto lo sviluppo di funzionalità secondarie quali l'autenticazione degli utenti, la gestione della rubrica, la visualizzazione di metadati sulle connessioni e la possibilità di mantenere un registro storico delle attività realizzate mediante il sistema.

\subsection{Assunzioni e prerequisiti}
In fase di \underline{deployment} si presuppone che l'installatore presso il committente sia in possesso delle competenze necessarie per l'installazione e la configurazione della parte server del sistema.

Si assume inoltre che l'utente sia in possesso di una connessione a internet funzionante al fine di instaurare la comunicazione. È altresì assunto implicitamente l'utilizzo di un browser con supporto a \underline{WebRTC}.

\clearpage
%da qua in poi mi affido al software di stampa automatico
\subsection{Tracciamenti Requisiti-Componenti}\label{sec:tracRecComp}

\begin{center}
\rowcolors{4}{lightblue}{llightblue}\begin{longtable}{lp{.55\textwidth}l}
\toprule Requisiti &  Componenti\\
\midrule
RUFO1.0.0 & CS07 -- Façade del server \\
 & CS02 -- Gestione connessione \\
 & CS01 -- Gestione database \\
 & CS04 -- Gestione autenticazione \\
RUFD1.1.0 & CS01 -- Gestione database \\
RUFD1.1.2 & CS01 -- Gestione database \\
RSQO1.2.0 & CS02 -- Gestione connessione \\
 & CS01 -- Gestione database \\
  & CS04 -- Gestione autenticazione \\
RUFO2.0.0 & CS01 -- Gestione database \\
 & CS07 -- Façade del server \\
  & CS04 -- Gestione autenticazione \\
RSQO2.1.0 & CS01 -- Gestione database \\
 & CS04 -- Gestione autenticazione \\
RSDD2.2.0 & CS01 -- Gestione database \\
RUFF3.0.0 & CS07 -- Façade del server \\
 & CS01 -- Gestione database \\
  & CS04 -- Gestione autenticazione \\
RUFF3.1.0 & CS01 -- Gestione database \\
 & CS04 -- Gestione autenticazione \\
RUFF3.2.0 & CS01 -- Gestione database \\
RUFF4.0.0 & CS03 -- Gestione rubrica \\
 & CS07 -- Façade del server \\
 & CS01 -- Gestione database \\
RUFF4.1.0 & CS01 -- Gestione database \\
 & CS03 -- Gestione rubrica \\
RUFF4.2.0 & CS01 -- Gestione database \\
 & CS03 -- Gestione rubrica \\
RUFF4.3.0 & CS01 -- Gestione database \\
 & CS03 -- Gestione rubrica \\
RUFF4.4.0 & CS01 -- Gestione database \\
 & CS03 -- Gestione rubrica \\
RUFF4.4.1 & CS01 -- Gestione database \\
 & CS03 -- Gestione rubrica \\
RUFF4.4.2 & CS03 -- Gestione rubrica \\
 & CS01 -- Gestione database \\
RUFF4.4.3 & CS01 -- Gestione database \\
 & CS03 -- Gestione rubrica \\
RUFF4.4.4 & CS01 -- Gestione database \\
 & CS03 -- Gestione rubrica \\
RUFF4.5.0 & CS01 -- Gestione database \\
 & CS03 -- Gestione rubrica \\
RUFF4.6.0 & CS03 -- Gestione rubrica \\
RUFF4.7.0 & CS01 -- Gestione database \\
 & CS03 -- Gestione rubrica \\
RUFO5.0.0 & CS01 -- Gestione database \\
 & CS07 -- Façade del server \\
RUFF5.1.0 & CS01 -- Gestione database \\
RSDO6.0.0 & CS02 -- Gestione connessione \\
RUFO6.1.0 & CS02 -- Gestione connessione \\
 & CS07 -- Façade del server \\
  & CS06 -- Gestione chiamata \\
RUFO6.1.1 & CS02 -- Gestione connessione \\
  & CS06 -- Gestione chiamata \\
RUFF6.1.2 & CS02 -- Gestione connessione \\
RUFO6.1.3 & CS02 -- Gestione connessione \\
  & CS06 -- Gestione chiamata \\
RUFF6.1.4 & CS02 -- Gestione connessione \\
RUFO6.2.0 & CS07 -- Façade del server \\
 & CS02 -- Gestione connessione \\
   & CS06 -- Gestione chiamata \\
RUFO6.2.1 & CS02 -- Gestione connessione \\
  & CS06 -- Gestione chiamata \\
RUFF6.2.2 & CS02 -- Gestione connessione \\
RUFO6.2.3 & CS02 -- Gestione connessione \\
  & CS06 -- Gestione chiamata \\
RUFF6.2.4 & CS02 -- Gestione connessione \\
RUFF6.2.5 & CS02 -- Gestione connessione \\
RUFF6.3.0 & CS02 -- Gestione connessione \\
RUFO6.4.0   & CS06 -- Gestione chiamata \\
RUFO 6.5.0  & CS06 -- Gestione chiamata \\
RUFO7.0.0 & CS02 -- Gestione connessione \\
  & CS06 -- Gestione chiamata \\
RUFO8.0.0 & CS02 -- Gestione connessione \\
  & CS06 -- Gestione chiamata \\
RUFO8.1.0 & CS02 -- Gestione connessione \\
  & CS06 -- Gestione chiamata \\
RUFO8.2.0 & CS02 -- Gestione connessione \\
  & CS06 -- Gestione chiamata \\
RUFO9.0.0 & CS02 -- Gestione connessione \\
  & CS06 -- Gestione chiamata \\
RUFO9.1.0 & CS02 -- Gestione connessione \\
  & CS06 -- Gestione chiamata \\
RUFO9.2.0 & CS02 -- Gestione connessione \\
  & CS06 -- Gestione chiamata \\
RUFF9.3.0 & CS02 -- Gestione connessione \\
RSFO11.0.0 & CS02 -- Gestione connessione \\
  & CS06 -- Gestione chiamata \\
RSFO11.1.0 & CS02 -- Gestione connessione \\
  & CS06 -- Gestione chiamata \\
RSFF11.2.0 & CS02 -- Gestione connessione \\
RSFO12.0.0 & CS02 -- Gestione connessione \\
 & CS07 -- Façade del server \\
   & CS06 -- Gestione chiamata \\
RSFF12.1.0 & CS02 -- Gestione connessione \\
RUFD12.2.0 & CS02 -- Gestione connessione \\
RUFO12.3.0 & CS02 -- Gestione connessione \\
  & CS06 -- Gestione chiamata \\
RUFF13.0.0 & CS07 -- Façade del server \\
 & CS02 -- Gestione connessione \\
RUFD13.1.0 & CS02 -- Gestione connessione \\
RUFF13.2.0 & CS02 -- Gestione connessione \\
RUFF14.0.0 & CS02 -- Gestione connessione \\
RUFF14.1.0 & CS02 -- Gestione connessione \\
RUFF14.2.0 & CS02 -- Gestione connessione \\
RUFF15.0.0 & CS05 -- Gestione rubrica \\
 & CS01 -- Gestione database \\
 & CS02 -- Gestione connessione \\
 & CS07 -- Façade del server \\
RUFF15.1.0 & CS05 -- Gestione rubrica \\
 & CS01 -- Gestione database \\
RUFF15.2.0 & CS05 -- Gestione rubrica \\
 & CS02 -- Gestione connessione \\
 & CS01 -- Gestione database \\
RUFF15.3.0 & CS05 -- Gestione rubrica \\
 & CS02 -- Gestione connessione \\
 & CS01 -- Gestione database \\
RUFF15.4.0 & CS01 -- Gestione database \\
 & CS05 -- Gestione rubrica \\
 & CS02 -- Gestione connessione \\
RUFF15.5.0 & CS05 -- Gestione rubrica \\
 & CS02 -- Gestione connessione \\
 & CS01 -- Gestione database \\
 & CS02 -- Gestione connessione \\
 & CS07 -- Façade del server \\
 & CS02 -- Gestione connessione \\
 & CS07 -- Façade del server \\
RUFF18.0.0 & CS07 -- Façade del server \\
 & CS01 -- Gestione database \\
 & CS02 -- Gestione connessione \\
RUFF19.0.0 & CS02 -- Gestione connessione \\
RUFF20.0.0 & CS02 -- Gestione connessione \\
RUFF20.1.0 & CS02 -- Gestione connessione \\
RUFF20.2.0 & CS02 -- Gestione connessione \\
RUFF20.3.0 & CS02 -- Gestione connessione \\
RUFF20.4.0 & CS02 -- Gestione connessione \\
\bottomrule
\end{longtable}
\end{center}
\subsection{Tracciamenti Componenti-Requisiti}\label{sec:tracCompRec}

\begin{center}
\rowcolors{4}{lightblue}{llightblue}\begin{longtable}{lp{.55\textwidth}l}
\toprule Componenti & Requisiti associati\\
\midrule
CS01 -- Gestione database & RUFD1.1.0 \\
 & RUFF18.0.0 \\
 & RUFF4.4.0 \\
 & RUFO1.0.0 \\
 & RUFD1.1.2 \\
 & RUFF3.0.0 \\
 & RUFF4.4.1 \\
 & RUFO2.0.0 \\
 & RUFO5.0.0 \\
 & RUFF15.0.0 \\
 & RUFF3.1.0 \\
 & RUFF4.4.2 \\
 & RUFF15.1.0 \\
 & RUFF3.2.0 \\
 & RUFF4.4.3 \\
 & RUFF15.2.0 \\
 & RUFF4.0.0 \\
 & RUFF4.4.4 \\
 & RSDD2.2.0 \\
 & RUFF15.3.0 \\
 & RUFF4.1.0 \\
 & RUFF4.5.0 \\
 & RSQO1.2.0 \\
 & RUFF15.4.0 \\
 & RUFF4.2.0 \\
 & RUFF4.7.0 \\
 & RSQO2.1.0 \\
 & RUFF15.5.0 \\
 & RUFF4.3.0 \\
 & RUFF5.1.0 \\
CS02 -- Gestione connessione & RSFO12.0.0 \\
 & RUFF14.2.0 \\
 & RUFF18.0.0 \\
 & RUFF6.1.4 \\
 & RUFO6.1.0 \\
 & RUFO8.1.0 \\
 & RSQO1.2.0 \\
 & RUFF15.0.0 \\
 & RUFF19.0.0 \\
 & RUFF6.2.2 \\
 & RUFO6.1.1 \\
 & RUFO8.2.0 \\
 & RUFD12.2.0 \\
 & RUFF15.2.0 \\
 & RUFF20.0.0 \\
 & RUFF6.2.4 \\
 & RUFO6.1.3 \\
 & RUFO9.0.0 \\
 & RSDO6.0.0 \\
 & RUFD13.1.0 \\
 & RUFF15.3.0 \\
 & RUFF20.1.0 \\
 & RUFF6.2.5 \\
 & RUFO6.2.0 \\
 & RUFO9.1.0 \\
 & RSFF11.2.0 \\
 & RUFF13.0.0 \\
 & RUFF15.4.0 \\
 & RUFF20.2.0 \\
 & RUFF6.3.0 \\
 & RUFO6.2.1 \\
 & RUFO9.2.0 \\
 & RSFF12.1.0 \\
 & RUFF13.2.0 \\
 & RUFF15.5.0 \\
 & RUFF20.3.0 \\
 & RUFF9.3.0 \\
 & RUFO6.2.3 \\
 & RSFO11.0.0 \\
 & RUFF14.0.0 \\
 & RUFF16.0.0 \\
 & RUFF20.4.0 \\
 & RUFO1.0.0 \\
 & RUFO7.0.0 \\
 & RSFO11.1.0 \\
 & RUFF14.1.0 \\
 & RUFF17.0.0 \\
 & RUFF6.1.2 \\
 & RUFO12.3.0 \\
 & RUFO8.0.0 \\
CS03 -- Gestione rubrica & RUFF4.3.0 \\
 & RUFF4.7.0 \\
 & RUFF4.4.0 \\
 & RUFF4.4.1 \\
 & RUFF4.4.2 \\
 & RUFF4.4.3 \\
 & RUFF4.0.0 \\
 & RUFF4.4.4 \\
 & RUFF4.1.0 \\
 & RUFF4.5.0 \\
 & RUFF4.2.0 \\
 & RUFF4.6.0 \\
 
CS04 -- Gestione autenticazione & RUFO1.0.0 \\
 & RUFO1.2.0 \\
 & RUFO2.0.0 \\
 & RSFD2.1.2 \\
 & RUFF3.0.0 \\
 & RUFF3.1.0 \\
 
CS05 -- Gestione rubrica & RUFF15.5.0 \\
 & RUFF15.0.0 \\
 & RUFF15.1.0 \\
 & RUFF15.2.0 \\
 & RUFF15.3.0 \\
 & RUFF15.4.0 \\
 
CS06 -- Gestione chiamata
 & RUFO6.1.0 \\
 & RUFO6.1.1 \\
 & RUFO6.1.3 \\
 & RUFO6.2.0 \\
 & RUFO6.2.1 \\
 & RUFO6.2.3 \\
 & RUFO6.4.0 \\
 & RUFO6.5.0 \\
 & RUFO7.0.0 \\
 & RUFO8.0.0 \\
 & RUFO8.1.0 \\
 & RUFO8.2.0 \\
 & RUFO9.0.0 \\
 & RUFO9.1.0 \\
 & RUFO9.2.0 \\
 & RSFO11.0.0 \\
 & RSFO11.1.0 \\
 & RSFO12.0.0 \\
 & RUFO12.3.0 \\
 
CS07 -- Façade del server & RUFF4.0.0 \\
 & RSFO12.0.0 \\
 & RUFO1.0.0 \\
 & RUFF13.0.0 \\
 & RUFO2.0.0 \\
 & RUFF15.0.0 \\
 & RUFO5.0.0 \\
 & RUFF16.0.0 \\
 & RUFO6.1.0 \\
 & RUFF17.0.0 \\
 & RUFO6.2.0 \\
 & RUFF18.0.0 \\
 & RUFF3.0.0 \\
 
CP01 -- Gestione comunicazione & RUFO6.1.0\\
& RUFO6.1.1\\
& RUFO6.1.3\\
& RUFO6.2.0\\
& RUFO6.2.1\\
& RUFO6.2.3\\
& RUFO6.4.0\\
& RUFO6.5.0\\
& RUFO7.0.0\\
& RUFO8.0.0\\
& RUFO8.1.0\\
& RUFO8.2.0\\
& RUFO9.0.0\\
& RUFO9.1.0\\
& RUFO9.2.0\\
& RUFF6.1.2\\
& RUFF6.1.4\\
& RUFF6.2.2\\
& RUFF6.2.4\\
& RUFF6.2.5\\
& RUFF6.3.0\\
& RSDO6.0.0\\
CP02 -- Rappresentazione dati & RUFO1.0.0\\
& RUFO1.2.0\\
& RUFO2.1.0\\
& RUFF3.0.0\\
& RUFF3.1.0\\
& RUFF3.2.0\\
CP03 -- Gestione GUI & RSFD21.0.0\\
& RSQF26.0.0\\
& RSDO10.0.0\\
& RSDO10.1.0\\
& RSFD21.0.0\\
CV02 -- Login & RUFO1.0.0\\
& RSFO1.2.0\\
& RSFO2.1.0\\
& RSDD2.2.0\\
\bottomrule
\end{longtable}
\end{center}
\subsection{Tracciamenti Componenti-DesignPattern}\label{sec:tracCompDp}

\begin{center}
\rowcolors{4}{lightblue}{llightblue}\begin{longtable}{lp{.55\textwidth}l}
\toprule Componenti & Design pattern utilizzati\\
\midrule
CP03 -- Gestione GUI & MVP \\
CS01 -- Gestione database & Data Access Object \\
& Singleton \\
& MVP \\
CS02 -- Gestione connessione & Factory Method \\
 & Singleton \\
CP01 -- Gestione comunicazione & Singleton \\
CS06 -- Gestione chiamate & MVP \\
CS05 -- Gestione rubrica & MVP \\
CS04 -- Gestione autenticazione & Strategy \\
CS03 -- Gestione rubrica & MVP \\
CS07 -- Façade del server & Façade \\
& MVP \\
\bottomrule
\end{longtable}
\end{center}
\subsection{Tracciamenti DesignPattern-Componenti}\label{sec:tracDpComp}

\begin{center}
\rowcolors{4}{lightblue}{llightblue}\begin{longtable}{lp{.55\textwidth}l}
\toprule Design pattern & Componenti\\
\midrule
Data Access Object & CS01 -- Gestione database\\
Façade & CS07 -- Façade del server\\
Factory Method & CS02 -- Gestione connessione\\
Singleton & CS02 -- Gestione connessione\\
Strategy & CS04 -- Gestione autenticazione\\
MVP & CS01 -- Gestione database\\
& CS03 -- Gestione rubrica\\
& CS05 -- Gestione rubrica\\
& CP01 -- Gestione comunicazione\\
& CS06 -- Gestione chiamate\\
& CP03 -- Gestione GUI\\
\bottomrule
\end{longtable}
\end{center}
\subsection{Tracciamenti Componenti-Classi}\label{sec:tracCompClass}

\begin{center}
\rowcolors{4}{lightblue}{llightblue}\begin{longtable}{lp{0.7\textwidth}l}
\toprule Componenti & Classi\\
\midrule
CS01 -- Gestione database
& server.dao.CallDAO\\
& server.dao.CallListDAO\\
& server.dao.GroupDAO\\
& server.dao.MessageDAO\\
& server.dao.UserDataDAO\\
& server.dao.AddressBookEntryDAO\\
& server.dao.HibernateUtil\\

CS03 -- Gestione rubrica & server.abook.AddressBookEntry\\
& server.abook.IAddressBookEntry\\
& server.abook.IGroup\\
& server.abook.Group\\
& server.abook.IUserData\\
& server.abook.UserData\\

CS05 -- Gestione rubrica & server.message.IMessage\\
& server.message.Message\\

CS06 -- Gestione chiamate
 & server.call.ICall\\
& server.call.Call\\
 & server.call.ICallList\\
& server.call.CallList\\

CS02 -- Gestione connessione & server.connection.PushInbound\\
& org.apache.catalina.websocket.MessageInbound\\

CS04 -- Gestione autenticazione & server.authentication.AuthenticationModule\\
& server.authentication.CredentialLoader\\
& server.authentication.PrincipalImpl\\
& server.authentication.IAuthenticationData\\
& server.authentication.AuthenticationData\\
& server.authentication.AESAlgorithm\\
& server.authentication.ISecurityStrategy\\
& javax.security.auth.spi.LoginModule\\
& javax.security.auth.callback.CallbackHandler\\
& javax.security.Principal\\

CS07 -- Façade del server & server.connection.ChannelServlet\\
& server.abook.servlet.AddressBookDoAddContactServlet\\
& server.abook.servlet.AddressBookDoRemoveContactServlet\\
& server.abook.servlet.AddressBookDoCreateGroupServlet\\
& server.abook.servlet.AddressBookDoDeletGroupServlet\\
& server.abook.servlet.AddressBookDoInsertInGroupServlet\\
& server.abook.servlet.AddressBookDoRemoveInGroupServlet\\
& server.abook.servlet.AddressBookDoBlockServlet\\
& server.abook.servlet.AddressBookDoUnblockServlet\\
& server.abook.servlet.AddressBookGetContactsServlet\\
& server.abook.servlet.AddressBookGetGroupsServlet\\
& server.abook.servlet.AddressBookDoSearchServlet\\
& server.authentication.servlet.LoginServlet\\
& server.authentication.servlet.LogoutServlet\\
& server.authentication.servlet.RegisterServlet\\
& server.message.servlet.InsertMessageServlet\\
& server.message.servlet.DeletMessageServlet\\
& server.message.servlet.UpdateStatusMessageServlet\\
& server.message.servlet.DownloadMessageListServlet\\
& server.call.servlet.DownloadCallHistoryManager\\
& javax.servlet.http.HttpServlet\\
& org.apache.catalina.websocket.WebSocketServlet\\

CP01 -- Gestione comunicazione & clientpresenter.kernel.CommunicationCenter\\
& PeerICECandidate\\
& WebKitRTCPeerConnection\\
& PeerSessionDescription\\

CP02 -- Rappresentazione dati & clientpresenter.data.JSCall\\
& clientpresenter.data.JSGroup\\
& clientpresenter.data.JSMessage\\
& clientpresenter.data.JSUserData\\

CP03 -- Gestione GUI & clientpresenter.guicontrol.AccountSettingsPanelPresenter\\
& clientpresenter.guicontrol.AddressBookPanelPresenter\\
& clientpresenter.guicontrol.CallHistoryPanelPresenter\\
& clientpresenter.guicontrol.CommunicationPanelPresenter\\
& clientpresenter.guicontrol.GroupPanelPresenter\\
& clientpresenter.guicontrol.SearchResultPanelPresenter\\
& clientpresenter.guicontrol.ContactPanelPresenter\\
& clientpresenter.guicontrol.LoginPanelPresenter\\
& clientpresenter.guicontrol.RegisterPanelPresenter\\
& clientpresenter.guicontrol.TopLevelPresenter\\
& clientpresenter.guicontrol.ChildPrenter\\
& clientpresenter.guicontrol.MainPanelPresenter\\
& clientpresenter.guicontrol.MessagePanelPresenter\\
& clientpresenter.guicontrol.PresenterMediator\\
& clientpresenter.guicontrol.ToolsPanelPresenter\\

CV01 -- GUI & clientview.MainPanel\\
& clientview.ToolsPanel\\
& clientview.AddressBookPanel\\
& clientview.ContactPanel\\
& clientview.MessagePanel\\
& clientview.GroupPanel\\
& clientview.SearchPanel\\
& clientview.AccountSettingsPanel\\
& clientview.CallHistoryPanel\\
& clientview.CommunicationPanel\\

CV02 -- Login & clientview.LoginPanel\\
& clientview.RegisterPanel\\

\bottomrule
\end{longtable}
\end{center}
\subsection{Tracciamenti Classi-Componenti}\label{sec:tracClassComp}

\begin{center}
\rowcolors{4}{lightblue}{llightblue}\begin{longtable}{lp{0.33\textwidth}l}
\toprule Classi & Componenti\\
\midrule

clientpresenter.data.JSCall & CP02 -- Rappresentazione dati\\
clientpresenter.data.JSGroup & CP02 -- Rappresentazione dati\\
clientpresenter.data.JSMessage & CP02 -- Rappresentazione dati\\
clientpresenter.data.JSUserData & CP02 -- Rappresentazione dati\\
clientpresenter.guicontrol.AccountSettingsPanelPresenter & CP03 -- Gestione GUI\\
clientpresenter.guicontrol.AddressBookPanelPresenter & CP03 -- Gestione GUI\\
clientpresenter.guicontrol.CallHistoryPanelPresenter & CP03 -- Gestione GUI\\
clientpresenter.guicontrol.ChildPrenter & CP03 -- Gestione GUI\\
clientpresenter.guicontrol.CommunicationPanelPresenter & CP03 -- Gestione GUI\\
clientpresenter.guicontrol.ContactPanelPresenter & CP03 -- Gestione GUI\\
clientpresenter.guicontrol.GroupPanelPresenter & CP03 -- Gestione GUI\\
clientpresenter.guicontrol.LoginPanelPresenter & CP03 -- Gestione GUI\\
clientpresenter.guicontrol.MainPanelPresenter & CP03 -- Gestione GUI\\
clientpresenter.guicontrol.MessagePanelPresenter & CP03 -- Gestione GUI\\
clientpresenter.guicontrol.PresenterMediator & CP03 -- Gestione GUI\\
clientpresenter.guicontrol.RegisterPanelPresenter & CP03 -- Gestione GUI\\
clientpresenter.guicontrol.SearchResultPanelPresenter & CP03 -- Gestione GUI\\
clientpresenter.guicontrol.ToolsPanelPresenter & CP03 -- Gestione GUI\\
clientpresenter.guicontrol.TopLevelPresenter & CP03 -- Gestione GUI\\
clientpresenter.kernel.CommunicationCenter & CP01 -- Gestione comunicazione\\
clientview.AccountSettingsPanel & CV01 -- GUI\\
clientview.AddressBookPanel & CV01 -- GUI\\
clientview.CallHistoryPanel & CV01 -- GUI\\
clientview.CommunicationPanel & CV01 -- GUI\\
clientview.ContactPanel & CV01 -- GUI\\
clientview.GroupPanel & CV01 -- GUI\\
clientview.LoginPanel & CV02 -- Login\\
clientview.MainPanel & CV01 -- GUI\\
clientview.MessagePanel & CV01 -- GUI\\
clientview.RegisterPanel & CV02 -- Login\\
clientview.SearchPanel & CV01 -- GUI\\
clientview.ToolsPanel & CV01 -- GUI\\
javax.security.auth.callback.CallbackHandler & CS04 -- Gestione autenticazione\\
javax.security.auth.spi.LoginModule & CS04 -- Gestione autenticazione\\
javax.security.Principal & CS04 -- Gestione autenticazione\\
javax.servlet.http.HttpServlet & CS07 -- Façade del server\\
org.apache.catalina.websocket.MessageInbound & CS02 -- Gestione connessione\\
org.apache.catalina.websocket.WebSocketServlet & CS07 -- Façade del server\\
PeerICECandidate & CP01 -- Gestione comunicazione\\
PeerSessionDescription & CP01 -- Gestione comunicazione\\
server.abook.AddressBookEntry & CS03 -- Gestione rubrica\\
server.abook.Group & CS03 -- Gestione rubrica\\
server.abook.IAddressBookEntry& CS03 -- Gestione rubrica\\
server.abook.IGroup & CS03 -- Gestione rubrica\\
server.abook.IUserData & CS03 -- Gestione rubrica\\
server.abook.servlet.AddressBookDoAddContactServlet & CS07 -- Façade del server\\
server.abook.servlet.AddressBookDoBlockServlet & CS07 -- Façade del server\\
server.abook.servlet.AddressBookDoCreateGroupServlet & CS07 -- Façade del server\\
server.abook.servlet.AddressBookDoDeletGroupServlet & CS07 -- Façade del server\\
server.abook.servlet.AddressBookDoInsertInGroupServlet & CS07 -- Façade del server\\
server.abook.servlet.AddressBookDoRemoveContactServlet & CS07 -- Façade del server\\
server.abook.servlet.AddressBookDoRemoveInGroupServlet & CS07 -- Façade del server\\
server.abook.servlet.AddressBookDoSearchServlet & CS07 -- Façade del server\\
server.abook.servlet.AddressBookDoUnblockServlet & CS07 -- Façade del server\\
server.abook.servlet.AddressBookGetContactsServlet & CS07 -- Façade del server\\
server.abook.servlet.AddressBookGetGroupsServlet & CS07 -- Façade del server\\
server.abook.UserData & CS03 -- Gestione rubrica\\
server.authentication.AESAlgorithm & CS04 -- Gestione autenticazione\\
server.authentication.AuthenticationData & CS04 -- Gestione autenticazione\\
server.authentication.AuthenticationModule & CS04 -- Gestione autenticazione\\
server.authentication.CredentialLoader & CS04 -- Gestione autenticazione\\
server.authentication.IAuthenticationData & CS04 -- Gestione autenticazione\\
server.authentication.ISecurityStrategy & CS04 -- Gestione autenticazione\\
server.authentication.PrincipalImpl & CS04 -- Gestione autenticazione\\
server.authentication.servlet.LoginServlet & CS07 -- Façade del server\\
server.authentication.servlet.LogoutServlet & CS07 -- Façade del server\\
server.authentication.servlet.RegisterServlet & CS07 -- Façade del server\\
server.dao.CallDAO & CS01 -- Gestione database\\
server.dao.CallListDAO & CS01 -- Gestione database\\
server.call.Call & CS06 -- Gestione chiamate\\
server.call.ICall & CS06 -- Gestione chiamate\\
server.call.CallList & CS06 -- Gestione chiamate\\
server.call.ICallList & CS06 -- Gestione chiamate\\
server.call.servlet.DownloadCallHistoryManager & CS07 -- Façade del server\\
server.connection.ChannelServlet & CS07 -- Façade del server\\
server.connection.PushInbound & CS02 -- Gestione connessione\\
server.dao.AddressBookEntryDAO & CS01 -- Gestione database\\
server.dao.GroupDAO & CS01 -- Gestione database\\
server.dao.HibernateUtil & CS01 -- Gestione database\\
server.dao.MessageDAO & CS01 -- Gestione database\\
server.dao.UserDataDAO & CS01 -- Gestione database\\
server.message.IMessage & CS05 -- Gestione rubrica\\
server.message.Message & CS05 -- Gestione rubrica\\
server.message.servlet.DeletMessageServlet & CS07 -- Façade del server\\
server.message.servlet.DownloadMessageListServlet & CS07 -- Façade del server\\
server.message.servlet.InsertMessageServlet & CS07 -- Façade del server\\
server.message.servlet.UpdateStatusMessageServlet & CS07 -- Façade del server\\
WebKitRTCPeerConnection & CP01 -- Gestione comunicazione\\

\bottomrule
\end{longtable}
\end{center}


\end{document}
% per tabelle con colori alternati inserire prima della tabella l'istruzione seguente:
% \rowcolors{3}{lightblue}{llightblue}