% analisi_dei_requisiti/analisi_dei_requisiti.tex

% **************************************************
% Macro specifiche per il documento corrente
% **************************************************
% Nome
\newcommand{\docName}{Analisi dei requisiti}
% Nome file
\newcommand{\docFileName}{analisi\_dei\_requisiti.4.0.pdf}
% Versione
\newcommand{\docVers}{4.0}
% Data creazione
\newcommand{\creationDate}{2012-12-03}
% Data ultima modifica
\newcommand{\modificationDate}{2013-05-05}
% Stato in {Approvato, Non approvato}
\newcommand{\docState}{Approvato}
% Uso in {Interno, Esterno}
\newcommand{\docUsage}{Esterno}
% Destinatari da specificare come nome1\\ &nome2\\ ecc.
\newcommand{\docDistributionList}{Prof. Tullio Vardanga\\ & Prof. Riccardo Cardin\\ & Dott. Gregorio Piccoli \\& Team \team{}}
% Redattori da specificare come nome1\\ &nome2\\ ecc.
\newcommand{\docAuthors}{Andrea Meneghinello \\& Andrea Rizzi\\& Diego Beraldin\\& Elena Zecchinato\\& Marco Schivo\\& Riccardo Tresoldi\\& Stefano Farronato}
% Approvato da
\newcommand{\approvedBy}{Marco Schivo}
% Verificatori
\newcommand{\verifiedBy}{Riccardo Tresoldi\\}
% Perscorso (relativo o assoluto) che punta alla directory contenente shared/
% come sua sottodirectory (per comodità chiamiamola 'doc root').
\newcommand{\docRoot}{..}
% definire se si vuole l'indice delle tabelle
\def\INDICETABELLE{false}
% definire se si vuole l'indice delle figure
\def\INDICEFIGURE{true}

% importa il preambolo condiviso da tutti i documenti
% shared/preamble.tex
%
% Questo documento contiene la parte del preambolo condivisa e viene pertanto
% richiamato nel 'master' di tutti i documenti di progetto.  Al suo interno
% contiene le inclusioni (e le configurazioni) di tutti i package richiesti per
% la compilazione dei documenti, le macro di carattere generale e la definizione
% degli stili di pagina.

\documentclass[a4paper,10pt]{article}

% **************************************************
% Macro generiche
% **************************************************
\newcommand{\team}{Software Synthesis}                    % chi siamo
\newcommand{\email}{info@softwaresynthesis.org}           % e-mail
\newcommand{\caName}{MyTalk}                              % titolo capitolato
\newcommand{\manager}{SynthesisRequirementManager}        % nome del sistema di tracciamento
\newcommand{\memberdata}[1]{%
  \texttt{\textcolor{RedOrange}{#1}}}                     % attributi di una classe
\newcommand{\method}[1]{\texttt{\textcolor{Emerald}{#1}}} % metodi di una classe
\newcommand{\exception}[1]{%
  \texttt{\textcolor{RedViolet}{#1}}}                     % eccezione
% \newcommand{\handler}[1]{\texttt{\textcolor{Maroon}{#1}}} % per gli event handler
\newcommand{\inglese}[1]{%
  \foreignlanguage{english}{\textit{#1}}}                 % per i testi in lingua inglese
\newcommand{\purpose}{%                                     scopo del prodotto
Con il progetto ``\caName'' si intende un sistema software di comunicazione tra utenti mediante \underline{browser} senza la necessit{\`a} di installazione di \underline{plugin} e/o software esterni. L'utilizzatore avr{\`a} la possibilit{\`a} di interagire con un altro utente tramite una comunicazione audio - audio/video - testuale e, inoltre, ottenere delle statistiche sull'attivit{\`a} in tempo reale.%
}
\newcommand{\glossaryIntro}{%                               introduzione al glossario
Al fine di evitare incomprensioni dovute all'uso di termini tecnici nei documenti, viene redatto e allegato il documento \textit{glossario.4.0.pdf} dove vengono definiti e descritti tutti i termini marcati con una sottolineatura.%
}


% **************************************************
% Codifica e lingua dei documenti
% **************************************************
\usepackage[utf8x]{inputenc}                              % codifica caratteri dei documenti sorgenti
\usepackage[english,italian]{babel}                       % localizzazione ai fini di sillabazione e cross-references
\usepackage[T1]{fontenc}                                  % codifica font di output

% **************************************************
% Definizione geometria della pagina
% **************************************************
\usepackage[a4paper,head=4cm,top=4.5cm,bottom=3cm,left=3cm,right=3cm,bindingoffset=5mm]{geometry}

% *************************************************
% Intestazioni e piè di pagina personalizzati
% *************************************************
\usepackage{fancyhdr}
% stile normale
\fancypagestyle{normal}{
\fancyhead{}                                              % intestazione
\fancyhead[RE,RO]{
\begin{picture}(0,0)
  \put(-410,0){\includegraphics[width=1.02\textwidth]{header_logo}}
  \put(-410,10){\sffamily\large\leftmark}
\end{picture}
\vspace{-4pt}
}
\renewcommand{\headrulewidth}{0pt}                       % riga sotto l'intestazione
\cfoot{}                                                  % piè di pagina
\fancyfoot[RO,LE]{\sffamily
  pag.~\thepage{} di \pageref{LastPage}}                  % a dx nelle pag. dispari e a sx in quelle pari
\fancyfoot[RE,LO]{\sffamily\docFileName{}}
\renewcommand{\footrulewidth}{.4pt}                       % riga sopra il piè di pagina
}
% stile per gli indici
\fancypagestyle{toc}{
\fancyhead{}                                              % intestazione
\fancyhead[RE,RO]{
\begin{picture}(0,0)
  \put(-410,0){\includegraphics[width=1.02\textwidth]{header_logo}}
\end{picture}
}
\renewcommand{\headrule}{}                                % nessuna riga sotto l'intestazione
\cfoot{}                                                  % piè di pagina
\fancyfoot[RO,LE]{\sffamily\thepage{}}                    % a dx nelle pag. dispari e a sx in quelle pari
\fancyfoot[RE,LO]{\sffamily\docFileName{} -- v.\docVers}
\renewcommand{\footrulewidth}{.4pt}                       % riga sopra il piè di pagina
}

\pagestyle{fancy}                                         % premetto: non so usare bene le marche:
\renewcommand{\sectionmark}[1]{\markboth{#1}{#1}}         % se qualcuno ha idee migliori si faccia avanti!

% **************************************************
% Tabelle
% **************************************************
\usepackage{tabularx}                                     % tabelle di larghezza fissa con una o più colonne variabili
\usepackage{multirow}                                     % colonne con colonne che si estendono per più righe
\usepackage{booktabs}                                     % per inserire l'ambiente table e le righe orizz. nelle tabelle
\usepackage{longtable}			                              % tabelle oltre i limiti di pagina

% **************************************************
% Cross-references e collegamenti ipertestuali
% **************************************************
\usepackage[hidelinks]{hyperref}
\hypersetup{%
  colorlinks=false, linktocpage=false, pdfborder={0,0,0}, pdfstartpage=1, pdfstartview=FitV,%
  urlcolor=Cyan, linkcolor=Cyan, citecolor=Black, %pagecolor=Black,%
  pdftitle={\docName}, pdfauthor={\team}, pdfsubject={}, pdfkeywords={},%
  pdfcreator={pdflatex}, pdfproducer={pdflatex with hyperref package}%
}

% **************************************************
% Immagini e grafica
% **************************************************
\usepackage{graphicx}                                     % supporto ad aspetti avanzati delle immagini
\usepackage[table,usenames,dvipsnames]{xcolor}            % tabelle con righe colorate e alternate
\graphicspath{{\docRoot/pics/}}                           % percorso contenente tutti i file immagini
\usepackage{float}                                        % per rendere non flottanti gli ambienti flottanti
\usepackage[italian]{varioref}                            % testo completo riferimenti in italiano

% **************************************************
% Definizioni di colori
% **************************************************
\definecolor{myBlue}{RGB}{1,167,236}
\definecolor{lightblue}{RGB}{213,243,253}%{119,218,247}
\definecolor{llightblue}{RGB}{229,255,255}

% **************************************************
% Altri pacchetti opzionali
% **************************************************     
\usepackage{lastpage}                                     % per sapere il numero totale di pagine
\usepackage{eurosym}                                      % per il simbolo dell'euro usare \EUR{x} dove x è l'importo
\usepackage{ifthen}                                       % permette la scelta di rami condizionali nella compilazione
\usepackage{enumitem}                                     % permette di configurare gli elenchi puntati e numerati


% Fine del preambolo e inizio del documento
\begin{document}

% Inclusione della prima pagina
% shared/firstpage.tex
%
% Questo documento definisce il contenuto della prima pagina, che si suppone
% essere uguale in tutti i documenti.  Oltre al logo e al titolo, la prima
% pagina contiene i metadati relativi al documento in cui viene inclusa.


% rimuove intestazioni e piè di pagina
\pagestyle{empty}

\begin{center}

% logo del gruppo
\includegraphics[width=1.5\textwidth]{logo}

\vspace{1in}

% titolo del documento
{\Huge\bfseries \docName}

\vspace{1in}

% tabella riepilogativa
\begin{tabularx}{.7\textwidth}{>{\bfseries\sffamily}l>{\sffamily}l}
\toprule
\multicolumn{2}{>{\sffamily}c}{Informazioni sul documento}\\
\midrule
Nome file:            & \docFileName\\
Versione:             & \docVers\\
Data creazione:       & \creationDate\\
Data ultima modifica: & \modificationDate\\
Stato:                & \docState\\
Uso:                  & \docUsage\\
Redattori:            & \docAuthors\\
Approvato da:         & \approvedBy\\
Verificatori:         & \verifiedBy\\
\bottomrule
\end{tabularx}

\end{center}

\newpage


% Storico delle modifiche
\section*{Storia delle modifiche}
\begin{center}
\begin{longtable}{lp{.32\textwidth}lll}
\toprule
Versione & Descrizione intervento & Membro & Ruolo & Data\\
\midrule % inserire qui il contenuto della tabella

4.0 & Approvazione documento. & Marco Schivo & Responsabile & 2013-05-05\\
3.5 & Verifica documento. & Riccardo Tresoldi & Verificatore & 2013-04-27\\
3.4 & Aggiunti requisiti di qualità di processo. & Stefano Farronato & Verificatore & 2013-04-25\\
3.3 & Aggiornati tracciamenti. & Andrea Rizzi & Verificatore & 2013-04-12\\
3.2 & Modificato il testo di alcuni requisiti funzionali, rendendolo più chiaro in riferimento ai sotto-requisiti. & Stefano Farronato & Verificatore & 2013-04-11 \\
3.1 & Migliorata l'atomicità dei requisiti: RSFO1.2.0, RSFO2.1.0, RUFF3.0.0, RUFF3.1.0, RUFF4.3.0, RUFF4.7.0, RUFF5.1.0 & Andrea Rizzi & Verificatore & 2013-04-10\\
3.0 & Approvazione documento & Diego Beraldin & Responsabile & 2013-02-15\\
2.3 & Verifica documento & Andrea Meneghinello & Verificatore & 2013-02-15\\
2.2 & Segmentazione dei reguisiti segnalati in RP & Marco Schivo & analista & 2013-02-14\\
2.1 & Correzione nomenclatura requisiti di qualità e correzione tabella tracciamenti componenti-test & Marco Schivo & analista & 2013-02-13\\
2.0 & Approvazione del documento & Andrea Meneghinello & Responsabile & 2013-01-18\\
1.11 & Verifica lessicale e grammaticale del documento & Andrea Rizzi & Verificatore & 2013-01-18\\
1.10 & Correzione errori rilevati dal verificatore & Marco Schivo & Analista & 2013-01-17\\
1.9 & Verifica corrispondenza e coerenza dei dati nelle tabelle e diagrammi & Andrea Rizzi & Verificatore & 2013-01-15\\
1.8 & Inserimento sezione Requisiti-Test & Marco Schivo & Analista & 2013-01-16\\
1.7 & Revisione modalità di visualizzazione codice dei requisiti, aggiunta tabella tracciamento requisiti-casi d'uso. & Marco Schivo & Analista & 2013-01-15\\
1.6 & Aggiunti vincoli riportati nel capitolato ma non riportati in RR, aggiunti riferimenti espliciti ai codici dei corrispondenti casi d'uso. & Diego Beraldin & Analista & 2013-01-15\\
1.5 & Correzione UC2.0 e UC2.5 Correzione RUFO 1.0.0 RUFO2.0.0 RUFO 3.0.0 & Marco Schivo & Analista & 2013-01-14\\
1.4 & Specializzato UC2.2. Correzione errori ortografici. Correzione UC2.1. Correzione UC2.5. & Diego Beraldin & Analista & 2013-01-12\\
1.3 &  Modificata Sez. ``Assunzioni e prerequisiti''. Modificato UC1 & Diego Beraldin & Analista & 2013-01-12\\
1.2 & Correzione errori rilevati in RR: aggiornamento tabella delle modifiche con l'aggiunta dei ruoli. Aggiunta la lista di distribuzione. Aggiunto sommario. Aggiornati i riferimenti normativi. & Diego Beraldin & Analista & 2013-01-12\\
1.1 & Correzione errori ortografici non rilevati. & Diego Beraldin & Analista & 2013-01-09\\
1.0 & Approvazione documento. & Elena Zecchinato & Responsabile & 2012-12-18\\
0.13 & Verifica dell'intero documento. & Diego Beraldin & Verificatore & 2012-12-18\\
0.12 & Correzione casi d'uso errati. Aggiunta requisiti mancanti. Fine stesura documento. & Andrea Rizzi & Analista & 2012-12-17\\
0.11 & Seconda stesura dei casi d'uso e loro inserimento nel sistema \manager. & Riccardo Tresoldi & Analista & 2012-12-15\\
0.10 & Correzione requisiti errati. Correzione casi d'uso errati. & Stefano Farronato & Analista & 2012-12-14\\
0.9 & Seconda stesura dei requisiti e loro inserimento nel sistema \manager. & Andrea Meneghinello & Analista & 2012-12-13\\
0.8 & Incontro con il proponente. Stesura dei requisiti emersi. Correzione requisiti errati. & Riccardo Tresoldi & Analista & 2012-12-10\\
0.7 & Verifica documento fino alla versione 0.4. Correzione errori nel punto ``funzionalità''. & Marco Schivo & Verificatore & 2012-12-08\\
0.6 & Prima stesura dei casi d'uso e loro inserimento nel sistema \manager. Stesura del punto ``Requisiti'' (introduzione) & Riccardo Tresoldi & Analista & 2012-12-07\\
0.5 & Continuazione stesura dei requisiti e loro inserimento nel sistema \manager. Correzione requisiti errati. & Diego Beraldin & Analista & 2012-12-06\\
0.4 & Prima stesura dei requisiti e loro inserimento nel sistema \manager. & Andrea Rizzi & Analista & 2012-12-05\\
0.3 & Stesura del punto ``Funzionalità'' e ``Assunzioni e prerequisiti''. & Diego Beraldin & Analista & 2012-12-04\\
0.2 & Stesura del punto ``Riferimenti'' e ``Descrizione generale''. & Riccardo Tresoldi & Analista & 2012-12-03\\
0.1 & Creazione del documento e stesura del punto ``Scopo del documento''. & Andrea Rizzi & Analista & 2012-12-03\\
\bottomrule
\end{longtable}
\end{center}
\newpage

% inclusione dell'indice
% shared/toc.tex
%
% Questo file contiene le istruzioni che generano l'indice o gli indici del
% documento (utile nel caso in cui decidessimo di avere anche un indice delle
% tabelle e/o un indice delle figure).

% imposta lo stile di pagina per i titoli definito nel preambolo
\pagestyle{toc}
% imposta i numeri di pagina romani minuscoli
\pagenumbering{roman}

% genera automaticamente l'indice di LaTeX
\tableofcontents

% se è true \INDICETABELLE allora genera l'indice delle tabelle, altrimenti non fa nulla
\ifthenelse{\equal{\INDICETABELLE}{true}}{%
  \clearpage % l'indice delle tabelle, se c'è, deve andare a pagina nuova
  \listoftables
}{}

% se è true |INDICEFIGURE allora genera l'indice delle figure, altrimenti non fa nulla
\ifthenelse{\equal{\INDICEFIGURE}{true}}{%
  \clearpage % l'indice delle figure, se c'è, deve andare a pagina nuova
  \listoffigures
}{}

%in ogni caso occorre andare a pagina nuova dopo gli indici
\clearpage


% Alcuni aggiustamenti per le pagine
\pagenumbering{arabic}
\setcounter{page}{1}
\pagestyle{normal}

% Qui ha inizio il documento vero e proprio...

\begin{abstract}
Il presente documento, nasce con la necessità di evidenziare i requisiti che costituiscono il sistema e i casi d'uso associati. Inoltre viene proposta la tabella di tracciamento requisiti-casi e requisiti-fonti.
\end{abstract}

\newpage

\section{Introduzione}
\subsection{Scopo del prodotto}
\purpose

\subsection{Scopo del documento}
Il presente documento riporta il risultato dell'attività di analisi dei requisiti svolta dal gruppo \team{} in fase preliminare, vale a dire un insieme di requisiti -- in parte esplicitati nel testo del capitolato C1, in parte emersi durante incontri con il committente Zucchetti S.r.l., in parte inferiti dal dominio e in parte auto-imposti dal gruppo -- che il prodotto software è tenuto a soddisfare in termini funzionali, prestazionali, qualitativi e dichiarativi.

Il comportamento del sistema osservabile dall'utente finale, in conformità con i requisiti sopra enunciati, è riportato secondo il formalismo noto come diagrammi dei casi d'uso. La corrispondenza fra casi d'uso e requisiti è illustrata mediante la tabella riportata nella sezione~\ref{sec:tracciamento}.
%TODO: aggiungere riferimenti incrociati e non basarsi su numeri hard-coded qui dentro

\subsection{Glossario}
\glossaryIntro

\clearpage
\section{Riferimenti}

\subsection{Normativi}
\begin{itemize}
%\item[] verbale \textit{verbale\_incontro\_2012-12-11.pdf} allegato;
\item[] \textit{piano\_di\_qualifica.3.0.pdf} allegato.
\item[] \textit{norme\_di\_progetto.3.0.pdf} allegato.
\end{itemize}

\subsection{Informativi}
\begin{itemize}
\item[] Capitolato d'appalto: \caName{}, v1.0, redatto e rilasciato dal proponente Zucchetti s.r.l. reperibile all'indirizzo \url{http://www.math.unipd.it/~tullio/IS-1/2012/Progetto/C1.pdf};
\item[] testo di consultazione: \textit{Software Engineering (8th edition) Ian Sommerville, Pearson Education | Addison Wesley};
\item[] \textit{glossario.3.0.pdf} allegato.
\end{itemize}

\clearpage
\section{Descrizione generale}

\subsection{Contesto di utilizzo}
Il sistema software realizzato nell'ambito del progetto \caName{} si configura come una piattaforma di comunicazione fra utenti connessi alla rete. Oggetto della condivisione -- poiché ``comunicare'' letteralmente significa ``mettere in comune'' -- saranno non solo flussi audio/video, secondo le modalità tipiche della (video)chiamata via \inglese{softphone}, ma anche file e documenti presenti in locale, per i quali è previsto tanto il trasferimento quanto la condivisione delle modifiche in tempo reale.\\
Dal punto di vista dell'utente, \caName{} si configura inoltre come un applicativo fruibile attraverso la sola mediazione di un \underline{browser} per la navigazione web, senza necessità di installazione di alcun programma \inglese{stand-alone} sul proprio sistema né di \underline{plugin} per il browser forniti da \team{} o da terze parti.\\
Al fine di gestire l'utenza è prevista la realizzazione di una parte di \inglese{back-end} da installare in un \underline{server} sotto il controllo del fornitore del servizio agli utenti. L'installazione dell'applicativo lato {server} è prevista in un ambiente \underline{TomCat}.

\subsection{Utente finale}
Il modello di utente finale di riferimento prevede la disponibilità di una connessione a Internet funzionante e di un browser di ultima generazione in grado di supportare lo standard \underline{WebRTC}.\\
Fra i prerequisiti non figura il possesso di conoscenze particolari estranee alla comune esperienza di navigazione web (in particolare, l'eventuale confidenza con sistemi esistenti di telefonia via internet non rientra nelle assunzioni).

\subsection{Funzionalità}
Le funzionalità offerte dal prodotto agli utenti finali possono essere suddivise in due categorie principali:
\begin{itemize}
  \item gestione di scambi di informazioni in tempo reale, in particolare
  \begin{itemize}
  \item[--] comunicazione audio e audio/video tra due (o più) utenti;
  \item[--] chat testuale;
  \item[--] condivisione di risorse (ad esempio lavagna grafica, schermo e documenti PDF). 
  \end{itemize}
  \item scambio di informazioni asincrone 
  \begin{itemize}
  \item[--] segreteria (sia per messaggi vocali che videomessaggi);
  \item[--] trasferimento file.
  \end{itemize}
\end{itemize}
A integrazione e supporto di tali funzionalità di base è previsto lo sviluppo di funzionalità secondarie quali l'autenticazione degli utenti, la gestione della rubrica, la visualizzazione di metadati sulle connessioni e la possibilità di mantenere un registro storico delle attività realizzate mediante il sistema.

\subsection{Assunzioni e prerequisiti}
%In fase di \underline{\inglese{deployment}} si presuppone che l'installatore presso il committente sia in possesso delle competenze necessarie per l'installazione e la configurazione della parte server del sistema.
Si assume che l'utente sia in possesso di una connessione a internet funzionante al fine di instaurare la comunicazione. È altresì assunto implicitamente l'utilizzo di un browser con supporto a WebRTC.

\clearpage
%da qua in poi mi affido al software di stampa automatico
\subsection{Tracciamenti Requisiti-Componenti}\label{sec:tracRecComp}

\begin{center}
\rowcolors{4}{lightblue}{llightblue}\begin{longtable}{lp{.55\textwidth}l}
\toprule Requisiti &  Componenti\\
\midrule
RUFO1.0.0 & CS07 -- Façade del server \\
 & CS02 -- Gestione connessione \\
 & CS01 -- Gestione database \\
 & CS04 -- Gestione autenticazione \\
RUFD1.1.0 & CS01 -- Gestione database \\
RUFD1.1.2 & CS01 -- Gestione database \\
RSQO1.2.0 & CS02 -- Gestione connessione \\
 & CS01 -- Gestione database \\
  & CS04 -- Gestione autenticazione \\
RUFO2.0.0 & CS01 -- Gestione database \\
 & CS07 -- Façade del server \\
  & CS04 -- Gestione autenticazione \\
RSQO2.1.0 & CS01 -- Gestione database \\
 & CS04 -- Gestione autenticazione \\
RSDD2.2.0 & CS01 -- Gestione database \\
RUFF3.0.0 & CS07 -- Façade del server \\
 & CS01 -- Gestione database \\
  & CS04 -- Gestione autenticazione \\
RUFF3.1.0 & CS01 -- Gestione database \\
 & CS04 -- Gestione autenticazione \\
RUFF3.2.0 & CS01 -- Gestione database \\
RUFF4.0.0 & CS03 -- Gestione rubrica \\
 & CS07 -- Façade del server \\
 & CS01 -- Gestione database \\
RUFF4.1.0 & CS01 -- Gestione database \\
 & CS03 -- Gestione rubrica \\
RUFF4.2.0 & CS01 -- Gestione database \\
 & CS03 -- Gestione rubrica \\
RUFF4.3.0 & CS01 -- Gestione database \\
 & CS03 -- Gestione rubrica \\
RUFF4.4.0 & CS01 -- Gestione database \\
 & CS03 -- Gestione rubrica \\
RUFF4.4.1 & CS01 -- Gestione database \\
 & CS03 -- Gestione rubrica \\
RUFF4.4.2 & CS03 -- Gestione rubrica \\
 & CS01 -- Gestione database \\
RUFF4.4.3 & CS01 -- Gestione database \\
 & CS03 -- Gestione rubrica \\
RUFF4.4.4 & CS01 -- Gestione database \\
 & CS03 -- Gestione rubrica \\
RUFF4.5.0 & CS01 -- Gestione database \\
 & CS03 -- Gestione rubrica \\
RUFF4.6.0 & CS03 -- Gestione rubrica \\
RUFF4.7.0 & CS01 -- Gestione database \\
 & CS03 -- Gestione rubrica \\
RUFO5.0.0 & CS01 -- Gestione database \\
 & CS07 -- Façade del server \\
RUFF5.1.0 & CS01 -- Gestione database \\
RSDO6.0.0 & CS02 -- Gestione connessione \\
RUFO6.1.0 & CS02 -- Gestione connessione \\
 & CS07 -- Façade del server \\
  & CS06 -- Gestione chiamata \\
RUFO6.1.1 & CS02 -- Gestione connessione \\
  & CS06 -- Gestione chiamata \\
RUFF6.1.2 & CS02 -- Gestione connessione \\
RUFO6.1.3 & CS02 -- Gestione connessione \\
  & CS06 -- Gestione chiamata \\
RUFF6.1.4 & CS02 -- Gestione connessione \\
RUFO6.2.0 & CS07 -- Façade del server \\
 & CS02 -- Gestione connessione \\
   & CS06 -- Gestione chiamata \\
RUFO6.2.1 & CS02 -- Gestione connessione \\
  & CS06 -- Gestione chiamata \\
RUFF6.2.2 & CS02 -- Gestione connessione \\
RUFO6.2.3 & CS02 -- Gestione connessione \\
  & CS06 -- Gestione chiamata \\
RUFF6.2.4 & CS02 -- Gestione connessione \\
RUFF6.2.5 & CS02 -- Gestione connessione \\
RUFF6.3.0 & CS02 -- Gestione connessione \\
RUFO6.4.0   & CS06 -- Gestione chiamata \\
RUFO 6.5.0  & CS06 -- Gestione chiamata \\
RUFO7.0.0 & CS02 -- Gestione connessione \\
  & CS06 -- Gestione chiamata \\
RUFO8.0.0 & CS02 -- Gestione connessione \\
  & CS06 -- Gestione chiamata \\
RUFO8.1.0 & CS02 -- Gestione connessione \\
  & CS06 -- Gestione chiamata \\
RUFO8.2.0 & CS02 -- Gestione connessione \\
  & CS06 -- Gestione chiamata \\
RUFO9.0.0 & CS02 -- Gestione connessione \\
  & CS06 -- Gestione chiamata \\
RUFO9.1.0 & CS02 -- Gestione connessione \\
  & CS06 -- Gestione chiamata \\
RUFO9.2.0 & CS02 -- Gestione connessione \\
  & CS06 -- Gestione chiamata \\
RUFF9.3.0 & CS02 -- Gestione connessione \\
RSFO11.0.0 & CS02 -- Gestione connessione \\
  & CS06 -- Gestione chiamata \\
RSFO11.1.0 & CS02 -- Gestione connessione \\
  & CS06 -- Gestione chiamata \\
RSFF11.2.0 & CS02 -- Gestione connessione \\
RSFO12.0.0 & CS02 -- Gestione connessione \\
 & CS07 -- Façade del server \\
   & CS06 -- Gestione chiamata \\
RSFF12.1.0 & CS02 -- Gestione connessione \\
RUFD12.2.0 & CS02 -- Gestione connessione \\
RUFO12.3.0 & CS02 -- Gestione connessione \\
  & CS06 -- Gestione chiamata \\
RUFF13.0.0 & CS07 -- Façade del server \\
 & CS02 -- Gestione connessione \\
RUFD13.1.0 & CS02 -- Gestione connessione \\
RUFF13.2.0 & CS02 -- Gestione connessione \\
RUFF14.0.0 & CS02 -- Gestione connessione \\
RUFF14.1.0 & CS02 -- Gestione connessione \\
RUFF14.2.0 & CS02 -- Gestione connessione \\
RUFF15.0.0 & CS05 -- Gestione rubrica \\
 & CS01 -- Gestione database \\
 & CS02 -- Gestione connessione \\
 & CS07 -- Façade del server \\
RUFF15.1.0 & CS05 -- Gestione rubrica \\
 & CS01 -- Gestione database \\
RUFF15.2.0 & CS05 -- Gestione rubrica \\
 & CS02 -- Gestione connessione \\
 & CS01 -- Gestione database \\
RUFF15.3.0 & CS05 -- Gestione rubrica \\
 & CS02 -- Gestione connessione \\
 & CS01 -- Gestione database \\
RUFF15.4.0 & CS01 -- Gestione database \\
 & CS05 -- Gestione rubrica \\
 & CS02 -- Gestione connessione \\
RUFF15.5.0 & CS05 -- Gestione rubrica \\
 & CS02 -- Gestione connessione \\
 & CS01 -- Gestione database \\
 & CS02 -- Gestione connessione \\
 & CS07 -- Façade del server \\
 & CS02 -- Gestione connessione \\
 & CS07 -- Façade del server \\
RUFF18.0.0 & CS07 -- Façade del server \\
 & CS01 -- Gestione database \\
 & CS02 -- Gestione connessione \\
RUFF19.0.0 & CS02 -- Gestione connessione \\
RUFF20.0.0 & CS02 -- Gestione connessione \\
RUFF20.1.0 & CS02 -- Gestione connessione \\
RUFF20.2.0 & CS02 -- Gestione connessione \\
RUFF20.3.0 & CS02 -- Gestione connessione \\
RUFF20.4.0 & CS02 -- Gestione connessione \\
\bottomrule
\end{longtable}
\end{center}
\subsection{Tracciamenti Componenti-Requisiti}\label{sec:tracCompRec}

\begin{center}
\rowcolors{4}{lightblue}{llightblue}\begin{longtable}{lp{.55\textwidth}l}
\toprule Componenti & Requisiti associati\\
\midrule
CS01 -- Gestione database & RUFD1.1.0 \\
 & RUFF18.0.0 \\
 & RUFF4.4.0 \\
 & RUFO1.0.0 \\
 & RUFD1.1.2 \\
 & RUFF3.0.0 \\
 & RUFF4.4.1 \\
 & RUFO2.0.0 \\
 & RUFO5.0.0 \\
 & RUFF15.0.0 \\
 & RUFF3.1.0 \\
 & RUFF4.4.2 \\
 & RUFF15.1.0 \\
 & RUFF3.2.0 \\
 & RUFF4.4.3 \\
 & RUFF15.2.0 \\
 & RUFF4.0.0 \\
 & RUFF4.4.4 \\
 & RSDD2.2.0 \\
 & RUFF15.3.0 \\
 & RUFF4.1.0 \\
 & RUFF4.5.0 \\
 & RSQO1.2.0 \\
 & RUFF15.4.0 \\
 & RUFF4.2.0 \\
 & RUFF4.7.0 \\
 & RSQO2.1.0 \\
 & RUFF15.5.0 \\
 & RUFF4.3.0 \\
 & RUFF5.1.0 \\
CS02 -- Gestione connessione & RSFO12.0.0 \\
 & RUFF14.2.0 \\
 & RUFF18.0.0 \\
 & RUFF6.1.4 \\
 & RUFO6.1.0 \\
 & RUFO8.1.0 \\
 & RSQO1.2.0 \\
 & RUFF15.0.0 \\
 & RUFF19.0.0 \\
 & RUFF6.2.2 \\
 & RUFO6.1.1 \\
 & RUFO8.2.0 \\
 & RUFD12.2.0 \\
 & RUFF15.2.0 \\
 & RUFF20.0.0 \\
 & RUFF6.2.4 \\
 & RUFO6.1.3 \\
 & RUFO9.0.0 \\
 & RSDO6.0.0 \\
 & RUFD13.1.0 \\
 & RUFF15.3.0 \\
 & RUFF20.1.0 \\
 & RUFF6.2.5 \\
 & RUFO6.2.0 \\
 & RUFO9.1.0 \\
 & RSFF11.2.0 \\
 & RUFF13.0.0 \\
 & RUFF15.4.0 \\
 & RUFF20.2.0 \\
 & RUFF6.3.0 \\
 & RUFO6.2.1 \\
 & RUFO9.2.0 \\
 & RSFF12.1.0 \\
 & RUFF13.2.0 \\
 & RUFF15.5.0 \\
 & RUFF20.3.0 \\
 & RUFF9.3.0 \\
 & RUFO6.2.3 \\
 & RSFO11.0.0 \\
 & RUFF14.0.0 \\
 & RUFF16.0.0 \\
 & RUFF20.4.0 \\
 & RUFO1.0.0 \\
 & RUFO7.0.0 \\
 & RSFO11.1.0 \\
 & RUFF14.1.0 \\
 & RUFF17.0.0 \\
 & RUFF6.1.2 \\
 & RUFO12.3.0 \\
 & RUFO8.0.0 \\
CS03 -- Gestione rubrica & RUFF4.3.0 \\
 & RUFF4.7.0 \\
 & RUFF4.4.0 \\
 & RUFF4.4.1 \\
 & RUFF4.4.2 \\
 & RUFF4.4.3 \\
 & RUFF4.0.0 \\
 & RUFF4.4.4 \\
 & RUFF4.1.0 \\
 & RUFF4.5.0 \\
 & RUFF4.2.0 \\
 & RUFF4.6.0 \\
 
CS04 -- Gestione autenticazione & RUFO1.0.0 \\
 & RUFO1.2.0 \\
 & RUFO2.0.0 \\
 & RSFD2.1.2 \\
 & RUFF3.0.0 \\
 & RUFF3.1.0 \\
 
CS05 -- Gestione rubrica & RUFF15.5.0 \\
 & RUFF15.0.0 \\
 & RUFF15.1.0 \\
 & RUFF15.2.0 \\
 & RUFF15.3.0 \\
 & RUFF15.4.0 \\
 
CS06 -- Gestione chiamata
 & RUFO6.1.0 \\
 & RUFO6.1.1 \\
 & RUFO6.1.3 \\
 & RUFO6.2.0 \\
 & RUFO6.2.1 \\
 & RUFO6.2.3 \\
 & RUFO6.4.0 \\
 & RUFO6.5.0 \\
 & RUFO7.0.0 \\
 & RUFO8.0.0 \\
 & RUFO8.1.0 \\
 & RUFO8.2.0 \\
 & RUFO9.0.0 \\
 & RUFO9.1.0 \\
 & RUFO9.2.0 \\
 & RSFO11.0.0 \\
 & RSFO11.1.0 \\
 & RSFO12.0.0 \\
 & RUFO12.3.0 \\
 
CS07 -- Façade del server & RUFF4.0.0 \\
 & RSFO12.0.0 \\
 & RUFO1.0.0 \\
 & RUFF13.0.0 \\
 & RUFO2.0.0 \\
 & RUFF15.0.0 \\
 & RUFO5.0.0 \\
 & RUFF16.0.0 \\
 & RUFO6.1.0 \\
 & RUFF17.0.0 \\
 & RUFO6.2.0 \\
 & RUFF18.0.0 \\
 & RUFF3.0.0 \\
 
CP01 -- Gestione comunicazione & RUFO6.1.0\\
& RUFO6.1.1\\
& RUFO6.1.3\\
& RUFO6.2.0\\
& RUFO6.2.1\\
& RUFO6.2.3\\
& RUFO6.4.0\\
& RUFO6.5.0\\
& RUFO7.0.0\\
& RUFO8.0.0\\
& RUFO8.1.0\\
& RUFO8.2.0\\
& RUFO9.0.0\\
& RUFO9.1.0\\
& RUFO9.2.0\\
& RUFF6.1.2\\
& RUFF6.1.4\\
& RUFF6.2.2\\
& RUFF6.2.4\\
& RUFF6.2.5\\
& RUFF6.3.0\\
& RSDO6.0.0\\
CP02 -- Rappresentazione dati & RUFO1.0.0\\
& RUFO1.2.0\\
& RUFO2.1.0\\
& RUFF3.0.0\\
& RUFF3.1.0\\
& RUFF3.2.0\\
CP03 -- Gestione GUI & RSFD21.0.0\\
& RSQF26.0.0\\
& RSDO10.0.0\\
& RSDO10.1.0\\
& RSFD21.0.0\\
CV02 -- Login & RUFO1.0.0\\
& RSFO1.2.0\\
& RSFO2.1.0\\
& RSDD2.2.0\\
\bottomrule
\end{longtable}
\end{center}
\subsection{Tracciamenti Componenti-DesignPattern}\label{sec:tracCompDp}

\begin{center}
\rowcolors{4}{lightblue}{llightblue}\begin{longtable}{lp{.55\textwidth}l}
\toprule Componenti & Design pattern utilizzati\\
\midrule
CP03 -- Gestione GUI & MVP \\
CS01 -- Gestione database & Data Access Object \\
& Singleton \\
& MVP \\
CS02 -- Gestione connessione & Factory Method \\
 & Singleton \\
CP01 -- Gestione comunicazione & Singleton \\
CS06 -- Gestione chiamate & MVP \\
CS05 -- Gestione rubrica & MVP \\
CS04 -- Gestione autenticazione & Strategy \\
CS03 -- Gestione rubrica & MVP \\
CS07 -- Façade del server & Façade \\
& MVP \\
\bottomrule
\end{longtable}
\end{center}
\subsection{Tracciamenti DesignPattern-Componenti}\label{sec:tracDpComp}

\begin{center}
\rowcolors{4}{lightblue}{llightblue}\begin{longtable}{lp{.55\textwidth}l}
\toprule Design pattern & Componenti\\
\midrule
Data Access Object & CS01 -- Gestione database\\
Façade & CS07 -- Façade del server\\
Factory Method & CS02 -- Gestione connessione\\
Singleton & CS02 -- Gestione connessione\\
Strategy & CS04 -- Gestione autenticazione\\
MVP & CS01 -- Gestione database\\
& CS03 -- Gestione rubrica\\
& CS05 -- Gestione rubrica\\
& CP01 -- Gestione comunicazione\\
& CS06 -- Gestione chiamate\\
& CP03 -- Gestione GUI\\
\bottomrule
\end{longtable}
\end{center}
\subsection{Tracciamenti Componenti-Classi}\label{sec:tracCompClass}

\begin{center}
\rowcolors{4}{lightblue}{llightblue}\begin{longtable}{lp{0.7\textwidth}l}
\toprule Componenti & Classi\\
\midrule
CS01 -- Gestione database
& server.dao.CallDAO\\
& server.dao.CallListDAO\\
& server.dao.GroupDAO\\
& server.dao.MessageDAO\\
& server.dao.UserDataDAO\\
& server.dao.AddressBookEntryDAO\\
& server.dao.HibernateUtil\\

CS03 -- Gestione rubrica & server.abook.AddressBookEntry\\
& server.abook.IAddressBookEntry\\
& server.abook.IGroup\\
& server.abook.Group\\
& server.abook.IUserData\\
& server.abook.UserData\\

CS05 -- Gestione rubrica & server.message.IMessage\\
& server.message.Message\\

CS06 -- Gestione chiamate
 & server.call.ICall\\
& server.call.Call\\
 & server.call.ICallList\\
& server.call.CallList\\

CS02 -- Gestione connessione & server.connection.PushInbound\\
& org.apache.catalina.websocket.MessageInbound\\

CS04 -- Gestione autenticazione & server.authentication.AuthenticationModule\\
& server.authentication.CredentialLoader\\
& server.authentication.PrincipalImpl\\
& server.authentication.IAuthenticationData\\
& server.authentication.AuthenticationData\\
& server.authentication.AESAlgorithm\\
& server.authentication.ISecurityStrategy\\
& javax.security.auth.spi.LoginModule\\
& javax.security.auth.callback.CallbackHandler\\
& javax.security.Principal\\

CS07 -- Façade del server & server.connection.ChannelServlet\\
& server.abook.servlet.AddressBookDoAddContactServlet\\
& server.abook.servlet.AddressBookDoRemoveContactServlet\\
& server.abook.servlet.AddressBookDoCreateGroupServlet\\
& server.abook.servlet.AddressBookDoDeletGroupServlet\\
& server.abook.servlet.AddressBookDoInsertInGroupServlet\\
& server.abook.servlet.AddressBookDoRemoveInGroupServlet\\
& server.abook.servlet.AddressBookDoBlockServlet\\
& server.abook.servlet.AddressBookDoUnblockServlet\\
& server.abook.servlet.AddressBookGetContactsServlet\\
& server.abook.servlet.AddressBookGetGroupsServlet\\
& server.abook.servlet.AddressBookDoSearchServlet\\
& server.authentication.servlet.LoginServlet\\
& server.authentication.servlet.LogoutServlet\\
& server.authentication.servlet.RegisterServlet\\
& server.message.servlet.InsertMessageServlet\\
& server.message.servlet.DeletMessageServlet\\
& server.message.servlet.UpdateStatusMessageServlet\\
& server.message.servlet.DownloadMessageListServlet\\
& server.call.servlet.DownloadCallHistoryManager\\
& javax.servlet.http.HttpServlet\\
& org.apache.catalina.websocket.WebSocketServlet\\

CP01 -- Gestione comunicazione & clientpresenter.kernel.CommunicationCenter\\
& PeerICECandidate\\
& WebKitRTCPeerConnection\\
& PeerSessionDescription\\

CP02 -- Rappresentazione dati & clientpresenter.data.JSCall\\
& clientpresenter.data.JSGroup\\
& clientpresenter.data.JSMessage\\
& clientpresenter.data.JSUserData\\

CP03 -- Gestione GUI & clientpresenter.guicontrol.AccountSettingsPanelPresenter\\
& clientpresenter.guicontrol.AddressBookPanelPresenter\\
& clientpresenter.guicontrol.CallHistoryPanelPresenter\\
& clientpresenter.guicontrol.CommunicationPanelPresenter\\
& clientpresenter.guicontrol.GroupPanelPresenter\\
& clientpresenter.guicontrol.SearchResultPanelPresenter\\
& clientpresenter.guicontrol.ContactPanelPresenter\\
& clientpresenter.guicontrol.LoginPanelPresenter\\
& clientpresenter.guicontrol.RegisterPanelPresenter\\
& clientpresenter.guicontrol.TopLevelPresenter\\
& clientpresenter.guicontrol.ChildPrenter\\
& clientpresenter.guicontrol.MainPanelPresenter\\
& clientpresenter.guicontrol.MessagePanelPresenter\\
& clientpresenter.guicontrol.PresenterMediator\\
& clientpresenter.guicontrol.ToolsPanelPresenter\\

CV01 -- GUI & clientview.MainPanel\\
& clientview.ToolsPanel\\
& clientview.AddressBookPanel\\
& clientview.ContactPanel\\
& clientview.MessagePanel\\
& clientview.GroupPanel\\
& clientview.SearchPanel\\
& clientview.AccountSettingsPanel\\
& clientview.CallHistoryPanel\\
& clientview.CommunicationPanel\\

CV02 -- Login & clientview.LoginPanel\\
& clientview.RegisterPanel\\

\bottomrule
\end{longtable}
\end{center}
\subsection{Tracciamenti Classi-Componenti}\label{sec:tracClassComp}

\begin{center}
\rowcolors{4}{lightblue}{llightblue}\begin{longtable}{lp{0.33\textwidth}l}
\toprule Classi & Componenti\\
\midrule

clientpresenter.data.JSCall & CP02 -- Rappresentazione dati\\
clientpresenter.data.JSGroup & CP02 -- Rappresentazione dati\\
clientpresenter.data.JSMessage & CP02 -- Rappresentazione dati\\
clientpresenter.data.JSUserData & CP02 -- Rappresentazione dati\\
clientpresenter.guicontrol.AccountSettingsPanelPresenter & CP03 -- Gestione GUI\\
clientpresenter.guicontrol.AddressBookPanelPresenter & CP03 -- Gestione GUI\\
clientpresenter.guicontrol.CallHistoryPanelPresenter & CP03 -- Gestione GUI\\
clientpresenter.guicontrol.ChildPrenter & CP03 -- Gestione GUI\\
clientpresenter.guicontrol.CommunicationPanelPresenter & CP03 -- Gestione GUI\\
clientpresenter.guicontrol.ContactPanelPresenter & CP03 -- Gestione GUI\\
clientpresenter.guicontrol.GroupPanelPresenter & CP03 -- Gestione GUI\\
clientpresenter.guicontrol.LoginPanelPresenter & CP03 -- Gestione GUI\\
clientpresenter.guicontrol.MainPanelPresenter & CP03 -- Gestione GUI\\
clientpresenter.guicontrol.MessagePanelPresenter & CP03 -- Gestione GUI\\
clientpresenter.guicontrol.PresenterMediator & CP03 -- Gestione GUI\\
clientpresenter.guicontrol.RegisterPanelPresenter & CP03 -- Gestione GUI\\
clientpresenter.guicontrol.SearchResultPanelPresenter & CP03 -- Gestione GUI\\
clientpresenter.guicontrol.ToolsPanelPresenter & CP03 -- Gestione GUI\\
clientpresenter.guicontrol.TopLevelPresenter & CP03 -- Gestione GUI\\
clientpresenter.kernel.CommunicationCenter & CP01 -- Gestione comunicazione\\
clientview.AccountSettingsPanel & CV01 -- GUI\\
clientview.AddressBookPanel & CV01 -- GUI\\
clientview.CallHistoryPanel & CV01 -- GUI\\
clientview.CommunicationPanel & CV01 -- GUI\\
clientview.ContactPanel & CV01 -- GUI\\
clientview.GroupPanel & CV01 -- GUI\\
clientview.LoginPanel & CV02 -- Login\\
clientview.MainPanel & CV01 -- GUI\\
clientview.MessagePanel & CV01 -- GUI\\
clientview.RegisterPanel & CV02 -- Login\\
clientview.SearchPanel & CV01 -- GUI\\
clientview.ToolsPanel & CV01 -- GUI\\
javax.security.auth.callback.CallbackHandler & CS04 -- Gestione autenticazione\\
javax.security.auth.spi.LoginModule & CS04 -- Gestione autenticazione\\
javax.security.Principal & CS04 -- Gestione autenticazione\\
javax.servlet.http.HttpServlet & CS07 -- Façade del server\\
org.apache.catalina.websocket.MessageInbound & CS02 -- Gestione connessione\\
org.apache.catalina.websocket.WebSocketServlet & CS07 -- Façade del server\\
PeerICECandidate & CP01 -- Gestione comunicazione\\
PeerSessionDescription & CP01 -- Gestione comunicazione\\
server.abook.AddressBookEntry & CS03 -- Gestione rubrica\\
server.abook.Group & CS03 -- Gestione rubrica\\
server.abook.IAddressBookEntry& CS03 -- Gestione rubrica\\
server.abook.IGroup & CS03 -- Gestione rubrica\\
server.abook.IUserData & CS03 -- Gestione rubrica\\
server.abook.servlet.AddressBookDoAddContactServlet & CS07 -- Façade del server\\
server.abook.servlet.AddressBookDoBlockServlet & CS07 -- Façade del server\\
server.abook.servlet.AddressBookDoCreateGroupServlet & CS07 -- Façade del server\\
server.abook.servlet.AddressBookDoDeletGroupServlet & CS07 -- Façade del server\\
server.abook.servlet.AddressBookDoInsertInGroupServlet & CS07 -- Façade del server\\
server.abook.servlet.AddressBookDoRemoveContactServlet & CS07 -- Façade del server\\
server.abook.servlet.AddressBookDoRemoveInGroupServlet & CS07 -- Façade del server\\
server.abook.servlet.AddressBookDoSearchServlet & CS07 -- Façade del server\\
server.abook.servlet.AddressBookDoUnblockServlet & CS07 -- Façade del server\\
server.abook.servlet.AddressBookGetContactsServlet & CS07 -- Façade del server\\
server.abook.servlet.AddressBookGetGroupsServlet & CS07 -- Façade del server\\
server.abook.UserData & CS03 -- Gestione rubrica\\
server.authentication.AESAlgorithm & CS04 -- Gestione autenticazione\\
server.authentication.AuthenticationData & CS04 -- Gestione autenticazione\\
server.authentication.AuthenticationModule & CS04 -- Gestione autenticazione\\
server.authentication.CredentialLoader & CS04 -- Gestione autenticazione\\
server.authentication.IAuthenticationData & CS04 -- Gestione autenticazione\\
server.authentication.ISecurityStrategy & CS04 -- Gestione autenticazione\\
server.authentication.PrincipalImpl & CS04 -- Gestione autenticazione\\
server.authentication.servlet.LoginServlet & CS07 -- Façade del server\\
server.authentication.servlet.LogoutServlet & CS07 -- Façade del server\\
server.authentication.servlet.RegisterServlet & CS07 -- Façade del server\\
server.dao.CallDAO & CS01 -- Gestione database\\
server.dao.CallListDAO & CS01 -- Gestione database\\
server.call.Call & CS06 -- Gestione chiamate\\
server.call.ICall & CS06 -- Gestione chiamate\\
server.call.CallList & CS06 -- Gestione chiamate\\
server.call.ICallList & CS06 -- Gestione chiamate\\
server.call.servlet.DownloadCallHistoryManager & CS07 -- Façade del server\\
server.connection.ChannelServlet & CS07 -- Façade del server\\
server.connection.PushInbound & CS02 -- Gestione connessione\\
server.dao.AddressBookEntryDAO & CS01 -- Gestione database\\
server.dao.GroupDAO & CS01 -- Gestione database\\
server.dao.HibernateUtil & CS01 -- Gestione database\\
server.dao.MessageDAO & CS01 -- Gestione database\\
server.dao.UserDataDAO & CS01 -- Gestione database\\
server.message.IMessage & CS05 -- Gestione rubrica\\
server.message.Message & CS05 -- Gestione rubrica\\
server.message.servlet.DeletMessageServlet & CS07 -- Façade del server\\
server.message.servlet.DownloadMessageListServlet & CS07 -- Façade del server\\
server.message.servlet.InsertMessageServlet & CS07 -- Façade del server\\
server.message.servlet.UpdateStatusMessageServlet & CS07 -- Façade del server\\
WebKitRTCPeerConnection & CP01 -- Gestione comunicazione\\

\bottomrule
\end{longtable}
\end{center}


%\newpage
\section{Tracciamento Requisiti - Test requisiti}\label{sec:tracciamento test}

\begin{center}
\rowcolors{2}{lightblue}{llightblue}\begin{longtable}{llp{.6\textwidth}}
\toprule Codice Requisito & Codice Test Requisito  & Descrizione Test sul Requisito\\
\midrule

RUFO1.0.0 & TUFO1.0.0 &tramite l'utente \inglese{test} si effettua una prova di login al sistema, sia con dati corretti che volutamente non corretti al fine di verificare la coerente risposta del sistema, verifica anche i sottorequisiti RSFO1.2.0, RSFO1.2.1 e RSFO1.2.2.\\
RUFO2.0.0  & TUFO2.0.0 & viene effettuata la procedura di registrazione sia con dati corretti che errati (o mancanti) al fine di verificare la risposta del sistema, comprende anche la verifica dei sottorequisiti RSFO2.1.0, RSFO2.1.3, RSFO2.1.4, RSFO2.1.5 e RSFO2.1.6. Si verificherà inoltre l'effettivo inserimento dell'utente nel \underline{database}.\\
RUFO6.1.0 & TUFO6.1.0& Comprende il test anche del requisito RUFO6.3.0. Effettua una chiamata audio con un utente \inglese{test} correttamente autenticato ad un utente \inglese{test} di cui sono note le caratteristiche (\inglese{hardware}/\inglese{software}) mantenendo una comunicazione attiva per una durata superiore a 3 minuti.\\
RUFO6.1.1 & TUFO6.1.1& Effettua una chiamata audio con un utente \inglese{test} correttamente autenticato ad un utente \inglese{test} di cui sono note le caratteristiche (\inglese{hardware}/\inglese{software}) e il suo indirizzo IP, mantenendo una comunicazione attiva per una durata superiore a 3 minuti.\\
RUFO6.1.3 & TUFO6.1.3& Effettua una chiamata audio con un utente \inglese{test} correttamente autenticato ad un utente \inglese{test} correttamente autenticato ad un utente \inglese{test} di cui sono note le caratteristiche (\inglese{hardware}/\inglese{software}), mantenendo una comunicazione attiva per una durata superiore a 3 minuti.\\
RUFO6.2.0 & TUFO6.2.0& Comprende il test anche del requisito RUFO6.4.0. Effettua una chiamata audio/video con un utente \inglese{test} correttamente autenticato ad un utente \inglese{test} di cui sono note le caratteristiche (\inglese{hardware}/\inglese{software}) mantenendo una comunicazione attiva per una durata superiore a 3 minuti.\\
RUFO6.2.1 & TUFO6.2.1& Effettua una chiamata audio/video audio con un utente \inglese{test} correttamente autenticato ad un utente \inglese{test} di cui sono note le caratteristiche (\inglese{hardware}/\inglese{software}) e il suo indirizzo IP, mantenendo una comunicazione attiva per una durata superiore a 3 minuti.\\
RUFO6.2.3 & TUFO6.2.3& Effettua una chiamata audio/video audio con un utente \inglese{test} correttamente autenticato ad un utente \inglese{test} di cui sono note le caratteristiche (\inglese{hardware}/\inglese{software}) mantenendo una comunicazione attiva per una durata superiore a 3 minuti.\\
RUFO7.0.0 & TUFO7.0.0& è verificato attraverso la presenza e la correttezza del codice di gestione del requisito all'interno del file di codifica, e il relativo funzionamento nell'applicazione tramite una prova di comunicazione tra due utenti \inglese{test}.\\
RUFO8.0.0 & TUFO8.0.0& è verificato attraverso la presenza e la correttezza del codice di gestione del requisito all'interno del file di codifica, e il relativo funzionamento nell'applicazione tramite una prova di comunicazione tra due utenti \inglese{test}.\\
RUFO8.1.0 & TUFO8.1.0& è verificato attraverso la presenza e la correttezza del codice di gestione del requisito all'interno del file di codifica, e il relativo funzionamento nell'applicazione tramite una prova di comunicazione tra due utenti \inglese{test}.\\
RUFO9.0.0 & TUFO9.0.0& è verificato attraverso la presenza e la correttezza del codice di gestione del requisito all'interno del file di codifica e il relativo funzionamento nell'applicazione tramite tre prove di comunicazione tra due utenti \inglese{test} di durate superiori a 3 minuti. Un utente \inglese{test} che riceverà la richiesta di connessione dovrà variare durante i tre test la capacità della propria banda, verificando di conseguenza l'effettivo cambiamento nella notifica di qualità di chiamata. \\
RUFO9.1.0 & TUFO9.1.0& è verificato attraverso la presenza e la correttezza del codice di gestione del requisito all'interno del file di codifica, e il relativo funzionamento nell'applicazione tramite una prova di comunicazione tra due utenti \inglese{test}, verranno effettuate tre prove di misurazione della latenza durante la connessione, dopo 10 secondi, un minuto, e a 2 minuti e 50 secondi. La comunicazione durerà in totale 3 minuti.\\
RUFO9.2.0 & TUFO9.2.0& verificata la presenza e la correttezza del codice di gestione del requisito all'interno del file di codifica, e il relativo funzionamento nell'applicazione tramite una prova di comunicazione tra due utenti \inglese{test}, verranno effettuate tre prove di rilevazione della velocità durante la connessione, dopo 10 secondi, un minuto, e a 2 minuti e 50 secondi. La comunicazione durerà in totale 3 minuti.\\
RSFO11.0.0 & TSFO11.0.0& confermare che i dati di richiesta di comunicazione inviati da un utente \inglese{test} siano ricevuti dal server e ritrasmessi correttamente al corrispondente utente \inglese{test} a cui il primo utente ha inviato la richiesta.\\
RSFO11.1.0 & TSFO11.1.0& confermare che i dati di richiesta di comunicazione siano correttamente ricevuti dal corrispondente utente \inglese{test} a cui il primo utente ha inviato la richiesta. Verrà inoltre confermata la presenza e la correttezza  del relativo codice all'interno dei corrispondenti file di codifica.\\
RSFO12.0.0 & TSFO12.0.0& confermare che le varie richieste di interazione durante la comunicazione tra due utenti \inglese{test} siano correttamente inviate e ricevute. Verrà inoltre confermata la presenza e la correttezza del relativo codice relativo a tali interazioni all'interno dei corrispondenti file di codifica.\\
RSFO12.3.0 & TSFO12.3.0& confermare che l'operazione di chiusura di una comunicazione richiesta da uno dei due utenti \inglese{test} che hanno stabilito una connessione sia gestita correttamente e riporti ad uno stato consistente per entrambi.Verrà inoltre confermata la presenza e la correttezza del relativo codice all'interno dei corrispondenti file di codifica.\\


RUFD1.1.0 & TUFD1.1.0& viene inviata da un utente \inglese{test} correttamente registrato la richiesta di invio tramite mail della propria password precedentemente memorizzata nel database. Verrà inoltre confermata la presenza e la correttezza del relativo codice all'interno dei corrispondenti file di codifica.\\
RUFD1.1.1 & TUFD1.1.1& viene eseguita la form di procedura d'impostazione della domanda segreta per il recupero password da parte di un utente \inglese{test} correttamente registrato, successivamente verrà verificato l'effettivo inserimento/modifica di tale campo all'interno del database mantenendolo in uno stato consistente. Verrà inoltre confermata la presenza e la correttezza del relativo codice all'interno dei corrispondenti file di codifica.\\
RUFD1.1.2 & TUFD1.1.2& viene inviata da un utente \inglese{test} correttamente registrato la richiesta di invio tramite mail della propria password precedentemente memorizzata nel database. Conseguentemente si verificherà la relativa ricezione del messaggio di posta nell'indirizzo indicato dall'utente stesso.Verrà inoltre confermata la presenza e la correttezza del relativo codice all'interno dei corrispondenti file di codifica.\\
RUFD12.2.0 & TUFD12.2.0& viene creata correttamente una connessione audio tra tre utenti \inglese{test} e mantenuta per un minuto, successivamente l'utente che ha avviato ed esteso la chiamata eliminerà un utente dalla conversazione. Verrà verificato che tutti che tutti gli utenti siano in uno stato del sistema consistente.\\
RUFD13.1.0 & TUFD13.1.0& viene avviata una comunicazione testuale tra due utenti \inglese{test}, verranno inviati da entrambi i soggetti cinque messaggi testuali e ne verrà verificata l'effettiva ricezione.\\


RUFF3.0.0 & TUFF3.0.0& vengono modificati singolarmente tutti i campi di un utente \inglese{test} correttamente registrato tramite la relativa form di modifica. Verrà successivamente verificato l'effettiva modifica di tale campo all'interno del database mantenendolo in uno stato consistente.\\
RUFF3.1.0 & TUFF3.1.0& vengono modificati singolarmente i campi password-domanda segreta-risposta alla domanda segreta di un utente \inglese{test} correttamente registrato tramite la relativa form di modifica. Verrà successivamente verificata l'effettiva modifica di tali campi all'interno del database e il suo mantenimento in uno stato consistente, comprende quindi i test relativi ai requisiti RUFF3.1.1, RUFF3.1.2, RUFF3.1.3.\\
RUFF3.2.0 & TUFF3.2.0& vengono modificati singolarmente tutti i campi classificati come ``facoltativi'' durante la registrazione di un utente \inglese{test} correttamente registrato tramite la relativa form di modifica. Verrà successivamente verificato l'effettiva modifica di tale campo all'interno del database mantenendolo in uno stato consistente, comprende quindi i test relativi ai requisiti RUFF3.2.1, RUFF3.2.2, RUFF3.2.3.\\
RUFF4.0.0 & TUFF4.0.0& viene verificata la presenza e il funzionamento mediante \inglese{test} specializzati nelle verifiche associate ai sotto-requisiti della rubrica (specificati nei codici RUFF4.X.0). Verrà inoltre confermata la presenza e la correttezza del relativo codice associato a tale funzionalità all'interno dei corrispondenti file di codifica.\\
RUFF4.1.0 & TUFF4.1.0& tramite un utente \inglese{test} correttamente registrato nel database e autenticato al sistema viene inserito un secondo utente \inglese{test} correttamente registrato nel database nella sua rubrica. Verrà verificato l'effettivo inserimento nella rubrica e che l'utente rimanga in uno stato consistente dopo tale inserimento.\\
RUFF4.2.0 & TUFF4.2.0& tramite un utente \inglese{test} correttamente registrato nel database e autenticato al sistema con un secondo utente \inglese{test} correttamente registrato nel database nella sua rubrica. Verrà eseguito il comando di rimozione sull'utente nella rubrica, verificata la corretta rimozione e che l'utente rimanga in uno stato consistente dopo tale comando.\\
RUFF4.3.0 & TUFF4.3.0& tramite un utente \inglese{test} correttamente registrato nel database e autenticato al sistema con un tre utenti \inglese{test} correttamente registrati nel database nella sua rubrica. Verrà eseguito il comando di ordinamento ``ordina per email'', successivamente per nome utente e per nome gruppi e riscontrato il corretto funzionamento dei suddetti. Verrà verificata la correttezza del codice di gestione del requisito all'interno del file di codifica. Verifica anche i sottorequisiti RUFF4.3.1 e RUFF4.3.2. \\
RUFF4.4.0 & TUFF4.4.0& viene verificata la presenza e il funzionamento mediante \inglese{test} specializzati nelle verifiche associate ai sotto-requisiti della funzionalità ``gruppi'' (specificati nei codici RUFF4.4.X). Verrà inoltre confermata la presenza e la correttezza del relativo codice associato a tale funzionalità all'interno dei corrispondenti file di codifica.\\
RUFF4.4.1 & TUFF4.4.1& tramite un utente \inglese{test} correttamente registrato nel database e autenticato al sistema con un tre utenti \inglese{test} correttamente registrati nel database nella sua rubrica con almeno un gruppo creato. Verrà eseguito il comando di rimozione sull'utente nel gruppo, verificata la corretta rimozione e che l'utente rimanga in uno stato consistente dopo tale comando.\\
RUFF4.4.2 & TUFF4.4.2& tramite un utente \inglese{test} correttamente registrato nel database e autenticato al sistema con un tre utenti \inglese{test} correttamente registrati nel database nella sua rubrica, verrà eseguito il comando di inserimento di sull'utente nel gruppo, verificata la corretta rimozione e che l'utente rimanga in uno stato consistente dopo tale comando.\\
RUFF4.4.3 & TUFF4.4.3& tramite un utente \inglese{test} correttamente registrato nel database e autenticato al sistema con un tre utenti \inglese{test} correttamente registrati nel database nella sua rubrica, verrà eseguito il comando relativo alla creazione di un nuovo gruppo e riscontrato il corretto funzionamento del suddetto. Verrà verificata la correttezza del codice di gestione del requisito all'interno del file di codifica.\\
RUFF4.4.4 & TUFF4.4.4& tramite un utente \inglese{test} correttamente registrato nel database e autenticato al sistema con un tre utenti \inglese{test} correttamente registrati nel database nella sua rubrica con almeno un gruppo creato. Verrà eseguito il comando relativo alla rimozione di un gruppo e riscontrato il corretto funzionamento del suddetto. Verrà verificata la correttezza del codice di gestione del requisito all'interno del file di codifica.\\
RUFF4.5.0 & TUFF4.5.0& tramite un utente \inglese{test} correttamente registrato nel database e autenticato al sistema con un tre utenti \inglese{test} correttamente registrati nel database nella sua rubrica, verrà eseguito il comando relativo all'esportazione della rubrica in formato XML e riscontrato il corretto funzionamento del suddetto verificando successivamente che il file generato risulti coerente con i dati della rubrica esportata. Verrà verificata infine la correttezza del codice di gestione del requisito all'interno del file di codifica.\\
RUFF4.6.0 & TUFF4.6.0& yramite un utente \inglese{test} correttamente registrato nel database e autenticato al sistema con un tre utenti \inglese{test} correttamente registrati nel database nella sua rubrica. Verrà eseguito il comando relativo all'importazione nella rubrica utenti da un file XML e riscontrato il corretto funzionamento del suddetto verificando successivamente che nella rubrica siano importati coerentemente i contatti presenti nel file XML. Verrà verificata infine la correttezza del codice di gestione del requisito all'interno del file di codifica.\\
RUFF4.7.0 & TUFF4.7.0& tramite un utente \inglese{test} correttamente registrato nel database e autenticato al sistema con un tre utenti \inglese{test} correttamente registrati nel database nella sua rubrica, verrà eseguito il comando relativo alla ricerca nella rubrica di un utente (sicuramente presente nella rubrica) e riscontrato il corretto funzionamento del suddetto (verifica implicitamente anche RUFF4.7.1 e RUFF4.7.2 e RUFF4.7.3). Verrà verificata infine la correttezza del codice di gestione del requisito all'interno del file di codifica.\\
RUFF5.1.0 & TUFF5.1.0& tramite un utente \inglese{test} correttamente registrato nel database e autenticato al sistema verrà eseguito il comando relativo alla ricerca nella di un utente nella lista degli utenti nel database (dove sono inseriti tre utenti prova) e riscontrato il corretto funzionamento del suddetto. Verrà verificata infine la correttezza del codice di gestione del requisito all'interno del file di codifica.\\
RUFF6.1.2 & TUFF6.1.2& si effettua una chiamata audio con un utente \inglese{test} correttamente autenticato ad un utente \inglese{test} presente in rubrica di cui sono note le caratteristiche (\inglese{hardware}/\inglese{software}) mantenendo una comunicazione attiva per una durata superiore a 3 minuti.\\
RUFF6.1.4 & TUFF6.1.4& effettuare una chiamata audio con un utente \inglese{test} correttamente autenticato ad un utente \inglese{test} di cui sono note le caratteristiche (\inglese{hardware}/\inglese{software}) mantenendo una comunicazione attiva per una durata di 30 secondi, successivamente verrà eseguito il comando relativo alla promozione alla comunicazione audio/video e mantenuta la comunicazione attiva per una durata di altri 30 secondi.\\
RUFF6.2.2 & TUFF6.2.2& si effettua una chiamata audio/video con un utente \inglese{test} correttamente autenticato ad un utente \inglese{test} presente in rubrica di cui sono note le caratteristiche (\inglese{hardware}/\inglese{software}) mantenendo una comunicazione attiva per una durata superiore a 3 minuti.\\
RUFF6.2.4 & TUFF6.2.4& si effettua una chiamata audio/video con un utente \inglese{test} correttamente autenticato ad un utente \inglese{test} presente in rubrica di cui sono note le caratteristiche (\inglese{hardware}/\inglese{software}) mantenendo una comunicazione attiva per una durata superiore a 30 secondi. Successivamente verrà eseguito il comando relativo alla declassazione alla comunicazione audio e mantenuta la comunicazione attiva per una durata di altri 30 secondi.\\
RUFF6.2.5 & TUFF6.2.5& si effettua una chiamata audio/video con un utente \inglese{test} correttamente autenticato ad un utente \inglese{test} presente in rubrica di cui sono note le caratteristiche (\inglese{hardware}/\inglese{software}) mantenendo una comunicazione attiva per una durata superiore a 30 secondi. Successivamente verrà eseguito il comando relativo alla disattivazione della webcam dell'utente \inglese{test} e mantenuta la comunicazione attiva per una durata di altri 30 secondi.\\
RUFF6.3.0 & TUFF6.3.0& si effettua una chiamata audio/video con un utente \inglese{test} correttamente autenticato ad un utente \inglese{test} presente in rubrica di cui sono note le caratteristiche (\inglese{hardware}/\inglese{software}) mantenendo una comunicazione attiva per una durata superiore a 30 secondi. Successivamente verrà eseguito il comando relativo alla disattivazione del microfono dell'utente \inglese{test} e mantenuta la comunicazione attiva per una durata di altri 30 secondi.\\
RUFF9.3.0 & TUFF9.3.0& si effettua una chiamata audio/video con un utente \inglese{test} correttamente autenticato ad un utente \inglese{test} presente in rubrica di cui sono note le caratteristiche (\inglese{hardware}/\inglese{software}) mantenendo una comunicazione attiva per una durata superiore a 30 secondi. Successivamente verrà eseguito il comando relativo alla visualizzazione dei frame per secondo sulla webcam dell'utente \inglese{test} e mantenuta la comunicazione attiva per una durata di altri 30 secondi. Verrà inoltre confermata la presenza e la correttezza  del relativo codice all'interno dei corrispondenti file di codifica.\\
RUFF11.2.0 & TUFF11.2.0& la verifica di tale requisito è implicitamente dimostrata dalla filosofia implementativa e progettuale del prodotto stesso.\\
RUFF12.1.0 & TUFF12.1.0& viene effettuata una chiamata audio con un utente \inglese{test} correttamente autenticato ad un utente \inglese{test} di cui sono note le caratteristiche (\inglese{hardware}/\inglese{software}), successivamente l'utente \inglese{test} che ha avviato la chiamata tramite il relativo comando aggiunge un terzo utente \inglese{test} (previa accettazione dello stesso) alla comunicazione. Tale connessione rimane attiva per 3 minuti totali.\\
RUFF13.0.0 & TUFF13.0.0& Il test eseguito è l'analogo di TUFF13.1.0.\\
RUFF13.2.0 & TUFF13.2.0& Il test eseguito è l'analogo di TUFF13.0.0, ma esteso a tre utenti correttamente autenticati al sistema.\\
RUFF14.0.0 & TUFF14.0.0& viene effettuata una chiamata audio con un utente \inglese{test} correttamente autenticato ad un utente \inglese{test} presente in rubrica di cui sono note le caratteristiche (\inglese{hardware}/\inglese{software}). A chiamata avviata l'utente \inglese{test} richiede mediante apposito comando la registrazione della chiamata, viene accettata tale richiesta e registrati 20 secondi. Al termine della comunicazione verrà verificata l'effettivo funzionamento del file audio risultante.\\
RUFF14.1.0 & TUFF14.1.0& viene effettuata una chiamata audio con un utente \inglese{test} correttamente autenticato ad un utente \inglese{test} presente in rubrica di cui sono note le caratteristiche (\inglese{hardware}/\inglese{software}). A chiamata avviata l'utente \inglese{test} richiede mediante apposito comando la registrazione della chiamata, viene rifiutata tale richiesta e verificato che tale negazione riporti entrambi gli utenti in uno stato consistente.\\
RUFF14.2.0 & TUFF14.2.0& Il test eseguito è l'analogo di TUFF14.0.0.\\
RUFF14.3.0 & TUFF14.3.0& verrà riascoltato il file prodotto dal test TUFF14.2.0 e constatata la coerenza con l'estratto di comunicazione svolto.\\
RUFF15.0.0 & TUFF15.0.0& è presente nella schermata home la presenza della gestione della segreteria telefonica personale. Verificato attraverso la presenza e la correttezza del codice di gestione del requisito all'interno dei file di codifica.\\
RUFF15.1.0 & TUFF15.1.0& viene lasciato un messaggio audio di 20 secondi con un utente \inglese{test} correttamente autenticato ad un utente \inglese{test} non autenticato. Successivamente verrà autenticato l'utente \inglese{test} destinatario del messaggio e verificata l'effettiva ricezione del messaggio.\\
RUFF15.2.0 & TUFF15.2.0& viene lasciato un messaggio audio/video di 20 secondi con un utente \inglese{test} correttamente autenticato ad un utente \inglese{test} non autenticato. Successivamente verrà autenticato l'utente \inglese{test} destinatario del messaggio e verificata l'effettiva ricezione del messaggio.\\
RUFF15.3.0 & TUFF15.3.0& il test di questo requisito è intrinseco nella correttezza dei test  TUFF15.1.0 o TUFF15.2.0.\\
RUFF15.4.0 & TUFF15.4.0& il test di questo requisito è successivo ai test TUFF15.1.0 o TUFF15.2.0, dopo aver ascoltato/visto il messaggio verrà eliminato tramite l'apposito comando nella form di gestione.\\
RUFF15.5.0 & TUFF15.5.0& il test di questo requisito è successivo ai test TUFF15.1.0 o TUFF15.2.0, dopo aver ascoltato/visto il messaggio verrà impostato come ``ascoltati'' tramite l'apposito comando nella form di gestione.\\
RUFF16.0.0 & TUFF16.0.0& comprende il test dei sotto-requisiti RUFF16.1.0 e RUFF16.2.0. Tramite un utente test correttamente registrato nel database e autenticato al sistema tramite l'apposito comando viene impostato a ``non disponibile'' il proprio stato. Tramite un ulteriore utente \inglese{test} correttamente registrato nel database e autenticato al sistema e munito dell'utente \inglese{test} principale nella propria rubrica, viene confermato il cambio di stato visualizzato, successivamente verrà ripristinato lo stato ``disponibile'' e verificato il cambio di stato visualizzato.\\
RUFF17.0.0 & TUFF17.0.0& Il test di questo requisito è intrinseco nella correttezza dei test TUFF16.0.0.\\
RUFF18.0.0 & TUFF18.0.0& tramite un utente \inglese{test} correttamente registrato nel database e autenticato al sistema si ha a disposizione nella rubrica un gruppo denominato blacklist (di default).\\
RUFF19.0.0 & TUFF19.0.0& tramite un utente \inglese{test} correttamente registrato nel database e autenticato al sistema e con tre utenti presenti in rubrica verranno effettuate 3 chiamate audio a tali utenti e riscontrata la corretta presenza nella sezione ''storico delle chiamate'' la presenza e i dettagli delle chiamate effettuate.\\
RUFF20.0.0& TUFF20.0.0& viene effettuata una chiamata audio o audio/video o una chat testuale tramite un utente \inglese{test} correttamente autenticato ad un utente \inglese{test} correttamente autenticato di cui sono note le caratteristiche (\inglese{hardware}/\inglese{software}). Successivamente sarà invitato l'utente che riceve la chiamata ad una condivisione di risorse mediante l'apposito comando (monitor-pdf-lavaglia grafica-file). I test specifici sono rimandati a TUFF20.1.0, TUFF20.2.0, TUFF20.3.0, TUFF20.4.0.\\
RUFF20.1.0& TUFF20.1.0 & Effettuare una chiamata audio o audio/video o una chat testuale tramite un utente \inglese{test} correttamente autenticato ad un utente \inglese{test} correttamente autenticato ad un utente \inglese{test} di cui sono note le caratteristiche (\inglese{hardware}/\inglese{software}). Successivamente sarà invitato l'utente che riceve la chiamata ad una condivisione del monitor, confermata la richiesta, ed effettuata la condivisione per 30 secondi. Al termine della condivisione gli utenti coinvolti dovranno tornare in uno stato consistente di connessione.\\
RUFF20.2.0& TUFF20.2.0 &Effettuare una chiamata audio o audio/video o una chat testuale tramite un utente \inglese{test} correttamente autenticato ad un utente \inglese{test} correttamente autenticato ad un utente \inglese{test} di cui sono note le caratteristiche (\inglese{hardware}/\inglese{software}). Successivamente sarà invitato l'utente che riceve la chiamata ad una condivisione di un file pdf, confermata la richiesta, ed effettuata la condivisione per 30 secondi. Al termine della condivisione gli utenti coinvolti dovranno tornare in uno stato consistente di connessione.\\
RUFF20.3.0& TUFF20.3.0 &Effettuare una chiamata audio o audio/video o una chat testuale tramite un utente \inglese{test} correttamente autenticato ad un utente \inglese{test} correttamente autenticato ad un utente \inglese{test} di cui sono note le caratteristiche (\inglese{hardware}/\inglese{software}). Successivamente sarà invitato l'utente che riceve la chiamata ad una condivisione di una lavaglia grafica, confermata la richiesta, ed effettuata la condivisione per 30 secondi. Al termine della condivisione gli utenti coinvolti dovranno tornare in uno stato consistente di connessione.\\
RUFF20.4.0& TUFF20.4.0 &Effettuare una chiamata audio o audio/video o una chat testuale tramite un utente \inglese{test} correttamente autenticato ad un utente \inglese{test} correttamente autenticato ad un utente \inglese{test} di cui sono note le caratteristiche (\inglese{hardware}/\inglese{software}). Successivamente sarà invitato l'utente che riceve la chiamata ad una ricezione di un file, confermata la richiesta, e sarà inviato il file. Al termine della condivisione gli utenti coinvolti dovranno tornare in uno stato consistente di connessione, il file sarà stato inviato correttamente al destinatario.\\

RSFO1.2.0 &TSFO1.2.0& il test eseguito è analogo a TUFO1.0.0.\\
RSFO2.1.0 &TSFO2.1.0& il test eseguito è analogo a TUFO2.0.0.\\
RSQO6.4.0 & TSQO6.4.0& la verifica di tale requisito è implicitamente dimostrata dalla filosofia progettuale e implementativa del prodotto stesso.\\
RSDD2.1.1 &TSDD2.1.1& durante l'inserimento della password desiderata nel form di registrazione (o nella form di modifica dei campi) di un utente \inglese{test} compare la stima di complessità (livello di qualità) della stringa inserita. Verrà inoltre confermata la presenza e la correttezza  del relativo codice all'interno dei corrispondenti file di codifica.\\
RSFD2.1.2 &TSFD2.1.2& durante la registrazione di un utente \inglese{test} verrà inserito uno username (e-mail) già inserita nel sistema, in tal caso il suddetto dovrà riportare un messaggio d'errore. Al contrario se lo username è disponibile verrà inserito correttamente, mantenendo integra la consistenza delle informazioni già presenti nel database. Verrà inoltre confermata la presenza e la correttezza  del relativo codice all'interno dei corrispondenti file di codifica.\\
RSDD2.2.0 & TSDD2.2.0& vengono inseriti singolarmente tutti i campi dati facoltativi di un utente \inglese{test} correttamente registrato (di cui sono stati lasciati vuoti i suddetti campi dati) tramite la relativa form di modifica. Verrà successivamente verificato l'effettiva modifica di tali campi all'interno del database mantenendolo in uno stato consistente.\\
RSFD2.1.2 & TSFD2.1.2&  il soddisfacimento è intrinseco nel test relativo alla registrazione dell'utente TUFO2.0.0.\\
RSFD21.0.0 & TSFD21.0.0 & tramite un utente \inglese{test} correttamente autenticato al sistema verrà modificata la lingua dell'interfaccia del sistema stesso, verificando che tale cambiamento non modifichi l'integrità del template e ci sia una corrispondenza nelle traduzioni linguistiche. Verrà infine reimpostata la lingua di default (italiano) e verificata la consistenza del sistema.\\
RSQF23.0.0  & TSQF23.0.0 & I test di tale requisito (e i suoi sotto-requisiti TSQF23.X.0) sono analoghi a quelli fatti su browser \underline{Chrome} presenti in questa tabella.\\
RSQF24.0.0 & TSQF24.0.0 & I test di tale requisito (e i suoi sotto-requisiti TSQF24.X.0) sono analoghi a quelli fatti su browser Chrome presenti in questa tabella.\\
RSQF25.0.0 & TSQF25.0.0 &I  test di tale requisito (e i suoi sotto-requisiti TSQF25.X.0) sono analoghi a quelli fatti su browser Chrome presenti in questa tabella.\\
RSDO6.0.0 & TSDO6.0.0 & la verifica di tale requisito è implicitamente dimostrata dalla filosofia implementativa del prodotto stesso.\\
RSDO10.0.0 & TSDO10.0.0 & la la verifica di tale requisito è implicitamente dimostrata al termine dei test sulle funzioni del prodotto.\\
RSDO10.1.0 & TSDO10.1..0 &  la verifica del requisito è implicitamente dimostrata nella codifica dell'interfaccia e nella fruizione del prodotto stesso.\\
RSDO11.3.0 & TSDO11.3.0 & verificato attraverso la correttezza del codice di gestione del requisito all'interno del file di codifica, e il relativo funzionamento nell'applicazione tramite una prova di comunicazione tra un utente autenticato al sistema e il server.\\
RSDO11.4.0 & TSDO11.4.0& verificato attraverso la presenza e la correttezza del codice di gestione del requisito all'interno del file di codifica, e il relativo funzionamento nell'applicazione tramite una prova di comunicazione tra due utenti \inglese{test}.\\
RSDO22.0.0 & TSDO22.0.0 & la verifica di tale requisito è implicitamente dimostrata dalla filosofia implementativa e progettuale del prodotto stesso.\\
RSDO22.1.0 & TSDO22.1.0 & la verifica di tale requisito è dimostrata dai test precedentemente effettuati (e superati) sotto \textit{browser} Google Chrome.\\
RSQO27.0.0 & TSQO27.0.0 & le modalità di verifica di tale requisito sono descritte nella sezione 3.1 del documento \textit{piano\_di\_qualifica\_4.0.pdf}.\\
RSQO28.0.0 & TSQO28.0.0 & le modalità di verifica di tale requisito sono descritte nella sezione 3.2 del documento \textit{piano\_di\_qualifica\_4.0.pdf}, verificando se si è in grado di soddisfare i requisiti in merito a funzionalità, portabilità, usabilità, affidabilità, efficienza e manutenibilità.\\
RSQO28.1.0 & TSQO28.1.0 & verificare che la pianificazione delle attività svolte, l'esecuzione e l'adozione delle misure correttive sulle stesse, rispettino quanto descritto nel ciclo di Deming.\\
RSQO29.0.0 & TSQO29.0.0 & il superamento di tale test è derivabile dalle descrizioni delle modalità di verifica elencate nel capitolo 4 (e dai sotto-capitoli) del documento \textit{piano\_di\_qualifica\_4.0.pdf}. Tale test comprende inoltre la soddisfacibilità dei sotto requisiti TSQO29.1.0, TSQO29.2.0, TSQO29.3.0, TSQO29.4.0, TSQO29.4.1 e TSQO29.5.0, che specializzano le modalità di verifica alle attività intermedie nel ciclo di vita del progetto.\\
RSQO30.0.0 & TSQO30.0.0 & il sistema di \textit{ticketing} adottato dal team di sviluppo è presentato e descritto nel documento \textit{norme\_di\_progetto.3.0.pdf}, la verifica è implicita in quanto le norme redatte sono state seguite scrupolosamente.\\
RSQO31.0.0 & TSQO31.0.0 & il superamento di tale requisito è implicito nella filosofia di sviluppo dell'applicazione, con una progettazione orientata appunto agli oggetti e descritta nei documenti \textit{specifica\_tecnica.3.0.pdf} e \textit{definizione\_di\_prodotto.2.0.pdf}.\\
RSQO31.1.0 & TSQO31.1.0 & il superamento di tale requisito è implicito nella filosofia di sviluppo dell'applicazione, l'utilizzo delle interfacce è descritto nei documenti \textit{specifica\_tecnica.3.0.pdf} e \textit{definizione\_di\_prodotto.2.0.pdf}.\\
RSQO31.2.0 & TSQO31.2.0 & il superamento di tale requisito viene verificato garantendo che le classi utilizzate siano coerenti con il loro effettivo inserimento nel contesto, e la loro utilità sia effettivamente coerente con lo stesso. Per fare ciò si fa riferimento a quanto descritto nei documenti \textit{specifica\_tecnica.3.0.pdf} e \textit{definizione\_di\_prodotto.2.0.pdf}.\\

\bottomrule
\end{longtable}
\end{center}

\end{document}
% per tabelle con colori alternati inserire prima della tabella l'istruzione seguente:
% \rowcolors{3}{lightblue}{llightblue}



% per ripetere le intestazioni ad ogni pagina inserire
% \endhead
% dopo la riga che delimita la fine delle intestazioni
% possiamo usare \hiderowcolors e \showrowcolors prima e dopo le intestazioni