% analisi_dei_requisiti/analisi_dei_requisiti.tex

%TODO:
% sistemare date e autori sulla base del PdP!

% **************************************************
% Macro specifiche per il documento corrente
% **************************************************
% Nome
\newcommand{\docName}{Analisi dei requisiti}
% Nome file
\newcommand{\docFileName}{analisi\_dei\_requisiti.tex}
% Versione
\newcommand{\docVers}{0.1}
% Data creazione
\newcommand{\creationDate}{20/11/2012}
% Data ultima modifica
\newcommand{\modificationDate}{20/11/2012}
% Stato in {Approvato, Non approvato}
\newcommand{\docState}{Non approvato}
% Uso in {Interno, Esterno}
\newcommand{\docUsage}{Esterno}
% Redattori da specificare come nome1\\ &nome2\\ ecc.
\newcommand{\docAuthors}{Andrea Rizzi\\}
% Approvato da
\newcommand{\approvedBy}{}
% Verificatori
\newcommand{\verifiedBy}{}
% Perscorso (relativo o assoluto) che punta alla directory contenente shared/
% come sua sottodirectory (per comodità chiamiamola 'doc root').
\newcommand{\docRoot}{..}
% definire se si vuole l'indice delle tabelle
\def\INDICETABELLE{false}
% definire se si vuole l'indice delle figure
\def\INDICEFIGURE{false}


% importa il preambolo condiviso da tutti i documenti
% shared/preamble.tex
%
% Questo documento contiene la parte del preambolo condivisa e viene pertanto
% richiamato nel 'master' di tutti i documenti di progetto.  Al suo interno
% contiene le inclusioni (e le configurazioni) di tutti i package richiesti per
% la compilazione dei documenti, le macro di carattere generale e la definizione
% degli stili di pagina.

\documentclass[a4paper,10pt]{article}

% **************************************************
% Macro generiche
% **************************************************
\newcommand{\team}{Software Synthesis}                    % chi siamo
\newcommand{\email}{info@softwaresynthesis.org}           % e-mail
\newcommand{\caName}{MyTalk}                              % titolo capitolato
\newcommand{\manager}{SynthesisRequirementManager}        % nome del sistema di tracciamento
\newcommand{\memberdata}[1]{%
  \texttt{\textcolor{RedOrange}{#1}}}                     % attributi di una classe
\newcommand{\method}[1]{\texttt{\textcolor{Emerald}{#1}}} % metodi di una classe
\newcommand{\exception}[1]{%
  \texttt{\textcolor{RedViolet}{#1}}}                     % eccezione
% \newcommand{\handler}[1]{\texttt{\textcolor{Maroon}{#1}}} % per gli event handler
\newcommand{\inglese}[1]{%
  \foreignlanguage{english}{\textit{#1}}}                 % per i testi in lingua inglese
\newcommand{\purpose}{%                                     scopo del prodotto
Con il progetto ``\caName'' si intende un sistema software di comunicazione tra utenti mediante \underline{browser} senza la necessit{\`a} di installazione di \underline{plugin} e/o software esterni. L'utilizzatore avr{\`a} la possibilit{\`a} di interagire con un altro utente tramite una comunicazione audio - audio/video - testuale e, inoltre, ottenere delle statistiche sull'attivit{\`a} in tempo reale.%
}
\newcommand{\glossaryIntro}{%                               introduzione al glossario
Al fine di evitare incomprensioni dovute all'uso di termini tecnici nei documenti, viene redatto e allegato il documento \textit{glossario.4.0.pdf} dove vengono definiti e descritti tutti i termini marcati con una sottolineatura.%
}


% **************************************************
% Codifica e lingua dei documenti
% **************************************************
\usepackage[utf8x]{inputenc}                              % codifica caratteri dei documenti sorgenti
\usepackage[english,italian]{babel}                       % localizzazione ai fini di sillabazione e cross-references
\usepackage[T1]{fontenc}                                  % codifica font di output

% **************************************************
% Definizione geometria della pagina
% **************************************************
\usepackage[a4paper,head=4cm,top=4.5cm,bottom=3cm,left=3cm,right=3cm,bindingoffset=5mm]{geometry}

% *************************************************
% Intestazioni e piè di pagina personalizzati
% *************************************************
\usepackage{fancyhdr}
% stile normale
\fancypagestyle{normal}{
\fancyhead{}                                              % intestazione
\fancyhead[RE,RO]{
\begin{picture}(0,0)
  \put(-410,0){\includegraphics[width=1.02\textwidth]{header_logo}}
  \put(-410,10){\sffamily\large\leftmark}
\end{picture}
\vspace{-4pt}
}
\renewcommand{\headrulewidth}{0pt}                       % riga sotto l'intestazione
\cfoot{}                                                  % piè di pagina
\fancyfoot[RO,LE]{\sffamily
  pag.~\thepage{} di \pageref{LastPage}}                  % a dx nelle pag. dispari e a sx in quelle pari
\fancyfoot[RE,LO]{\sffamily\docFileName{}}
\renewcommand{\footrulewidth}{.4pt}                       % riga sopra il piè di pagina
}
% stile per gli indici
\fancypagestyle{toc}{
\fancyhead{}                                              % intestazione
\fancyhead[RE,RO]{
\begin{picture}(0,0)
  \put(-410,0){\includegraphics[width=1.02\textwidth]{header_logo}}
\end{picture}
}
\renewcommand{\headrule}{}                                % nessuna riga sotto l'intestazione
\cfoot{}                                                  % piè di pagina
\fancyfoot[RO,LE]{\sffamily\thepage{}}                    % a dx nelle pag. dispari e a sx in quelle pari
\fancyfoot[RE,LO]{\sffamily\docFileName{} -- v.\docVers}
\renewcommand{\footrulewidth}{.4pt}                       % riga sopra il piè di pagina
}

\pagestyle{fancy}                                         % premetto: non so usare bene le marche:
\renewcommand{\sectionmark}[1]{\markboth{#1}{#1}}         % se qualcuno ha idee migliori si faccia avanti!

% **************************************************
% Tabelle
% **************************************************
\usepackage{tabularx}                                     % tabelle di larghezza fissa con una o più colonne variabili
\usepackage{multirow}                                     % colonne con colonne che si estendono per più righe
\usepackage{booktabs}                                     % per inserire l'ambiente table e le righe orizz. nelle tabelle
\usepackage{longtable}			                              % tabelle oltre i limiti di pagina

% **************************************************
% Cross-references e collegamenti ipertestuali
% **************************************************
\usepackage[hidelinks]{hyperref}
\hypersetup{%
  colorlinks=false, linktocpage=false, pdfborder={0,0,0}, pdfstartpage=1, pdfstartview=FitV,%
  urlcolor=Cyan, linkcolor=Cyan, citecolor=Black, %pagecolor=Black,%
  pdftitle={\docName}, pdfauthor={\team}, pdfsubject={}, pdfkeywords={},%
  pdfcreator={pdflatex}, pdfproducer={pdflatex with hyperref package}%
}

% **************************************************
% Immagini e grafica
% **************************************************
\usepackage{graphicx}                                     % supporto ad aspetti avanzati delle immagini
\usepackage[table,usenames,dvipsnames]{xcolor}            % tabelle con righe colorate e alternate
\graphicspath{{\docRoot/pics/}}                           % percorso contenente tutti i file immagini
\usepackage{float}                                        % per rendere non flottanti gli ambienti flottanti
\usepackage[italian]{varioref}                            % testo completo riferimenti in italiano

% **************************************************
% Definizioni di colori
% **************************************************
\definecolor{myBlue}{RGB}{1,167,236}
\definecolor{lightblue}{RGB}{213,243,253}%{119,218,247}
\definecolor{llightblue}{RGB}{229,255,255}

% **************************************************
% Altri pacchetti opzionali
% **************************************************     
\usepackage{lastpage}                                     % per sapere il numero totale di pagine
\usepackage{eurosym}                                      % per il simbolo dell'euro usare \EUR{x} dove x è l'importo
\usepackage{ifthen}                                       % permette la scelta di rami condizionali nella compilazione
\usepackage{enumitem}                                     % permette di configurare gli elenchi puntati e numerati


% Fine del preambolo e inizio del documento
\begin{document}

% Inclusione della prima pagina
% shared/firstpage.tex
%
% Questo documento definisce il contenuto della prima pagina, che si suppone
% essere uguale in tutti i documenti.  Oltre al logo e al titolo, la prima
% pagina contiene i metadati relativi al documento in cui viene inclusa.


% rimuove intestazioni e piè di pagina
\pagestyle{empty}

\begin{center}

% logo del gruppo
\includegraphics[width=1.5\textwidth]{logo}

\vspace{1in}

% titolo del documento
{\Huge\bfseries \docName}

\vspace{1in}

% tabella riepilogativa
\begin{tabularx}{.7\textwidth}{>{\bfseries\sffamily}l>{\sffamily}l}
\toprule
\multicolumn{2}{>{\sffamily}c}{Informazioni sul documento}\\
\midrule
Nome file:            & \docFileName\\
Versione:             & \docVers\\
Data creazione:       & \creationDate\\
Data ultima modifica: & \modificationDate\\
Stato:                & \docState\\
Uso:                  & \docUsage\\
Redattori:            & \docAuthors\\
Approvato da:         & \approvedBy\\
Verificatori:         & \verifiedBy\\
\bottomrule
\end{tabularx}

\end{center}

\newpage


% Storico delle modifiche
\section*{Storia delle modifiche}
\begin{tabularx}{\textwidth}{lXll}
\toprule
Versione & Descrzione intervento & Redattore & Data\\
\midrule % inserire qui il contenuto della tabella
0.1 & Versione iniziale & Andrea Rizzi & 20/11/2012\\
\bottomrule
\end{tabularx}
\newpage

% inclusione dell'indice
% shared/toc.tex
%
% Questo file contiene le istruzioni che generano l'indice o gli indici del
% documento (utile nel caso in cui decidessimo di avere anche un indice delle
% tabelle e/o un indice delle figure).

% imposta lo stile di pagina per i titoli definito nel preambolo
\pagestyle{toc}
% imposta i numeri di pagina romani minuscoli
\pagenumbering{roman}

% genera automaticamente l'indice di LaTeX
\tableofcontents

% se è true \INDICETABELLE allora genera l'indice delle tabelle, altrimenti non fa nulla
\ifthenelse{\equal{\INDICETABELLE}{true}}{%
  \clearpage % l'indice delle tabelle, se c'è, deve andare a pagina nuova
  \listoftables
}{}

% se è true |INDICEFIGURE allora genera l'indice delle figure, altrimenti non fa nulla
\ifthenelse{\equal{\INDICEFIGURE}{true}}{%
  \clearpage % l'indice delle figure, se c'è, deve andare a pagina nuova
  \listoffigures
}{}

%in ogni caso occorre andare a pagina nuova dopo gli indici
\clearpage


% Alcuni aggiustamenti per le pagine
\pagenumbering{arabic}
\setcounter{page}{1}
\pagestyle{normal}

% Qui ha inizio il documento vero e proprio...

\section{Introduzione}
\subsection{Scopo del prodotto}
\purpose

\subsection{Scopo del documento}
Il presente documento riporta il risultato dell'attività di analisi dei requisiti svolta dal gruppo \team{} in fase preliminare, vale a dire un insieme di requisiti -- in parte esplicitati nel testo del capitolato C1, in parte emersi durante incontri con il committente Zucchetti Srl, in parte inferiti dal dominio e in parte auto-imposti dal gruppo -- che il prodotto software è tenuto a soddisfare in termini funzionali, prestazionali, qualitativi e dichiarativi.

Il comportamento del sistema osservabile dall'utente finale, in conformità con i requisiti sopra enunciati, è riportato secondo il formalismo noto come diagrammi dei casi d'uso. La corrispondenza fra casi d'uso e requisiti è illustrata mediante la tabella riportata nella sezione~5.
%TODO: aggiungere riferimenti incrociati e non basarsi su numeri hard-coded qui dentro

\subsection{Glossario}
\glossaryIntro

\subsection{Riferimenti}
\begin{itemize}
\item Capitolato d'appalto C1\\ (\url{http://www.math.unipd.it/~tullio/IS-1/2012/Progetto/C1.pdf});
\item verbale relativo all'incontro con il proponente tenutosi in data 2012/12/11\\ (verbale\_incontro\_2012-12-11.pdf);
\item Piano di qualifica\\ (piano\_di\_qualifica.pdf).
\end{itemize}

\section{Descrizione generale}

\subsection{Contesto di utilizzo}
Il sistema software realizzato nell'ambito del progetto \caName{} si configura come una piattaforma di comunicazione fra utenti connessi alla rete. Oggetto della condivisione -- poiché `comunicare' letteralmente significa `mettere in comune' -- saranno non solo flussi audio/video, secondo le modalità tipiche della (video)chiamata via \inglese{softphone}, ma anche file e documenti presenti in locale, per i quali è previsto tanto il trasferimento quanto la condivisione delle modifiche in tempo reale.

Dal punto di vista dell'utente, \caName{} si configura inoltre come un applicativo fruibile attraverso la sola mediazione di un \underline{browser}, senza necessità di installazione di alcun programma \inglese{stand-alone} sul proprio sistema né di \underline{plugin} per il \underline{browser} forniti da \team{} o da terze parti.

\subsection{Utente finale}
Il modello implicitamente assunto come utente finale del prodotto prevede che l'utilizzatore del prodotto \caName{} abbia disponibilità di una connessione ad Internet funzionante e di un \underline{browser} di ultima generazione in grado di supportare lo standard \underline{WebRTC}\@. Fra i prerequisiti non figura il possesso di conoscenze particolari estranee alla comune esperienza di navigazione web (in particolare, l'eventuale confidenza con sistemi esistenti di telefonia via internet non rientra nelle assunzioni).

\subsection{Funzionalità}
Le funzionalità offerte dal prodotto agli utenti finali possono essere raggruppate in due macro-categorie: gestione di scambi di informazioni sincroni (quali comunicazione multimediale fra due o più utenti o condivisione di risorse in tempo reale) da una parte, e scambio di informazioni asincrone (segreteria, chat testuale, trasferimento file) dall'altra. A supporto di tali funzionalità di base è previsto lo sviluppo di funzionalità secondarie quali la gestione di una rubrica utente, la visualizzazione di metadati sulle connessioni e la possibilità di mantenere un registro storico delle attività realizzate mediante il sistema.

%da qua in poi mi affido al software di stampa automatico
% TODO: sono gestiti i riferimenti incrociati?
\section{Diagrammi dei casi d'uso}

\section{Requisiti}

RSDD2.1.0 & L'utente dovrebbe avere una stima della complessità per la propria password. & Capitolato d'appalto \\
RSDD2.2.0 & Convalidare username (e-mail) dell'utente & Interno \\
RSDD2.3.0 & Inserimento dati anagrafici (nome, cognome), immagine e indirizzo email (utilizzato anche come username) & Interno \\
RSDO10.0.0 & L'intero sistema deve essere contenuto in un unica pagina Web & Capitolato d'appalto \\
RSDO10.1.0 & L'interfaccia grafica non deve subire refresh per ogni operazione dell'utente & Interno \\
RSDO22.0.0 & L'applicativo deve funzionare sotto l'ultima versione (23.0.1271.97m) del browser Google Chrome & Capitolato d'appalto \\
RSDO6.0.0 & Gestire le comunicazioni utente tramite WebRTC & Capitolato d'appalto \\
RSFF11.2.0 & Utilizzo del protocollo webSocket per creare una connessione tra più di 2 utenti & Interno \\
RSFF12.1.0 & Estendere la connessione ad altri client & Interno \\
RSFO11.0.0 & Creare una connessione tra client, mediante l'utilizzo di un server & Capitolato d'appalto \\
RSFO11.1.0 & Utilizzo del protocollo webSocket per creare una connessione tra 2 utenti & Interno \\

RSFO12.0.0 & Gestire gli eventi dell'utente durante la connessione & Interno \\
RSQD21.0.0 & Gestione interfaccia grafica in più lingue & Interno \\
RSQF23.0.0 & Verificare che l'applicativo funzioni anche sotto gli altri browser del S.O. Windows & Interno \\
RSQF23.1.0 & Verificare che funzioni con Opera & Capitolato d'appalto \\
RSQF23.2.0 & Verificare che funzioni con Firefox & Capitolato d'appalto \\
RSQF23.3.0 & Verificare che funzioni con Internet Explorer (v9 e superiori) & Capitolato d'appalto \\
RSQF23.4.0 & Verificare che funzioni con Safari & Capitolato d'appalto \\
RSQF24.0.0 & Verificare che l'applicativo funzioni anche sotto gli altri browser del S.O. Linux & Interno \\
RSQF24.1.0 & Verificare che funzioni con Opera & Capitolato d'appalto \\
RSQF24.2.0 & Verificare che funzioni con Firefox & Capitolato d'appalto \\
RSQF24.3.0 & Verificare che funzioni con Chromium & Interno \\
RSQF25.0.0 & Verificare che l'applicativo funzioni anche sotto gli altri browser del S.O. Macintosh & Interno \\
RSQF25.1.0 & Verificare che funzioni con Opera & Capitolato d'appalto \\
RSQF25.2.0 & Verificare che funzioni con Firefox & Capitolato d'appalto \\
RSQF25.3.0 & Verificare che funzioni con Safari & Capitolato d'appalto \\
RUFD1.1.0 & Gestione password dimenticata & Interno \\
RUFD1.1.1 & Proporre la domanda segreta all'utente & Interno \\
RUFD1.1.2 & Invio di una mail all'utente contenete la password dimenticata & Interno \\
RUFD12.2.0 & Chi crea la connessione può eliminare i membri del gruppo & Interno \\
RUFD13.1.0 & Servizio chat tra 2 utenti & Interno \\
RUFF13.0.0 & Servizio chat testuale & Capitolato d'appalto \\
RUFF13.2.0 & Servizio chat tra più di 2 utenti & Interno \\
RUFF14.0.0 & Registrazione della chiamata & Capitolato d'appalto \\
RUFF14.1.0 & Necessità di autorizzazione dagli utenti della chiamata per poter avviare la registrazione & Interno \\
RUFF14.2.0 & Registrazione audio & Interno \\
RUFF14.3.0 & Possibilità di riascoltare la registrazione & Interno \\

RUFF15.0.0 & L'utente avrà a disposizione una segreteria telefonica & Interno \\
RUFF15.1.0 & Possibilità di lasciare un audio messaggio in segreteria & Interno \\
RUFF15.2.0 & Possibilità di lasciare un audio/video messaggio in segreteria & Capitolato d'appalto \\
RUFF15.3.0 & Possibilità di ascoltare la propria segreteria & Interno \\
RUFF15.4.0 & Possibilità di cancellare messaggi della segretaria & Interno \\
RUFF15.5.0 & Possibilità di cambiare lo stato del messaggio tra ascoltato/non ascoltato & Interno \\
RUFF16.0.0 & Possibilità di impostare uno stato utente & Interno \\
RUFF17.0.0 & Dare la possibilità di vedere gli stati personali altrui. & Interno \\
RUFF18.0.0 & Creazione di una Blacklist & Interno \\
RUFF19.0.0 & Possibilità di tenere uno storico delle chiamate & Interno \\
RUFF20.0.0 & Condividere risorse & Interno \\
RUFF20.1.0 & Condividere monitor & Interno \\
RUFF20.2.0 & Condivisione pdf & Interno \\

RUFF20.3.0 & Condivisione lavagna grafica & Interno \\
RUFF20.4.0 & Invio file & Interno \\
RUFF3.0.0 & Modifica dati utente (ad eccezione dell'indirizzo email) & Interno \\
RUFF4.0.0 & L'utente dovrebbe poter tenere tenere una rubrica personale dove raggruppare i propri contatti. & Interno \\
RUFF4.1.0 & Un utente deve poter inserire utenti dalla propria rubrica. & Capitolato d'appalto \\
RUFF4.2.0 & Eliminare un utente dalla propria rubrica & Interno \\
RUFF4.3.0 & Possibilità di ordinare la rubrica su alcuni parametri rilevanti & Interno \\
RUFF4.4.0 & Possibilità di suddividere la rubrica in gruppi & Interno \\
RUFF4.4.1 & Possibilità di togliere un elemento da un gruppo & Interno \\
RUFF4.4.2 & Possibilità di aggiungere un elemento in un gruppo & Interno \\
RUFF4.4.3 & Possibilità di creare un gruppo & Interno \\
RUFF4.4.4 & Possibilità di eliminare un gruppo & Interno \\
RUFF4.5.0 & Possibilità di esportare in xml la rubrica personale & Interno \\
RUFF4.6.0 & Possibilità di modificare la rubrica importando un file xml & Interno \\
RUFF4.7.0 & Ricerca di un utente nella propria rubrica & Interno \\
RUFF5.1.0 & Possibilità di cercare un utente dalla lista & Interno \\
RUFF6.1.2 & Stabilire una comunicazione audio con un utente registrato e presente nella rubrica & Capitolato d'appalto \\
RUFF6.1.4 & Promuovere una comunicazione audio avviata con un utente in una comunicazione audio/video & Interno \\
RUFF6.2.2 & Stabilire una comunicazione audio/video con un utente registrato e presente nella rubrica & Capitolato d'appalto \\
RUFF6.2.4 & Declassare una comunicazione audio/video avviata con un utente in una comunicazione solo audio & Interno \\
RUFF6.2.5 & Disattivare la webcam utente pur continuando a ricevere il segnale video proveniente dall'altro capo della comunicazione & Interno \\
RUFF6.3.0 & Disattivare il microfono utente pur continuando a ricevere il segnale video proveniente dall'altro capo della comunicazione & Interno \\
RUFF9.3.0 & Rilevazione frame per secondo & Interno \\
RUFO1.0.0 & L'utente deve potersi autentificare nel server, cosi da permettere a quest'ultimo di rilevare la sua presenza nel sistema. Il Login dovrà essere gestito con username (indirizzo mail) e password. & Interno \\
RUFO12.3.0 & Chi partecipa alla connessione può togliersi da essa & Interno \\
RUFO2.0.0 & Registrazione del nuovo utente & Capitolato d'appalto \\
RUFO5.0.0 & Possibilità di visualizzare la lista di utenti registrati nel sistema & Capitolato d'appalto \\
RUFO6.1.0 & Stabilire e gestire una comunicazione audio con un utente in linea & Capitolato d'appalto \\
RUFO6.1.1 & Stabilire una comunicazione audio mediante inserimento d'indirizzo IP & Capitolato d'appalto \\
RUFO6.1.3 & Stabilire una comunicazione audio con un utente registrato e NON presente nella rubrica & Capitolato d'appalto \\
RUFO6.2.0 & Stabilire e gestire  una comunicazione audio/video con un utente & Capitolato d'appalto \\
RUFO6.2.1 & Stabilire una comunicazione audio/video mediante inserimento d'indirizzo IP & Capitolato d'appalto \\
RUFO6.2.3 & Stabilire una comunicazione audio/video con un utente registrato e NON presente nella rubrica & Capitolato d'appalto \\
RUFO7.0.0 & Indicare il tempo di comunicazione & Capitolato d'appalto \\
RUFO8.0.0 & Valutare il numero di byte trasmessi & Capitolato d'appalto \\
RUFO8.1.0 & Valutare il numero di byte inviati & Interno \\
RUFO8.2.0 & Valutare il numero di byte ricevuti & Interno \\
RUFO9.0.0 & Indicare la qualità della linea di trasmissione & Capitolato d'appalto \\
RUFO9.1.0 & Rilevare latenza & Capitolato d'appalto \\
RUFO9.2.0 & Rilevare velocità di trasmissione & Capitolato d'appalto \\
\bottomrule
\end{longtable}

\section{Tracciamento dei requisiti}

\begin{tabularx}{\textwidth}{lXl}
\toprule Codice UC & Nome UC  & Requisito\\
\midrule
\bottomrule
\end{tabularx}


\end{document}
