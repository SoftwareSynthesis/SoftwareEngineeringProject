% ATTENZIONE!!! 
% Per far funzionare i collegamenti ipertestuali si raccomanda di usare
%	\classname{nomedellaclasse}
% per le classi dello stesso package mentre invece 
%	\hyperref[nomedellaclasse]{\ttfamily{}nomequalificatodellaclasse}
% per le classi che non sono dello stesso package e che hanno il nome completo

% **************************************************
% Macro specifiche per il documento corrente
% **************************************************
% Nome
\newcommand{\docName}{Definizione di prodotto}
% Nome file
\newcommand{\docFileName}{definizione\_di\_prodotto.1.0.pdf}
% Versione
\newcommand{\docVers}{1.0}
% Data creazione
\newcommand{\creationDate}{2013-02-03}
% Data ultima modifica
\newcommand{\modificationDate}{2013-02-03}
% Stato in {Approvato, Non approvato}
\newcommand{\docState}{Non approvato}
% Uso in {Interno, Esterno}
\newcommand{\docUsage}{Esterno}
% Destinatari da specificare come nome1\\ &nome2\\ ecc.
\newcommand{\docDistributionList}{Prof. Tullio Vardanega\\&Prof. Riccardo Cardin\\&Dott. Gregorio Piccoli\\&Team SoftwareSynthesis}
% Redattori da specificare come nome1\\ &nome2\\ ecc.
\newcommand{\docAuthors}{Elena Zecchinato\\&Andrea Meneghinello\\&Riccardo Tresoldi}
% Approvato da
\newcommand{\approvedBy}{Andrea Rizzi}
% Verificatori
\newcommand{\verifiedBy}{Stefano Farronato\\&Marco Schivo}
% Perscorso (relativo o assoluto) che punta alla directory contenente shared/
% come sua sottodirectory (per comodità chiamiamola 'doc root').
\newcommand{\docRoot}{..}
% definire se si vuole l'indice delle tabelle
\def\INDICETABELLE{false}
% definire se si vuole l'indice delle figure
\def\INDICEFIGURE{false}

% importa il preambolo condiviso da tutti i documenti
% shared/preamble.tex
%
% Questo documento contiene la parte del preambolo condivisa e viene pertanto
% richiamato nel 'master' di tutti i documenti di progetto.  Al suo interno
% contiene le inclusioni (e le configurazioni) di tutti i package richiesti per
% la compilazione dei documenti, le macro di carattere generale e la definizione
% degli stili di pagina.

\documentclass[a4paper,10pt]{article}

% **************************************************
% Macro generiche
% **************************************************
\newcommand{\team}{Software Synthesis}                    % chi siamo
\newcommand{\email}{info@softwaresynthesis.org}           % e-mail
\newcommand{\caName}{MyTalk}                              % titolo capitolato
\newcommand{\manager}{SynthesisRequirementManager}        % nome del sistema di tracciamento
\newcommand{\memberdata}[1]{%
  \texttt{\textcolor{RedOrange}{#1}}}                     % attributi di una classe
\newcommand{\method}[1]{\texttt{\textcolor{Emerald}{#1}}} % metodi di una classe
\newcommand{\exception}[1]{%
  \texttt{\textcolor{RedViolet}{#1}}}                     % eccezione
% \newcommand{\handler}[1]{\texttt{\textcolor{Maroon}{#1}}} % per gli event handler
\newcommand{\inglese}[1]{%
  \foreignlanguage{english}{\textit{#1}}}                 % per i testi in lingua inglese
\newcommand{\purpose}{%                                     scopo del prodotto
Con il progetto ``\caName'' si intende un sistema software di comunicazione tra utenti mediante \underline{browser} senza la necessit{\`a} di installazione di \underline{plugin} e/o software esterni. L'utilizzatore avr{\`a} la possibilit{\`a} di interagire con un altro utente tramite una comunicazione audio - audio/video - testuale e, inoltre, ottenere delle statistiche sull'attivit{\`a} in tempo reale.%
}
\newcommand{\glossaryIntro}{%                               introduzione al glossario
Al fine di evitare incomprensioni dovute all'uso di termini tecnici nei documenti, viene redatto e allegato il documento \textit{glossario.4.0.pdf} dove vengono definiti e descritti tutti i termini marcati con una sottolineatura.%
}


% **************************************************
% Codifica e lingua dei documenti
% **************************************************
\usepackage[utf8x]{inputenc}                              % codifica caratteri dei documenti sorgenti
\usepackage[english,italian]{babel}                       % localizzazione ai fini di sillabazione e cross-references
\usepackage[T1]{fontenc}                                  % codifica font di output

% **************************************************
% Definizione geometria della pagina
% **************************************************
\usepackage[a4paper,head=4cm,top=4.5cm,bottom=3cm,left=3cm,right=3cm,bindingoffset=5mm]{geometry}

% *************************************************
% Intestazioni e piè di pagina personalizzati
% *************************************************
\usepackage{fancyhdr}
% stile normale
\fancypagestyle{normal}{
\fancyhead{}                                              % intestazione
\fancyhead[RE,RO]{
\begin{picture}(0,0)
  \put(-410,0){\includegraphics[width=1.02\textwidth]{header_logo}}
  \put(-410,10){\sffamily\large\leftmark}
\end{picture}
\vspace{-4pt}
}
\renewcommand{\headrulewidth}{0pt}                       % riga sotto l'intestazione
\cfoot{}                                                  % piè di pagina
\fancyfoot[RO,LE]{\sffamily
  pag.~\thepage{} di \pageref{LastPage}}                  % a dx nelle pag. dispari e a sx in quelle pari
\fancyfoot[RE,LO]{\sffamily\docFileName{}}
\renewcommand{\footrulewidth}{.4pt}                       % riga sopra il piè di pagina
}
% stile per gli indici
\fancypagestyle{toc}{
\fancyhead{}                                              % intestazione
\fancyhead[RE,RO]{
\begin{picture}(0,0)
  \put(-410,0){\includegraphics[width=1.02\textwidth]{header_logo}}
\end{picture}
}
\renewcommand{\headrule}{}                                % nessuna riga sotto l'intestazione
\cfoot{}                                                  % piè di pagina
\fancyfoot[RO,LE]{\sffamily\thepage{}}                    % a dx nelle pag. dispari e a sx in quelle pari
\fancyfoot[RE,LO]{\sffamily\docFileName{} -- v.\docVers}
\renewcommand{\footrulewidth}{.4pt}                       % riga sopra il piè di pagina
}

\pagestyle{fancy}                                         % premetto: non so usare bene le marche:
\renewcommand{\sectionmark}[1]{\markboth{#1}{#1}}         % se qualcuno ha idee migliori si faccia avanti!

% **************************************************
% Tabelle
% **************************************************
\usepackage{tabularx}                                     % tabelle di larghezza fissa con una o più colonne variabili
\usepackage{multirow}                                     % colonne con colonne che si estendono per più righe
\usepackage{booktabs}                                     % per inserire l'ambiente table e le righe orizz. nelle tabelle
\usepackage{longtable}			                              % tabelle oltre i limiti di pagina

% **************************************************
% Cross-references e collegamenti ipertestuali
% **************************************************
\usepackage[hidelinks]{hyperref}
\hypersetup{%
  colorlinks=false, linktocpage=false, pdfborder={0,0,0}, pdfstartpage=1, pdfstartview=FitV,%
  urlcolor=Cyan, linkcolor=Cyan, citecolor=Black, %pagecolor=Black,%
  pdftitle={\docName}, pdfauthor={\team}, pdfsubject={}, pdfkeywords={},%
  pdfcreator={pdflatex}, pdfproducer={pdflatex with hyperref package}%
}

% **************************************************
% Immagini e grafica
% **************************************************
\usepackage{graphicx}                                     % supporto ad aspetti avanzati delle immagini
\usepackage[table,usenames,dvipsnames]{xcolor}            % tabelle con righe colorate e alternate
\graphicspath{{\docRoot/pics/}}                           % percorso contenente tutti i file immagini
\usepackage{float}                                        % per rendere non flottanti gli ambienti flottanti
\usepackage[italian]{varioref}                            % testo completo riferimenti in italiano

% **************************************************
% Definizioni di colori
% **************************************************
\definecolor{myBlue}{RGB}{1,167,236}
\definecolor{lightblue}{RGB}{213,243,253}%{119,218,247}
\definecolor{llightblue}{RGB}{229,255,255}

% **************************************************
% Altri pacchetti opzionali
% **************************************************     
\usepackage{lastpage}                                     % per sapere il numero totale di pagine
\usepackage{eurosym}                                      % per il simbolo dell'euro usare \EUR{x} dove x è l'importo
\usepackage{ifthen}                                       % permette la scelta di rami condizionali nella compilazione
\usepackage{enumitem}                                     % permette di configurare gli elenchi puntati e numerati


% macro specifiche per il documento corrente
\newcommand{\memberdata}[1]{\texttt{\textcolor{RedOrange}{#1}}}
\newcommand{\method}[1]{\texttt{\textcolor{Emerald}{#1}}}
\newcommand{\exception}[1]{\texttt{\textcolor{RedViolet}{#1}}}
\newcommand{\classsection}[1]{\subsubsection{#1}\label{#1}}
\newcommand{\classname}[1]{\hyperref[#1]{\ttfamily#1}}

% Fine del preambolo e inizio del documento
\begin{document}

% Inclusione della prima pagina
% shared/firstpage.tex
%
% Questo documento definisce il contenuto della prima pagina, che si suppone
% essere uguale in tutti i documenti.  Oltre al logo e al titolo, la prima
% pagina contiene i metadati relativi al documento in cui viene inclusa.


% rimuove intestazioni e piè di pagina
\pagestyle{empty}

\begin{center}

% logo del gruppo
\includegraphics[width=1.5\textwidth]{logo}

\vspace{1in}

% titolo del documento
{\Huge\bfseries \docName}

\vspace{1in}

% tabella riepilogativa
\begin{tabularx}{.7\textwidth}{>{\bfseries\sffamily}l>{\sffamily}l}
\toprule
\multicolumn{2}{>{\sffamily}c}{Informazioni sul documento}\\
\midrule
Nome file:            & \docFileName\\
Versione:             & \docVers\\
Data creazione:       & \creationDate\\
Data ultima modifica: & \modificationDate\\
Stato:                & \docState\\
Uso:                  & \docUsage\\
Redattori:            & \docAuthors\\
Approvato da:         & \approvedBy\\
Verificatori:         & \verifiedBy\\
\bottomrule
\end{tabularx}

\end{center}

\newpage


%---------------------------RUOLI----------------------------
%FASE 1:
%Programmatori: DIEGO, STEFANO, SCHIVO
%FASE 2:
%Programmatori: MENE, TRES, ELENA

%Verificatore: SCHIVO, STEFANO, RIZZI, DIEGO (dobbiamo fare un sacco di test)
%Responsabile finale supremo: RIZZI
%------------------------------------------------------------

% Storico delle modifiche
\section*{Storia delle modifiche}
\begin{center}
\begin{longtable}{lp{.32\textwidth}lll}
\toprule
Versione & Descrizione intervento & Membro & Ruolo & Data\\
\midrule % inserire qui il contenuto della tabella
0.1 & Creazione del documento e stesura delle sezioni ``Introduzione'' e ``Riferimenti''. & &  & 2013-02-03\\
\bottomrule
\end{longtable}
\end{center}
\newpage

% inclusione dell'indice
% shared/toc.tex
%
% Questo file contiene le istruzioni che generano l'indice o gli indici del
% documento (utile nel caso in cui decidessimo di avere anche un indice delle
% tabelle e/o un indice delle figure).

% imposta lo stile di pagina per i titoli definito nel preambolo
\pagestyle{toc}
% imposta i numeri di pagina romani minuscoli
\pagenumbering{roman}

% genera automaticamente l'indice di LaTeX
\tableofcontents

% se è true \INDICETABELLE allora genera l'indice delle tabelle, altrimenti non fa nulla
\ifthenelse{\equal{\INDICETABELLE}{true}}{%
  \clearpage % l'indice delle tabelle, se c'è, deve andare a pagina nuova
  \listoftables
}{}

% se è true |INDICEFIGURE allora genera l'indice delle figure, altrimenti non fa nulla
\ifthenelse{\equal{\INDICEFIGURE}{true}}{%
  \clearpage % l'indice delle figure, se c'è, deve andare a pagina nuova
  \listoffigures
}{}

%in ogni caso occorre andare a pagina nuova dopo gli indici
\clearpage


% Alcuni aggiustamenti per le pagine
\pagenumbering{arabic}
\setcounter{page}{1}
\pagestyle{normal}

% Qui ha inizio il documento vero e proprio...
\newpage

\section{Introduzione}
\subsection{Scopo del prodotto}
\purpose

\subsection{Scopo del documento}
Il presente documento presenta una descrizione dettagliata dell'architettura del sistema software destinata a costituire il prodotto \caName{}, coerentemente con la progettazione ad alto livello contenuta nell'allegato \textit{specifica\_tecnica.2.0.pdf}.

A tal fine, si riporta per ognuno dei componenti definiti nel documento di specifica tecnica una descrizione delle classi in termini di operazioni disponibili, proprietà, responsabilità e collaborazioni. Il contenuto del presente documento ha inoltre valore vincolante per i programmatori, che avranno l'obbligo di attenersi alle disposizioni in esso contenute senza alcuna possibilità di deroga.

\subsection{Glossario}
\glossaryIntro

\subsection{Convenzioni di scrittura}
% vedere issue #56 al riguardo
Al fine di rendere quanto più agevole possibile la consultazione di questo documento da parte dei programmatori e del committente, è stata adottata una serie di accorgimenti. In primo luogo, sono stati inseriti riferimenti incrociati sui nomi delle classi aventi come destinazione la sottosezione che descrive la classe stessa.

In secondo luogo, per i membri di classi e interfacce, oltre a una famiglia di font a spaziatura fissa, sono state adottate le seguenti convenzioni cromatiche:
\begin{itemize}[noitemsep,nolistsep]
  \item \memberdata{\bfseries rosso} per i campi dati;
  \item \method{\bfseries verde} per i metodi.
\end{itemize}

I differenti livelli di accessibilità sono contrassegnati, in maniera analoga a UML, come:
\begin{itemize}
  \item \texttt{\ttfamily +} per la parte pubblica;
  \item \texttt{\ttfamily \#} per la parte protetta;
  \item \texttt{\ttfamily \textasciitilde} per  l'accessibilità di package;
  \item \texttt{\ttfamily --} per la parte privata.
\end{itemize}

Infine, i membri statici di classi e interfacce sono contrassegnati con una \underline{\texttt{sottolineatura}}.
\clearpage

\section{Riferimenti}
\subsection{Normativi}
\begin{itemize}
\item[] \textit{piano\_di\_qualifica.3.0.pdf} allegato.
\item[] \textit{norme\_di\_progetto.3.0.pdf} allegato.
\item[] \textit{specifica\_tecnica.2.0.pdf} allegato
\end{itemize}

\subsection{Informativi}
\begin{itemize}
\item[] Capitolato d'appalto: \caName{}, v1.0, redatto e rilasciato dal proponente Zucchetti s.r.l. reperibile all'indirizzo \url{http://www.math.unipd.it/~tullio/IS-1/2012/Progetto/C1.pdf};
\item[] testo di consultazione: \textit{Software Engineering (8th edition) Ian Sommerville, Pearson Education | Addison Wesley};
\item[] manuale all'utilizzo dei design pattens: \textit{Design Patterns, Elementi per il riuso di software a oggetti -- (1/Ed. italiana) Eric Gamma, Richard Helm, Ralph Johnson, John Vlissides, Pearson Education};
\item[] \textit{glossario.3.0.pdf} allegato.
\end{itemize}
\clearpage

\section{Standard di progetto}

\subsection{Standard di progettazione architetturale}

\subsection{Standard di documentazione del codice}
% dobbiamo pensare se enumerarle qui oppure se rimandare a una sezione delle NdP
% vedere issue #54 al riguardo

\subsection{Standard di denominazione di entità e relazioni}

\subsection{Standard di programmazione}

\subsection{Strumenti di lavoro}

\clearpage

\section{Specifica sotto-architettura sever}\label{sec:serverarchitecture}

\subsection{Package org.softwaresynthesis.mytalk.server.dao}\label{sec:dao}

\classsection{IMessage}

\subsubsection*{Funzione}
L'interfaccia rappresenta un generico messaggio della segreteria.

\subsubsection*{Relazioni d'uso}

\subsubsection*{Metodi}
\begin{description}
	\item{\method{+ getSender(): IUserData}}\\
Restituisce il mittente del messaggio, sotto forma di un sottotipo (design pattern Proxy) di \classname{IUserData}.
	\item{\method{+ getDate(): Date}}\\
Restituisce la data di creazione del messaggio (appoggiandosi a \texttt{java.util.Date}).
	\item{\method{+ isNew(): boolean}}\\
Restituisce un valore booleano che fornisce le informazioni relative all'ascolto del messaggio: \texttt{true} se è stato ascoltato, \texttt{false} altrimenti.
	\item{\method{+ getLength(): int}}\\
Restituisce la lunghezza del messaggio espressa in secondi.
	\item{\method{+ getStream(): byte[]}}\\
Restituisce lo stream dati del messaggio.
	\item{\method{+ getSize(): int}}\\
Restituisce la dimensione del messaggio espressa in byte.
	\item{\method{+ setNew(value: boolean): void}}\\
Imposta il valore che determina se un messaggio è stato ascoltato o meno.

\end{description}

\classsection{IAudioMessage}

\subsubsection*{Interfacce estese}
\begin{itemize}[noitemsep,nolistsep]
  \item[-] \classname{IMessage}
\end{itemize}

\subsubsection*{Funzione}
L'interfaccia rappresenta un messaggio audio lasciato nella segreteria dell'utente.

\subsubsection*{Relazioni d'uso}

\subsubsection*{Metodi}
\begin{description}
  \item{\method{+ getAudio(): byte[]}}\\
Restituisce il contenuto audio del messaggio attraverso il metodo \method{getStream()} dell'interfaccia \classname{IMessage}.
\end{description}

\classsection{IAudioVideoMessage}

\subsubsection*{Interfacce estese}
\begin{itemize}[noitemsep,nolistsep]
  \item[-] \classname{IMessage}
\end{itemize}

\subsubsection*{Funzione}
L'interfaccia rappresenta un messaggio audio/video lasciato nella segreteria dell'utente.

\subsubsection*{Metodi}
\begin{description}
  \item{\method{+ getAudioVideo(): byte[]}}\\
Restituisce il contenuto audio-video del messaggio attraverso il metodo \method{getStream()} dell'interfaccia \classname{IMessage}.
\end{description}

\classsection{AudioMessage}

\subsubsection*{Interfacce implementate}
\begin{itemize}[noitemsep,nolistsep]
  \item[-] \classname{IAudioMessage}
\end{itemize}

\subsubsection*{Funzionalità}
La classe rappresenta il messaggio audio vero e proprio memorizzato nella segreteria dell'utente.

\subsubsection*{Relazioni d'uso}

\subsubsection*{Attributi}
\begin{description}
  \item{\memberdata{-- sender: IUserData}}\\
Riferimento all'utente che costituisce il mittente del messaggio audio.
  \item{\memberdata{-- date: Date}}\\
Data in cui è stato creato il messaggio audio.
  \item{\memberdata{-- path: String}}\\
Percorso che punta alla rappresentazione in forma persistente del messaggio audio.
  \item{\memberdata{-- new: boolean}}\\
Flag booleano che permette di discriminare i messaggi audio ascoltati da quelli non ascoltati.
  \item{\memberdata{-- content: byte[]}}\\
Rappresentazione del contenuto del messaggio audio.
\end{description}

\subsubsection*{Metodi}
Oltre alle implementazioni dell'interfaccia \classname{IAudioMessage}, la classe contiene i seguenti metodi:
\begin{description}
  \item{\method{AudioMessage(path: String)}}\\
Costruttore dell'oggetto \classname{AudioMessage}.
\end{description}

\classsection{AudioVideoMessage}

\subsubsection*{Interfacce implementate}
\begin{itemize}[noitemsep,nolistsep]
  \item[-] \classname{IAudioVideoMessage}
\end{itemize}

\subsubsection*{Funzionalità}
La classe rappresenta il messaggio audio/video vero e proprio memorizzato nella segreteria dell'utente.

\subsubsection*{Relazioni d'uso}

\subsubsection*{Attributi}
\begin{description}
  \item{\memberdata{-- sender: IUserData}}\\
Riferimento all'utente che costituisce il mittente del messaggio audio/video.
  \item{\memberdata{-- date: Date}}\\
Data in cui è stato creato il messaggio audio/video.
  \item{\memberdata{-- path: String}}\\
Percorso che punta alla rappresentazione in forma persistente del messaggio audio/video.
  \item{\memberdata{-- new: boolean}}\\
Flag booleano che permette di discriminare i messaggi audio/video ascoltati da quelli non ascoltati.
  \item{\memberdata{-- content: byte[]}}\\
Rappresentazione del contenuto del messaggio audio/video.
\end{description}

\subsubsection*{Metodi}
Oltre alle implementazioni dell'interfaccia \classname{IAudioVideoMessage}, la classe contiene i seguenti metodi:
\begin{description}
  \item{\method{AudioVideoMessage(path: String)}}\\
Costruttore dell'oggetto \classname{AudioVideoMessagio()}.
\end{description}

\classsection{IGroup}

\subsubsection*{Interfacce estese}
\begin{itemize}[noitemsep,nolistsep]
  \item[-] \hyperref[IContact]{\ttfamily{}org.softwaresynthesis.mytalk.abook.IContact}
\end{itemize}

\subsubsection*{Funzionalità}

\subsubsection*{Relazioni d'uso}

\subsubsection*{Metodi}
\begin{description}
	\item{\method{+ addContact(contact: IContact): void}}\\
Aggiunge al gruppo un nuovo oggetto IContact, può essere un sottogruppo oppure un singolo utente.
	\item{\method{+ getOwner(): IUserData}}\\
Restituisce l'utente proprietario del gruppo.
	\item{\method{+ getContact(): List<IContact>}}\\
Restituisce i contatti, sottogruppi oppure utenti, appartenenti al gruppo.
	\item{\method{+ setOwner(user: IUserData): void}}\\
Imposta l'utente proprietario del gruppo.
\end{description}

\classsection{StandardGroup}

\subsubsection*{Interfacce implementate}
\begin{itemize}[noitemsep,nolistsep]
  \item[-] \hyperref[IContact]{\ttfamily{}org.softwaresynthesis.mytalk.abook.IContact}
  \item[-] \classname{IGroup}
\end{itemize}

\subsubsection*{Funzionalità}

\subsubsection*{Relazioni d'uso}

\subsubsection*{Attributi}

\subsubsection*{Metodi}

\classsection{IUserData}
\subsubsection*{Interfacce estese}
\begin{itemize}[noitemsep,nolistsep]
  \item[-] \hyperref[IContact]{\ttfamily{}org.softwaresynthesis.mytalk.abook.IContact}
\end{itemize}

\subsubsection*{Funzionalità}

\subsubsection*{Relazioni d'uso}

\subsubsection*{Metodi}
\begin{description}
	\item{\method{+ getEMail(): string}}\\
Restituisce l'indirizzo e-mail con cui l'utente si è registrato.
	\item{\method{+ getPassword(): string}}\\
Restituisce la password, non criptata, dell'utente.
	\item{\method{+ getQuestion(): string}}\\
Restituisce la domanda, scelta dall'utente in fase di registrazione, per il recupero della password.
	\item{\method{+ getName(): string}}\\
Restituisce il nome dell'utente.
	\item{\method{+ getSurname(): string}}\\
Restituisce il cognome dell'utente.
	\item{\method{+ getPicture(): Image}}\\
Restituisce l'immagine che l'utente ha associato al proprio account.
	\item{\method{+ getAddressBook(): IAddressBook}}\\
Restituisce la rubrica dell'utente.
	\item{\method{+ getCallList(): List<ICall>}}\\
Restituisce la lista delle chiamate effettuate e ricevute dall'utente.
	\item{\method{+ getMessages(): List<IMessage>}}\\
Restituisce la lista di tutti i messaggi memorizzati nella segreteria dell'utente
	\item{\method{+ getMessages(news: Boolean): List<IMessage>}}\\
Restituisce la lista dei nuovi messaggi, in segreteria, se news è impostato a true altrimenti ritorna tutti quelli memorizzati.
	\item{\method{+ setEMail(mail: string): void}}\\
Imposta l'indirizzo e-mail di un utente.
	\item{\method{+ setPassword(password): void}}\\
Imposta la password che l'utente utilizza per accedere al sistema \caName.
	\item{\method{+ setQuestion(question: string): void}}\\
Imposta la domanda, scelta dall'utente utilizzata nella procedura di recupero della password.
	\item{\method{+ setAnswer(answer: string): void}}\\
Imposta la risposta alla domanda segreta utilizzata per la procedura di recupero della password.
	\item{\method{+ setName(name: string): void}}\\
Imposta il nome dell'utente.
	\item{\method{+ setSurname(surname: string): void}}\\
Imposta il cognome dell'utente.
	\item{\method{+ setPicture(image: Image): void}}\\
Imposta l'immagine relativa all'utente.
\end{description}

\classsection{StandardUserData}
\subsubsection*{Interfacce implementate}
\begin{itemize}[noitemsep,nolistsep]
  \item[-] \hyperref[IContact]{\ttfamily{}org.softwaresynthesis.mytalk.abook.IContact}
  \item[-] \classname{IUserData}
\end{itemize}

\subsection{Package org.softwaresynthesis.mytalk.server.connection}\label{sec:connection}

\classsection{ICommunicationHandler}

\subsubsection*{Funzionalità}

\subsubsection*{Relazioni d'uso}


\subsubsection*{Metodi}

\classsection{StandardCommunicationHandler}

\subsubsection*{Interfacce implementate}
\begin{itemize}[noitemsep,nolistsep]
  \item[-] \classname{ICommunicationHandler}
\end{itemize}

\subsubsection*{Funzionalità}

\subsubsection*{Relazioni d'uso}

\subsubsection*{Attributi}

\subsubsection*{Metodi}

\classsection{IConnection}

\subsubsection*{Funzionalità}

\subsubsection*{Relazioni d'uso}

\subsubsection*{Metodi}

\classsection{WebRTCInfo}

\subsubsection*{Interfacce implementate}
\begin{itemize}[noitemsep,nolistsep]
  \item[-] \classname{IConnection}
\end{itemize}

\subsubsection*{Funzionalità}

\subsubsection*{Relazioni d'uso}

\subsubsection*{Attributi}

\subsubsection*{Metodi}

\subsection{Package org.softwaresynthesis.mytalk.server.abook}\label{sec:abook}

\classsection{IContact}

\subsubsection*{Funzionalità}

\subsubsection*{Relazioni d'uso}

\subsubsection*{Metodi}
\begin{description}
  \item{\method{+ getName(): String}}\\
Restituisce il nome del contatto.
  \item{\method{+ setName(String name): void}}\\
Imposta il nome del contatto.
  \item{\method{+ getChildrenNumber(): int}}\\
Restituisce il numero di figli di un determinato nodo.
  \item{\method{+ addChild(contact: IContact): voi}}\\
Aggiunge un contatto (utente singolo o gruppo) come figlio del nodo selezionato.
  \item{\method{+ deleteChild(contact: IContact): void}}\\
Cancella dai figli del contatto di invocazione il contatto \texttt{contact}.
\end{description}

\classsection{IAddressBook}

\subsubsection*{Funzionalità}

\subsubsection*{Relazioni d'uso}

\subsubsection*{Metodi}

\classsection{AddressBook}

\subsubsection*{Interfacce implementate}
\begin{itemize}[noitemsep,nolistsep]
  \item[-] \classname{IAddressBook}
\end{itemize}

\subsubsection*{Funzionalità}

\subsubsection*{Relazioni d'uso}

\subsubsection*{Attributi}

\subsubsection*{Metodi}

\classsection{UserDataProxy}

\subsubsection*{Interfacce implementate}
\begin{itemize}[noitemsep,nolistsep]
  \item[-] \hyperref[IUserData]{\ttfamily{}org.softwaresynthesis.mytalk.server.dao.IUserData}
\end{itemize}

\subsubsection*{Funzionalità}

\subsubsection*{Relazioni d'uso}

\subsubsection*{Attributi}

\subsubsection*{Metodi}

\subsection{Package org.softwaresynthesis.mytalk.server.state}\label{sec:state}

\classsection{IState}

\subsubsection*{Funzionalità}

\subsubsection*{Relazioni d'uso}

\subsubsection*{Metodi}

\classsection{StateOnline}

\subsubsection*{Interfacce implementate}
\begin{itemize}[noitemsep,nolistsep]
  \item[-] \classname{IState}
\end{itemize}

\subsubsection*{Funzionalità}

\subsubsection*{Relazioni d'uso}

\subsubsection*{Attributi}

\subsubsection*{Metodi}

\classsection{StateOffline}

\subsubsection*{Interfacce implementate}
\begin{itemize}[noitemsep,nolistsep]
  \item[-] \classname{IState}
\end{itemize}

\subsubsection*{Funzionalità}

\subsubsection*{Relazioni d'uso}

\subsubsection*{Attributi}

\subsubsection*{Metodi}

\classsection{StateAvailable}

\subsubsection*{Interfacce implementate}
\begin{itemize}[noitemsep,nolistsep]
  \item[-] \classname{IState}
\end{itemize}

\subsubsection*{Funzionalità}

\subsubsection*{Relazioni d'uso}

\subsubsection*{Attributi}

\subsubsection*{Metodi}

\classsection{StateOccupied}

\subsubsection*{Interfacce implementate}
\begin{itemize}[noitemsep,nolistsep]
  \item[-] \classname{IState}
\end{itemize}

\subsubsection*{Funzionalità}

\subsubsection*{Relazioni d'uso}

\subsubsection*{Attributi}

\subsubsection*{Metodi}

\subsection{Package org.softwaresynthesis.mytalk.server.message}\label{sec:message}

\classsection{IMessageBox}

\subsubsection*{Funzionalità}

\subsubsection*{Relazioni d'uso}

\subsubsection*{Metodi}

\classsection{StandardMessageBox}

\subsubsection*{Interfacce implementate}
\begin{itemize}[noitemsep,nolistsep]
  \item[-] \classname{IMessageBox}
\end{itemize}

\subsubsection*{Funzionalità}

\subsubsection*{Relazioni d'uso}

\subsubsection*{Attributi}

\subsubsection*{Metodi}

\classsection{AudioMessageProxy}

\subsubsection*{Interfacce implementate}
\begin{itemize}[noitemsep,nolistsep]
  \item[-] \hyperref[IAudioMessage]{\ttfamily{}org.softwaresynthesis.mytalk.IAudioMessage}
\end{itemize}

\subsubsection*{Funzionalità}

\subsubsection*{Relazioni d'uso}

\subsubsection*{Attributi}
\begin{description}
  \item{\memberdata{-- sender: IUserData}}\\
Riferimento all'utente che costituisce il mittente del messaggio audio.
  \item{\memberdata{-- date: Date}}\\
Data in cui è stato creato il messaggio audio.
  \item{\memberdata{-- path: String}}\\
Percorso che punta alla rappresentazione in forma persistente del messaggio audio.
  \item{\memberdata{-- new: boolean}}\\
Flag booleano che permette di discriminare i messaggi audio ascoltati da quelli non ascoltati.
  \item{\memberdata{-- msg: AudioMessage}}\\
Riferimento di tipo \hyperref[AudioMessage]{\texttt{org.softwaresynthesis.mytalk.server.dao.AudioMessage}} al messaggio reale  costruito mediante la tecnica di lazy initialization.
\end{description}

\subsubsection*{Metodi}
Oltre alle implementazioni dell'interfaccia \texttt{org.mytalk.server.dao.IAudioMessage}, la classe prevede i seguenti metodi:
\begin{description}
\item{\method{+ AudioMessageProxy(path: String)}}\\
\end{description}

\classsection{AudioVideoMessageProxy}

\subsubsection*{Interfacce implementate}
\begin{itemize}[noitemsep,nolistsep]
  \item[-] \hyperref[IAudioVideoMessage]{\ttfamily{}org.softwaresynthesis.mytalk.IAudioVideoMessage}
\end{itemize}

\subsubsection*{Funzionalità}

\subsubsection*{Relazioni d'uso}

\subsubsection*{Attributi}
\begin{description}
  \item{\memberdata{-- sender: IUserData}}\\
Riferimento all'utente che costituisce il mittente del messaggio audio/video.
  \item{\memberdata{-- date: Date}}\\
Data in cui è stato creato il messaggio audio/video.
  \item{\memberdata{-- path: String}}\\
Percorso che punta alla rappresentazione in forma persistente del messaggio audio/video.
  \item{\memberdata{-- new: boolean}}\\
Flag booleano che permette di discriminare i messaggi audio/video ascoltati da quelli non ascoltati.
  \item{\memberdata{-- msg: AudioVideoMessage}}\\
Riferimento di tipo \hyperref[AudioVideoMessage]{\ttfamily{}org.softwaresynthesis.mytalk.server.dao.AudioVideoMessage} al messaggio reale costruito mediante la tecnica di lazy initialization.
\end{description}

\subsubsection*{Metodi}
Oltre alle implementazioni dell'interfaccia \hyperref[IAudioVideoMessage]{\ttfamily{}org.mytalk.server.dao.IAudioVideoMessage}, la classe prevede i seguenti metodi:
\begin{description}
\item{\method{+ AudioMessageProxy(path: String)}}\\
\end{description}

\subsection{Package org.softwaresynthesis.mytalk.server}\label{sec:server}

\classsection{IServerFacade}

\subsubsection*{Funzionalità}

\subsubsection*{Relazioni d'uso}

\subsubsection*{Metodi}

\classsection{StandardServerFacade}

\subsubsection*{Interfacce implementate}
\begin{itemize}[noitemsep,nolistsep]
  \item[-] \classname{IServerFacade}
\end{itemize}

\subsubsection*{Funzionalità}

\subsubsection*{Relazioni d'uso}

\subsubsection*{Attributi}

\subsubsection*{Metodi}

\clearpage

\section{Specifica sotto-archiettura clientpresenter}\label{clientpresenterarchitecture}

\subsection{Package org.softwaresynthesis.mytalk.clientpresenter}\label{sec:clientpresetner}

\classsection{IClient}

\subsubsection*{Funzionalità}

\subsubsection*{Relazioni d'uso}

\subsubsection*{Metodi}

\classsection{StandardClient}

\subsubsection*{Interfacce implementate}
\begin{itemize}[noitemsep,nolistsep]
  \item[-] \classname{IClient}
\end{itemize}

\subsubsection*{Funzionalità}

\subsubsection*{Relazioni d'uso}

\subsubsection*{Attributi}

\subsubsection*{Metodi}

\classsection{IPresenterFacade}

\subsubsection*{Funzionalità}

\subsubsection*{Relazioni d'uso}

\subsubsection*{Metodi}

\classsection{StandardPresenterFacade}

\subsubsection*{Interfacce implementate}
\begin{itemize}[noitemsep,nolistsep]
  \item[-] \classname{IPresenterFacade}
\end{itemize}

\subsubsection*{Funzionalità}

\subsubsection*{Relazioni d'uso}

\subsubsection*{Attributi}

\subsubsection*{Metodi}

\clearpage

\section{Specifica sotto-architettura clientview}\label{sec:clientviewarchitecture}

\subsection{Package org.softwaresynthesis.mytalk.clientview}\label{sec:clientview}

\classsection{IViewFacade}

\classsection{StandardViewFacade}

\subsection{Package org.softwaresynthesis.clientview.gui}\label{sec:gui}

\classsection{GUIHandler}

\classsection{AddressBookPanel}

\classsection{MainPanel}

\classsection{ToolsPanel}

\classsection{SearchResultPanel}

\classsection{ContactPanel}

\classsection{MessagePanel}

\classsection{LanguagePanel}

\classsection{AccountSettingsPanel}

\classsection{CallHistoryPanel}

\clearpage

\section{Tracciamento della relazione componenti-requisiti}

\end{document}
