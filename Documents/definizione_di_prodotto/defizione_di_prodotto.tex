% ATTENZIONE!!! 
% Per far funzionare i collegamenti ipertestuali si raccomanda di usare
%	\classname{nomedellaclasse}
% per le classi dello stesso package mentre invece 
%	\hyperref[nomedellaclasse]{\ttfamily{}nomequalificatodellaclasse}
% per le classi che non sono dello stesso package e che hanno il nome completo

% **************************************************
% Macro specifiche per il documento corrente
% **************************************************
% Nome
\newcommand{\docName}{Definizione di prodotto}
% Nome file
\newcommand{\docFileName}{definizione\_di\_prodotto.1.0.pdf}
% Versione
\newcommand{\docVers}{1.0}
% Data creazione
\newcommand{\creationDate}{2013-02-03}
% Data ultima modifica
\newcommand{\modificationDate}{2013-02-03}
% Stato in {Approvato, Non approvato}
\newcommand{\docState}{Non approvato}
% Uso in {Interno, Esterno}
\newcommand{\docUsage}{Esterno}
% Destinatari da specificare come nome1\\ &nome2\\ ecc.
\newcommand{\docDistributionList}{Prof. Tullio Vardanega\\&Prof. Riccardo Cardin\\&Dott. Gregorio Piccoli\\&Team SoftwareSynthesis}
% Redattori da specificare come nome1\\ &nome2\\ ecc.
\newcommand{\docAuthors}{Elena Zecchinato\\&Andrea Meneghinello\\&Riccardo Tresoldi}
% Approvato da
\newcommand{\approvedBy}{Andrea Rizzi}
% Verificatori
\newcommand{\verifiedBy}{Stefano Farronato\\&Marco Schivo}
% Perscorso (relativo o assoluto) che punta alla directory contenente shared/
% come sua sottodirectory (per comodità chiamiamola 'doc root').
\newcommand{\docRoot}{..}
% definire se si vuole l'indice delle tabelle
\def\INDICETABELLE{false}
% definire se si vuole l'indice delle figure
\def\INDICEFIGURE{false}

% importa il preambolo condiviso da tutti i documenti
% shared/preamble.tex
%
% Questo documento contiene la parte del preambolo condivisa e viene pertanto
% richiamato nel 'master' di tutti i documenti di progetto.  Al suo interno
% contiene le inclusioni (e le configurazioni) di tutti i package richiesti per
% la compilazione dei documenti, le macro di carattere generale e la definizione
% degli stili di pagina.

\documentclass[a4paper,10pt]{article}

% **************************************************
% Macro generiche
% **************************************************
\newcommand{\team}{Software Synthesis}                    % chi siamo
\newcommand{\email}{info@softwaresynthesis.org}           % e-mail
\newcommand{\caName}{MyTalk}                              % titolo capitolato
\newcommand{\manager}{SynthesisRequirementManager}        % nome del sistema di tracciamento
\newcommand{\memberdata}[1]{%
  \texttt{\textcolor{RedOrange}{#1}}}                     % attributi di una classe
\newcommand{\method}[1]{\texttt{\textcolor{Emerald}{#1}}} % metodi di una classe
\newcommand{\exception}[1]{%
  \texttt{\textcolor{RedViolet}{#1}}}                     % eccezione
% \newcommand{\handler}[1]{\texttt{\textcolor{Maroon}{#1}}} % per gli event handler
\newcommand{\inglese}[1]{%
  \foreignlanguage{english}{\textit{#1}}}                 % per i testi in lingua inglese
\newcommand{\purpose}{%                                     scopo del prodotto
Con il progetto ``\caName'' si intende un sistema software di comunicazione tra utenti mediante \underline{browser} senza la necessit{\`a} di installazione di \underline{plugin} e/o software esterni. L'utilizzatore avr{\`a} la possibilit{\`a} di interagire con un altro utente tramite una comunicazione audio - audio/video - testuale e, inoltre, ottenere delle statistiche sull'attivit{\`a} in tempo reale.%
}
\newcommand{\glossaryIntro}{%                               introduzione al glossario
Al fine di evitare incomprensioni dovute all'uso di termini tecnici nei documenti, viene redatto e allegato il documento \textit{glossario.4.0.pdf} dove vengono definiti e descritti tutti i termini marcati con una sottolineatura.%
}


% **************************************************
% Codifica e lingua dei documenti
% **************************************************
\usepackage[utf8x]{inputenc}                              % codifica caratteri dei documenti sorgenti
\usepackage[english,italian]{babel}                       % localizzazione ai fini di sillabazione e cross-references
\usepackage[T1]{fontenc}                                  % codifica font di output

% **************************************************
% Definizione geometria della pagina
% **************************************************
\usepackage[a4paper,head=4cm,top=4.5cm,bottom=3cm,left=3cm,right=3cm,bindingoffset=5mm]{geometry}

% *************************************************
% Intestazioni e piè di pagina personalizzati
% *************************************************
\usepackage{fancyhdr}
% stile normale
\fancypagestyle{normal}{
\fancyhead{}                                              % intestazione
\fancyhead[RE,RO]{
\begin{picture}(0,0)
  \put(-410,0){\includegraphics[width=1.02\textwidth]{header_logo}}
  \put(-410,10){\sffamily\large\leftmark}
\end{picture}
\vspace{-4pt}
}
\renewcommand{\headrulewidth}{0pt}                       % riga sotto l'intestazione
\cfoot{}                                                  % piè di pagina
\fancyfoot[RO,LE]{\sffamily
  pag.~\thepage{} di \pageref{LastPage}}                  % a dx nelle pag. dispari e a sx in quelle pari
\fancyfoot[RE,LO]{\sffamily\docFileName{}}
\renewcommand{\footrulewidth}{.4pt}                       % riga sopra il piè di pagina
}
% stile per gli indici
\fancypagestyle{toc}{
\fancyhead{}                                              % intestazione
\fancyhead[RE,RO]{
\begin{picture}(0,0)
  \put(-410,0){\includegraphics[width=1.02\textwidth]{header_logo}}
\end{picture}
}
\renewcommand{\headrule}{}                                % nessuna riga sotto l'intestazione
\cfoot{}                                                  % piè di pagina
\fancyfoot[RO,LE]{\sffamily\thepage{}}                    % a dx nelle pag. dispari e a sx in quelle pari
\fancyfoot[RE,LO]{\sffamily\docFileName{} -- v.\docVers}
\renewcommand{\footrulewidth}{.4pt}                       % riga sopra il piè di pagina
}

\pagestyle{fancy}                                         % premetto: non so usare bene le marche:
\renewcommand{\sectionmark}[1]{\markboth{#1}{#1}}         % se qualcuno ha idee migliori si faccia avanti!

% **************************************************
% Tabelle
% **************************************************
\usepackage{tabularx}                                     % tabelle di larghezza fissa con una o più colonne variabili
\usepackage{multirow}                                     % colonne con colonne che si estendono per più righe
\usepackage{booktabs}                                     % per inserire l'ambiente table e le righe orizz. nelle tabelle
\usepackage{longtable}			                              % tabelle oltre i limiti di pagina

% **************************************************
% Cross-references e collegamenti ipertestuali
% **************************************************
\usepackage[hidelinks]{hyperref}
\hypersetup{%
  colorlinks=false, linktocpage=false, pdfborder={0,0,0}, pdfstartpage=1, pdfstartview=FitV,%
  urlcolor=Cyan, linkcolor=Cyan, citecolor=Black, %pagecolor=Black,%
  pdftitle={\docName}, pdfauthor={\team}, pdfsubject={}, pdfkeywords={},%
  pdfcreator={pdflatex}, pdfproducer={pdflatex with hyperref package}%
}

% **************************************************
% Immagini e grafica
% **************************************************
\usepackage{graphicx}                                     % supporto ad aspetti avanzati delle immagini
\usepackage[table,usenames,dvipsnames]{xcolor}            % tabelle con righe colorate e alternate
\graphicspath{{\docRoot/pics/}}                           % percorso contenente tutti i file immagini
\usepackage{float}                                        % per rendere non flottanti gli ambienti flottanti
\usepackage[italian]{varioref}                            % testo completo riferimenti in italiano

% **************************************************
% Definizioni di colori
% **************************************************
\definecolor{myBlue}{RGB}{1,167,236}
\definecolor{lightblue}{RGB}{213,243,253}%{119,218,247}
\definecolor{llightblue}{RGB}{229,255,255}

% **************************************************
% Altri pacchetti opzionali
% **************************************************     
\usepackage{lastpage}                                     % per sapere il numero totale di pagine
\usepackage{eurosym}                                      % per il simbolo dell'euro usare \EUR{x} dove x è l'importo
\usepackage{ifthen}                                       % permette la scelta di rami condizionali nella compilazione
\usepackage{enumitem}                                     % permette di configurare gli elenchi puntati e numerati


% macro specifiche per il documento corrente
\newcommand{\classsection}[1]{\subsubsection{#1}\label{#1}}
\newcommand{\classname}[1]{\hyperref[#1]{\ttfamily#1}}

% Fine del preambolo e inizio del documento
\begin{document}

% Inclusione della prima pagina
% shared/firstpage.tex
%
% Questo documento definisce il contenuto della prima pagina, che si suppone
% essere uguale in tutti i documenti.  Oltre al logo e al titolo, la prima
% pagina contiene i metadati relativi al documento in cui viene inclusa.


% rimuove intestazioni e piè di pagina
\pagestyle{empty}

\begin{center}

% logo del gruppo
\includegraphics[width=1.5\textwidth]{logo}

\vspace{1in}

% titolo del documento
{\Huge\bfseries \docName}

\vspace{1in}

% tabella riepilogativa
\begin{tabularx}{.7\textwidth}{>{\bfseries\sffamily}l>{\sffamily}l}
\toprule
\multicolumn{2}{>{\sffamily}c}{Informazioni sul documento}\\
\midrule
Nome file:            & \docFileName\\
Versione:             & \docVers\\
Data creazione:       & \creationDate\\
Data ultima modifica: & \modificationDate\\
Stato:                & \docState\\
Uso:                  & \docUsage\\
Redattori:            & \docAuthors\\
Approvato da:         & \approvedBy\\
Verificatori:         & \verifiedBy\\
\bottomrule
\end{tabularx}

\end{center}

\newpage


%---------------------------RUOLI----------------------------
%FASE 1:
%Programmatori: DIEGO, STEFANO, SCHIVO
%FASE 2:
%Programmatori: MENE, TRES, ELENA

%Verificatore: SCHIVO, STEFANO, RIZZI, DIEGO (dobbiamo fare un sacco di test)
%Responsabile finale supremo: RIZZI
%------------------------------------------------------------

% Storico delle modifiche
\section*{Storia delle modifiche}
\begin{center}
\begin{longtable}{lp{.32\textwidth}lll}
\toprule
Versione & Descrizione intervento & Membro & Ruolo & Data\\
\midrule % inserire qui il contenuto della tabella
0.1 & Creazione del documento e stesura delle sezioni ``Introduzione'' e ``Riferimenti''. & &  & 2013-02-03\\
\bottomrule
\end{longtable}
\end{center}
\newpage

% inclusione dell'indice
% shared/toc.tex
%
% Questo file contiene le istruzioni che generano l'indice o gli indici del
% documento (utile nel caso in cui decidessimo di avere anche un indice delle
% tabelle e/o un indice delle figure).

% imposta lo stile di pagina per i titoli definito nel preambolo
\pagestyle{toc}
% imposta i numeri di pagina romani minuscoli
\pagenumbering{roman}

% genera automaticamente l'indice di LaTeX
\tableofcontents

% se è true \INDICETABELLE allora genera l'indice delle tabelle, altrimenti non fa nulla
\ifthenelse{\equal{\INDICETABELLE}{true}}{%
  \clearpage % l'indice delle tabelle, se c'è, deve andare a pagina nuova
  \listoftables
}{}

% se è true |INDICEFIGURE allora genera l'indice delle figure, altrimenti non fa nulla
\ifthenelse{\equal{\INDICEFIGURE}{true}}{%
  \clearpage % l'indice delle figure, se c'è, deve andare a pagina nuova
  \listoffigures
}{}

%in ogni caso occorre andare a pagina nuova dopo gli indici
\clearpage


% Alcuni aggiustamenti per le pagine
\pagenumbering{arabic}
\setcounter{page}{1}
\pagestyle{normal}

% Qui ha inizio il documento vero e proprio...
\newpage

\section{Introduzione}
\subsection{Scopo del prodotto}
\purpose

\subsection{Scopo del documento}
Il presente documento presenta una descrizione dettagliata dell'architettura del sistema software destinata a costituire il prodotto \caName{}, coerentemente con la progettazione ad alto livello contenuta nell'allegato \textit{specifica\_tecnica.2.0.pdf}.

A tal fine, si riporta per ognuno dei componenti definiti nel documento di specifica tecnica una descrizione delle classi in termini di operazioni disponibili, proprietà, responsabilità e collaborazioni. Il contenuto del presente documento ha inoltre valore vincolante per i programmatori, che avranno l'obbligo di attenersi alle disposizioni in esso contenute senza alcuna possibilità di deroga.

\subsection{Glossario}
\glossaryIntro

\subsection{Convenzioni di scrittura}
% vedere issue #56 al riguardo
Al fine di rendere quanto più agevole possibile la consultazione di questo documento da parte dei programmatori e del committente, è stata adottata una serie di accorgimenti sia a livello di riferimenti sulla nomenclatura delle classi sia a livello cromatico per campi dati e metodi. 
Tali norme possono essere consultate in dettaglio nel documento \textit{norme\_di\_progetto.3.0.pdf} allegato.
\clearpage

\section{Riferimenti}
\subsection{Normativi}
\begin{itemize}
\item[] \textit{piano\_di\_qualifica.3.0.pdf} allegato.
\item[] \textit{norme\_di\_progetto.3.0.pdf} allegato.
\item[] \textit{specifica\_tecnica.2.0.pdf} allegato
\end{itemize}

\subsection{Informativi}
\begin{itemize}
\item[] Capitolato d'appalto: \caName{}, v1.0, redatto e rilasciato dal proponente Zucchetti s.r.l. reperibile all'indirizzo \url{http://www.math.unipd.it/~tullio/IS-1/2012/Progetto/C1.pdf};
\item[] testo di consultazione: \textit{Software Engineering (8th edition) Ian Sommerville, Pearson Education | Addison Wesley};
\item[] manuale all'utilizzo dei design pattens: \textit{Design Patterns, Elementi per il riuso di software a oggetti -- (1/Ed. italiana) Eric Gamma, Richard Helm, Ralph Johnson, John Vlissides, Pearson Education};
\item[] \textit{glossario.3.0.pdf} allegato.
\end{itemize}
\clearpage

\section{Standard di progetto}

\subsection{Standard di progettazione architetturale}
Lo sviluppo del progetto ha seguito le regole architetturali specificate nel documento  \textit{norme\_di\_progetto.3.0.pdf} allegato.

\subsection{Standard di documentazione del codice}
Le regole che definiscono la documentazione del codice relativo al funzionamento del prodotto sono specificate nel documento \textit{norme\_di\_progetto.3.0.pdf} allegato.

\subsection{Standard di denominazione di entità e relazioni}
Le convenzioni relative alla denominazione delle entità e le relative relazioni sono specificate nel documento \textit{norme\_di\_progetto.3.0.pdf} allegato.

\subsection{Standard di programmazione}
Le regole relative agli standard di programmazione sono enunciate nel documento \textit{norme\_di\_progetto.3.0.pdf} allegato.

\subsection{Strumenti di lavoro}
Gli strumenti utilizzati per la stesura e lo sviluppo sono specificati nei documenti \textit{norme\_di\_progetto.3.0.pdf} e \textit{piano\_di\_qualifica.3.0.pdf} allegati.

\clearpage

\section{Specifica sotto-architettura sever}\label{sec:serverarchitecture}

\subsection{Package org.softwaresynthesis.mytalk.server.dao}\label{sec:dao}

\subsection{Package org.softwaresynthesis.mytalk.server.abook}\label{sec:abook}

\classsection{IUserData}

\subsubsection*{Funzione}
Interfaccia rappresentante il comportamento di un generico utente del sistema

\subsubsection*{Relazioni d'uso}

L'interfaccia estende \texttt{java.util.Observer}.

\subsubsection*{Metodi}
\begin{description}
	\item{\method{+ getId(): Long}}\\
	Restituisce l'identificatore univoco di uno \classname{IUserData}.
	\item{\method{+ getEmail(): String}}\\
	Restituisce l'indirizzo e-mail con cui uno \classname{IUserData} si è registrato nel sistema mytalk.
	\item{\method{+ setEmail(mail: String): void}}\\ 
	Imposta l'indirizzo e-mail con cui si registra nel sistema mytalk uno \classname{IUserData}.
	\item{\method{+ getPassword(): String}}\\
	Restituisce la password di accesso al sistema mytalk di uno \classname{IUserData}.
	\item{\method{+ setPassword(password: String): void}}\\
	Imposta la password di accesso al sistema di uno \classname{IUserData}.
	\item{\method{+ getQuestion(): String}}\\
	Restituisce la domanda segreta, scelta da uno \classname{IUserData}, per il recupero della password smarrita di accesso al sistema mytalk.
	\item{\method{+ setQuestion(question: String): void}}\\
	Imposta la domanda segreta, scelta da uno \classname{IUserData}, per il recupero della password smarrita di accesso al sistema mytalk.
	\item{\method{+ getAnswer(): String}}\\
	Restituisce la risposta alla domanda per il recupero della password smarrita di accesso al sistema mytalk.
	\item{\method{+ setAnswer(answer: String): void}}\\
	Imposta la risposta alla domanda segreta per il recupero della password di accesso al sistema mytalk.
	\item{\method{+ getName(): String}}\\
	Restituisce il nome di uno \classname{IUserData}.
	\item{\method{+ setName(name: String): void}}\\
	Imposta il nome di uno \classname{IUserData}.
	\item{\method{+ getSurname(): String}}\\
	Resituisce il cognome di uno \classname{IUserData}.
	\item{\method{+ setSurname(surname: String): void}}\\
	Imposta il cognome di uno \classname{IUserData}.
	\item{\method{+ getPicturePath(): String}}\\
	Restituisce una stringa con il percorso dell'immagine del profilo di uno \classname{IUserData}.
	\item{\method{+ setPicturePath(path: String): void}}\\
	Imposta il percorso dell'immagine profilo di uno \classname{IUserData}.
\end{description}

\classsection{IGroup}

\subsubsection*{Funzione}
Interfaccia rappresentante un gruppo di una rubrica utente del sistema mytalk

\subsubsection*{Metodi}
\begin{description}
	\item{\method{+ getId(): Long}}\\
	Restituisce l'identificativo univoco di uno gruppo di una rubrica utente.
	\item{\method{+ getName(): String}}\\
	Restituisce il nome di un gruppo di una rubrica utente.
	\item{\method{+ setName(name: String): void}}\\
	Imposta il nome di un gruppo di una rubrica utente.
\end{description}

\classsection{IAddressBookEntry}

\subsubsection*{Funzione}
Interfaccia rappresentante una entry di una rubrica utente del sistema mytalk

\subsubsection*{Relazioni d'uso}
\begin{itemize}
	\item \classname{IUserData}: l'interfaccia \classname{IAddressBookEntry} definisce più metodi che restituiscono oggetti aventi tipo di ritorno \classname{IUserData}. Tali sono i metodi \inglese{get} per ottenere il ``possessore'' della rubrica e per ottenere l'utente registrato nella rubrica. Analogamente \classname{IUserData} viene usato come parametro d'ingresso per i metodi \inglese{set} collegati ai metodi già citati.
\end{itemize}

\subsubsection*{Metodi}
\begin{description}
	\item{\method{+ getId(): Long}}\\
	Resituisce l'identificativo univoco di una \inglese{entry} di una rubrica utente del sistema mytalk.
	\item{\method{+ getEntry(): IUserData}}\\
	Restituisce un istanza di un oggetto avente tipo \classname{IUserData}, e rappresentante un contatto della rubrica.
	\item{\method{+ setEntry(contact: IUserData): void}}\\
	Imposta l'utente \classname{IUserData} (passato come parametro d'ingresso) come contatto della rubrica.
	\item{\method{+ getGroup(): IGroup}}\\
	Restituisce il gruppo a cui appartiene lo \classname{IUserData} registrato nella rubrica.
	\item{\method{+ setGroup(group: IGroup): void}}\\
	Imposta il gruppo di appartenenza dello \classname{IUserData} registrato nella rubrica.
	\item{\method{+ getOwner(): IUserData}}\\
	Restituisce lo \classname{IUserData} possesore di questa \inglese{entry} della rubrica
	\item{\method{+ setOwner(owner: IUserData ): void}}\\
	Imposta l'utente \classname{IUserData} possessore della entry della rubrica.
\end{description}

\subsection{Package org.softwaresynthesis.mytalk.server.call}\label{sec:call}

\classsection{ICall}

\subsubsection*{Funzione}
Interfaccia che rappresenta una chiamata effettuata dal sistema mytalk. Si ricorda che gli oggetti che implementano tale interfaccia vengono usati per rappresentare lo storico delle chiamate di un utente.

\subsubsection*{Relazioni d'uso}
\begin{itemize}
	\item \texttt{java.util.Date}: tipo utilizzato per definire la data d'inizio e fine di una chiamata.
\end{itemize}

\subsubsection*{Metodi}
\begin{description}
	\item{\method{+ getId(): Long}}\\
	Restituisce l'identificativo univoco della chiamata.
	\item{\method{+ getStartDate(): Date}}\\
	Restituisce un istanza di \texttt{java.util.Date} di avvio della chiamata.
	\item{\method{+ setStartDate(startDate: Date): void}}\\
	Imposta la data di avvio della chiamata.
	\item{\method{+ getEndDate(): Date}}\\
	Restituisce la data di terminazione della chiamata
	\item{\method{+ setEndDate(endDate: Date): void}}\\
	Imposta la data di terminazione della chiamata.
\end{description}

\subsection{Package org.softwaresynthesis.mytalk.server.message}\label{sec:message}

\classsection{IMessage}

\subsubsection*{Funzione}
Interfaccia che rappresenta un messaggio di segreteria del sistema mytalk.

\subsubsection*{Relazioni d'uso}
\begin{itemize}
	\item \texttt{java.util.Date}: tipo utilizzato per definire la data in cui è stato registrato un messaggio.
	\item \classname{abook.IUserData}: usata per rappresentare il mittente e il destinatario del messaggio di segreteria.
\end{itemize}

\subsubsection*{Metodi}
\begin{description}
	\item{\method{+ getId(): Long}}\\
	Restituisce l'identificativo univoco del messaggio.
	\item{\method{+ getSender(): IUserData}}\\
	Restituisce un istanza di tipo classname{abook.IUserData} che rappresenta il mittente del messaggio.
	\item{\method{+ setSender(sender: IUserData): void}}\\
	Imposta il mittente del messaggio.
	\item{\method{+ getReceiver(): IUserData}}\\
	Restituisce un istanza di tipo classname{abook.IUserData} che rappresenta il destinatario del messaggio.
	\item{\method{+ setReceiver(receiver: IUserData): void}}\\
	Imposta il destinatario del messaggio.
	\item{\method{+ isNew(): boolean}}\\
	Imposta se il messaggio deve risultare come già ascoltato o meno.
	\item{\method{+ setNew(status: boolean): void}}\\
	Imposta il messaggio come ``già letto'' o come ``da leggere''.
	\item{\method{+ isVideo(): boolean}}\\
	Restituisce un booleano che determina se si tratta di un messaggio audio oppure audio-video.
	\item{\method{+ setVideo(video: boolean): void }}\\	
	Imposta il messaggio come messaggio audio-video.
	\item{\method{+ getDate(): Date}}\\
	Restituisce la data in cui il mittente ha lasciato il messaggio nella segreteria del destinatario.
	\item{\method{+ setDate(date: Date): void}}\\
	Imposta la data in cui il mittente ha lasciato il messaggio nella segreteria del destinatario.

\end{description}

\subsection{Package org.softwaresynthesis.mytalk.server.authentication}\label{sec:authentication}

\classsection{ISecurityStrategy}

\subsubsection*{Funzione}
Interfaccia che identifica il comportamento di una strategia generica di crittografia dei dati.

\subsubsection*{Relazioni d'uso}
Nessuna relazione d'uso evidenziata.

\subsubsection*{Metodi}
\begin{description}
	\item{\method{+ encrypt(plainText: String): String}}\\
	Cripta la stringa di testo ricevuta come parametro.\\\\
	Il metodo può lanciare eccezioni:
	\begin{itemize}
		\item \exception{Exception}: il metodo può lanciare un'eccezione generica.
	\end{itemize}
	\item{\method{+ decrypt(encryptedText: String)}}\\
	Decripta la stringa di testo ricevuta come parametro.\\\\
	Il metodo può lanciare eccezioni:
	\begin{itemize}
		\item \exception{Exception}: il metodo può lanciare un'eccezione generica.
	\end{itemize}
\end{description}

\classsection{AESAlgorithm}

\subsubsection*{Funzione}
Implementazione della strategia di codifica/decodifica con l'uso dell'algoritmo AES a 128bit.

\subsubsection*{Relazioni d'uso}
\begin{itemize}
	\item \classname{ISecurityStrategy}: interfaccia d'imlementazione
	\item \texttt{java.security.Key}: usata per creare un istanza di una chiave durante il processo di criptazione.
	\item \texttt{javax.crypto.Cipher}: usata per creare un istanza di un oggetto di criptazione che implementa l'algoritmo AES a 128 bit.
	\item \texttt{javax.crypto.spec.SecretKeySpec}: usata dall'algoritmo di per costruire una chiave segreta a partire da un array di byte.
	\item \texttt{sun.misc.BASE64Encoder}: utilizzata per eseguire una trasformazione da stringa a byte.
	\item \texttt{sun.misc.BASE64Decoder}: utilizzata per eseguire una trasformazione da byte in stringa.
\end{itemize}

\subsubsection*{Attributi}
\begin{description}
  \item{\memberdata{-- \{frozen\} \underline{keyValue}: byte[]}}\\
  Array di byte usato per definire il valore della chiave su cui si basa l'algoritmo AES di criptazione. La chiave effettiva sarà creata a partire da tale attributo, per mezzo della classe \texttt{javax.crypto.spec.SecretKeySpec}.
  \item{\memberdata{-- \{frozen\} \underline{algorithm}: String}}\\
  Stringa costante che identifica il nominativo dell'algoritmo usato, e che dovrà essere specificato durante l'uso di \texttt{javax.crypto.Cipher}. l'attributo avrà valore ``AES''.
\end{description}

\subsubsection*{Metodi}
\begin{description}
	\item{\method{- generateKey(): Key}}\\
	Metodo che restituisce una chiave di tipo \texttt{java.security.Key}, a partire dall'array \memberdata{keyValue}. Per fare ciò, il metodo usa il costruttore di \texttt{javax.crypto.spec.SecretKeySpec}.
	Il metodo può lanciare eccezioni:
	\begin{itemize}
		\item \exception{Exception}: il metodo può lanciare un'eccezione generica.
	\end{itemize}
	\item{\method{+ encrypt(plainText: String): String}}\\
	Metodo usato per criptare un testo di tipo String passato come parametro d'ingresso. Il metodo usa una chiave ottenuta a partire dal metodo \method{generateKey()} in associazione alla classe \texttt{javax.crypto.Cipher} per criptare il testo tramite l'algoritmo AES.
	\begin{itemize}
		\item \exception{Exception}: il metodo può lanciare un'eccezione generica.
	\end{itemize}
	\item{\method{+ decrypt(encryptedText: String)}}\\
	Decripta la stringa di testo ricevuta come parametro, attuando una procedura inversa a quella presentata nel metodo \method{encrypt(plainText: String)}\\\\
	Il metodo può lanciare eccezioni:
	\begin{itemize}
		\item \exception{Exception}: il metodo può lanciare un'eccezione generica.
	\end{itemize}
\end{description}

\classsection{PrincipalImpl}

\subsubsection*{Funzione}
Oggetto che permette di identificate univocamente uno IUserData memorizzato nel database del sistema mytalk.

\subsubsection*{Relazioni d'uso}
\begin{itemize}
	\item \texttt{java.io.Serializable}: interfaccia d'implementazione usata per rendere serializzabili le istanze della classe.
	\item \texttt{java.security.Principal}: interfaccia d'implementazione usata per rendere caratterizzante le istanze della classe.
\end{itemize}

\subsubsection*{Attributi}
\begin{description}
  \item{\memberdata{-- \{frozen\} \underline{serialVersionUID}: long}}\\
  Identificativo univoco per la classe, usato al fine di rendere l'oggetto serializzabile.
  \item{\memberdata{-- mail: String}}\\
  Attributo che rappresenta l'indirizzo e-mail dell'utente.
\end{description}

\subsubsection*{Metodi}
\begin{description}
	\item{\method{+ PrincipalImpl(mail: String)}}\\
	Classe costruttore. crea un oggetto PrincipalImpl che permetterà di determinare univocamente lo IUserData che ha effettuato il login.
	\item{\method{+ getName(): String}}\\
	Restituisce l'elemento identificativo (mail dell'utente) dello IUserData che ha effettuato la procedura di login.
	\item{\method{+ equals(obj: Object): boolean}}\\
	Verifica l'uguaglianza di due oggetti PrincipalImpl sulla base di un confronto tra gli indirizzi mail degli utenti confrontati.
	\item{\method{+ hashCode(): int}}\\
	Restituisc eil codice hash dell'oggetto di invocazione.				\item{\method{+ toString(): String}}\\
	Restituisce l'istanza dell'oggetto sotto forma di stringa. In particolare evidenziando l'indirizzo e-mail dell'utente.
\end{description}

\classsection{AuthenticationData}

\subsubsection*{Funzione}
Oggetto contenente i dati di accesso utilizzati da un utente che vuole accedere al sistema mytalk.

\subsubsection*{Relazioni d'uso}

Nessuna relazione evidenziata.

\subsubsection*{Attributi}
\begin{description}
  \item{\memberdata{-- username: String}}\\
	Attributo contenente l'username dell'utente che richiede l'autenticazione.
  \item{\memberdata{-- password: String}}\\
	Attributo contenente la password dell'utente che richiede l'autenticazione.
\end{description}

\subsubsection*{Metodi}
\begin{description}
	\item{\method{+ AuthenticationData(username: String, password: String)}}\\
	Costruttore pubblico che per creare un istanza di AuthenticationData, si basa sulla password e lo username dell'utente da autenticare.

	\item{\method{+ getUsername(): String}}\\
	Metodo che restituisce lo username dell'oggetto di autenticazione.
	
	\item{\method{+ getPassword(): String}}\\
	Metodo che Restituisce la password dell'utente che sta effettuando la prodedura di login.

	\item{\method{+ hashCode(): int}}\\
	Restituisce il codice hash di questa istanza.
	
	\item{\method{+ equals(Object obj): boolean}}\\
	Determina l'uguaglianza di due istanze, effettuando un confronto tra l'oggetto stesso da cui è chiamato il metodo, e un Object obj ricevuto come parametro d'ingresso.
	
	\item{\method{+ toString: String}}\\
	Metodo che Restituisce l'istanza sotto forma di stringa. Si osservi che il metodo non dovrà ritornare la password, ma solo lo username dell'utente da autenticare.

\end{description}

\classsection{CredentialLoader}

\subsubsection*{Funzione}
Permette di caricare le credenziali di autenticazione, fornite dall'utente, per preparare la fase di login.

\subsubsection*{Relazioni d'uso}
\begin{itemize}
	\item \texttt{java.io.IOException}:
	\item \texttt{javax.security.auth.callback.Callback}:
	\item \texttt{javax.security.auth.callback.CallbackHandler}:
	\item \texttt{javax.security.auth.callback.NameCallback}:
	\item \texttt{javax.security.auth.callback.PasswordCallback}:
	\item \texttt{javax.security.auth.callback.UnsupportedCallbackException}:
\end{itemize}

\subsubsection*{Attributi}
\begin{description}
  \item{\memberdata{-- credential: AuthenticationData}}\\
  Attributo contenente i dati da autenticare per la login dell'utente.
  \item{\memberdata{-- security: ISecurityStrategy}}\\
  Attributo che contiene un oggetto che definisce un algoritmo di criptazione per i dati. Necessario per criptare i dati di autenticazione.
\end{description}

\subsubsection*{Metodi}
\begin{description}
	\item{\method{+ CredentialLoader(credential: AuthenticationData, security: ISecurityStrategy)}}\\
	Costruttore pubblico. Crea un istanza con le credenziali fornite dall'utente (fornite in fase di login).

	\item{\method{+ handle(callbacks: Callback[]): void}}\\
	Effettua il caricamento e crittografa le credenziali fornite dall'utente per la fase di login\\\\
	Il metodo può lanciare eccezioni:
	\begin{itemize}
		\item \exception{IOException}
		\item \exception{UnsupportedCallbackException}
	\end{itemize}

\end{description}


\clearpage

\section{Specifica sotto-architettura clientpresenter}\label{sec:clientpresenterarchitecture}

\clearpage

\section{Specifica sotto-architettura clientview}\label{sec:clientviewarchitecture}

\clearpage

\section{Tracciamento della relazione componenti-requisiti}

\end{document}
