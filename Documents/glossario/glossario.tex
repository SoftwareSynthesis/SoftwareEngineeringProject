% **************************************************
% Macro specifiche per il documento corrente
% **************************************************
% Nome
\newcommand{\docName}{Glossario}
% Nome file
\newcommand{\docFileName}{glossario.tex}
% Versione
\newcommand{\docVers}{1.0}
% Data creazione
\newcommand{\creationDate}{2012-12-17}
% Data ultima modifica
\newcommand{\modificationDate}{2012-12-18}
% Stato in {Approvato, Non approvato}
\newcommand{\docState}{Approvato}
% Uso in {Interno, Esterno}
\newcommand{\docUsage}{Esterno}
% Redattori da specificare come nome1\\ &nome2\\ ecc.
\newcommand{\docAuthors}{Diego Beraldin\\ &Riccardo Tresoldi\\}
% Approvato da
\newcommand{\approvedBy}{Elena Zecchinato\\}
% Verificatori
\newcommand{\verifiedBy}{Marco Schivo\\}
% Perscorso (relativo o assoluto) che punta alla directory contenente shared/
% come sua sottodirectory (per comodità chiamiamola 'doc root').
\newcommand{\docRoot}{..}

% importa il preambolo condiviso da tutti i documenti
% shared/preamble.tex
%
% Questo documento contiene la parte del preambolo condivisa e viene pertanto
% richiamato nel 'master' di tutti i documenti di progetto.  Al suo interno
% contiene le inclusioni (e le configurazioni) di tutti i package richiesti per
% la compilazione dei documenti, le macro di carattere generale e la definizione
% degli stili di pagina.

\documentclass[a4paper,10pt]{article}

% **************************************************
% Macro generiche
% **************************************************
\newcommand{\team}{Software Synthesis}                    % chi siamo
\newcommand{\email}{info@softwaresynthesis.org}           % e-mail
\newcommand{\caName}{MyTalk}                              % titolo capitolato
\newcommand{\manager}{SynthesisRequirementManager}        % nome del sistema di tracciamento
\newcommand{\memberdata}[1]{%
  \texttt{\textcolor{RedOrange}{#1}}}                     % attributi di una classe
\newcommand{\method}[1]{\texttt{\textcolor{Emerald}{#1}}} % metodi di una classe
\newcommand{\exception}[1]{%
  \texttt{\textcolor{RedViolet}{#1}}}                     % eccezione
% \newcommand{\handler}[1]{\texttt{\textcolor{Maroon}{#1}}} % per gli event handler
\newcommand{\inglese}[1]{%
  \foreignlanguage{english}{\textit{#1}}}                 % per i testi in lingua inglese
\newcommand{\purpose}{%                                     scopo del prodotto
Con il progetto ``\caName'' si intende un sistema software di comunicazione tra utenti mediante \underline{browser} senza la necessit{\`a} di installazione di \underline{plugin} e/o software esterni. L'utilizzatore avr{\`a} la possibilit{\`a} di interagire con un altro utente tramite una comunicazione audio - audio/video - testuale e, inoltre, ottenere delle statistiche sull'attivit{\`a} in tempo reale.%
}
\newcommand{\glossaryIntro}{%                               introduzione al glossario
Al fine di evitare incomprensioni dovute all'uso di termini tecnici nei documenti, viene redatto e allegato il documento \textit{glossario.4.0.pdf} dove vengono definiti e descritti tutti i termini marcati con una sottolineatura.%
}


% **************************************************
% Codifica e lingua dei documenti
% **************************************************
\usepackage[utf8x]{inputenc}                              % codifica caratteri dei documenti sorgenti
\usepackage[english,italian]{babel}                       % localizzazione ai fini di sillabazione e cross-references
\usepackage[T1]{fontenc}                                  % codifica font di output

% **************************************************
% Definizione geometria della pagina
% **************************************************
\usepackage[a4paper,head=4cm,top=4.5cm,bottom=3cm,left=3cm,right=3cm,bindingoffset=5mm]{geometry}

% *************************************************
% Intestazioni e piè di pagina personalizzati
% *************************************************
\usepackage{fancyhdr}
% stile normale
\fancypagestyle{normal}{
\fancyhead{}                                              % intestazione
\fancyhead[RE,RO]{
\begin{picture}(0,0)
  \put(-410,0){\includegraphics[width=1.02\textwidth]{header_logo}}
  \put(-410,10){\sffamily\large\leftmark}
\end{picture}
\vspace{-4pt}
}
\renewcommand{\headrulewidth}{0pt}                       % riga sotto l'intestazione
\cfoot{}                                                  % piè di pagina
\fancyfoot[RO,LE]{\sffamily
  pag.~\thepage{} di \pageref{LastPage}}                  % a dx nelle pag. dispari e a sx in quelle pari
\fancyfoot[RE,LO]{\sffamily\docFileName{}}
\renewcommand{\footrulewidth}{.4pt}                       % riga sopra il piè di pagina
}
% stile per gli indici
\fancypagestyle{toc}{
\fancyhead{}                                              % intestazione
\fancyhead[RE,RO]{
\begin{picture}(0,0)
  \put(-410,0){\includegraphics[width=1.02\textwidth]{header_logo}}
\end{picture}
}
\renewcommand{\headrule}{}                                % nessuna riga sotto l'intestazione
\cfoot{}                                                  % piè di pagina
\fancyfoot[RO,LE]{\sffamily\thepage{}}                    % a dx nelle pag. dispari e a sx in quelle pari
\fancyfoot[RE,LO]{\sffamily\docFileName{} -- v.\docVers}
\renewcommand{\footrulewidth}{.4pt}                       % riga sopra il piè di pagina
}

\pagestyle{fancy}                                         % premetto: non so usare bene le marche:
\renewcommand{\sectionmark}[1]{\markboth{#1}{#1}}         % se qualcuno ha idee migliori si faccia avanti!

% **************************************************
% Tabelle
% **************************************************
\usepackage{tabularx}                                     % tabelle di larghezza fissa con una o più colonne variabili
\usepackage{multirow}                                     % colonne con colonne che si estendono per più righe
\usepackage{booktabs}                                     % per inserire l'ambiente table e le righe orizz. nelle tabelle
\usepackage{longtable}			                              % tabelle oltre i limiti di pagina

% **************************************************
% Cross-references e collegamenti ipertestuali
% **************************************************
\usepackage[hidelinks]{hyperref}
\hypersetup{%
  colorlinks=false, linktocpage=false, pdfborder={0,0,0}, pdfstartpage=1, pdfstartview=FitV,%
  urlcolor=Cyan, linkcolor=Cyan, citecolor=Black, %pagecolor=Black,%
  pdftitle={\docName}, pdfauthor={\team}, pdfsubject={}, pdfkeywords={},%
  pdfcreator={pdflatex}, pdfproducer={pdflatex with hyperref package}%
}

% **************************************************
% Immagini e grafica
% **************************************************
\usepackage{graphicx}                                     % supporto ad aspetti avanzati delle immagini
\usepackage[table,usenames,dvipsnames]{xcolor}            % tabelle con righe colorate e alternate
\graphicspath{{\docRoot/pics/}}                           % percorso contenente tutti i file immagini
\usepackage{float}                                        % per rendere non flottanti gli ambienti flottanti
\usepackage[italian]{varioref}                            % testo completo riferimenti in italiano

% **************************************************
% Definizioni di colori
% **************************************************
\definecolor{myBlue}{RGB}{1,167,236}
\definecolor{lightblue}{RGB}{213,243,253}%{119,218,247}
\definecolor{llightblue}{RGB}{229,255,255}

% **************************************************
% Altri pacchetti opzionali
% **************************************************     
\usepackage{lastpage}                                     % per sapere il numero totale di pagine
\usepackage{eurosym}                                      % per il simbolo dell'euro usare \EUR{x} dove x è l'importo
\usepackage{ifthen}                                       % permette la scelta di rami condizionali nella compilazione
\usepackage{enumitem}                                     % permette di configurare gli elenchi puntati e numerati



% **************************************************
% configurazioni specifiche per il glossario
% **************************************************
\usepackage{fancybox}        % per la creazione delle scatoline
\setlength{\shadowsize}{2pt} % dimensione dell'ombra
% comando per generare i separatori di lettera
\newcommand{\letterbox}[1]{\noindent\shadowbox{\begin{minipage}[t]{.98\linewidth}\centering\large\uppercase{$\mathcal{#1}$}\end{minipage}}\newline}
% comando di formattazione di tutte le voci del glossario
\newcommand{\glossaryItem}[2]{\noindent\textbf{#1}\newline{}#2\newline}

% Fine del preambolo e inizio del documento
\begin{document}

% Inclusione della prima pagina
% shared/firstpage.tex
%
% Questo documento definisce il contenuto della prima pagina, che si suppone
% essere uguale in tutti i documenti.  Oltre al logo e al titolo, la prima
% pagina contiene i metadati relativi al documento in cui viene inclusa.


% rimuove intestazioni e piè di pagina
\pagestyle{empty}

\begin{center}

% logo del gruppo
\includegraphics[width=1.5\textwidth]{logo}

\vspace{1in}

% titolo del documento
{\Huge\bfseries \docName}

\vspace{1in}

% tabella riepilogativa
\begin{tabularx}{.7\textwidth}{>{\bfseries\sffamily}l>{\sffamily}l}
\toprule
\multicolumn{2}{>{\sffamily}c}{Informazioni sul documento}\\
\midrule
Nome file:            & \docFileName\\
Versione:             & \docVers\\
Data creazione:       & \creationDate\\
Data ultima modifica: & \modificationDate\\
Stato:                & \docState\\
Uso:                  & \docUsage\\
Redattori:            & \docAuthors\\
Approvato da:         & \approvedBy\\
Verificatori:         & \verifiedBy\\
\bottomrule
\end{tabularx}

\end{center}

\newpage


\pagestyle{normal}

% Storico delle modifiche
\section*{Storia delle modifiche}
\begin{tabularx}{\textwidth}{lXll}
\toprule
Versione & Descrizione intervento & Redattore & Data\\
\midrule % inserire qui il contenuto della tabella
1.0 & Approvazione & Elena Zecchinato & 2012-12-17\\
0.3 & Validazione & Marco Schivo & 2012-12-17\\
0.2 & Stesura definitiva & Riccardo Tresoldi & 2012-12-17\\
0.1 & Stesura preliminare & Diego Beraldin & 2012-12-17\\
\bottomrule
\end{tabularx}
\newpage

% Alcuni aggiustamenti per le pagine
\pagenumbering{arabic}
\setcounter{page}{1}
\pagestyle{normal}

% Qui ha inizio il documento vero e proprio...
\letterbox{a}

\glossaryItem{AJAX}{%
acronimo di \inglese{Asynchronous JavaScript and XML}, rappresenta una serie di tecniche lato client per sviluppo di \inglese{web application} asincrone. L'utilizzo di questa tecnologia permette di interrogare strutture dati esterne alla pagina web senza dover ricaricare la pagina stessa.
}

%\glossaryItem{anomalia}{%
%vd. \underline{difetto}
%}

\glossaryItem{API}{%
acronimo di \inglese{Application Programming Interface}, designa l'interfaccia esposta agli utenti programmatori da parte di una libreria software al fine di rendere pi{\`u } agevole lo sviluppo di applicazioni astraendo una serie di dettagli di basso livello e centralizzando l'accesso a determinate procedure.
}

\letterbox{b}

\glossaryItem{best practice}{%
prassi (dal greco $\pi\rho\acute{\alpha}\xi\iota\varsigma$, `azione') consolidata sulla base dell'esperienza comune in un determinato dominio e la cui validità è universalmente o in larga misura riconosciuta dagli esperti di tale dominio. In genere le best practices sono formalizzate in una serie di regole la cui applicazione offre buone garanzie di riuscita del prodotto finale.
}

\glossaryItem{\inglese{Brainstorm}}{%
parola inglese che tradotta in italiano significa letteralmente ``tempesta di cervelli'', sta ad indicare una riunione, solitamente con un discreto numero di persone. Nella sudetta riunione ogni membro del gruppo propone liberamente idee anche apparentemente poco inerenti all'argomento.
}

\glossaryItem{breakpoint}{%
punto di interruzione marcato nel codice sorgente, generalmente impiegato per finalità di debugging una volta raggiunto il quale l'esecuzione è momentaneamente sospesa ed è possibile, a seconda del debugger utilizzato, acquisire informazioni inerenti lo stato del calcolo (ad es.~il valore delle variabili in un determinato scope).
}

\glossaryItem{browser}{%
programma utilizzato per attingere, elaborare, visualizzare e trasmettere informazioni su internet affidandosi a protocolli di livello applicazione (HTTP, FTP). I browser oltre a integrare un motore di rendering necessario a interpretare e tradurre il codice (X)HTML in forma di ipertesto offrono la possibilità di realizzare computazione lato client grazie, ad esempio, ad un interprete JavaScript.
}

\letterbox{c}

\glossaryItem{Chrome}{%
browser moderno lanciato da Google Inc. nel 2008 e basato sul motore di rendering WebKit.
}

\glossaryItem{ciclo di Deming}{%
strategia di gestione dei processi che persegue il miglioramento continuo della qualità ideato da W.~E.~Deming negli anni '50. Tale modello di gestione prevede la suddivisione delle attività in quattro fasi da attuarsi ciclicamente e corrispondenti alla pianificazione preliminare (P), esecuzione (D) conforme ai piani stabiliti in precedenza, controllo del risultato (C) ed eventuali azioni correttive (A). È altrimenti detto ciclo PDCA\@.
}

\glossaryItem{client}{%
in un'architettura \inglese{client-server}, corrisponde alla componente che richiede e ottiene accesso a risorse e/o servizi attraverso un canale di comunicazione (quale una rete informatica).
}

\glossaryItem{CSS}{%
acronimo di \inglese{Cascading Style Sheets} (lett. `fogli di stile a cascata'), la tecnogia web che associata al linguaggio di HTML permette la definizione
delle pagine web. Le ragioni alla base dell’introduzione dei CSS sono da ravvisarsi nell'esigenza di mantenere quanto più netta la separazione fra i contenuti e le specifiche di formattazione, lasciando al markup (X)HTML la descrizione puramente semantico-strutturale dei contenuti e delegando le istruzioni di formattazione grafica al foglio di stile.
}

\letterbox{d}

\glossaryItem{database}{%
raccolta di informazioni (dati e metadati) permanenti gestiti da un elaboratore elettronico. In particolare, i metadati descrivono lo schema secondo cui sono organizzati i dati mentre questi ultimi corrispondono alla rappresentazione di una porzione di realtà, sono persistenti, hanno una cardinalità molto maggiore rispetto ai metadati e sono accessibili mediante transazioni.
}

\glossaryItem{deadline}{%
termine ultimo entro e non oltre il quale deve essere raggiunto un certo obiettivo o consegnato un determinato \inglese{deliverable} di progetto, vale a dire il risultato materiale o immateriale di un'attività di progetto.
}

\glossaryItem{debugger}{%
strumento di sviluppo la cui finalità è facilitare l'analisi e l'eliminazione dei bug nel software. Tipicamente un debugger permette di visualizzare contestualmente all'esecuzione del codice oggetto la porzione di codice sorgente corrispondente in una qualche forma di più alto livello, di sospendere e riprendere l'esecuzione un'istruzione alla volta (\inglese{step-by-step execution}) mostrando inoltre informazioni sul contenuto dell'area di memoria riferito dalle variabili in uso.
}

\glossaryItem{deployment}{%
indichiamo con questo termine la consegna al cliente, installazione e messa in funzione di un applicativo che senza il passaggio dalla fase di sviluppo alla fase di manutenzione. 
}

\glossaryItem{design pattern}{%
soluzione progettuale che si presta ad essere scelta per le risposte che fornisce a determinate esigenze e le sue proprietà ben testate e documentate di efficienza e manutenibilità. A differenza di un framework, in cui le soluzioni progettuali sono già implementate in un determinato linguaggio e pronte all'uso, un \inglese{design pattern} descrive la struttura della soluzione ad alto livello e si presta ad essere adattato alle specifiche esigenze del progetto e/o composto con altri pattern.
}

\glossaryItem{difetto}{%
altrimenti detto `anomalia' è il risultato di un errore operativo ed è la causa di un malfunzionamento del programma. Mentre l'errore alla radice del difetto avviene in un punto preciso dello sviluppo, il difetto persiste fintantoché non viene corretto e può dare luogo a malfunzionamenti in determinate condizioni. Se un difetto non è rilevato e non dà origine a malfunzionamenti osservabili rimane quiescente e può passare molto tempo dalla sua introduzione alla sua correzione.
}

\glossaryItem{DOM}{%
acronimo di \inglese{Document Object Model}, è lo standard ufficiale del W3C per la rappresentazione logica di documenti, neutrali sia per linguaggio che per piattaforma. Letteralmente si traduce con ''modello a oggetti del documento'' e nell'ambito del progeto si riferisce al documento HTML.
}

\glossaryItem{driver}{%
in fase di analisi dinamica, definisce una componente avente un ruolo attivo durante il test e in grado di richiamare le componenti (passive) da sottoporre a verifica e verificarne l'output in risposta a dati di test, assicurando in tal modo che forniscano un risultato conforme alle aspettative.
}

\letterbox{f}
\glossaryItem{flag}{%
variabile a due soli possibili stati corrispondenti ai valori di verità dell'algebra booleana utile a rappresentare una possibilità dicotomica.
}

\glossaryItem{form}{%
parte dell'interfaccia utente che, all'interno di un'architettura \inglese{client-server}, permette agli utenti del primo di inserire i dati da inviare successivamente al secondo per l'elaborazione in remoto.
}

\letterbox{h}

\glossaryItem{HTML}{%
acronimo di \inglese{HyperText Markup Language} designa un linguaggio di markup semantico per definire la struttura di contenuti in forma di ipertesti. Le istruzioni di marcatura, la cui sintassi basata su `tag' deriva dall'SGML (\inglese{Standard Generalized Markup Language}) più che fornire in modo `procedurale' istruzioni sulla formattazione dichiarano la forma logica del documento creando una struttura arboriforme di elementi innestati gli uni agli altri.
}

\glossaryItem{HTML5}{%
ultima revisione dello standard HTML (ancora in fase di definizione) che si propone di integrare in un unico standard le ultime versioni di HTML e XHTML ed è fortemente orientata all'integrazione di elementi multimediali e allo sviluppo di applicazioni web grazie all'introduzione di numerose migliorie e aggiunte rispetto alle precedenti versioni dello standard.
}

\letterbox{i}
\glossaryItem{IDE}{%
acronimo di \inglese{Integrated Development Environment} (ambiente di sviluppo integrato) rappresenta un applicativo che ha il compito di facilitare la codifica del software integrando tipicamente al proprio interno sotto un'unica interfaccia un editor per i sorgenti, un compilatore/interprete per il linguaggio prescelto e un debugger.
}

\glossaryItem{InnoDB}{%
motore per il salvataggio di dati del DBMS MySQL in grado di supportare il meccanismo di integrità referenziale mediante vincolo di chiave esterna e le proprietà ACID per le transazioni.
}

\glossaryItem{inspection}{%
metodo di analisi statica (e, in particolare, di `desk check') che prevede un controllo con obiettivi ristretti e stabiliti a priori prima che la verifica abbia luogo. Tale tecnica consiste nella ricerca mirata di possibili errori o incongruenze limitando l'analisi (e quindi il lavoro richiesto) a una sola porzione del documento/codice secondo una lista di controllo predeterminata, escludendo le parti in cui il verificatore stesso ritiene sia più improbabile l'insorgenza di problematiche.
}

\letterbox{j}
\glossaryItem{Java}{%
Linguaggio di programmazione interpretato, completamente orientato agli oggetti e fortemente tipizzato ideato negli anni `90 da J.~Gosling et al.~per Sun Microsystems (oggi Oracle America, Inc.). Assieme alla JVM (\inglese{Java Virtual Machine}) e alle estese API che lo corredano, il linguaggio è parte della piattaforma Java.
}

\glossaryItem{JavaScript}{%
linguaggio di scripting interpretato, debolmente orientato agli oggetti con una sintassi ispirata al C utilizzato prevalentemente come linguaggio di scripting all'interno di un browser per la realizzazione di pagine web dinamiche.
}

\glossaryItem{JQuery}{%
libreria JavaScript che ha il compito di semplificare l'attraversamento del codice HTML, la gestione degli eventi, le animazioni e le interazioni AJAX (ossia chiamate asincrone). Si tratta di un framework sviluppato con il preciso intento di rendere il codice più sintetico e limitare al minimo l'estensione degli oggetti globali per ottenere la massima compatibilità con altre librerie.
}

\letterbox{m}
\glossaryItem{malfunzionamento}{
comportamento del software non conforme ai requisiti impliciti o espliciti di progetto, si manifesta quando in assenza di malfunzionamenti della piattaforma sottostante il sistema software non soddisfa le aspettative. I malfunzionamenti sono dovuti alla presenza di difetti (vd. sopra) nel software non rilevati fino alla fase di test (nel caso peggiore, test di sistema).
}

\glossaryItem{milestone}{%
letteralmente `pietra miliare', indica il raggiungimento di determinati obiettivi prefissati in fase di definizione del progetto. Tali eventi presentano particolare importanza e meritano specifico risalto tanto all'interno dei documenti di progetto quanto eventualmente all'esterno con il committente.
}

\glossaryItem{MySQL}{%
DBMS relazionale open source ideato e realizzato da M.~Widenius, largamente utilizzato in ambito Unix e per la realizzazione di web applications.
}

\letterbox{p}
\glossaryItem{PDCA}{%
vd. \underline{ciclo di Deming}
}

\glossaryItem{phpMyAdmin}{%
utilità di amministrazione implementata in PHP per MySQL mediante il browser che consente di avere un'interfaccia grafica per interagire con il DBMS alternativa a quella testuale normalmente distribuita con quest'ultimo.
}

\glossaryItem{plugin}{%
programma non dotato di una propria autonomia (non \inglese{stand-alone}) ma che al contrario può essere eseguito esclusivamente dall'interno di un altro programma di cui rappresenta un'estensione modulare.
}

\glossaryItem{processo}{%
insieme di attività coordinate e coese finalizzate al raggiungimento di un obiettivo comune: nel corso del ciclo di vita di un software i processi sono responsabili del transito del prodotto da uno stadio all'altro e sono composti da più `attività' di livello inferiore. Un processo necessita di risorse materiali, umane o immateriali per il proprio svolgimento, è associato a un insieme di risultati ed è sottoposto a controllo da parte di un'entità superiore (\inglese{process owner}). I processi di ciclo di vita software sono oggetto dello standard ISO12207.
}

\letterbox{q}

\glossaryItem{QA}{%
acronimo di \inglese{Quality Assurance}, si riferisce all'insieme di attività sistematiche volte ad assicurare la qualità di un determinato prodotto o servizio mediante la misurazione rigorosa di metriche sul prodotto, il confronto con gli standard di qualità imposti, il controllo costante delle attività di processo e l'adozione di misure correttive nel caso in cui dovessero essere rilevate potenziali minacce alla qualità del risultato finale.
}

\letterbox{r}

\glossaryItem{repository}{%
servizio di storage in grado di fornire non solo una semplice copia di backup (condivisa) dei dati, ma un sistema di versionamento controllato che permette di ricostruire la storia delle versioni precedenti, annullare selettivamente determinate modifiche e agevolare il lavoro in parallelo dei membri del team di sviluppo.
}

\glossaryItem{roadmap}{%
pianificazione a termine variabile che descrive delle linee guida imposte per raggiungere un determinato obiettivo tecnologico.
}

\letterbox{s}

\glossaryItem{server}{%
in un'architettura \inglese{client-server}, componente che fornisce un servizio in risposta alle richieste ricevute dalla controparte (\inglese{client}).
}

\glossaryItem{sistema di controllo versione}{%
sistema di gestione delle modifiche destinato ad essere usato in ambienti collaborativi che associa a ogni modifica un numero di versione e consente di tenere traccia della storia delle modifiche, annullarne gli effetti o sviluppare in parallelo più rami dello stesso progetto.
}

\glossaryItem{sistema di ticketing}{%
sistema di gestione per le segnalazioni di discrepanze e malfunzionamenti. Ha lo scopo di agevolare e tenere traccia delle attività di correzione definendo tempistiche, eventuali suggerimenti e aggiornamenti sullo stato in cui si trova l'attività presa in esame.
}

\glossaryItem{softphone}{%
applicativo software che permette di effettuare chiamate telefoniche attraverso Internet mediante l'utilizzo di un personale computer \inglese{general purpose} e prevede funzionalità simili a quelle di un telefono tradizionale tramite l'hardware di un computer.
}

\glossaryItem{stand-alone}{%
oggetto o software il cui funzionamento è indipendente da altri oggetti o applicativi, in particolare, un programma stand-alone è autonomo (\textit{versus} un \underline{plugin}) e non richiede altri componenti una volta installato.
}

\glossaryItem{stub}{%
in fase di analisi dinamica, la componente passiva richiamata in fase di verifica da una parte attiva che costituisce l'oggetto del test. Si tratta di componenti di semplice implementazione in quanto hanno solo il ruolo di fornire in output risultati plausibili senza alcuna necessità di applicare gli algoritmi della `reale' logica di business.
}

\letterbox{t}

\glossaryItem{TCO}{%
acronimo di \inglese{Total Cost of Ownership}, è un approccio di gestione dei costi per apparecchiatura informatica (\inglese{Software} e \inglese{Hardware}) che tiene conto delle varie fasi: acquisto, installazione, gestione, manutenzione e smantellamento.
}

\glossaryItem{TomCat}{%
contenitore per servlet Java e motore JSP (\inglese{Java Server Pages}) sviluppato dalla Apache Foundation in grado di fornire un ambiente di esecuzione per le pagine dinamiche e le servlet realizzate sulla base della tecnologia Java.
}

\glossaryItem{ticket}{%
parte fondamentale del sistema di ticketing, rappresenta uno specifico messaggio redatto secondo apposite schematiche rivolto ad un determinato soggetto.
}

\letterbox{v}

\glossaryItem{VoIP}{%
acronimo di \inglese{Voice over IP}, si intende una tecnologia formata da un insieme si protocolli che rende possibile effettuare una conversazione telefonica sfruttando una qualsiasi rete di dati a pacchetto che utilizzi il protocollo IP.
}

\letterbox{w}

\glossaryItem{walkthrough}{
metodo di analisi statica che il verificatore esegue durante lo svolgimento del suo compito. Questo metodo prende in esame tutto il file, , estendendo alla massima ampiezza la ricerca degli errori.
}

\glossaryItem{web application}{%
applicativo in genere fruibile tramite un browser il cui accesso e la cui distribuzione sono basate essenzialmente sul web, il cui vantaggio principale è il costo di setup iniziale e manutenzione pressoché nullo per la parte client.
}

\enlargethispage{\baselineskip}
\glossaryItem{web host}{%
definisce un servizio che permette di `ospitare' (dall'inglese \inglese{hosting}) su un server web le pagine di un sito realizzato per renderlo accessibile agli utenti tramite browser e connessione alla rete Internet. Per garantire l'identificazione dei contenuti viene assegnato al sito un dominio e un indirizzo IP.
}

\glossaryItem{WebRTC}{%
progetto open source di comunicazione audio/video real time proposto da Google che consiste in una serie di API per JavaScript. Si tratta di un progetto tutt'ora in fase di sviluppo che non ha ancora raggiunto una versione stabile e convalidata dal W3C. Nelle intenzioni degli sviluppatori, WebRTC dovrà fornire la possibilità di creare applicazioni web per il realtime multimedia (ad esempio: video chat), senza richiedere l'installazione di plugin o contenuti aggiuntivi.
}

\glossaryItem{worst practice}{%
in antitesi al concetto di `best practice', pratica fuorviante che può essere potenziale fonte di errori (e, pertanto, di anomalie) e come tale è riconosciuta collettivamente da una comunità di esperti del settore.
}

\end{document}

% CANDIDATI PER AGGIUNTE FUTURE
% interfaccia web
% logger
% processo
% SWEBOK
% tracciamento
% UML
% W3C