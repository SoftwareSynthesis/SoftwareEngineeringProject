% ATTENZIONE!!! 
% Per far funzionare i collegamenti ipertestuali si raccomanda di usare
%	\classname{nomedellaclasse}
% per le classi dello stesso package mentre invece 
%	\hyperref[nomedellaclasse]{\ttfamily{}nomequalificatodellaclasse}
% per le classi che non sono dello stesso package e che hanno il nome completo

% **************************************************
% Macro specifiche per il documento corrente
% **************************************************
% Nome
\newcommand{\docName}{Test e misurazioni}
% Nome file
\newcommand{\docFileName}{test\_e\_misurazioni.1.0.pdf}
% Versione
\newcommand{\docVers}{1.0}
% Data creazione
\newcommand{\creationDate}{2013-02-28}
% Data ultima modifica
\newcommand{\modificationDate}{2013-02-28}
% Stato in {Approvato, Non approvato}
\newcommand{\docState}{Non approvato}
% Uso in {Interno, Esterno}
\newcommand{\docUsage}{Esterno}
% Destinatari da specificare come nome1\\ &nome2\\ ecc.
\newcommand{\docDistributionList}{Prof. Tullio Vardanega\\&Prof. Riccardo Cardin\\&Dott. Gregorio Piccoli\\&Team SoftwareSynthesis}
% Redattori da specificare come nome1\\ &nome2\\ ecc.
\newcommand{\docAuthors}{Stefano Farronato}
% Approvato da
\newcommand{\approvedBy}{Andrea Rizzi}
% Verificatori
\newcommand{\verifiedBy}{Diego Beraldin}
% Perscorso (relativo o assoluto) che punta alla directory contenente shared/
% come sua sottodirectory (per comodità chiamiamola 'doc root').
\newcommand{\docRoot}{..}
% definire se si vuole l'indice delle tabelle
\def\INDICETABELLE{false}
% definire se si vuole l'indice delle figure
\def\INDICEFIGURE{false}

% importa il preambolo condiviso da tutti i documenti
% shared/preamble.tex
%
% Questo documento contiene la parte del preambolo condivisa e viene pertanto
% richiamato nel 'master' di tutti i documenti di progetto.  Al suo interno
% contiene le inclusioni (e le configurazioni) di tutti i package richiesti per
% la compilazione dei documenti, le macro di carattere generale e la definizione
% degli stili di pagina.

\documentclass[a4paper,10pt]{article}

% **************************************************
% Macro generiche
% **************************************************
\newcommand{\team}{Software Synthesis}                    % chi siamo
\newcommand{\email}{info@softwaresynthesis.org}           % e-mail
\newcommand{\caName}{MyTalk}                              % titolo capitolato
\newcommand{\manager}{SynthesisRequirementManager}        % nome del sistema di tracciamento
\newcommand{\memberdata}[1]{%
  \texttt{\textcolor{RedOrange}{#1}}}                     % attributi di una classe
\newcommand{\method}[1]{\texttt{\textcolor{Emerald}{#1}}} % metodi di una classe
\newcommand{\exception}[1]{%
  \texttt{\textcolor{RedViolet}{#1}}}                     % eccezione
% \newcommand{\handler}[1]{\texttt{\textcolor{Maroon}{#1}}} % per gli event handler
\newcommand{\inglese}[1]{%
  \foreignlanguage{english}{\textit{#1}}}                 % per i testi in lingua inglese
\newcommand{\purpose}{%                                     scopo del prodotto
Con il progetto ``\caName'' si intende un sistema software di comunicazione tra utenti mediante \underline{browser} senza la necessit{\`a} di installazione di \underline{plugin} e/o software esterni. L'utilizzatore avr{\`a} la possibilit{\`a} di interagire con un altro utente tramite una comunicazione audio - audio/video - testuale e, inoltre, ottenere delle statistiche sull'attivit{\`a} in tempo reale.%
}
\newcommand{\glossaryIntro}{%                               introduzione al glossario
Al fine di evitare incomprensioni dovute all'uso di termini tecnici nei documenti, viene redatto e allegato il documento \textit{glossario.4.0.pdf} dove vengono definiti e descritti tutti i termini marcati con una sottolineatura.%
}


% **************************************************
% Codifica e lingua dei documenti
% **************************************************
\usepackage[utf8x]{inputenc}                              % codifica caratteri dei documenti sorgenti
\usepackage[english,italian]{babel}                       % localizzazione ai fini di sillabazione e cross-references
\usepackage[T1]{fontenc}                                  % codifica font di output

% **************************************************
% Definizione geometria della pagina
% **************************************************
\usepackage[a4paper,head=4cm,top=4.5cm,bottom=3cm,left=3cm,right=3cm,bindingoffset=5mm]{geometry}

% *************************************************
% Intestazioni e piè di pagina personalizzati
% *************************************************
\usepackage{fancyhdr}
% stile normale
\fancypagestyle{normal}{
\fancyhead{}                                              % intestazione
\fancyhead[RE,RO]{
\begin{picture}(0,0)
  \put(-410,0){\includegraphics[width=1.02\textwidth]{header_logo}}
  \put(-410,10){\sffamily\large\leftmark}
\end{picture}
\vspace{-4pt}
}
\renewcommand{\headrulewidth}{0pt}                       % riga sotto l'intestazione
\cfoot{}                                                  % piè di pagina
\fancyfoot[RO,LE]{\sffamily
  pag.~\thepage{} di \pageref{LastPage}}                  % a dx nelle pag. dispari e a sx in quelle pari
\fancyfoot[RE,LO]{\sffamily\docFileName{}}
\renewcommand{\footrulewidth}{.4pt}                       % riga sopra il piè di pagina
}
% stile per gli indici
\fancypagestyle{toc}{
\fancyhead{}                                              % intestazione
\fancyhead[RE,RO]{
\begin{picture}(0,0)
  \put(-410,0){\includegraphics[width=1.02\textwidth]{header_logo}}
\end{picture}
}
\renewcommand{\headrule}{}                                % nessuna riga sotto l'intestazione
\cfoot{}                                                  % piè di pagina
\fancyfoot[RO,LE]{\sffamily\thepage{}}                    % a dx nelle pag. dispari e a sx in quelle pari
\fancyfoot[RE,LO]{\sffamily\docFileName{} -- v.\docVers}
\renewcommand{\footrulewidth}{.4pt}                       % riga sopra il piè di pagina
}

\pagestyle{fancy}                                         % premetto: non so usare bene le marche:
\renewcommand{\sectionmark}[1]{\markboth{#1}{#1}}         % se qualcuno ha idee migliori si faccia avanti!

% **************************************************
% Tabelle
% **************************************************
\usepackage{tabularx}                                     % tabelle di larghezza fissa con una o più colonne variabili
\usepackage{multirow}                                     % colonne con colonne che si estendono per più righe
\usepackage{booktabs}                                     % per inserire l'ambiente table e le righe orizz. nelle tabelle
\usepackage{longtable}			                              % tabelle oltre i limiti di pagina

% **************************************************
% Cross-references e collegamenti ipertestuali
% **************************************************
\usepackage[hidelinks]{hyperref}
\hypersetup{%
  colorlinks=false, linktocpage=false, pdfborder={0,0,0}, pdfstartpage=1, pdfstartview=FitV,%
  urlcolor=Cyan, linkcolor=Cyan, citecolor=Black, %pagecolor=Black,%
  pdftitle={\docName}, pdfauthor={\team}, pdfsubject={}, pdfkeywords={},%
  pdfcreator={pdflatex}, pdfproducer={pdflatex with hyperref package}%
}

% **************************************************
% Immagini e grafica
% **************************************************
\usepackage{graphicx}                                     % supporto ad aspetti avanzati delle immagini
\usepackage[table,usenames,dvipsnames]{xcolor}            % tabelle con righe colorate e alternate
\graphicspath{{\docRoot/pics/}}                           % percorso contenente tutti i file immagini
\usepackage{float}                                        % per rendere non flottanti gli ambienti flottanti
\usepackage[italian]{varioref}                            % testo completo riferimenti in italiano

% **************************************************
% Definizioni di colori
% **************************************************
\definecolor{myBlue}{RGB}{1,167,236}
\definecolor{lightblue}{RGB}{213,243,253}%{119,218,247}
\definecolor{llightblue}{RGB}{229,255,255}

% **************************************************
% Altri pacchetti opzionali
% **************************************************     
\usepackage{lastpage}                                     % per sapere il numero totale di pagine
\usepackage{eurosym}                                      % per il simbolo dell'euro usare \EUR{x} dove x è l'importo
\usepackage{ifthen}                                       % permette la scelta di rami condizionali nella compilazione
\usepackage{enumitem}                                     % permette di configurare gli elenchi puntati e numerati


% macro specifiche per il documento corrente
\newcommand{\classsection}[1]{\subsubsection{#1}\label{#1}}
\newcommand{\classname}[1]{\hyperref[#1]{\ttfamily#1}}

% Fine del preambolo e inizio del documento
\begin{document}

% Inclusione della prima pagina
% shared/firstpage.tex
%
% Questo documento definisce il contenuto della prima pagina, che si suppone
% essere uguale in tutti i documenti.  Oltre al logo e al titolo, la prima
% pagina contiene i metadati relativi al documento in cui viene inclusa.


% rimuove intestazioni e piè di pagina
\pagestyle{empty}

\begin{center}

% logo del gruppo
\includegraphics[width=1.5\textwidth]{logo}

\vspace{1in}

% titolo del documento
{\Huge\bfseries \docName}

\vspace{1in}

% tabella riepilogativa
\begin{tabularx}{.7\textwidth}{>{\bfseries\sffamily}l>{\sffamily}l}
\toprule
\multicolumn{2}{>{\sffamily}c}{Informazioni sul documento}\\
\midrule
Nome file:            & \docFileName\\
Versione:             & \docVers\\
Data creazione:       & \creationDate\\
Data ultima modifica: & \modificationDate\\
Stato:                & \docState\\
Uso:                  & \docUsage\\
Redattori:            & \docAuthors\\
Approvato da:         & \approvedBy\\
Verificatori:         & \verifiedBy\\
\bottomrule
\end{tabularx}

\end{center}

\newpage


%---------------------------RUOLI----------------------------
%FASE 1:
%Programmatori: DIEGO, STEFANO, SCHIVO
%FASE 2:
%Programmatori: MENE, TRES, ELENA

%Verificatore: SCHIVO, STEFANO, RIZZI, DIEGO (dobbiamo fare un sacco di test)
%Responsabile finale supremo: RIZZI
%------------------------------------------------------------

% Storico delle modifiche
\section*{Storia delle modifiche}
\begin{center}
\begin{longtable}{lp{.32\textwidth}lll}
\toprule
Versione & Descrizione intervento & Membro & Ruolo & Data\\
\midrule % inserire qui il contenuto della tabella
1.0 & Approvazione documento& Andrea Rizzi & Responsabile & 2013-02-28\\
0.8 & correzione errori segnalati dal verificatore& Stefano Farronato & Verificatore & 2013-02-28\\
0.7 & verifica correttezza lessico ortografica e corrispondenza test con quanto descritto& Diego Beraldin & Verificatore & 2013-02-28\\
0.6 & aggiunti test su capitolo 4.2 e test coverage& Stefano Farronato & Verificatore & 2013-02-28\\
0.5 & stesura capitoli sui test di sistema e di utilizzo, aggiunti test su capitolo 4.1& Stefano Farronato & Verificatore & 2013-02-28\\
0.4 & stesura capitolo 4.2, relativo ai test sul \textit{presenter} & Stefano Farronato & Verificatore & 2013-02-28\\
0.3 & stesura capitolo 4.1, relativo ai test sul \textit{model} & Stefano Farronato & Verificatore & 2013-02-28\\
0.2 & stesura introduzione con ``metriche sul codice'' e preambolo su ``test e risultati''& Stefano Farronato & Verificatore & 2013-02-28\\
0.1 & Creazione del documento e stesura delle sezioni ``Introduzione'' e ``Riferimenti''. & Stefano Farronato & Verificatore & 2013-02-28\\
\bottomrule
\end{longtable}
\end{center}
\newpage

% inclusione dell'indice
% shared/toc.tex
%
% Questo file contiene le istruzioni che generano l'indice o gli indici del
% documento (utile nel caso in cui decidessimo di avere anche un indice delle
% tabelle e/o un indice delle figure).

% imposta lo stile di pagina per i titoli definito nel preambolo
\pagestyle{toc}
% imposta i numeri di pagina romani minuscoli
\pagenumbering{roman}

% genera automaticamente l'indice di LaTeX
\tableofcontents

% se è true \INDICETABELLE allora genera l'indice delle tabelle, altrimenti non fa nulla
\ifthenelse{\equal{\INDICETABELLE}{true}}{%
  \clearpage % l'indice delle tabelle, se c'è, deve andare a pagina nuova
  \listoftables
}{}

% se è true |INDICEFIGURE allora genera l'indice delle figure, altrimenti non fa nulla
\ifthenelse{\equal{\INDICEFIGURE}{true}}{%
  \clearpage % l'indice delle figure, se c'è, deve andare a pagina nuova
  \listoffigures
}{}

%in ogni caso occorre andare a pagina nuova dopo gli indici
\clearpage


% Alcuni aggiustamenti per le pagine
\pagenumbering{arabic}
\setcounter{page}{1}
\pagestyle{normal}

% Qui ha inizio il documento vero e proprio...
\newpage

\section{Introduzione}
\subsection{Scopo del prodotto}
\purpose

\subsection{Scopo del documento}
Il documento ha lo scopo di definire e riportare l'esito dei test effettuati sul codice prodotto, attraverso misurazioni quantitative valutate attraverso metriche definite al fine di assicurare che la qualità del prodotto risultante sia coerente con i requisiti redatti e rispetti quanto redatto nel documento \textit{piano\_di\_qualifica.3.0.pdf}.
I test predisposti saranno descritti esaustivamente sia nella forma che nei risultati ottenuti, al fine di renderene più chiara e oggettiva possibile la comprensione.

\subsection{Glossario}
\glossaryIntro
\clearpage

\section{Riferimenti}
\subsection{Normativi}
\begin{itemize}
\item[] \textit{piano\_di\_qualifica.3.0.pdf} allegato;
\item[] \textit{norme\_di\_progetto.3.0.pdf} allegato;
\item[] \textit{specifica\_tecnica.2.0.pdf} allegato;
\item[] \textit{definizione\_di\_prodotto.1.0.pdf} allegato.
\item[] \textit{analisi\_dei\_requsiti.3.0.pdf} allegato;
\end{itemize}

\subsection{Informativi}
\begin{itemize}
\item[] Capitolato d'appalto: \caName{}, v1.0, redatto e rilasciato dal proponente Zucchetti s.r.l. reperibile all'indirizzo \url{http://www.math.unipd.it/~tullio/IS-1/2012/Progetto/C1.pdf};
\item[] testo di consultazione: \textit{Software Engineering (8th edition) Ian Sommerville, Pearson Education | Addison Wesley};
\item[] \textit{glossario.3.0.pdf} allegato.
\end{itemize}
\clearpage

\section{Metriche sul codice}
Prima di tutto sono state calcolate, al fine di quantificare la ``dimensione'' di una componente, due metriche: il \inglese{Number Of Classes} (NOC) e il \inglese{Number Of Method} (NOM).
NOC rappresenta il numero totale delle classi che sono rappresentate all'interno del package, mentre il NOM rappresenta il numero dei metodi conteggiati, sempre per package.

La tabella X riporta le misurazioni su tali metriche per i package del prodotto \caName{}, evidenziandone i più corposi dal punto di vista implementativo.


\begin{center}
\rowcolors{2}{lightblue}{llightblue}\begin{longtable}{p{.55\textwidth}ll}
\toprule Package & TLOC  & MLOC\\
nome.package.esaminato & XXXX & XXXX\\
\midrule

\bottomrule
\end{longtable}
\end{center}


Spingendosi più nel dettaglio si è deciso di esporre il \inglese{Total Line Of Code} (TLOC) e il \inglese{Method Lines Of Code} (MLOC).
Come suggeriscono i nomi, TLOC rappresenta il numero totale di linee di codice (effettive, ovvero senza contare quelle non vuote o commentate) all'interno di un'unità di compilazione, MLOC definisce invece il numero totale di linee di codice (sempre effettive) che definiscono i corpi dei metodi. 

Come per NOC e NOM riportiamo la tabella X con le misurazioni su tali metriche relative ai package di \caName{}.


\begin{center}
\rowcolors{2}{lightblue}{llightblue}\begin{longtable}{p{.55\textwidth}ll}
\toprule Package & NOC  & NOM\\
\midrule
nome.package.esaminato& XXXX & XXXX\\
\bottomrule
\end{longtable}
\end{center}


Coerentemente con le Norme di Progetto consultabili nel documento allegato \textit{norme\_di\_progetto.3.0.pdf} si riporta l'indice di complessità ciclomatica (media) per ogni package. Tale misura è direttamente legata al numero di cammini linearmente indipendenti che compongono il grafo di controllo di flusso. Tale indice dovrebbe essere compreso tra 0 e 10 per definire il codice nella norma.
Infine il \inglese(Lack Of Cohesion Of Methods) (LCOM3) indica il liello (medio) di coesione dei metodi, più tale livello risulta basso migliore risulterà essere la progettazione della classe. I valori possono quindi variare tra lo 0, che indica la massima coesione possibile, e 2, che ne identifica il valore minimo.


\begin{center}
\rowcolors{2}{lightblue}{llightblue}\begin{longtable}{p{.55\textwidth}ll}
\toprule Package & CC  & LCOM3\\
nome.package.esaminato& XXXX & XXXX\\
\midrule

\bottomrule
\end{longtable}
\end{center}


\section{Test e risultati}
I test riportati hanno lo scopo di dimostrare in modo oggettivo la bontà e la qualità di quanto prodotto a livello di codice. Tali test servono inoltre a rilevare \inglese{bug} non evidenziati (o non possibili da evidenziare) durante l'analisi statica del codice e vengono effettuati sui package relativi alla parte di \inglese{Model} e \inglese{Presenter} del modello MVP generale.
Le verifiche relative alla parte \inglese{View} saranno al contrario effettuati successivamente a quelli sui componenti sopracitati, in quanto sarà necessario avere la certezza che tali componenti risultino stabili e privi (per quanto possibile) di \inglese{bug}. 
Specifichiamo inoltre che tali verifiche saranno effettuati tramite i test di sistema specificati nel documento \textit{anilisi\_dei\_requisiti.3.0.pdf} allegato.

Il team ha seguito la seguente sequenza di passi per la realizzazione e l'utilizzo delle varie funzioni di test:
\begin{itemize}
\item individuazione preliminare delle funzionalità da testare;
\item analisi, progettazione e successiva implementazione delle funzioni di test per le funzionalità individuate, stabilendo i dati di input e specificati gli output attesi;
\item esecuzione delle funzioni di test create.
\item analisi dei risultati dei test, confrontato i risultati con quelli attesi ed effettuato la correzione di eventuali anomalie riscontrate, verificata infine la ripetibilità del test stesso;
\item registrati i risultati ottenuti.
\end{itemize}

I test per la parte \inglese{Model} sono stati predisposti ed eseguiti usufruendo della libreria JUnit, in quanto questa parte della struttura è implementata mediante codice \underline{Java}.
Per il componente \inglese{Presenter} si è deciso invece di usufruire di QUnit, coerentemente con il linguaggio scelto (\underline{JavaScript}) che lo implementa. Tale \underline{franework} risulta pratico in quanto permette di individuare facilmente quale metodo analizzato ha portato al fallimento del test, inoltre permette una netta separazione tra codice testato e i test effettivi.

\subsection{Model}
I test su questa parte del sistema mirano a verificare che la persistenza dei dati all'interno del \underline{server} sia gestita correttamente dal \underline{database}.
Per l'esecuzione di questi test si è usata, coerentemente con le norme di progetto, la libreria \textit{JUnit}.
Ogni test è inoltre stato eseguito più volte in modo da avere uno spettro più ampio di misurazioni e produrre una verifica più stabile e oggettiva possibile.

\subsubsection{AESAlgorithmTest}
\textbf{Verifica la Classe:}\textit{AESAlgorithm}.\\
\textbf{Descrizione}:l'obbiettivo di questo test è assicurare che l'algrotimo di cittografia AES utilizzato dal sistema \caName risulti affidabile e riesca a codificare e decodificare una stringa passata.\\
Tale test si compone di un solo metodo:
\begin{itemize}
\item \textit{testEncodeAndDecode() } effettua la codifica di una stringa passata, effettua successivamente la decodifica della stessa e la controlla con l'originale per verificarne l'uguaglianza.
\end{itemize}
\textbf{Risultato del test:} non sono stati rilevati errori.

\begin{table}
\end{table}

\subsubsection{AdressBookEntryTest}
\textbf{Verifica la Classe:} \textit{AdressBookEntry}.\\
\textbf{Descrizione}:l'obbiettivo di questo test è verificare la corretta conversione di un istanza di \textit{AddressBookEntry}.\\
Tale test si compone di un solo metodo:
\begin{itemize}
\item \textit{testToJson() } tale metodo riceve una stringa in formato json di una 	\textit{entry} della rubrica e verifica che quest'ultima sia equivalente a quella ottenuta dalla chiamata al metodo \textit{ToJson} sull'istanza della classe \textit{AddressBookEntry}.
\item{testEquals()} viene creato un oggetto \textit{AdressBookEntry} e viene confrontato con l'istanza dell'oggetto che ha invocato il metodo, restituendo \textit{true} se e solo se sono uguali.
\end{itemize}
\textbf{Risultato del test:} non sono stati rilevati errori.

\subsubsection{AuthenticationDataTest}
\textbf{Verifica la Classe:} \textit{AuthenticationData}.\\
\textbf{Descrizione}:l'obbiettivo di questo test è assicurare l'uguaglianza tra due oggetti che rappresentano le credenziali dell'utente inserite volutamente con gli stessi dati. Successivamente verrà effettuata anche una negazione di tale test per assicurarne l'efficacia.\\
Tale test si compone di due metodi:
\begin{itemize}
\item \textit{testEquals() } testa che due oggetti diversi contenenti gli stessi dati passati al metodo restituiscano \textit{true} solo se sono effettivamente uguali;
\item \textit{testDifferent()}testa che due oggetti diversi contenenti diversi dati passati al metodo restituiscano \textit{true} solo se sono effettivamente differenti.
\end{itemize}
\textbf{Risultato del test:} non sono stati rilevati errori.

\begin{table}
\end{table}

\subsubsection{AuthenticationModuleTest}
*Descrizione*
Tale test si compone di X metodi:
\begin{itemize}
\item \textit{testLogin() }
\item \textit{testLogout() }
\item \textit{testAbort() }
\end{itemize}
\textbf{Risultato del test:} non sono stati rilevati errori.

\begin{table}
\end{table}

\subsubsection{CredentialLoaderTest}
\textbf{Verifica la Classe:} \textit{CredentialLoarder}.\\
\textbf{Descrizione}:l'obbiettivo di questo test è assicurare che il vettore in cui celle contiente tutte le credenziali d'accesso dell'utente (coppia \textit{username-password}) venga popolato correttamente.\\
Tale test si compone di un solo metodo metodi:
\begin{itemize}
\item \textit{testLoarder() } tale metodo carica nel vettore i campi \textit{username} e \textit{password} e controlla che siano effettivamente stati inseriti e i dati siano coerenti (uguali) con quelli iniziali.
\end{itemize}
\textbf{Risultato del test:} non sono stati rilevati errori.

\subsubsection{GroupTest}
\textbf{Verifica la Classe:} \textit{Group}.\\
\textbf{Descrizione}:l'obbiettivo di questo test è verificare la corretta conversione di un istanza di \textit{Group}.\\
Tale test si compone di un solo metodo:
\begin{itemize}
\item \textit{testToJson() } tale metodo riceve una stringa in formato json di una 	\textit{entry} della rubrica e verifica che quest'ultima sia equivalente a quella ottenuta dalla chiamata al metodo \textit{ToJson} sull'istanza della classe \textit{Group}.
\item \textit{testId() } verifica l'effettivo inserimento del campo \textit{id} mediante il metodo \textit{getId()};
\item \textit{testName() } verifica l'effettivo inserimento del campo \textit{name} mediante il metodo \textit{getName()};
\item{testEquals()} viene creato un oggetto \textit{AdressBookEntry} e viene confrontato con l'istanza dell'oggetto che ha invocato il metodo, restituendo \textit{true} se e solo se sono uguali.
\end{itemize}
\textbf{Risultato del test:} non sono stati rilevati errori.

\subsubsection{HybernateUtilTest}
\textbf{Verifica la Classe:} \textit{HybernateUtil}.\\
\textbf{Descrizione}:l'obbiettivo di questo test è assicurare la corretta esecuzione dei metodi della classe \textit{HibernateUtil}. \\
Tale test si compone di due metodi:
\begin{itemize}
\item \textit{testIstance() } l'obbiettivo di questo test è testare l'effettiva presenza di un unica istanza della classe textit{HibernateUtil} mediante la creazione di due istanze e verificando che entrambi puntino alla stessa istanza dell'oggetto;
\item \textit{testGetFactory()} ha lo scopo di verifcare la presenza di una \textit{sessionFactory} configurata correttamente per la comunicazione con il database.
\end{itemize}
\textbf{Risultato del test:} non sono stati rilevati errori.

\subsubsection{UserDataDAOTest}
\textbf{Verifica la Classe:} \textit{UserDataDAO}.\\
\textbf{Descrizione}:l'obbiettivo di questo test è assicurare la corretta esecuzione delle operazioni CRUD (\textit{Create, Read, Update, Delete}) per gli oggetti di tipo \textit{UserData} verso il database. \\
Tale test si compone di 5 metodi:
\begin{itemize}
\item \textit{testInsert()} tale metodo ha lo scopo di verificare il corretto inserimento di uno \textit{userdata} all'interno del database;
\item \textit{testGetByEmail()} ha lo scopo di verificare il prelevamento dal database di uno \textit{userdata} tramite l'indirizzo mail con cui si è registrato nel sistema;
\item \textit{testUpdate() } verifica se le modifiche ad un oggetto \textit{userdata} siano effettive nel database;
\item \textit{testDelete() } verifica l'effettiva cancellazione di uno \textit{userdata} dal database.
\item \textit{testSearchGeneric()} verifica che sia presente uno \textit{userdata} nel database mediante la ricerca fatta cercando un parametro di input che ha una corrispondenza con uno o più campi tra \textit{nome, cognome, email}.
\end{itemize}
\textbf{Risultato del test:} non sono stati rilevati errori.

\subsubsection{UserDataTest}
\textbf{Verifica la Classe:} \textit{UserData}.\\
\textbf{Descrizione}:l'obbiettivo di questo test è verificare la corretta conversione di un istanza di \textit{DataTest}. La classe inizializza inoltre un istanza di tale classe, impostandone i campi membro attraverso i metodi \textit{set}. Successivamente mediante i metodi \textit{get} vengono verificati gli effettivi inserimenti.\\
Tale test si compone di otto metodi:
\begin{itemize}
\item \textit{testToJson() } tale metodo riceve una stringa in formato json di una 	\textit{entry} della rubrica e verifica che quest'ultima sia equivalente a quella ottenuta dalla chiamata al metodo \textit{ToJson} sull'istanza della classe \textit{DataTest};
\item \textit{testId() } verifica l'effettivo inserimento del campo \textit{id} mediante il metodo \textit{getId()};
\item \textit{testEmail() } verifica l'effettivo inserimento del campo \textit{email} mediante il metodo \textit{getEmail()};
\item \textit{testPassword() } verifica l'effettivo inserimento del campo \textit{password} mediante il metodo \textit{getPassword()};
\item \textit{testQuestion() } verifica l'effettivo inserimento del campo \textit{question} mediante il metodo \textit{getQuestion()};
\item \textit{testAnswer() } verifica l'effettivo inserimento del campo \textit{answer} mediante il metodo \textit{getAnswer()};
\item \textit{testName() } verifica l'effettivo inserimento del campo \textit{name} mediante il metodo \textit{getName()};
\item \textit{testSurname() } verifica l'effettivo inserimento del campo \textit{surname} mediante il metodo \textit{getSurname()};
\item \textit{testPicturePath() } verifica l'effettivo inserimento del campo \textit{picturePath} mediante il metodo \textit{getPicturePath()};
\item{\textit{testEquals()}} verranno creati due oggetti \textit{UserData} contenenti gli stessi dati, tale metodo restituirà \textit{true} se e solo se sono effettivamente uguali.
\end{itemize}
\textbf{Risultato del test:} non sono stati rilevati errori.

\subsubsection{}
\textbf{Verifica la Classe:} \textit{}.\\
\textbf{Descrizione}:\\
Tale test si compone di X metodi:
\begin{itemize}
\item \textit{() }
\end{itemize}
\textbf{Risultato del test:} non sono stati rilevati errori.


\subsection{Presenter}
I test sul \textit{presenter} sono mirati a verificare che le classi di tale componente rispettino il comportamento atteso, interagendo i dati tramite l'interazione con il \textit{model}, componente testato nella sezione precedente. Questa serie di test, come anticipato nelle prime pagine del documento, è stata eseguita mediante il \underline{\textit{framework}} \textit{Qunit}. Tali verifiche sono stati implementate allo scopo di non testare soltanto il comportamento della specifica funzione, ma anche verificare lo stato di avanzamento del prodotto e il numero di bug evidenziati ancora esistenti.
Ovviamente al termine di tali verifiche il risultato dovrà risultare positivo, ovvero i test dovranno essere \textit{tutti} superati.

\subsubsection{AccountSettingsPanelPresenterTest}
\textbf{Verifica l'oggetto:} \textit{AccountSettingsPanelPresenter}.\\
\textbf{Descrizione}: verifica la corretta inizializzazione di \textit{AccountSettingPanel}, ovvero che l'albero sia stato costruito correttamente, e che il contenuto dei vari nodi sia stato inserito coerentemente.\\
Tale verifica è composta da un solo test:
\begin{itemize}
\item \textit{testCreatePanel() } crea l'elemento \textit{AccountSettingPanel}, estrae la lista dei suoi figli, ne controlla il numero (deve essere \textit{tre} ) e che il primo sia un immagine, il secondo una lista e il terzo un \inglese{button}.
Successivamente controlla che gli elementi della lista vengano visualizzati correttamente.
\end{itemize}
\textbf{Risultato del test:} non sono stati rilevati errori.

\subsubsection{AddressBookPanelPresenterTest}
\textbf{Verifica l'oggetto:} \textit{AdressBookPanelPresenter}.\\
\textbf{Descrizione}:\\
Tale verifica è composta da un solo test:
\begin{itemize}
\item \textit{() }
\end{itemize}
\textbf{Risultato del test:} non sono stati rilevati errori.

\subsubsection{CallHistoryPanelPresenterTest}
\textbf{Verifica l'oggetto:} \textit{CallHistoryPanelPresenter}.\\
\textbf{Descrizione}: l'obbiettivo di questo test è verificare la corretta creazione del sotto-albero che ha radice nell'elemento \textit{CallHistoryPanel}.\\
Tale verifica è composta da un solo test:
\begin{itemize}
\item \textit{testCreatePanel() } controlla che CallHistoryPanel abbia un unico figlio, e che quest'ultimo sia effettivamente una lista.
\end{itemize}
\textbf{Risultato del test:} non sono stati rilevati errori.

\subsubsection{ComunicationPanelPresenterTest}
\textbf{Verifica l'oggetto:} \textit{ComunicationPanelPresenter}.\\
\textbf{Descrizione}: l'obbiettivo di questo test è verificare la corretta creazione del sotto-albero che ha radice nell'elemento \textit{ComunicationPanel}.\\
Tale verifica è composta da un solo test:
\begin{itemize}
\item \textit{testCreatePanel() } controlla che il numero dei figli del \textit{ComunicationPanel} siano esattamente due, che entrambi siano \textit{<div>} e che il valore dell'attributo \textit{id} del primo figlio sia ``divCall''. mentre del secondo ``divChat''. 
\item \textit{()} %TODO MANCANO TEST
\end{itemize}
\textbf{Risultato del test:} non sono stati rilevati errori.

\subsubsection{ContactPanelPresenterTest}
\textbf{Verifica l'oggetto:} \textit{ContactPanelPresenter}.\\
\textbf{Descrizione}: l'obbiettivo di questo test è verificare la corretta creazione del sotto-albero che ha radice nell'elemento \textit{ContactPanel}.\\
Tale verifica è composta da un solo test:
\begin{itemize}
\item \textit{testCreatePanel() }crea l'elemento \textit{ContactPanel}, estrae la lista dei suoi figli e controlla per ogni figlio che le informazioni siano coerenti con quelle inserite in input.
\item \textit{()} %TODO MANCANO TEST
\end{itemize}
\textbf{Risultato del test:} non sono stati rilevati errori.

\subsubsection{GroupPanelPresenterTest}
\textbf{Verifica l'oggetto:} \textit{GroupPanelPresenter}.\\
\textbf{Descrizione}:\\
Tale verifica è composta da un solo test:
\begin{itemize}
\item \textit{() } %TODO MANCANO TEST
\end{itemize}
\textbf{Risultato del test:} non sono stati rilevati errori.

\subsubsection{LoginPanelPresenterTest}
\textbf{Verifica l'oggetto:} \textit{LoginPanelPresenter}.\\
\textbf{Descrizione}: l'obbiettivo di questo test è verificare la corretta istanziazione del \textit{testLoginPanel} popolato con dei dati coerenti rispetto allo scopo della \underline{\inglese{form}}.
Tale verifica è composta da quattro test:
\begin{itemize} 
\item \textit{testInitialize() } mediante diversi metodi \textit{equal} controlla l'effettiva correttezza della \inglese{form} creata, sia nella forma che nel funzionamento;
\item \textit{testLogin() } effettua una prova di login, mediante l'inserimento di uno \textit{username} e una \textit{password} corretti, verificando che tali dati siano accettati correttamente;
\item \textit{testGetUsername() } verifica che, dato uno \textit{username} in input corretto generato come \underline{\inglese{stub}}, sia possibile recuperarlo correttamente. Se lo \textit{username} non è una mail valida, o non è inserito nella \inglese{form}, solleva un'eccezione.
\item \textit{testGetPassword() } verifica che, data una \textit{password} in input corretta generata come \inglese{stub}, sia possibile recuperarla correttamente.Se la \textit{password} non è inserita nella \inglese{form}, solleva un'eccezione.
\item \textit{testHide()} controlla che venga correttamente nascosto il \inglese{form} di \textit{login} dopo l'effettiva autenticazione o di richiesta di registrazione.
\item \textit{testBuildRetrivePasswordForm()} tale test verifica la corretta creazione della form richiamata per il recupero della password tramite risposta segreta, successivamente controlla che venga richiamata la domanda correttamente impostata a cui rispondere.
\item \textit{testSendAnswer} %TODO DA DESCRIVERE
\end{itemize}
\textbf{Risultato del test:} non sono stati rilevati errori.

\subsubsection{MainPanelPresenterTest}
\textbf{Verifica l'oggetto:} \textit{MainPanelPresenter}.\\
\textbf{Descrizione}:\\
Tale verifica è composta da un solo test:
\begin{itemize}
\item \textit{() } %TODO MANCANO TEST
\end{itemize}

\subsubsection{MessagePanelPresenterTest}
\textbf{Verifica l'oggetto:} \textit{MessagePanelPresenter}.\\
\textbf{Descrizione}: l'obbiettivo di questo test è verificare la corretta creazione del sotto-albero che ha radice nell'elemento \textit{MessagePanel} e che il contenuto dei vari nodi sia stato inserito coerentemente.\\
Tale verifica è composta da un solo test:
\begin{itemize}
\item \textit{testCreatePanel()} crea l'elemento \textit{MessagePanel} e estrae la lista dei figli del pannello verificando che contenga due figli:il primo dev'essere di tipo \textit{video} mentre il secondo di tipo \textit{div}.
Successivamente estraggo la lista dei figli del primo figlio (il suo figlio deve essere \textit{source}) e del secondo figlio (il cui figlio deve essere \textit{ul}).
\end{itemize}
\textbf{Risultato del test:} non sono stati rilevati errori.

\subsubsection{RegisterPanelPresenterTest}
\textbf{Verifica l'oggetto:} \textit{RegisterPanelPresenter}.\\
\textbf{Descrizione}:l'obbiettivo di questo test è verificare la corretta creazione del sotto-albero che ha radice nell'elemento \textit{RegisterPanel}, la possibilità di nascondere correttamente l'elemento \textit{RegisterPanel}, il corretto recupero dei dati inseriti nella \inglese{form} d'iscrizione e infine i test relativi al recupero delle singole informazioni passate in fase di registrazione.\\
Tale verifica è composta da dieci test:
\begin{itemize}
\item \textit{testInitialize() } mediante diversi metodi \textit{equal} controlla l'effettiva correttezza della \inglese{form} creata, sia nella forma che nel funzionamento;
\item \textit{testHide() } controlla che venga correttamente nascosto il \inglese{form} di \textit{login} dopo l'effettiva richiesta di registrazione.
\item \textit{testRegister() } mediante diversi metodi \textit{equal} controlla l'effettiva registrazione dei dati inseriti nella \inglese{form}, confrontandoli con quelli passati in ingresso;
\item \textit{testGetPicturePath() } crea una \inglese{form} compilata correttamente con l'immagine utente da inserire, successivamente prova a recuperarla mediante \textit{GetPicturePath}, verificando se è stata inserita correttamente (o meno). 
\item \textit{testGetUsername() } crea una \inglese{form} compilata correttamente con lo \textit{username}, e prova a recuperarlo mediante \textit{GetUsername}, verificando se è stato inserito correttamente (o meno).
\item \textit{testGetPassword() } crea una \inglese{form} compilata correttamente con il campo \textit{password}, e prova a recuperarlo mediante \textit{GetPassword}, verificando se è stata inserita correttamente (o meno).
\item \textit{testGetQuestion() }  crea una \inglese{form} compilata correttamente con il campo \textit{question}, e prova a recuperarlo mediante \textit{GetQuestion}, verificando se è stata inserita correttamente (o meno).
\item \textit{testGetAnswer() } crea una \inglese{form} compilata correttamente con il campo \textit{answer}, e prova a recuperarlo mediante \textit{GetAnswer}, verificando se è stata inserita correttamente (o meno).
\item \textit{testGetName() }  crea una \inglese{form} compilata correttamente con il campo \textit{name}, e prova a recuperarlo mediante \textit{GetName}, verificando se è stato inserito correttamente (o meno).
\item \textit{testGetSurname() }  crea una \inglese{form} compilata correttamente con il campo \textit{surname}, e prova a recuperarlo mediante \textit{GetSurname}, verificando se è stata inserita correttamente (o meno).

\end{itemize}
\textbf{Risultato del test:} non sono stati rilevati errori.

\subsubsection{SearchResultPanelPresenterTest}
\textbf{Verifica l'oggetto:} \textit{SearchResultPanelPresenter}.\\
\textbf{Descrizione}:\\
Tale verifica è composta da un solo test:
\begin{itemize}
\item \textit{() } %TODO MANCANO TEST
\end{itemize}

\subsubsection{ToolsPanelPresenterTest}
\textbf{Verifica l'oggetto:} \textit{ToolsPanelPresenter}.\\
\textbf{Descrizione}:\\
Tale verifica è composta da un solo test:
\begin{itemize}
\item \textit{() } %TODO MANCANO TEST
\end{itemize}


\section{Test Coverage}
Tale test è stato effettuato mediante il \underline{\inglese{plugin}} per Eclipse \textit{eclemma} per la parte Java (quindi nel \textit{Model}) e con \textit{JSCoverage} per la parte \textit{JavaScript} (per il \textit{Presenter}) definendo quindi le percentuali di codice che è stato sottoposto a test per i singoli componenti dell'applicazione.\\\\

\begin{center}
\rowcolors{2}{lightblue}{llightblue}\begin{longtable}{p{.30\textwidth}ll}
\toprule Componente & Copertura\\
\midrule
Model & X\%\\
Controller & X\% \\
\bottomrule
\end{longtable}
\end{center}

La percentuale di copertura rilevata è relativamente bassa nella parte \textit{Model}, tale risultato si giustifica dal numero elevato di metodi \textit{getter} e \textit{setter} che non sono stati tutti analizzati tramite test di unità (ma prelevati e verificati a campione), ma solo tramite analisi statica del codice. Per la parte \textit{Presenter} al contrario si può ricondurre il risultato alle numerose funzioni anonime impossibili da testare e invocate soltanto durante l'effettivo funzionamento dell'applicazione.

\section{Test di Sistema}
Nell'attività di validazione che verrà affrontata nella successiva (e ultima) fase del progetto, saranno effettuati i test per accertare le effettive funzionalità offerte dal prodotto \caName.\\ Tali verifiche sono state definite e pianificate già durante l'attività di analisi dei requisiti e sono reperibili nel documento relativo allegato (\textit{analisi\_dei\_requisiti.3.0.pdf}) in cui sono elencati in una pratica tabella che ne descrive (per ognuno) le modalità di verifica e i requisiti ad essi associati. In questo modo verrà assicurato il tracciamento tra requisiti e test, pertanto tutti i requisiti soddisfatti saranno verificati mediante un test specifico.

\section{Test di Utilizzo}
Al fine di verificare che l'interfaccia del prodotto sia più intuitiva e \inglese{user-friendly} possibile, sono stati predisposte delle simulazioni da proporre ad un campione di utenza selezionata (cinque persone con conoscenze informatiche medio-basse).
Essendo test non oggettivi non verranno catalogati come vere e proprie verifiche, ma saranno semplici \inglese{feedback} su come migliorare o confermare la parte visiva di \caName in base ai comportamenti assunti dai \inglese{tester} durante la prova.
\end{document}