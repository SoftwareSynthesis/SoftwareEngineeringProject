% ATTENZIONE!!! 
% Per far funzionare i collegamenti ipertestuali si raccomanda di usare
%	\classname{nomedellaclasse}
% per le classi dello stesso package mentre invece 
%	\hyperref[nomedellaclasse]{\ttfamily{}nomequalificatodellaclasse}
% per le classi che non sono dello stesso package e che hanno il nome completo

% **************************************************
% Macro specifiche per il documento corrente
% **************************************************
% Nome
\newcommand{\docName}{Test e misurazioni}
% Nome file
\newcommand{\docFileName}{test\_e\_misurazioni.1.0.pdf}
% Versione
\newcommand{\docVers}{1.0}
% Data creazione
\newcommand{\creationDate}{2013-02-28}
% Data ultima modifica
\newcommand{\modificationDate}{2013-02-28}
% Stato in {Approvato, Non approvato}
\newcommand{\docState}{Non approvato}
% Uso in {Interno, Esterno}
\newcommand{\docUsage}{Esterno}
% Destinatari da specificare come nome1\\ &nome2\\ ecc.
\newcommand{\docDistributionList}{Prof. Tullio Vardanega\\&Prof. Riccardo Cardin\\&Dott. Gregorio Piccoli\\&Team SoftwareSynthesis}
% Redattori da specificare come nome1\\ &nome2\\ ecc.
\newcommand{\docAuthors}{Stefano Farronato\\&Diego Beraldin\\&Marco Schivo}
% Approvato da
\newcommand{\approvedBy}{Andrea Rizzi}
% Verificatori
\newcommand{\verifiedBy}{Stefano Farronato\\&Marco Schivo}
% Perscorso (relativo o assoluto) che punta alla directory contenente shared/
% come sua sottodirectory (per comodità chiamiamola 'doc root').
\newcommand{\docRoot}{..}
% definire se si vuole l'indice delle tabelle
\def\INDICETABELLE{false}
% definire se si vuole l'indice delle figure
\def\INDICEFIGURE{false}

% importa il preambolo condiviso da tutti i documenti
% shared/preamble.tex
%
% Questo documento contiene la parte del preambolo condivisa e viene pertanto
% richiamato nel 'master' di tutti i documenti di progetto.  Al suo interno
% contiene le inclusioni (e le configurazioni) di tutti i package richiesti per
% la compilazione dei documenti, le macro di carattere generale e la definizione
% degli stili di pagina.

\documentclass[a4paper,10pt]{article}

% **************************************************
% Macro generiche
% **************************************************
\newcommand{\team}{Software Synthesis}                    % chi siamo
\newcommand{\email}{info@softwaresynthesis.org}           % e-mail
\newcommand{\caName}{MyTalk}                              % titolo capitolato
\newcommand{\manager}{SynthesisRequirementManager}        % nome del sistema di tracciamento
\newcommand{\memberdata}[1]{%
  \texttt{\textcolor{RedOrange}{#1}}}                     % attributi di una classe
\newcommand{\method}[1]{\texttt{\textcolor{Emerald}{#1}}} % metodi di una classe
\newcommand{\exception}[1]{%
  \texttt{\textcolor{RedViolet}{#1}}}                     % eccezione
% \newcommand{\handler}[1]{\texttt{\textcolor{Maroon}{#1}}} % per gli event handler
\newcommand{\inglese}[1]{%
  \foreignlanguage{english}{\textit{#1}}}                 % per i testi in lingua inglese
\newcommand{\purpose}{%                                     scopo del prodotto
Con il progetto ``\caName'' si intende un sistema software di comunicazione tra utenti mediante \underline{browser} senza la necessit{\`a} di installazione di \underline{plugin} e/o software esterni. L'utilizzatore avr{\`a} la possibilit{\`a} di interagire con un altro utente tramite una comunicazione audio - audio/video - testuale e, inoltre, ottenere delle statistiche sull'attivit{\`a} in tempo reale.%
}
\newcommand{\glossaryIntro}{%                               introduzione al glossario
Al fine di evitare incomprensioni dovute all'uso di termini tecnici nei documenti, viene redatto e allegato il documento \textit{glossario.4.0.pdf} dove vengono definiti e descritti tutti i termini marcati con una sottolineatura.%
}


% **************************************************
% Codifica e lingua dei documenti
% **************************************************
\usepackage[utf8x]{inputenc}                              % codifica caratteri dei documenti sorgenti
\usepackage[english,italian]{babel}                       % localizzazione ai fini di sillabazione e cross-references
\usepackage[T1]{fontenc}                                  % codifica font di output

% **************************************************
% Definizione geometria della pagina
% **************************************************
\usepackage[a4paper,head=4cm,top=4.5cm,bottom=3cm,left=3cm,right=3cm,bindingoffset=5mm]{geometry}

% *************************************************
% Intestazioni e piè di pagina personalizzati
% *************************************************
\usepackage{fancyhdr}
% stile normale
\fancypagestyle{normal}{
\fancyhead{}                                              % intestazione
\fancyhead[RE,RO]{
\begin{picture}(0,0)
  \put(-410,0){\includegraphics[width=1.02\textwidth]{header_logo}}
  \put(-410,10){\sffamily\large\leftmark}
\end{picture}
\vspace{-4pt}
}
\renewcommand{\headrulewidth}{0pt}                       % riga sotto l'intestazione
\cfoot{}                                                  % piè di pagina
\fancyfoot[RO,LE]{\sffamily
  pag.~\thepage{} di \pageref{LastPage}}                  % a dx nelle pag. dispari e a sx in quelle pari
\fancyfoot[RE,LO]{\sffamily\docFileName{}}
\renewcommand{\footrulewidth}{.4pt}                       % riga sopra il piè di pagina
}
% stile per gli indici
\fancypagestyle{toc}{
\fancyhead{}                                              % intestazione
\fancyhead[RE,RO]{
\begin{picture}(0,0)
  \put(-410,0){\includegraphics[width=1.02\textwidth]{header_logo}}
\end{picture}
}
\renewcommand{\headrule}{}                                % nessuna riga sotto l'intestazione
\cfoot{}                                                  % piè di pagina
\fancyfoot[RO,LE]{\sffamily\thepage{}}                    % a dx nelle pag. dispari e a sx in quelle pari
\fancyfoot[RE,LO]{\sffamily\docFileName{} -- v.\docVers}
\renewcommand{\footrulewidth}{.4pt}                       % riga sopra il piè di pagina
}

\pagestyle{fancy}                                         % premetto: non so usare bene le marche:
\renewcommand{\sectionmark}[1]{\markboth{#1}{#1}}         % se qualcuno ha idee migliori si faccia avanti!

% **************************************************
% Tabelle
% **************************************************
\usepackage{tabularx}                                     % tabelle di larghezza fissa con una o più colonne variabili
\usepackage{multirow}                                     % colonne con colonne che si estendono per più righe
\usepackage{booktabs}                                     % per inserire l'ambiente table e le righe orizz. nelle tabelle
\usepackage{longtable}			                              % tabelle oltre i limiti di pagina

% **************************************************
% Cross-references e collegamenti ipertestuali
% **************************************************
\usepackage[hidelinks]{hyperref}
\hypersetup{%
  colorlinks=false, linktocpage=false, pdfborder={0,0,0}, pdfstartpage=1, pdfstartview=FitV,%
  urlcolor=Cyan, linkcolor=Cyan, citecolor=Black, %pagecolor=Black,%
  pdftitle={\docName}, pdfauthor={\team}, pdfsubject={}, pdfkeywords={},%
  pdfcreator={pdflatex}, pdfproducer={pdflatex with hyperref package}%
}

% **************************************************
% Immagini e grafica
% **************************************************
\usepackage{graphicx}                                     % supporto ad aspetti avanzati delle immagini
\usepackage[table,usenames,dvipsnames]{xcolor}            % tabelle con righe colorate e alternate
\graphicspath{{\docRoot/pics/}}                           % percorso contenente tutti i file immagini
\usepackage{float}                                        % per rendere non flottanti gli ambienti flottanti
\usepackage[italian]{varioref}                            % testo completo riferimenti in italiano

% **************************************************
% Definizioni di colori
% **************************************************
\definecolor{myBlue}{RGB}{1,167,236}
\definecolor{lightblue}{RGB}{213,243,253}%{119,218,247}
\definecolor{llightblue}{RGB}{229,255,255}

% **************************************************
% Altri pacchetti opzionali
% **************************************************     
\usepackage{lastpage}                                     % per sapere il numero totale di pagine
\usepackage{eurosym}                                      % per il simbolo dell'euro usare \EUR{x} dove x è l'importo
\usepackage{ifthen}                                       % permette la scelta di rami condizionali nella compilazione
\usepackage{enumitem}                                     % permette di configurare gli elenchi puntati e numerati


% macro specifiche per il documento corrente
\newcommand{\classsection}[1]{\subsubsection{#1}\label{#1}}
\newcommand{\classname}[1]{\hyperref[#1]{\ttfamily#1}}

% Fine del preambolo e inizio del documento
\begin{document}

% Inclusione della prima pagina
% shared/firstpage.tex
%
% Questo documento definisce il contenuto della prima pagina, che si suppone
% essere uguale in tutti i documenti.  Oltre al logo e al titolo, la prima
% pagina contiene i metadati relativi al documento in cui viene inclusa.


% rimuove intestazioni e piè di pagina
\pagestyle{empty}

\begin{center}

% logo del gruppo
\includegraphics[width=1.5\textwidth]{logo}

\vspace{1in}

% titolo del documento
{\Huge\bfseries \docName}

\vspace{1in}

% tabella riepilogativa
\begin{tabularx}{.7\textwidth}{>{\bfseries\sffamily}l>{\sffamily}l}
\toprule
\multicolumn{2}{>{\sffamily}c}{Informazioni sul documento}\\
\midrule
Nome file:            & \docFileName\\
Versione:             & \docVers\\
Data creazione:       & \creationDate\\
Data ultima modifica: & \modificationDate\\
Stato:                & \docState\\
Uso:                  & \docUsage\\
Redattori:            & \docAuthors\\
Approvato da:         & \approvedBy\\
Verificatori:         & \verifiedBy\\
\bottomrule
\end{tabularx}

\end{center}

\newpage


%---------------------------RUOLI----------------------------
%FASE 1:
%Programmatori: DIEGO, STEFANO, SCHIVO
%FASE 2:
%Programmatori: MENE, TRES, ELENA

%Verificatore: SCHIVO, STEFANO, RIZZI, DIEGO (dobbiamo fare un sacco di test)
%Responsabile finale supremo: RIZZI
%------------------------------------------------------------

% Storico delle modifiche
\section*{Storia delle modifiche}
\begin{center}
\begin{longtable}{lp{.32\textwidth}lll}
\toprule
Versione & Descrizione intervento & Membro & Ruolo & Data\\
\midrule % inserire qui il contenuto della tabella
0.1 & Creazione del documento e stesura delle sezioni ``Introduzione'' e ``Riferimenti''. & Stefano Farronato & Verificatore & 2013-02-28\\
\bottomrule
\end{longtable}
\end{center}
\newpage

% inclusione dell'indice
% shared/toc.tex
%
% Questo file contiene le istruzioni che generano l'indice o gli indici del
% documento (utile nel caso in cui decidessimo di avere anche un indice delle
% tabelle e/o un indice delle figure).

% imposta lo stile di pagina per i titoli definito nel preambolo
\pagestyle{toc}
% imposta i numeri di pagina romani minuscoli
\pagenumbering{roman}

% genera automaticamente l'indice di LaTeX
\tableofcontents

% se è true \INDICETABELLE allora genera l'indice delle tabelle, altrimenti non fa nulla
\ifthenelse{\equal{\INDICETABELLE}{true}}{%
  \clearpage % l'indice delle tabelle, se c'è, deve andare a pagina nuova
  \listoftables
}{}

% se è true |INDICEFIGURE allora genera l'indice delle figure, altrimenti non fa nulla
\ifthenelse{\equal{\INDICEFIGURE}{true}}{%
  \clearpage % l'indice delle figure, se c'è, deve andare a pagina nuova
  \listoffigures
}{}

%in ogni caso occorre andare a pagina nuova dopo gli indici
\clearpage


% Alcuni aggiustamenti per le pagine
\pagenumbering{arabic}
\setcounter{page}{1}
\pagestyle{normal}

% Qui ha inizio il documento vero e proprio...
\newpage

\section{Introduzione}
\subsection{Scopo del prodotto}
\purpose

\subsection{Scopo del documento}
Il documento ha lo scopo di definire e riportare l'esito dei test effettuati sul codice prodotto, attraverso misurazioni quantitative valutate attraverso metriche definite al fine di assicurare che la qualità del prodotto risultante sia coerente con i requisiti redatti e rispetti quanto redatto nel documento \textit{piano\_di\_qualifica.3.0.pdf}.
I test predisposti saranno descritti esaustivamente sia nella forma che nei risultati ottenuti, al fine di renderene più chiara e oggettiva possibile la comprensione.

\subsection{Glossario}
\glossaryIntro
\clearpage

\section{Riferimenti}
\subsection{Normativi}
\begin{itemize}
\item[] \textit{piano\_di\_qualifica.3.0.pdf} allegato;
\item[] \textit{norme\_di\_progetto.3.0.pdf} allegato;
\item[] \textit{specifica\_tecnica.2.0.pdf} allegato;
\item[] \textit{definizione\_di\_prodotto.1.0.pdf} allegato.
\item[] \textit{analisi\_dei\_requsiti.3.0.pdf} allegato;
\end{itemize}

\subsection{Informativi}
\begin{itemize}
\item[] Capitolato d'appalto: \caName{}, v1.0, redatto e rilasciato dal proponente Zucchetti s.r.l. reperibile all'indirizzo \url{http://www.math.unipd.it/~tullio/IS-1/2012/Progetto/C1.pdf};
\item[] testo di consultazione: \textit{Software Engineering (8th edition) Ian Sommerville, Pearson Education | Addison Wesley};
\item[] \textit{glossario.3.0.pdf} allegato.
\end{itemize}
\clearpage

\section{Metriche sul codice}
Prima di tutto sono state calcolate, al fine di quantificare la ``dimensione'' di una componente, due metriche: il \inglese{Number Of Classes} (NOC) e il \inglese{Number Of Method} (NOM).
NOC rappresenta il numero totale delle classi che sono rappresentate all'interno del package, mentre il NOM rappresenta il numero dei metodi conteggiati, sempre per package.

La tabella X riporta le misurazioni su tali metriche per i package del prodotto \caName{}, evidenziandone i più corposi dal punto di vista implementativo.
\\\\
***TABELLA NOC E NOM***
\\\\
Spingendosi più nel dettaglio si è deciso di esporre il \inglese{Total Line Of Code} (TLOC) e il \inglese{Method Lines Of Code} (MLOC).
Come suggeriscono i nomi, TLOC rappresenta il numero totale di linee di codice (effettive, ovvero senza contare quelle non vuote o commentate) all'interno di un'unità di compilazione, MLOC definisce invece il numero totale di linee di codice (sempre effettive) che definiscono i corpi dei metodi. 

Come per NOC e NOM riportiamo la tabella X con le misurazioni su tali metriche relative ai package di \caName{}.
\\\\
***TABELLA TLOC E MLOC***
\\\\
Coerentemente con le Norme di Progetto consultabili nel documento allegato \textit{norme\_di\_progetto.3.0.pdf} si riporta l'indice di complessità ciclomatica (media) per ogni package. Tale misura è direttamente legata al numero di cammini linearmente indipendenti che compongono il grafo di controllo di flusso. Tale indice dovrebbe essere compreso tra 0 e 10 per definire il codice nella norma.
Infine il \inglese(Lack Of Cohesion Of Methods) (LCOM3) indica il liello (medio) di coesione dei metodi, più tale livello risulta basso migliore risulterà essere la progettazione della classe. I valori possono quindi variare tra lo 0, che indica la massima coesione possibile, e 2, che ne identifica il valore minimo.
\\\\
***TABELLA CC E LCOM3***
\\\
\section{Test e risultati}
I test riportati hanno lo scopo di dimostrare in modo oggettivo la bontà e la qualità di quanto prodotto a livello di codice. Tali test servono inoltre a rilevare \inglese{bug} non evidenziati (o non possibili da evidenziare) durante l'analisi statica del codice e vengono effettuati sui package relativi alla parte di \inglese{Model} e \inglese{Presenter} del modello MVP generale.
Le verifiche relative alla parte \inglese{View} saranno al contrario effettuati successivamente a quelli sul \inglese{Model} e sul \inglese{Presenter}, in quanto sarà nesessario avere la certezza che tali componenti risultino stabili e privi (per quanto possibile) di \inglese{bug}, specifichiamo inoltre che tali test saranno effettuati tramite i test di sistema specificati nel documento \textit{anilisi\_dei\_requisiti.3.0.pdf} allegato.

I test per la parte \inglese{Model} sono stati predisposti ed eseguiti usufruendo della libreria JUnit, in quanto questa parte della struttura è implementata mediante codice java.
Per il componente \inglese{Presenter} si è deciso invece di usufruire di QUnit, coerentemente con il linguaggio scelto (JavaScript) che lo implementa.

\subsection{Model}
I test su questa parte del sistema mirano a verificare che la persistenza dei dati all'interno del server sia gestita correttamente dal database.
Per l'esecuzione di questi test si è usata, coerentemente con le norme di progetto, la libreria \textit{JUnit}.
Ogni test è inoltre stato eseguito più volte in modo da avere uno spettro più ampio di misurazioni e produrre una verifica più stabile e oggettiva possibile.

\subsubsection{AESAlgorithmTest.java}
L'obbiettivo di questo test è assicurare che l'algrotimo di cittografia AES utilizzato dal sistema \caName risulti affidabile e riesca a codificare e decodificare una stringa passata.
Tale test si compone di un solo metodo:

\begin{itemize}
\item \textit{testEncodeAndDecode() } effettua la codifica di una stringa passata, effettua successivamente la decodifica della stessa e la controlla con l'originale per verificarne l'uguaglianza.
\end{itemize}
\textbf{Risultato del test:} non sono stati rilevati errori.

\begin{table}
\end{table}

\subsubsection{AuthenticationDataTest.java}
L'obbiettivo di questo test è assicurare l'uguaglianza tra due oggetti che rappresentano le credenziali dell'utente inserite volutamente con gli stessi dati. Successivamente verrà effettuata anche una negazione di tale test per assicurarne l'efficacia.
Tale test si compone di due metodi:
\begin{itemize}
\item \textit{equalsTest() } testa che due oggetti diversi contenenti gli stessi dati passati al metodo restituiscano \textit{true} solo se sono effettivamente uguali.
\item \textit{differentTest()}testa che due oggetti diversi contenenti diversi dati passati al metodo restituiscano \textit{true} solo se sono effettivamente differenti.
\end{itemize}
\textbf{Risultato del test:} non sono stati rilevati errori.

\begin{table}
\end{table}

\subsubsection{AuthenticationModuleTest.java}
*Descrizione*
Tale test si compone di X metodi:
\begin{itemize}
\item \textit{loginTest() }
\item \textit{logoutTest() }
\item \textit{abortTest() }
\end{itemize}
\textbf{Risultato del test:} non sono stati rilevati errori.

\begin{table}
\end{table}

\subsubsection{CredentialLoaderTest.java}
L'obbiettivo di questo test è assicurare che il vettore in cui celle contiente tutte le credenziali d'accesso dell'utente (coppia \textit{username-password}) venga popolato correttamente.
Tale test si compone di un solo metodo metodi:
\begin{itemize}
\item \textit{testloarder() } tale metodo carica nel vettore i campi \textit{username} e \textit{password} e controlla che siano effettivamente stati inseriti e i dati siano coerenti (uguali) con quelli iniziali.
\end{itemize}
\textbf{Risultato del test:} non sono stati rilevati errori.


\subsubsection{1}
*Descrizione*
Tale test si compone di X metodi:
\begin{itemize}
\item \textit{() }
\end{itemize}
Nella tabella seguente sono riportati i dati rilevati in fase di test.



\end{document}