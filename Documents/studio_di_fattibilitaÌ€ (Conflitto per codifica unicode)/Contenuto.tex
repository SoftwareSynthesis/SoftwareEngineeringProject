\begin{abstract}
Con il presente documento, il gruppo Software Synthesis, itende dimostrare la fattibilità della realizzazione del progetto MyTalk. Si cercherà di stabilire quali tecnologie sono necessarie al conseguimento dell'obbiettivo, e le problematiche insite nell'affrontarlo, sia sul piano dei requisiti che sul piano delle tecnologie.
\end{abstract}
\newpage

\section{Descrizione sommaria del capitolato scelto}
Il capitolato C1 (denominato MyTalk) proposto dall'azienda italiana Zucchetti, prevede la creazione di un software di comunicazione audio/video, basato sul progetto (attualmente ancora in fase di sviluppo) WebRTC. Proprio perchè nato di recente, WebRTC è facilemtne soggetto a modifiche (si pensi che in relazione alla data in cui è scritto il presente, l'ultima modifica risale al 15 novembre). Di conseguenza il committente ha precisato i seguenti punti fondamentali:

\begin{itemize}
	\item l'architettura software deve basarsi su un modello elastico e facilmente scalabile in seguito a modifiche del pacchetto WebRTC;
	\item i requisiti opzionali sono modificabili/eliminabili/aggiungibili in corso d'opera, proprio perchè a priori non è evidentemente chiara la loro fattibilità.
\end{itemize}

Senza prendere in considerazione la completezza dei requisiti obbligatori (scopo del documento ''Analisi dei requisiti''), riportiamo di seguito alcuni punti fondamentali da cui è possibile trarre spunti per la determinazione di problemi tecnici e progettuali:

\begin{itemize}
	\item l'applicativo non dovrà richiedere installazione ne di plugin ne di componenti aggiuntivi. Sfrutterà semplicemente il browser Chrome;
	\item di base, vi è un server scritto in java con l'implementazione di webSocket, al quale i client dovranno connettersi, ma che non dovrà partecipare alla comunicazione tra gli utenti;
	\item comunicazione audio;
	\item comunicazione video.
\end{itemize}

\section{Studio del dominio}

\subsection{Dominio tecnologico}
La realizzazione del progetto MyTalk richiede la conoscenza di alcuni strumenti tecnologici obbligatori, senza i quali sarebbe impossibile rispondere alle esigenze del committente. Di seguito riportiamo un elenco delle conoscenze richieste, affiancate da:

\begin{itemize}
	\item uno o più riferimenti alle parti che potrebbero (sempre in termini di anlisi preliminare) richiedere una conoscenza più o meno elevata di tale tecnica/tecnologia (estratte dal capitolato o da considerazioni interne al gruppo);
	\item i vantaggi che si otterrebbero dall'uso di tale tecnologia;
	\item una valutazione delle conoscenze di tale dominio da parte dei membri del gruppo, quantificate su una scala di autovalutazione che va da 1 (nessuna conoscenza) a 5 (Conoscenza totale della tecnologia).
\end{itemize}
Per una descrizione più accurata delle varie tecnologie, rimandiamo al glossario allegato.

\subsubsection{WebRTC}
\begin{description}
	\item{\scshape\bfseries Riferimenti:} ''Il progetto deve essere basato sulla tecnologia WebRTC, parte delle proposte di evoluzione dell'HTML5.'' - estratto dal capitolato C1.

	\item{\scshape\bfseries Vantaggi d'implementazione:} l'applicativo renderebbe quasi nullo il TCO. Esso funzionerebbe solo con l'apposità installazione del brower Google Chrome, e non richiederebbe l'installazione di componenti aggiuntivi o plugin.
	
	\item{\scshape\bfseries Conoscenza attuale:} ogni componente del gruppo si trova ad affontare per la prima volta tale tecnologia, definendo un livello di conoscenza pari a 1. Sara quindi necessario formare il personale sulle caratteristiche della tecnologia.
\end{description}

\subsubsection{Java - webSoket}
\begin{description}
	\item{\scshape\bfseries Riferimenti:} dal capitolato: ''La parte server, necessaria solo nella fase di inizializzazione della chiamata, dovrà essere realizzata in Java e utilizzare il protocollo di comunicazione WebSocket.'' - estratto dal capitolato C1.
	
	\item{\scshape\bfseries Vantaggi d'implementazione:} le webSoket permettono ai browser e server di ''parlare'' in maniera asincrona e senza bisogno dell'interazione dell'utente.

	\item{\scshape\bfseries Conoscenza attuale:} in merito a Java, ogni componente del gruppo ha già un livello di formazione pari a 4. Per quanto conprende la libreria  WebSocket, il gruppo ha una conoscenza basilare (livello 2).  
\end{description}

\subsubsection{HTML5}
\begin{description}
	\item{\scshape\bfseries Riferimenti:} ''Il progetto deve essere basato sulla tecnologia WebRTC, parte delle proposte di evoluzione dell'HTML5.'' - estratto dal capitolato C1.
	
	\item{\scshape\bfseries Vantaggi d'implementazione:} tra le caratteristiche più interessanti, e che offrono spunti per l'implementazione di nuovi requisiti facoltativi, html5 supporta Canvas, che permette di utilizzare JavaScript per creare animazioni e grafica bitmap. Per esempio rende possibile la condivisione di una lavagna grafica.
	
	\item{\scshape\bfseries Conoscenza attuale:} tutti i componenti del gruppo presentano un livello medio di conoscenza pari a 5, sull'utilizzo base di HTML. Per quanto riguarda l'utilizzo di HTML5, in particolar modo delle componenti grafiche offerte (Canvas), solo tre componenti del gruppo hanno un livello pari a 3. Per i restanti cinque sarà necessario organizzare un corso di formazione.
\end{description}

\subsubsection{CSS3}
\begin{description}
	\item{\scshape\bfseries Riferimenti:} non pervenuti nel capitolato. Inteso in relazione ad una considerazione del gruppo, risulta interessante il loro utillo al fine di gestire l'apparato grafico dell'applicativo, in modo sistematico e performante.
	
\item{\scshape\bfseries Vantaggi d'implementazione:} permette una maggiore manuntenibilità del sorgente HTML, perché le istruzioni di formattazione sono accentrate in un solo punto, sono richiamabili in più punti della stessa pagina (si pensi alle classi di elementi HTML) e possono essere condivise anche fra più pagine se si collega ad esse lo stesso CSS\@. Ne deriva inoltre un certo incremento prestazionale dal momento che le pagine web diventano di minore dimensione, e occorre dunque meno tempo per il loro download, mentre il CSS può risiedere nalla cache del browser ed avere pertanto tempi di accesso rapidi e senza comportare ulteriori richieste di trasmissione di dati dal server.
	
	\item{\scshape\bfseries Conoscenza attuale:} cinque componenti del gruppo presentano un livello di formazione medio pari a 3. Per gli altri due componenti si dovrà imporre uno studio autodidattico della tecnologia, al più affiancatato dalla collaborazione con i colleghi più esperti.
\end{description}

\subsubsection{Javascript}
\begin{description} 
	\item{\scshape\bfseries Riferimenti:}
  ''il programma che sarà realizzato non deve essere inteso come una pagina Web ma come un software che per l'occasione utilizzi il linguaggio Javascript e le librerie contenute nel browser'' - estratto dal capitolato C1.

	\item{\scshape\bfseries Vantaggi d'implementazione:} javascript è alla base di altre tecnologie d'interesse, come AJAX e JQuery. Una sua conoscenza risulta pertanto d'obbligo.

	\item{\scshape\bfseries Conoscenza attuale:} quattro componenti del gruppo hanno già un livello di formazione pari a 3. Per gli altri tre sarà necessario organizzare un corso di formazione.
\end{description}

\subsubsection{JQuery}
\begin{description}
	\item{\scshape\bfseries Riferimenti:} non pervenuti nel capitolato. Inteso in relazione ad una considerazione del gruppo, risulta interessante il loro utillo al fine di riutilizzare funzionalità javascript già presenti.
	
	\item{\scshape\bfseries Vantaggi d'implementazione:} permette di semplificare l'attraversamento del codice HTML, la gestione degli eventi, le animazioni e le interazioni Ajax (ossia chiamate asincrone). Il framework rende il codice più sintetico e limita al minimo l’estensione degli oggetti globali per ottenere la massima compatibilità con altre librerie. Da questo principio è nata una libreria in grado di offrire un'ampia gamma di funzionalità, che vanno dalla manipolazione degli stili CSS e degli elementi HTML, agli effetti grafici per passare a comodi metodi per chiamate AJAX cross-browser. Il tutto, appunto, senza toccare nessuno degli oggetti nativi JavaScript.
	
	\item{\scshape\bfseries Conoscenza attuale:} due componenti del gruppo hanno già un livello di formazione pari a 3. Per gli altri cinque, si prevede che sarà sufficiente diffondere del materiale per uno studio autodidattico.
\end{description}

\subsubsection{AJAX}
\begin{description}
	\item{\scshape\bfseries Riferimenti:} ''L'intero sistema deve essere contenuto in un unica pagina Web.'' - estratto dal capitolato C1.

	\item{\scshape\bfseries Vantaggi d'implementazione:} tale tecnologia consente l'aggiornamento dinamico di una pagina web senza esplicito ricaricamento da parte dell'utente. Si osservi che AJAX è asincrono nel senso che i dati extra sono richiesti al server e caricati in background senza interferire con il comportamento della pagina esistente.
	
	\item{\scshape\bfseries Conoscenza attuale:} tutti i componenti del gruppo si trovano ad affrontare tale tecnologia per la prima volta (livello 1). Si suggerisce quindi uno studio collettivo, e la ricerca di script o funzioni che implementino già le funzionalità desiderate.
\end{description}

\subsubsection{Protocolli e funzionalità di Google Chrome}
\begin{description}
	\item{\scshape\bfseries Riferimenti:}
  ''L'estensione dell'HTML5 WebRTC presente nel browser Chrome si propone di rendere semplice la realizzazione di questi programmi e di far sì che le componenti necessarie siano installate praticamente in ogni computer.'' - estratto dal capitolato C1.
  
 	\item{\scshape\bfseries Vantaggi d'implementazione:} Il supporto di Google Chrome a WebRTC è realizzato attraverso una serie di API accessibili ai programmi JavaScript. In particolare citiamo: PeerConnection, MediaStream e DataChannel. Inoltre si ha che la possibilità di registrare gli stream trasmessi e di condividere lo schermo sono fra le funzionalità che è in programma di integrare nel prossimo futuro. Tali quindi ci permettono di prendere in considerazione la possibilità di scegliere alcuni interessanti requisiti facoltativi.
	
	\item{\scshape\bfseries Conoscenza attuale:} tutti i componenti del gruppo si trovano ad affrontare tale tecnologia per la prima volta (livello 1). Si suggerisce quindi uno studio collettivo. Inoltre è consigliabile che il gruppo si tenga aggiornato sul rilasio di ulteriori funzionalità.
\end{description}

\subsubsection{Database relazionali}
\begin{description}
	\item{\scshape\bfseries Riferimenti:} non pervenuti nel capitolato. Inteso in relazione ad una considerazione del gruppo, risulta interessante il loro utillo al fine di creare un sistema per memorizzare gli utenti registrati e le loro impostazioni dell'applicativo, cosi chè abbiano possibilità di impostare le proprie configurazioni (per esempio linguistiche) semplicemente eseguendo un login. Ciò comporterà un'ulteriore riduzione del TCO.
  
 	\item{\scshape\bfseries Vantaggi d'implementazione:} la realizzazione di un DB relazionale è necessaria per la gestione della lista utenti. Per la creazione di tale DB sotto tecnologia MySQL, il team conta di potersi appoggiare al proprio spazio web.
	
	\item{\scshape\bfseries Conoscenza attuale:} tutti i componenti del gruppo hanno già una buona conoscenza dell'argomento. In particolare in merito alla tecnologia MySQL e SQLite, il livello di formazione medio è 4.
\end{description}

\subsection{Dominio applicativo}
Essenziale, per analizzare la fattibilità del progetto, è stabilire se esiste o meno una prova documentata che un qualcosa di simile sia già stato realizzato. Sotto tale tema, si riporta che il primo progetto di comunicazione audio/video tramite WebRTC è stato creato dalla Dubango Telecom, e prende il nome di sipML5. SipML5 è il primo client SIP che si basa su WebRTC scritto totalmente in JavaScript e completamente Open Source. La notizia, evidenziata dal sito:\\\\ http://www.html5today.it/link/sipml5-primo-client-sip-scritto-interamente-html5\\\\ dimostra come il progetto sia fattibile, quanto meno, nella realizzazione dei requisiti obbligatori. Si ribadisce che il progetto sipML5 è open source! Gli sviluppatori ne permettono l'accesso in lettura ai membri "non partecipanti". Per accedere ai sorgenti sarà sufficiente seguire le indicazioni riportate alla pagina:\\\\ http://code.google.com/p/sipml5/source/checkout\\\\ In merito ad altri applicativi software di voip, è possibile trarre spunti per l'implementazione di requisiti opzionali, analizzando programmi come Skype, seguendo il principio ''impariamo dai migliori!''.

\subsection{Conclusioni sul dominio}
Per quanto riguarda il dominio tecnologico:
\begin{itemize}
	\item[•] Dall'analisi del capitolato è emerso l'uso obbligatorio di 5 tecnologie: WebRTC, HTML5, Protocolli di Google Crome, WebSoket, Javascript. Alcuni di tali tecnologie sono nuove per alcuni componenti del gruppo. Tuttavia la ricerca di informazioni riguardanti il loro utilizzo è facilitata dalla mole di riferimenti a tutorial riscontrati nella fase di ''Dominio tecnologico'', e aggiunti nelle rispettive voci del glossario. Molti di questi linguaggi sono inoltre fonte di studio per il corso di studi ''Tecnologie Web''. Ne consegue che il gruppo avrà modo di approfondire il loro utilizzo durante il secondo trimestre dell'anno 2012-2013, affiancando ad uno studio autodidattico, le conoscenze apprese durante il già citato insegnamento.
	\item[•] Alle tecnologie obbligatorie se ne affiancano altre facoltative, il cui vantanggio è garantire un riuso sostanziale di funzionalità rese già disponibili (vedi JQuery).
	\item[•] Il dominio applicativo dimostra che un progetto simile essite già. Il team intende studiare i sorgenti resi disponibili dal gruppo Dubango Telecom, al fine di trarre spunti sull'utilizzo del WebRTC e, se si riscontrasse la possibilità, riutilizzare parte delle funzionalità proposte dal progeto SipML5.
	\item[•] L'ormai consolidato utilizzo di programmi di comunicazione RT, permette ai membri del gruppo di avere una conoscenza basilare sui requisiti utente, desiderabili per un end user.
\end{itemize}

\section{Valutazione del capitolato}

\subsection{Valutazione dei rischi}
\begin{itemize}
	\item Le tecnologie richieste per la creazione dell'applicativo, sono attualmente in via di definizione. La tecnologia WebRTC non è ancora uno standard, e il fatto che le sue API vengano modificate in corso d'opera del progetto, è una certezza. Si pensi che nella data attuale, l'ultima modifica apportata all'architettura risale al 15 novembre. Tuttavia "sembrerebbe" sensato supporre che le modifiche non intaccheranno in termini distruttivi le basi già istanziate. Questa è ovviamente una congettura personale, che non rappresenta necessariamente la realtà dei fatti. Tuttavia come  si evidenzierà nel paragrafo successivo, ciò può comportare anche un punto a nostro favore.
	\item Malgrado siano state rilevate diverse tecnologie d'implementazione, il livello medio di formazione, dei componenti del gruppo, è relativamente basso. Sarà quindi necessario puntare molto sulla formazione del team. Ciò può influenzare negativamente le tempstiche di sviluppo, e inoltre può comportare errori di valutazione sulla fattibilità di alcuni requisiti.
\end{itemize}

\subsection{Vantaggi potenziali}
\begin{itemize}
	\item Estratto dal capitolato: ''In corso d'opera non sarà possibile variare/modificare i requisiti minimi (obbligatori per accettare il prodotto). Sarà invece possibile variare i requisiti opzionali, in quanto saranno i gruppi vincitori dell'appalto a modificarli/eliminarli/aggiungerli''. Di conseguenza, non occorre farsi intimorire dall'applicazione del WebRTC, in quanto se si evidenzia l'impossibilità di soddisfare alcuni requisiti obbligatori, potremmo tranquillamente segnalarlo e abbandonarne lo sviluppo. Va da se, che è una pratica sconsigliata. Se mai è più ragionevole partire con meno requisiti opzionali, e aggiungerne in seguito se si crede che il tutto sia fattibile.
	\item Il software da sviluppare si poggia completamente sul browser Chrome. Di conseguenza si sarà completamente svincolati dalla gestione delle dipendenze dei sistemi operativi sottostanti.
	\item L'applicativo software poggia le sue funzionalità su un ''sito web'' che dovrà fungere da server. Per la creazione di tale sito si potrà sfruttare lo spazio web del gruppo, sia per sperimentare il corretto funzionamento delle parti (in fase di sviluppo), sia per permettere al committente di accedere egli stesso all'ambiente, al fine di verificare lo stato del prodotto.
	\item Riprendendo quanto citato nel paragrafo  "Valutazione rischi", la possibilità che alcune strutture logiche varino durante lo sviluppo del progetto, imporrà al nostro team di creare un architettura sensata e ben curata. Dando quindi un occhio di riguardo al riuso e alla "scalabilità" della struttura, incentiveremo l'utilizzo di pattern appropriati e ricercati. Tali considerazioni dovranno incentivare i membri del gruppo alla ricerca di un architettura il più performante possibile.
	\item Come riportato alla voce "Dominio Applicativo" si evidenzia che il progetto è fattibile. Inoltre è possibile accedere a dei sorgenti che mostrano come operare con la tecnologia WebRTC mediante Javascript.
\end{itemize}

\section{Fattibilità del progetto}
Il progetto MyTalk ha colpito innanzitutto il gruppo, per l'interesse sulle tecnologie d'implementazione. Malgrado il livello di formazione su tali, non sia attualmente sufficiente alla realizzazione del progetto, il team è sicuro di poter apprendere quanto necessario in tempo per iniziare la fase di progettazione. Per tale motivo, e per quanto già riportato in ''conclusioni sul dominio'' e ''valutazione del capitolato'', Il gruppo Software Synthesis ha ritenuto il progetto C1 fattibile, ed è pertanto intenzionato a realizzato nei tempi e nei costi previsti.

\section{Confronto con gli altri capitolati}

\subsection{Capitolato C2}
Nel valutare il capitolato C2, il gruppo si è trovato d'accordo nel ritenere il progetto fattibile. Tuttavia i rischi individuati hanno portato il team a scartare tale progetto in favore del capitolato C1

\subsubsection{Individuazione rischi}
\begin{itemize}
	\item Conoscenze totalmente assenti dei formati JSON, 3DS, OBJ \& MTL.
	\item Dubbi sulla facilità di creare opportuni algoritmi di conversione e ottimizzazione dei formati 3D.
\end{itemize}

\subsection{Capitolato C3}
Nel valutare il capitolato C3, il gruppo si è trovato d'accordo nel ritenere il progetto fattibile. Tuttavia sono sorte alcune ambiguità nell'analisi del capitolato. Alla richiesta pervenuta presso il committente, di organizzare una riunione per chiarire i dubbi, il gruppo si è trovato in difficoltà nel osservare le date proposte per l'incontro. Ritenendo che le tempistiche non fossero favorevoli allo sviluppo di C3, il gruppo Software Synthesis ha deciso di scartare il progetto C3.

\subsubsection{Individuazione rischi}
\begin{itemize}
	\item Dubbia comprensione di alcuni passaggi essenziali, in relazione all'imposibilità di chiarirli in breve tempo, hanno fatto del capitolato C3, un progetto a rischio. Il timore era quello di rendersi conto troppo tardi d certe difficoltà d'implementazione, e dell'impossibilità temporale di considerare un altro progetto in sostituzione a questo.
	\item Scarse conoscenze dei componenti del team in merito al dominio tecnologico.
\end{itemize}

\subsection{Capitolato C4}
Il capitolato C4 è stato valutato, in seguito ad un'analisi preliminare, fattibile ma con un forte rischio di sforare nelle tempistiche di consegna. I rischi evidenziati sono proposti nel punto seguente.

\subsubsection{Individuazione rischi}
\begin{itemize}
	\item Eccessiva complessità di alcuni requisiti obbligatori, in particolare legati alla richiesta di fornire un applicativo funzionante sia su dispositivi mobile che desktop.
	\item Scarse conoscenze dei componenti del team in merito al dominio tecnologico (programmazione per dispositivi mobili).
\end{itemize}

