\subsection{Risorse necessarie e disponibili}
L'utilizzo di risorse umane e tecnologiche è fondamentale per la verifica di qualità del prodotto e dei processi. 
\subsubsection{Umane}
Software Synthesis si è imposta per garantire un elevato standard qualitativo un team di sviluppo comprendente i seguenti ruoli:
\begin{itemize}
\item Responsabile: responsabile della corretta realizzazione del prodotto secondo gli standard e le richieste commissionate, designato all'allocazione corretta delle risorse umane ai rispettivi compiti e stimolarne il coordinamento. 
Controlla inoltre la qualità dei processi interni mediante le attività di verifica da lui predisposte. Infine ha la facoltà di approvare o meno ogni proposta di correzione (migliorativa o di modifica generica) avanzata.\\

\item Amministratore:
\item Verificatore:
\item Programmatore:
\end{itemize}
\subsubsection{Software}
A livello software risulteranno necessari strumenti per permettere l'automatizzazione nell'analisi statica del codice prodotto, ai fini di ricavarne il maggiorn numero possibile di informazioni.\\
Risulteranno altrettanto utili ai fini dei test di unità legati al linguaggio di programmazione scelto (HTML 5 e Java?) dei Frameworks specializzati, e degli strumenti per standardizzare (e automatizzare) i test sul prodotto finale nonchè produtte dei resoconti appropriati sulle eventuali anomalie riscontrate.\\
Infine il team ha deciso di produrre un semplice programma basato su interfaccia Web per il tracciamento e la gestione dei requisiti in modo da rendere più stardard e automatizzata possibile questa fase del progetto.
\subsubsection{Hardware}
Software Synthesis ha a disposizione oltre al materiale personale di ogni componente del team (Computer portatili e fissi) le strutture fisiche ed informatiche messe a disposizione dall'Università degli Studi di Padova per il dipartimento di Matematica Pura ed Applicata, quali laboratori didattici e le aule studio allocate negli stabili della Facoltà stessa.

\section{Qualità}
Al fine di garantire un elevato standard qualitativo, sia per ovvia scelta del team di sviluppo, sia per implicita richiesta da parte del cliente stesso, Software Synthesis ha preso come riferimento lo standard ISO/IEC 9126 le cui specifiche primarie sono consultabili al sito XXX.
 
\subsection{Funzionalità}
Il prodotto MyTalk deve soddisfare nelle sue funzionalità tutti i requisiti individuati nella fase di Analisi, garantendone il funzionamento e l'aderenza alle richieste specifiche del committente. Tutto questo sarà svolto nel modo meno oneroso sia dal punto di vista economico che di sfruttamento delle risorse disponibili.
Per valutare il grado di funzionalità raggiunto dal prodotto si valuterà la quantità di requisiti che sono stati correttamente implementati all'interno del software finale. La soglia minima di soddisfacimento risulta essere l'assoluta copertura dei requisiti obbligatori imposti dal committente, tuttavia è parso chiaro che la fantasia del team in questa specifica area sarà ben valutata.

\subsection{Portabilità}
Il prodotto finale dovrà per vincoli di capitolato essere pianamente usufruibile mediante browser Chrome, prodotto da Google, su tutti i sistemi operativi sui quali questo browser risulta compatibile. 
A dimostrazione di tale soddisfacimento ci si affida alla dimostrazione del superamento della validazione del codice del front-end web (?) e dell'assoluta coerenza con lo standard WebRTC, proposta evolutiva di HTML5.
All'approvazione del superamento di tale requisito si procederà con lo stesso metodo testando browser alternativi a quello imposto al fine di rendere MyTalk usufruibile da un bacino d'utenza più vasto possibile.
\subsection{Usabilità}
Il prodotto deve risultare facile ed intuitivo da parte dell'utenza che dispone di una conoscenza medio-bassa del web e dell'informatica generica.
Utenti che hanno familiarità con programmi per la gestione di chiamate mediante VOIP (Skype, ...) non dovranno trovare alcuna difficoltà o iniziale disorientamento nell'utilizzo di MyTalk.
Data l'aleatorietà di tale qualità di prodotto e la non "oggettività" nella misurazione di tale caratteristica si cercherà semplicemente di raggiungere tale risultato basandosi su esperienze personali o brevi test su specifici utenti selezionati.
\subsection{Affidabilità}
L'applicazione deve riuscire a stabilire e mantenere stabile una comunicazione tra due o più utenti, senza mostrare problemi di natura tecnica se non imputabili alla qualità della linea di cui dispongono gli utenti stessi. Deve dimostrarsi altresì robusta nella sua struttura e facile da ripristinare in caso di errori di varia natura.
Al fine di garantire queste caratteristiche verrà utilizzata come unità di misura la quantità di interazioni tra utenti con esito positivo, tenendo conto di tutti i parametri che concorrono ad una corretta comunicazione (qualità audio, video, messaggi testuali correttamente inviati/ricevuti, etc.).
\subsection{Efficienza}
MyTalk si pone come obbiettivo oltre alla corretta ed appagante esperienza comunicativa, anche di non risultare particolarmente esosa dal punto di vista harware, sia dal punto di vista puramente componentistico dell'unità dalla quale si accede al prodotto, sia dal punto di vista dell'uso di banda a disposizione della rete.
Verranno pertanto monitorate in fase di test sia le percentuali d'utilizzo di memoria e processore della macchina, sia la quantità di kb/s trasmessi e ricevuti durante l'esecuzione del programma. I test verranno eseguiti su varie tipologie di hardware e linee di diverse velocità, al fine di rendere il software usufruibile dalla più vasta fetta d'utenza possibile.
I test risulteranno superati se nei momenti di massimi consumi di risorse il programma riuscirà a garantire un utilizzo fluido e una discreta navigabilità nel web dalla macchina soggetta al test. (ApacheBench citano 7Seeds per prove).

\subsection{Manutenibilità}
Il capitolato specifica esplicitamente che la modifica e la manutenibilità del software sono una caratteristica fondamentale dell'intero progetto, questa necessità nasce dal costante utilizzo di linguaggi non ancora qualificati come standard, pertanto soggetti a continua (ma fortunatamente non radicale) evoluzione.
(Aggiungi-vedi 7Seeds)
\subsection{Altro (?)}