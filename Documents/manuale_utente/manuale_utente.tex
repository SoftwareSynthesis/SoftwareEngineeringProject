% **************************************************
% Macro specifiche per il documento corrente
% **************************************************
% Nome
\newcommand{\docName}{ManualeUtente}
% Nome file
\newcommand{\docFileName}{manuale\_utente.1.0.pdf}
% Versione
\newcommand{\docVers}{1.0}
% Data creazione
\newcommand{\creationDate}{2013-03-07}
% Data ultima modifica
\newcommand{\modificationDate}{2013-03-07}
% Stato in {Approvato, Non approvato}
\newcommand{\docState}{Non approvato}
% Uso in {Interno, Esterno}
\newcommand{\docUsage}{Esterno}
% Destinatari da specificare come nome1\\ &nome2\\ ecc.
\newcommand{\docDistributionList}{Prof. Tullio Vardanega\\ & Prof. Riccardo Cardin}
% Redattori da specificare come nome1\\ &nome2\\ ecc.
\newcommand{\docAuthors}{Elena Zecchinato}
% Approvato da
\newcommand{\approvedBy}{Andrea Rizzi}
% Verificatori
\newcommand{\verifiedBy}{---}
% Perscorso (relativo o assoluto) che punta alla directory contenente shared/
% come sua sottodirectory (per comodità chiamiamola 'doc root').
\newcommand{\docRoot}{..}
% definire se si vuole l'indice delle tabelle
\def\INDICETABELLE{true}
% definire se si vuole l'indice delle figure
\def\INDICEFIGURE{true}

% importa il preambolo condiviso da tutti i documenti
% shared/preamble.tex
%
% Questo documento contiene la parte del preambolo condivisa e viene pertanto
% richiamato nel 'master' di tutti i documenti di progetto.  Al suo interno
% contiene le inclusioni (e le configurazioni) di tutti i package richiesti per
% la compilazione dei documenti, le macro di carattere generale e la definizione
% degli stili di pagina.

\documentclass[a4paper,10pt]{article}

% **************************************************
% Macro generiche
% **************************************************
\newcommand{\team}{Software Synthesis}                    % chi siamo
\newcommand{\email}{info@softwaresynthesis.org}           % e-mail
\newcommand{\caName}{MyTalk}                              % titolo capitolato
\newcommand{\manager}{SynthesisRequirementManager}        % nome del sistema di tracciamento
\newcommand{\memberdata}[1]{%
  \texttt{\textcolor{RedOrange}{#1}}}                     % attributi di una classe
\newcommand{\method}[1]{\texttt{\textcolor{Emerald}{#1}}} % metodi di una classe
\newcommand{\exception}[1]{%
  \texttt{\textcolor{RedViolet}{#1}}}                     % eccezione
% \newcommand{\handler}[1]{\texttt{\textcolor{Maroon}{#1}}} % per gli event handler
\newcommand{\inglese}[1]{%
  \foreignlanguage{english}{\textit{#1}}}                 % per i testi in lingua inglese
\newcommand{\purpose}{%                                     scopo del prodotto
Con il progetto ``\caName'' si intende un sistema software di comunicazione tra utenti mediante \underline{browser} senza la necessit{\`a} di installazione di \underline{plugin} e/o software esterni. L'utilizzatore avr{\`a} la possibilit{\`a} di interagire con un altro utente tramite una comunicazione audio - audio/video - testuale e, inoltre, ottenere delle statistiche sull'attivit{\`a} in tempo reale.%
}
\newcommand{\glossaryIntro}{%                               introduzione al glossario
Al fine di evitare incomprensioni dovute all'uso di termini tecnici nei documenti, viene redatto e allegato il documento \textit{glossario.4.0.pdf} dove vengono definiti e descritti tutti i termini marcati con una sottolineatura.%
}


% **************************************************
% Codifica e lingua dei documenti
% **************************************************
\usepackage[utf8x]{inputenc}                              % codifica caratteri dei documenti sorgenti
\usepackage[english,italian]{babel}                       % localizzazione ai fini di sillabazione e cross-references
\usepackage[T1]{fontenc}                                  % codifica font di output

% **************************************************
% Definizione geometria della pagina
% **************************************************
\usepackage[a4paper,head=4cm,top=4.5cm,bottom=3cm,left=3cm,right=3cm,bindingoffset=5mm]{geometry}

% *************************************************
% Intestazioni e piè di pagina personalizzati
% *************************************************
\usepackage{fancyhdr}
% stile normale
\fancypagestyle{normal}{
\fancyhead{}                                              % intestazione
\fancyhead[RE,RO]{
\begin{picture}(0,0)
  \put(-410,0){\includegraphics[width=1.02\textwidth]{header_logo}}
  \put(-410,10){\sffamily\large\leftmark}
\end{picture}
\vspace{-4pt}
}
\renewcommand{\headrulewidth}{0pt}                       % riga sotto l'intestazione
\cfoot{}                                                  % piè di pagina
\fancyfoot[RO,LE]{\sffamily
  pag.~\thepage{} di \pageref{LastPage}}                  % a dx nelle pag. dispari e a sx in quelle pari
\fancyfoot[RE,LO]{\sffamily\docFileName{}}
\renewcommand{\footrulewidth}{.4pt}                       % riga sopra il piè di pagina
}
% stile per gli indici
\fancypagestyle{toc}{
\fancyhead{}                                              % intestazione
\fancyhead[RE,RO]{
\begin{picture}(0,0)
  \put(-410,0){\includegraphics[width=1.02\textwidth]{header_logo}}
\end{picture}
}
\renewcommand{\headrule}{}                                % nessuna riga sotto l'intestazione
\cfoot{}                                                  % piè di pagina
\fancyfoot[RO,LE]{\sffamily\thepage{}}                    % a dx nelle pag. dispari e a sx in quelle pari
\fancyfoot[RE,LO]{\sffamily\docFileName{} -- v.\docVers}
\renewcommand{\footrulewidth}{.4pt}                       % riga sopra il piè di pagina
}

\pagestyle{fancy}                                         % premetto: non so usare bene le marche:
\renewcommand{\sectionmark}[1]{\markboth{#1}{#1}}         % se qualcuno ha idee migliori si faccia avanti!

% **************************************************
% Tabelle
% **************************************************
\usepackage{tabularx}                                     % tabelle di larghezza fissa con una o più colonne variabili
\usepackage{multirow}                                     % colonne con colonne che si estendono per più righe
\usepackage{booktabs}                                     % per inserire l'ambiente table e le righe orizz. nelle tabelle
\usepackage{longtable}			                              % tabelle oltre i limiti di pagina

% **************************************************
% Cross-references e collegamenti ipertestuali
% **************************************************
\usepackage[hidelinks]{hyperref}
\hypersetup{%
  colorlinks=false, linktocpage=false, pdfborder={0,0,0}, pdfstartpage=1, pdfstartview=FitV,%
  urlcolor=Cyan, linkcolor=Cyan, citecolor=Black, %pagecolor=Black,%
  pdftitle={\docName}, pdfauthor={\team}, pdfsubject={}, pdfkeywords={},%
  pdfcreator={pdflatex}, pdfproducer={pdflatex with hyperref package}%
}

% **************************************************
% Immagini e grafica
% **************************************************
\usepackage{graphicx}                                     % supporto ad aspetti avanzati delle immagini
\usepackage[table,usenames,dvipsnames]{xcolor}            % tabelle con righe colorate e alternate
\graphicspath{{\docRoot/pics/}}                           % percorso contenente tutti i file immagini
\usepackage{float}                                        % per rendere non flottanti gli ambienti flottanti
\usepackage[italian]{varioref}                            % testo completo riferimenti in italiano

% **************************************************
% Definizioni di colori
% **************************************************
\definecolor{myBlue}{RGB}{1,167,236}
\definecolor{lightblue}{RGB}{213,243,253}%{119,218,247}
\definecolor{llightblue}{RGB}{229,255,255}

% **************************************************
% Altri pacchetti opzionali
% **************************************************     
\usepackage{lastpage}                                     % per sapere il numero totale di pagine
\usepackage{eurosym}                                      % per il simbolo dell'euro usare \EUR{x} dove x è l'importo
\usepackage{ifthen}                                       % permette la scelta di rami condizionali nella compilazione
\usepackage{enumitem}                                     % permette di configurare gli elenchi puntati e numerati


\usepackage[italian]{varioref}

% Fine del preambolo e inizio del documento
\begin{document}

% Inclusione della prima pagina
% shared/firstpage.tex
%
% Questo documento definisce il contenuto della prima pagina, che si suppone
% essere uguale in tutti i documenti.  Oltre al logo e al titolo, la prima
% pagina contiene i metadati relativi al documento in cui viene inclusa.


% rimuove intestazioni e piè di pagina
\pagestyle{empty}

\begin{center}

% logo del gruppo
\includegraphics[width=1.5\textwidth]{logo}

\vspace{1in}

% titolo del documento
{\Huge\bfseries \docName}

\vspace{1in}

% tabella riepilogativa
\begin{tabularx}{.7\textwidth}{>{\bfseries\sffamily}l>{\sffamily}l}
\toprule
\multicolumn{2}{>{\sffamily}c}{Informazioni sul documento}\\
\midrule
Nome file:            & \docFileName\\
Versione:             & \docVers\\
Data creazione:       & \creationDate\\
Data ultima modifica: & \modificationDate\\
Stato:                & \docState\\
Uso:                  & \docUsage\\
Redattori:            & \docAuthors\\
Approvato da:         & \approvedBy\\
Verificatori:         & \verifiedBy\\
\bottomrule
\end{tabularx}

\end{center}

\newpage


% Storico delle modifiche
\section*{Storia delle modifiche}
\begin{longtable}{lp{.3\textwidth}lll}
\toprule
Versione & Descrizione intervento & Redattore & Ruolo & Data\\
\midrule % inserire qui il contenuto della tabella

0.1 & Approvazione del documento. & Andrea Rizzi & Responsabile & 2013-03-25\\
0.1 & Verifica lessico ortografica del documento. & Diego Beraldin & Verificatore & 2013-03-24\\
0.1 & Inserito glossario e \textit{screen} delle schermate dell'applicazione.& Stefano Farronato & Verificatore & 2013-03-23\\
0.1 & Stesura istruzioni per l'uso e completamento sezione relativa alle istruzioni per l'accesso. & Stefano Farronato & Verificatore & 2013-03-23\\
0.1 & Stesura istruzioni per l'accesso & Stefano Farronato & Verificatore & 2013-03-22\\
0.1 & Stesura scheletro documento e introduzione. & Elena Zecchinato & Programmatore & 2013-03-22\\
\bottomrule
\end{longtable}
\newpage

% inclusione dell'indice
% shared/toc.tex
%
% Questo file contiene le istruzioni che generano l'indice o gli indici del
% documento (utile nel caso in cui decidessimo di avere anche un indice delle
% tabelle e/o un indice delle figure).

% imposta lo stile di pagina per i titoli definito nel preambolo
\pagestyle{toc}
% imposta i numeri di pagina romani minuscoli
\pagenumbering{roman}

% genera automaticamente l'indice di LaTeX
\tableofcontents

% se è true \INDICETABELLE allora genera l'indice delle tabelle, altrimenti non fa nulla
\ifthenelse{\equal{\INDICETABELLE}{true}}{%
  \clearpage % l'indice delle tabelle, se c'è, deve andare a pagina nuova
  \listoftables
}{}

% se è true |INDICEFIGURE allora genera l'indice delle figure, altrimenti non fa nulla
\ifthenelse{\equal{\INDICEFIGURE}{true}}{%
  \clearpage % l'indice delle figure, se c'è, deve andare a pagina nuova
  \listoffigures
}{}

%in ogni caso occorre andare a pagina nuova dopo gli indici
\clearpage



% Alcuni aggiustamenti per le pagine
\pagenumbering{arabic}
\setcounter{page}{1}
\pagestyle{normal}

% Qui ha inizio il documento vero e proprio...
\section{Introduzione}
\subsection{Scopo del prodotto}
Il prodotto ``MyTalk'' è un sistema software di comunicazione tra utenti mediante \underline{browser} internet, senza la necessità di installazione di \underline{plugin} e/o software esterni. Gli utenti avranno la possibilità di interagire mediante una comunicazione audio - audio/video - testuale e, inoltre, ottenere le statistiche sull'attività in tempo reale.

\subsection{Scopo del Manuale}
Lo scopo del presente manuale è quello di fornire una guida per comprendere e utilizzare al meglio l'applicazione ``\caName''.
Tale documento contiene la descrizione delle principali funzionalità del prodotto ed il modo
per utilizzarle.

\subsection{Destinatario del manuale}
L'applicazione  ``\caName'' è indirizzata agli utenti che desiderano effettuare una comunicazione audio/video/testuale tramite un computer in modo semplice  e senza dover installare nessun software esterno.  

\subsection{Come leggere il manuale}
Il manuale è un integrazione illustrativa e di supporto nell'uso dell'applicazione \caName e ha lo scopo di illustrarne le funzionalità nel modo più chiaro e semplice possibile, al fine di godere della massima gratificazione durante l'esperienza d'uso.

Il manuale è diviso essenzialmente in due parti.
Nella prima vengono descritte le istruzioni d'accesso, i requisiti del sistema,le modalità per riportare eventuali malfunzionamenti, la modalità di registrazione e l'autenticazione all'applicazione stessa. 
Nella seconda vengono descritte invece le varie funzioni che \caName offre: le modalità di comunicazione con gli altri utenti, l'interazione con la rubrica personale,la gestione del proprio \underline{\inglese{account}}, l'utilizzo della segreteria e la visualizzazione dello storico delle chiamate.

%Sarà riportata inoltre una sezione relativa agli errori, che presenta un elenco dei messaggi %d'errore che il sistema può restituire in caso di problemi di natura tecnica e la possibile %risoluzione degli stessi. 

Al termine sarà inoltre riportato un breve glossario contenente i termini utilizzati in questo manuale e che potrebbero causare difficoltà all'utente nella comprensione del testo. Tali termini saranno contrassegnati tramite sottolineatura. 

\section{Istruzioni per l'accesso}

\subsection{Installazione}
L'applicazione non necessita di installazione, è sufficiente aprire il browser, digitare nella barra relativa all'inserimento dell'\underline{\inglese{url}} e digitare
\begin{center}
 \url{http://www.softwaresynthesis.org/MyTalk}
\end{center}
Una volta raggiunto il sito sarà possibile iniziare ad usufruire del prodotto senza ulteriori operazioni.

\subsection{Requisiti del sistema}
\subsubsection{Requisiti software}
Il corretto funzionamento del software  ``\caName'' è assicurato solo con i seguenti browser:
\begin{itemize}
  \item Google Chrome (versione minima 23.0.1271.97m)
  \item Firefox (versione 17.0 minima)
\end{itemize}

Tuttavia, pur utilizzando tali browser non è garantita la possibilità di controllo delle statistiche. Infatti, tale funzionalità attualmente è disponibile solo per la \underline{versione sviluppatori} dei browser, tuttavia molto probabilmente sarà possibile disporre di tali funzionalità offerte dall'applicativo nelle prossime (e imminenti) versioni degli stessi.
L'utente deve inoltre disporre delle seguenti risorse \inglese{hardware}, oltre allo scontato computer dell'utente:
\begin{itemize}
  \item connessione a internet;
  \item microfono e/o  \inglese{webcam}.
 \end{itemize}

\subsubsection{Prerequisiti}
All'utente non sono richieste conoscenze particolari oltre alla comune esperienza di navigazione web.

\subsection{Comunicare il rilevamento di problemi e malfunzionamenti}
Nell'eventualità in cui durante l'utilizzo del software \caName{} si dovessero riscontrare malfunzionamenti o comportamenti inattesi che non corrispondono a quanto descritto nel presente manuale, si invita ad inviare le proprie segnalazioni all'indirizzo:
\begin{center}
  \email{}
\end{center}
specificando le seguenti informazioni:
\begin{itemize}[noitemsep,nolistsep]
  \item[-] il browser utilizzato, la versione dello stesso e il sistema operativo;
  \item[-] una descrizione più dettagliata possibile del problema riscontrato e le circostanze in cui si è verificato;
  \item[-] i messaggi d'errore (e i codici associati) che sono stati presentati.
\end{itemize}

\subsection{Registrazione}
Qual'ora un utente desideri usufruire del servizi offerti dall'applicazione ``\caName'' dovrà effettuare la registrazione al sistema.\\
L'utente può accedere alla pagina registrazione premendo il pulsante \texttt{Registrati} visualizzato in figura:

\begin{figure}[H]
  \includegraphics[width=\textwidth]{manual_register}
\caption{Form di registrazione all'applicazione}\label{fig:register}
\end{figure}

Per effettuare la registrazione l'utente deve inserire obbligatoriamente uno \underline{\inglese{username}}, che corrisponderà alla sua mail, e una \inglese{password}.
L'utente per motivi di sicurezza deve inoltre scegliere una ``domanda segreta'' e fornirne la risposta a tale domanda, in modo che nel caso venga dimenticata la password si riesca a recuperarla, dopo aver superata questo vincolo di sicurezza.
Successivamente si potranno inserire alcuni dati facoltativi e atti al completamento informativo del proprio account: tra questi troviamo il nome, il cognome e l'immagine di profilo.

Una volta inseriti i dati richiesti e premuto il pulsante \texttt{registrati} la registrazione sarà avvenuta e si verrà re-indirizzati alla pagina principale dell'allicazione.

\subsection{Autenticazione}
L'utente registrato al sistema ha la possibilità di autenticarsi al server \caName{} ed accedere al servizio, inserendo il suo \inglese{username} e la \inglese{password} associata.
Dopo un attesa di pochi secondi si verrà ri-indirizzati automaticamente alla schermata principale dell'applicazione, da dove è possibile raggiungere ogni servizio disponibile.

\begin{figure}[H]
  \includegraphics[width=\textwidth]{manual_login}
\caption{Form di login all'applicazione}\label{fig:login}
\end{figure}

\subsection{Recupero Password}
Tramite il pulsante \texttt{Recupera password} è possibile recuperare la stringa scelta per autenticarsi al sistema. Nella schermata visualizzata sarà necessario inserire la risposta alla domanda segreta precedentemente impostata, una volta inserita correttamente verrà inviato al proprio indirizzo \texttt{mail} (che sarà identificato dal nome utente) la password smarrita per effettuare l'autenticazione.

\begin{figure}[H]
  \includegraphics[width=\textwidth]{manual_answer}
\caption{Form di recupero della password}\label{fig:answer}
\end{figure}


\section{Istruzioni per l'uso}
\subsection{Home Screen dell'applicativo }
La \underline{schermata Home} di \caName{} è caratterizzata essenzialmente da tre aree principali.
\\\\*****FIGURA 5 HOME SCREEN*****

\begin{description}
\item \textbf{Area centrale:} sarà visualizzabile ogni chiamata e/o interazione con un utente con cui si desidera comunicare nonché le varie finestre relative alla segreteria e alla visualizzazione del registro chiamate. Quest'area sarà in sostanza il fulcro dell'applicazione stessa, in quanto attraverso di essa vengono erogate tutte le modalità di comunicazione.

\item \textbf{Area laterale destra:} area dedicata ai servizi a disposizione dell'utente. È possibile accedere a tutte le funzionalità collegate alla gestione del proprio account che verranno successivamente espanse nell'area centrale descritta precedentemente:
\begin{description}
\item \textbf{Stato personale}: lo stato (\texttt{disponibile-occupato-non disponibile}) che verrà visualizzato dai contatti che hanno il nostro riferimento nella loro rubrica, è possibile impostarlo mediante il pratico menù a tendina.
\item \textbf{Segreteria}: viene visualizzata nell'area centrale la gestione della segreteria.
\item \textbf{Lista Chiamate}: viene visualizzata nell'area centrale la lista delle chiamate effettuate e ricevute in ordine cronologico (dalla più recente).
\item \textbf{Impostazioni}: vengono visualizzate nell'area centrale le informazioni relative al proprio account, con la possibilità di effettuare eventuali modifiche ai campi dati.
\item \textbf{Gruppi rubrica}: vengono creati/eliminati tramite la \inglese{form} associata i gruppi in cui vengono catalogati i contatti presenti nella rubrica.
\end{description}
\item \textbf{Area laterale sinistra} area dedicata alla rubrica dell'utente.  
\end{description}

\subsection{Chiamata}
Per iniziare una chiamata audio è necessario selezionare dalla rubrica il contatto con cui si desidera comunicare. Nell'area centrale della pagina appariranno i dati dell'utente selezionato, e tramite un \inglese{click} sul pulsante \texttt{chiama} è possibile iniziare la chiamata con il suddetto utente.
\\\\*****FIGURA 6 CHIAMATA(UTENTE DA CHIAMARE)*****
\\\\*****FIGURA 7 CHIAMATA AVVIATA*****
\subsection{Video chiamata}
Per iniziare una chiamata audio video, analogamente alla chiamata audio, è necessario selezionare dalla rubrica il contatto con cui si desidera comunicare. Nell'area centrale della pagina appariranno i dati dell'utente selezionato, e tramite un \inglese{click} sul pulsante 
\texttt{video-chiama} è possibile iniziare la chiamata con il suddetto utente.
\\\\*****FIGURA 8 VIDEO-CHIAMATA(UTENTE DA CHIAMARE)*****
\\\\*****FIGURA 9 VIDEO-CHIAMATA AVVIATA*****
\\È ovviamente possibile estendere una chiamata audio ad una chiamata audio-video: durante una conversazione avviata con un contatto sarà necessario premere il pulsante \texttt{attiva video} per promuovere automaticamente la chiamata senza bisogno di instaurarne una nuova.
\\\\*****FIGURA 10 VIDEO-CHIAMATA(PROMOZIONE)*****
\subsection{Chat}
La chat testuale è disponibile con ogni utente contatto presente in rubrica e non necessita di ulteriori strumenti (microfono-webcam) da parte degli utenti coinvolti per essere avviata. Analogamente alle chiamate ``classiche'', anche per la chat è necessario selezionare dalla rubrica il contatto con cui si desidera comunicare. Nell'area centrale della pagina appariranno i dati dell'utente selezionato, e tramsite un \inglese{click} sul pulsante \texttt{chat} è possibile iniziare la conversazione testuale con il suddetto utente.
\\\\*****FIGURA 11 CHAT(UTENTE DA CHIAMARE)*****
\\\\*****FIGURA 12 CHAT AVVIATA*****
\subsection{Rubrica}
Nella parte sinistra della schermata home di \caName{} è possibile visualizzare e interagire con i contatti presenti in rubrica, nonché eliminarli da essa. 
Viene inoltre resa disponibile una pratica \inglese{form} di ricerca di un contatto, la possibilità di ordinare la lista in base a determinate regole organizzative. e di visualizzare i gruppi in cui sono inseriti (o eliminati) i contatti presenti tramite l'apposito pulsante disponibile nell'area dedicata ai servizi a disposizione dell'utente.
I contatti vengono inseriti in rubrica mediante la ricerca tra gli utenti iscritti al sistema nel pannello strumenti dell'utente tramite il pulsante \texttt{Gestione contatti}.
\\\\*****FIGURA 13 RUBRICA GENERALE*****
\subsection{Impostazioni Account}
La schermata \texttt{Impostazioni} è dedicata alla visualizzazione e alle modifiche dei dati del proprio account. Tali modifiche consentono il cambiamento della \textit{domanda segreta} (e relativa risposta) per il recupero della \textit{password} personale necessaria al \underline{\inglese{login}} al sistema e la modifica dei propri dati anagrafici.
È infine possibile caricare una nuova immagine da abbinare al proprio profilo, sostituendo quella (eventualmente) precedentemente associata.
\\\\*****FIGURA 14 IMPOSTAZIONI UTENTE*****
 
\subsection{Segreteria}
La segreteria personale consente di ricevere e quindi ascoltare (o vedere) i messaggi lasciati dai contatti in rubrica mentre l'utente non risultava disponibile per una conversazione con essi.
Ogni messaggio può essere acceduto mediante il pulsante \texttt{riproduci}, al termine del quale viene automaticamente impostato come ``visualizzato''(è possibile impostare anche manualmente tale stato).
I messaggi vengono salvati nella memoria centrale dell'applicazione \caName{} finché l'utente non desidera eliminare esplicitamente lo stesso mediante il pulsante \texttt{elimina}. Una volta eliminato un messaggio non può essere ripristinato in alcun modo.
\\\\*****FIGURA 15 SEGRETERIA GENERALE*****

\subsection{Gestione contatti}
Permette di aggiungere un contatto alla propria lista utenti presenente in rubrica. Tale inserimento avviene da una ricerca preliminare tra tutti gli utenti prensenti nel sistema. Una volta individuato l'utente da aggiungere, sarà necessario premere il pulsante \texttt{+} presente a destra del nome selezionato, che invierà la richiesta e attenderà la conferma di accettazione da parte dell'utente stesso.
\\\\*****FIGURA 16 GESTIONE CONTATTI*****
\subsection{Storico delle chiamate}
Lo storico presenta una lista delle chiamate effettuate e ricevute dall'utente durante l'utilizzo di \caName{}. Ogni chiamata sarà caratterizzata da informazioni dettagliate su di essa, quali: \texttt{chiamante,destinatario,data(gg/mm/aaa), orario d'inizio, orario di fine, durata chiamata}.
È possibile eliminare una singola chiamata dallo storico mediante l'apposito pulsante rappresentato con un \textit{cestino}.
\\\\*****FIGURA 17 STORICO CHIAMATE*****


%\section{Errori}
%Di seguito viene riportata una tabella relativa agli errori e le relative soluzioni che il prodotto \caName{} può presentare durante il suo funzionamento.

%\begin{center}
%\rowcolors{2}{lightblue}{llightblue}\begin{longtable}{llp{.6\textwidth}}
%\toprule Codice Errore & Nome Errore  & Possibili soluzioni\\
%\midrule
%\bottomrule
%\end{longtable}
%\end{center}

\section{Glossario}
\begin{description}
\item\textbf{Account:} termine usato per identificare le credenziali d'accesso di un utente.È composto dal \textit{nome utente} e la \textit{password} impostate in fase di registrazione, nonché tutte le informazioni relative alle misure di sicurezza per il cambio password e i propri dati anagrafici.
\item\textbf{Browser:} è un programma che consente di usufruire dei servizi di connettività in rete e di navigare sul \inglese{World Wide Web} (internet ndr.), permettendo di visualizzare i contenuti delle pagine dei siti web, specificandone l'\inglese{url} e di interagire con essi.
\item\textbf{Url:} indirizzo identificativo di ogni sito web, in modo da raggiungerlo e visualizzarlo. La barra di inserimento di tale indirizzo generalmente è posizionata nella parte superiore del browser.
\item\textbf{Login:} termine inglese per definire ``autenticazione''.
\item\textbf{Plugin:} è un programma non autonomo che interagisce con un altro programma per ampliarne le funzioni o offrire più servizi.
\item\textbf{Schermata Home:} scheramata iniziale e principale, da essa è possibile accedere tramite uno o più passi a tutte le funzioni messe a disposizione dal prodotto.
\item\textbf{Username:} termine inglese per definire ``nome utente''. Definisce il nome con il quale l'utente viene riconosciuto da un computer, da un programma o da un \inglese{server}. In altre parole, esso è un identificativo che, insieme alla password, rappresenta le credenziali per accedere alle risorse o in un sistema.
\item\textbf{Versione sviluppatori:} versione di un programma o di un generico prodotto non ancora disponibile per l'utenza finale (\inglese{consumer}) in quanto non ancora sufficientemente  stabile o priva di malfunzionamenti. Alla correzione degli stessi generalmente viene rilasciata la versione pubblica.
\end{description}

\end{document}