\documentclass[a4paper,10pt,openright]{article}

\usepackage[utf8x]{inputenc}
\usepackage[english,italian]{babel}
\usepackage[T1]{fontenc} 
\usepackage{graphicx}
\usepackage{booktabs}
\graphicspath{{../pics/}}

\begin{document}

% logo del gruppo
\includegraphics[width=1.5\textwidth]{logo}

\begin{center}

\begin{Large}
\hspace{1.2cm}{Verbale d'incontro 2012/12/12}
\newline
\end{Large}

\begin{small}
	Zecchinato Elena
\end{small}

\noindent\rule{\textwidth}{0.4pt}
\newline

\begin{tabular}{ll}
\toprule
\multicolumn{2}{c}{\sffamily Informazioni sull'incontro}\\
\midrule
Data & 2012/12/12 \\
Ora & 9:30 \\
Luogo & Padova, Padova, via Paolotti, Aula Luf1 \\
Partecipanti & Beraldin Diego \\ & Farronato Stefano \\ & Meneghinello Andrea \\ & Rizzi Andrea \\ & Schivo Marco \\ & Tresoldi Riccardo \\ & Zecchinato Elena\\
\bottomrule
\end{tabular}

\end{center}

\section*{Oggetto}
\begin{itemize}
	\item Emersa la necessità di aggiungere nuovi requisiti facoltativi, inerenti la possibilità di testare l'applicativo su altri browser, e una considerazione inerente la privacy per il requisito di registrazione chiamata.
	\item Ripasso di gruppo sulla formulazione dei casi d'uso.
	\item Identificazione dei casi d'uso, previa rilettura dei requisiti identificati.
	\item Assegnazione dei casi d'uso agli amministratori delegati.
\end{itemize}
\noindent\rule{\textwidth}{0.4pt}

\end{document}