\documentclass[a4paper,10pt,openright]{article}

\usepackage[utf8x]{inputenc}
\usepackage[english,italian]{babel}
\usepackage[T1]{fontenc} 
\usepackage{graphicx}
\usepackage{booktabs}
\graphicspath{{../pics/}}

\begin{document}

% logo del gruppo
\includegraphics[width=1.5\textwidth]{logo}

\begin{center}

\begin{Large}
\hspace{1.2cm}{Verbale d'incontro 2012/12/03}
\newline
\end{Large}

\begin{small}
	Zecchinato Elena
\end{small}

\noindent\rule{\textwidth}{0.4pt}
\newline

\begin{tabular}{ll}
\toprule
\multicolumn{2}{c}{\sffamily Informazioni sull'incontro}\\
\midrule
Data & 2012/12/03 \\
Ora & 10:30 \\
Luogo & Padova, Padova, via Paolotti, Aula Luf1 \\
Partecipanti & Beraldin Diego \\ & Farronato Stefano \\ & Meneghinello Andrea \\ & Rizzi Andrea \\ & Schivo Marco \\ & Tresoldi Riccardo \\ & Zecchinato Elena\\
\bottomrule
\end{tabular}

\end{center}

\section*{Oggetto}
\begin{itemize}
	\item Dopo un'attenta discussione, e un'analisi dello studio di fattibilità del capitolato C1, e un'analisi dei domini applicativi degli altri capitolati, il gruppo ha deciso di partecipare alal gara d'appalto per la realizzazione dell'applicativo MyTalk (capitolato C1).
	\item Creazione del repository privato su GitHub, e breve introduzione al gruppo del suo sistema da parte di Andrea Meneghinello.
	\item {E'} stato analizzato rivaloutato il dominio tecnologico proprosto nello studio di fattibilità, aggiungendo anche considerazioni sull'uso di AJAX e JQuery.
	\item Sono state considerati alcuni aspetti essenziali per la stesura dei documenti, per la gestione dell'ambiente di lavoro ele impostazioni degli strumenti software. Successiva stesura di una bozza per le norme di progetto.
	\item Test d'inserimento dei requisiti sul sistema informatico ''Tracciamenti RQ''.
	\item Test d'inserimento dei casi d'uso sul sistema informatico ''Tracciamenti RQ''.
	\item Test di stampa dei requisiti sul sistema informatico ''Tracciamenti RQ''.
\end{itemize}
\noindent\rule{\textwidth}{0.4pt}

\end{document}