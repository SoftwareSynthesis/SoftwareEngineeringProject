\documentclass[a4paper,10pt,openright]{article}

\usepackage[utf8x]{inputenc}
\usepackage[english,italian]{babel}
\usepackage[T1]{fontenc} 
\usepackage{graphicx}
\usepackage{booktabs}
\graphicspath{{../pics/}}

\begin{document}

% logo del gruppo
\includegraphics[width=1.5\textwidth]{logo}

\begin{center}

\begin{Large}
\hspace{1.2cm}{Verbale d'incontro 2013/01/15}
\newline
\end{Large}

\begin{small}
	Andrea Meneghinello
\end{small}

\noindent\rule{\textwidth}{0.4pt}
\newline

\begin{tabular}{ll}
\toprule
\multicolumn{2}{c}{\sffamily Informazioni sull'incontro}\\
\midrule
Data & 2013/01/15 \\
Ora & 11:30 \\
Luogo & Padova, Padova, via Paolotti, Aula Luf1 \\
Partecipanti & Andrea Meneghinello \\ & Andrea Rizzi \\& Diego Beraldin \\& Elena Zecchinato\\   & Marco Schivo \\ & Riccardo Tresoldi \\ & Stefano Farronato \\
\bottomrule
\end{tabular}

\end{center}

\section*{{\textsc{Oggetto:} \\Considerazioni generali su esito RR}}
Il team si riunice per analizzare l'esito della revisione dei requisiti svolta in data 2013-01-10 e le correzioni proposte dal docente in merito alla documentazione consegnata.\\
Nell'analisi dell'elaborato del docente sono state analizzate e sollevate alcune perplessità in merito ai commenti:
\begin{itemize}
\item Alcuni use case evidenziati come errati o da sottoporre a modifica non vengono ritenuti tali da parte del team, o quantomeno non sono state colte a pieno le mancanze in tali diagrammi. Verrà pertanto portata l'attenzione al docente della perplessità e sottoposta a conseguenti chiarimenti.
\item Il documento inerente allo studio di fattibilità risulta insufficiente in una sua parte, risulta doverosa una spiegazione più dettagliata in merito.
\item Il documento inerente al piano di qualifica risulta errato nella sua interpretazione, ciò denota che non è stata colta l'essenza da parte del team del documento stesso. Constatando che il docente stesso ha ammesso che l'argomento è stato mal compreso dalla maggior parte dei gruppi di lavoro ha deciso di dedicare del tempo alla chiarificazione dello stesso. Se eventualmente nel correggere l'elaborato dovessero sorgere ulteriori dubbi si esporranno al docente che ha concesso la relativa disponibilità.
\end{itemize}
Il team ha terminato l'incontro facendo il punto della situazione valutando la di ogni componente per le settimane successive e le conoscenze in merito agli argomenti teorici necessari per proseguire con la progettazione architetturale, riscontrando una già buona preparazione da parte di alcuni membri che si rendono disponibili ad assistere i colleghi in caso di eventuali difficoltà.
\noindent\rule{\textwidth}{0.4pt}
\end{document}