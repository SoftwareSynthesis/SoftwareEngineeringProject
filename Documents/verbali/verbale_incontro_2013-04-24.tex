\documentclass[a4paper,10pt,openright]{article}

\usepackage[utf8x]{inputenc}
\usepackage[english,italian]{babel}
\usepackage[T1]{fontenc} 
\usepackage{graphicx}
\usepackage{booktabs}
\graphicspath{{../pics/}}

\begin{document}

% logo del gruppo
\includegraphics[width=1.5\textwidth]{logo}

\begin{center}

\begin{Large}
\hspace{1.2cm}{Verbale d'incontro 2013/04/24}
\newline
\end{Large}

\begin{small}
	Marco Schivo
\end{small}

\noindent\rule{\textwidth}{0.4pt}
\newline

\begin{tabular}{ll}
\toprule
\multicolumn{2}{c}{\sffamily Informazioni sull'incontro}\\
\midrule
Data & 2013/04/24 \\
Ora & 14:30 \\
Luogo & Padova, Padova, via Paolotti, Aula Luf1 \\
Partecipanti & Beraldin Diego \\ & Farronato Stefano \\ & Meneghinello Andrea \\ & Rizzi Andrea \\ & Schivo Marco \\ & Tresoldi Riccardo \\ & Zecchinato Elena\\
\bottomrule
\end{tabular}

\end{center}

\section*{{\textsc{Oggetto:} \\Modifiche richieste a seguito del colloquio via mail}}
Il team si riunisce per analizzare le richieste di modifica e i commenti derivanti dallo scambio di mail con il professor Riccardo Cardin avute nei giorni precedenti all'incontro, allo scopo di valutare le risorse richieste e le tempistiche relative al compimento delle stesse.
Vengono inoltre affrontate e discusse alcune perplessità e richieste di modifica in sezioni del server e la situazione relativa ai requisiti soddisfatti e la pianificazione degli stessi da soddisfare per la revisione d'accettazione.

\begin{itemize}
\item I requisiti qualitativi elencati non risultano corretti nel loro essere, dopo una chiarificazione tramite un esempio sugli stesso è emerso come descriverli correttamente.
Sarà dunque compito di due componenti del gruppo (\textit{Stefano Farronato} e \textit{Andrea Rizzi} ) di classificare i requisiti qualitativi e, come richiesto, resi ancor più atomici i requisiti già classificati.

\item Esposti e discussi dubbi sulla sezione relativa ai DAO del server, rivedendo di fatto la forma e la realizzazione degli stessi per ottimizzarli a livello di codice e funzionalità. 

\item Discussa e valutata l'implementazione del \textit{design pattern Front Controller}, valutandone i vantaggi e le modifiche necessarie per implementarlo nel modo più efficiente.

\item Analizzato e discusso il \textit{pattern} MVP, supportati dal \textit{link} suggerito dal docente e cercando di correggere quanto realizzato in modo da aderire, diversamente da quanto presentato in sede di revisione di qualifica, correttamente al \textit{design pattern}.

\end{itemize}


Il team ha terminato l'incontro facendo il punto della situazione valutando la disponibilità di ogni componente per le settimane successive. Emerge che il lavoro richiesto e la disponibilità produttiva risultano sufficienti alla realizzazione progettuale definitiva.\\
\noindent\rule{\textwidth}{0.4pt}
\end{document}