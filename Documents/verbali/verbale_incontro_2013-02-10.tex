\documentclass[a4paper,10pt,openright]{article}

\usepackage[utf8x]{inputenc}
\usepackage[english,italian]{babel}
\usepackage[T1]{fontenc} 
\usepackage{graphicx}
\usepackage{booktabs}
\graphicspath{{../pics/}}

\begin{document}

% logo del gruppo
\includegraphics[width=1.5\textwidth]{logo}

\begin{center}

\begin{Large}
\hspace{1.2cm}{Verbale d'incontro 2013/02/10}
\newline
\end{Large}

\begin{small}
	Diego Beraldin
\end{small}

\noindent\rule{\textwidth}{0.4pt}
\newline

\begin{tabular}{ll}
\toprule
\multicolumn{2}{c}{\sffamily Informazioni sull'incontro}\\
\midrule
Data & 2013/02/10 \\
Ora & 11:30 \\
Luogo & Padova, via Paolotti, Aula Luf1 \\
Partecipanti & Andrea Meneghinello \\ & Andrea Rizzi \\& Diego Beraldin \\& Elena Zecchinato\\   & Marco Schivo \\ & Riccardo Tresoldi \\ & Stefano Farronato \\
\bottomrule
\end{tabular}

\end{center}

\section*{{\textsc{Oggetto:} \\Considerazioni generali su esito RP}}
Il team si riunisce per analizzare l'esito della revisione di progettazione svolta in data 2013-02-11 e le correzioni proposte dal docente in merito alla documentazione consegnata.\\
Dopo un attenta analisi dell'elaborato prodotto dal docente sono state sollevate alcune perplessità in merito:
\begin{itemize}
\item E' nuovamente stato segnalato l'errore relativo all'approvazione di un documento fornita dal suo stesso autore, violando il principio che regola il conflitto di interessi interno al team. Il team è d'accordo che tale discrepanza nasce dalla persistente rotazione dei ruoli interni dei membri stessi, questa necessità (non presente nel mondo reale) fa si che la maggior parte dei documenti sia modificato nelle varie fasi di progetta da tutti i componenti del team, che si ritrovano obbligatoriamente ad essere verificatori del documento che essi stessi hanno (in parte) redatto. Tale situazione necessita un chiarimento con il docente per evitare che venga recepita come una discrepanza dovuta a qualche errore interno, ma dovuta appunto ad una specifica richiesta dei vincoli progettuali.
\item Discussioni in merito alla presentazione generale avvenuta in fase di revisione di progettazione. Il team si trova d'accordo nell'affermare che le piccole ingenuità d'esposizione e di descrizione delle parti dell'architettura necessitano uno studio più approfondito a livello espositivo della stessa, al fine di renderla più chiara possibile in relazione al tempo disponibile.
\item Il documento \textit{norme\_di\_progetto.2.0.pdf} espone contenuti non ancora sufficientemente procedurali, verrà richiesta una delucidazione in merito qualora non si giunga ad una soluzione autonoma.
\item Il documento \textit{specifica\_tecnica.1.0.pdf} pur contenendo tutte le informazioni necessarie, non si spinge ad un livello di dettaglio opportuno inoltre viene sottolineata la poca chiarezza nell'adozione dei vari design pattern. Il team ha deciso di impegnarsi per una profonda revisione di tale documento, confrontandosi ulteriormente sull'architettura stessa del prodotto studiato. Molto probabilmente saranno necessario confrontarsi ulteriormente con il docente per chiarire i punti segnalati e/o segnalare ulteriori dubbi emersi durante la revisione del documento stesso.
\end{itemize}
Il team ha terminato l'incontro facendo il punto della situazione valutando la disponibilità di ogni componente per le settimane successive e le conoscenze in merito agli argomenti teorici e pratici per la prosecuzione della progettazione, la futura attività di codifica e la predisposizione ai test sul prodotto. I membri del team si impegnano pertanto ad approfondire le tematiche e gli strumenti che si renderanno necessari nel corso dei prossimi giorni.
\noindent\rule{\textwidth}{0.4pt}
\end{document}