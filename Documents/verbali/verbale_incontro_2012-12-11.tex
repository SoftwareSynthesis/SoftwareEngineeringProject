\documentclass[a4paper,10pt,openright]{article}

\usepackage[utf8x]{inputenc}
\usepackage[english,italian]{babel}
\usepackage[T1]{fontenc} 
\usepackage{graphicx}
\usepackage{booktabs}
\graphicspath{{../pics/}}

\begin{document}

% logo del gruppo
\includegraphics[width=1.5\textwidth]{logo}

\begin{center}

\begin{Large}
\hspace{1.1cm}{Verbale d'incontro 2012/12/11}
\newline
\end{Large}

\begin{small}
Elena Zecchinato
\end{small}

\noindent\rule{\textwidth}{0.4pt}
\newline

\begin{tabular}{ll}
\toprule
\multicolumn{2}{c}{\sffamily Informazioni sull'incontro}\\
\midrule
Data & 2012/12/11 \\
Ora & 9:30 \\
Luogo & Padova, Padova, via Paolotti, Aula Luf1 \\
Partecipanti & Beraldin Diego \\ & Farronato Stefano \\ & Meneghinello Andrea \\ & Rizzi Andrea \\ & Schivo Marco \\ & Tresoldi Riccardo \\ & Zecchinato Elena\\
\bottomrule
\end{tabular}

\end{center}

\section*{Oggetto}
In seguio all'incontro con il proponente del capitolato C1, Gregorio Piccoli, tenutosi presso l'ufficio dell'azienda Zucchetti (Padova, Via Giuseppe Cittadella n°7), il gruppo ha ritenuto necessario riunirsi in tal data, al fine di mettere per iscritto i punti emersi:
\begin{itemize}
	\item {È} emersa la necessità del web server Tomcat per la gestione del sistema di connessioni del progetto.
	\item Inseguito all'incontro con il proponente è emerso che l'applicativo può strutturarsi su più pagine web. Tuttavia il gruppo non dovrà abusare di tale possibilità. L'idea sarebbe che al più si può aprire una pagina successivamente allo stabilirsi di una connessione tra client, per ospitarvi la chat e/o comunicazione audio/video.
	\item {È} emersa la possibilità (sotto consiglio del committente) di creare un proptotipo ''usa e getta'', per stabilire una comunicazione point-to-point tra 2 utenti, mediante la conoscenza dei corrispettivi indirizzi ip.
	\item {È} emersa la possibilità di creare una rete di server comunicanti tra loro, al fine di ridurre il carico di lavoro su di un unica linea ed ottimizzare cosi il servizio proposto.
	\item {È} emersa la possibilitaà di fornire con l'applicativo, la possibilità di eseguire una condivisione di schermo. Si è vagliata anche la possibilità di condividere documenti pdf mediante la libreria pdf.js.
	\item Revisione dei requisiti con aggiunta di nuovi requisiti opzionali.
\end{itemize}
\noindent\rule{\textwidth}{0.4pt}

\end{document}