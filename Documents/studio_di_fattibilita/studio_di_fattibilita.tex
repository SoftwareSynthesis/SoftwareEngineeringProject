% **************************************************
% Macro specifiche per il documento corrente
% **************************************************
% Nome
\newcommand{\docName}{Studio di fattibilit\'a}
% Nome file
\newcommand{\docFileName}{studio\_di\_fattibilita.1.0.pdf}
% Versione
\newcommand{\docVers}{0.8}
% Data creazione
\newcommand{\creationDate}{2012-11-29}
% Data ultima modifica
\newcommand{\modificationDate}{2012-12-01}
% Stato in {Approvato, Non approvato}
\newcommand{\docState}{Approvato}
% Uso in {Interno, Esterno}
\newcommand{\docUsage}{Interno}
% Redattori da specificare come nome1\\ &nome2\\ ecc.
\newcommand{\docAuthors}{Andrea Rizzi\\ &Riccardo Tresoldi\\}
% Approvato da
\newcommand{\approvedBy}{Stefano Farronato}
% Verificatori
\newcommand{\verifiedBy}{Marco Schivo}
% Perscorso (relativo o assoluto) che punta alla directory contenente shared/
% come sua sottodirectory (per comodità chiamiamola 'doc root').
\newcommand{\docRoot}{..}
% definire se si vuole l'indice delle tabelle
\def\INDICETABELLE{false}
% definire se si vuole l'indice delle figure
\def\INDICEFIGURE{false}

% importa il preambolo condiviso da tutti i documenti
% shared/preamble.tex
%
% Questo documento contiene la parte del preambolo condivisa e viene pertanto
% richiamato nel 'master' di tutti i documenti di progetto.  Al suo interno
% contiene le inclusioni (e le configurazioni) di tutti i package richiesti per
% la compilazione dei documenti, le macro di carattere generale e la definizione
% degli stili di pagina.

\documentclass[a4paper,10pt]{article}

% **************************************************
% Macro generiche
% **************************************************
\newcommand{\team}{Software Synthesis}                    % chi siamo
\newcommand{\email}{info@softwaresynthesis.org}           % e-mail
\newcommand{\caName}{MyTalk}                              % titolo capitolato
\newcommand{\manager}{SynthesisRequirementManager}        % nome del sistema di tracciamento
\newcommand{\memberdata}[1]{%
  \texttt{\textcolor{RedOrange}{#1}}}                     % attributi di una classe
\newcommand{\method}[1]{\texttt{\textcolor{Emerald}{#1}}} % metodi di una classe
\newcommand{\exception}[1]{%
  \texttt{\textcolor{RedViolet}{#1}}}                     % eccezione
% \newcommand{\handler}[1]{\texttt{\textcolor{Maroon}{#1}}} % per gli event handler
\newcommand{\inglese}[1]{%
  \foreignlanguage{english}{\textit{#1}}}                 % per i testi in lingua inglese
\newcommand{\purpose}{%                                     scopo del prodotto
Con il progetto ``\caName'' si intende un sistema software di comunicazione tra utenti mediante \underline{browser} senza la necessit{\`a} di installazione di \underline{plugin} e/o software esterni. L'utilizzatore avr{\`a} la possibilit{\`a} di interagire con un altro utente tramite una comunicazione audio - audio/video - testuale e, inoltre, ottenere delle statistiche sull'attivit{\`a} in tempo reale.%
}
\newcommand{\glossaryIntro}{%                               introduzione al glossario
Al fine di evitare incomprensioni dovute all'uso di termini tecnici nei documenti, viene redatto e allegato il documento \textit{glossario.4.0.pdf} dove vengono definiti e descritti tutti i termini marcati con una sottolineatura.%
}


% **************************************************
% Codifica e lingua dei documenti
% **************************************************
\usepackage[utf8x]{inputenc}                              % codifica caratteri dei documenti sorgenti
\usepackage[english,italian]{babel}                       % localizzazione ai fini di sillabazione e cross-references
\usepackage[T1]{fontenc}                                  % codifica font di output

% **************************************************
% Definizione geometria della pagina
% **************************************************
\usepackage[a4paper,head=4cm,top=4.5cm,bottom=3cm,left=3cm,right=3cm,bindingoffset=5mm]{geometry}

% *************************************************
% Intestazioni e piè di pagina personalizzati
% *************************************************
\usepackage{fancyhdr}
% stile normale
\fancypagestyle{normal}{
\fancyhead{}                                              % intestazione
\fancyhead[RE,RO]{
\begin{picture}(0,0)
  \put(-410,0){\includegraphics[width=1.02\textwidth]{header_logo}}
  \put(-410,10){\sffamily\large\leftmark}
\end{picture}
\vspace{-4pt}
}
\renewcommand{\headrulewidth}{0pt}                       % riga sotto l'intestazione
\cfoot{}                                                  % piè di pagina
\fancyfoot[RO,LE]{\sffamily
  pag.~\thepage{} di \pageref{LastPage}}                  % a dx nelle pag. dispari e a sx in quelle pari
\fancyfoot[RE,LO]{\sffamily\docFileName{}}
\renewcommand{\footrulewidth}{.4pt}                       % riga sopra il piè di pagina
}
% stile per gli indici
\fancypagestyle{toc}{
\fancyhead{}                                              % intestazione
\fancyhead[RE,RO]{
\begin{picture}(0,0)
  \put(-410,0){\includegraphics[width=1.02\textwidth]{header_logo}}
\end{picture}
}
\renewcommand{\headrule}{}                                % nessuna riga sotto l'intestazione
\cfoot{}                                                  % piè di pagina
\fancyfoot[RO,LE]{\sffamily\thepage{}}                    % a dx nelle pag. dispari e a sx in quelle pari
\fancyfoot[RE,LO]{\sffamily\docFileName{} -- v.\docVers}
\renewcommand{\footrulewidth}{.4pt}                       % riga sopra il piè di pagina
}

\pagestyle{fancy}                                         % premetto: non so usare bene le marche:
\renewcommand{\sectionmark}[1]{\markboth{#1}{#1}}         % se qualcuno ha idee migliori si faccia avanti!

% **************************************************
% Tabelle
% **************************************************
\usepackage{tabularx}                                     % tabelle di larghezza fissa con una o più colonne variabili
\usepackage{multirow}                                     % colonne con colonne che si estendono per più righe
\usepackage{booktabs}                                     % per inserire l'ambiente table e le righe orizz. nelle tabelle
\usepackage{longtable}			                              % tabelle oltre i limiti di pagina

% **************************************************
% Cross-references e collegamenti ipertestuali
% **************************************************
\usepackage[hidelinks]{hyperref}
\hypersetup{%
  colorlinks=false, linktocpage=false, pdfborder={0,0,0}, pdfstartpage=1, pdfstartview=FitV,%
  urlcolor=Cyan, linkcolor=Cyan, citecolor=Black, %pagecolor=Black,%
  pdftitle={\docName}, pdfauthor={\team}, pdfsubject={}, pdfkeywords={},%
  pdfcreator={pdflatex}, pdfproducer={pdflatex with hyperref package}%
}

% **************************************************
% Immagini e grafica
% **************************************************
\usepackage{graphicx}                                     % supporto ad aspetti avanzati delle immagini
\usepackage[table,usenames,dvipsnames]{xcolor}            % tabelle con righe colorate e alternate
\graphicspath{{\docRoot/pics/}}                           % percorso contenente tutti i file immagini
\usepackage{float}                                        % per rendere non flottanti gli ambienti flottanti
\usepackage[italian]{varioref}                            % testo completo riferimenti in italiano

% **************************************************
% Definizioni di colori
% **************************************************
\definecolor{myBlue}{RGB}{1,167,236}
\definecolor{lightblue}{RGB}{213,243,253}%{119,218,247}
\definecolor{llightblue}{RGB}{229,255,255}

% **************************************************
% Altri pacchetti opzionali
% **************************************************     
\usepackage{lastpage}                                     % per sapere il numero totale di pagine
\usepackage{eurosym}                                      % per il simbolo dell'euro usare \EUR{x} dove x è l'importo
\usepackage{ifthen}                                       % permette la scelta di rami condizionali nella compilazione
\usepackage{enumitem}                                     % permette di configurare gli elenchi puntati e numerati


% Fine del preambolo e inizio del documento
\begin{document}

% Inclusione della prima pagina
% shared/firstpage.tex
%
% Questo documento definisce il contenuto della prima pagina, che si suppone
% essere uguale in tutti i documenti.  Oltre al logo e al titolo, la prima
% pagina contiene i metadati relativi al documento in cui viene inclusa.


% rimuove intestazioni e piè di pagina
\pagestyle{empty}

\begin{center}

% logo del gruppo
\includegraphics[width=1.5\textwidth]{logo}

\vspace{1in}

% titolo del documento
{\Huge\bfseries \docName}

\vspace{1in}

% tabella riepilogativa
\begin{tabularx}{.7\textwidth}{>{\bfseries\sffamily}l>{\sffamily}l}
\toprule
\multicolumn{2}{>{\sffamily}c}{Informazioni sul documento}\\
\midrule
Nome file:            & \docFileName\\
Versione:             & \docVers\\
Data creazione:       & \creationDate\\
Data ultima modifica: & \modificationDate\\
Stato:                & \docState\\
Uso:                  & \docUsage\\
Redattori:            & \docAuthors\\
Approvato da:         & \approvedBy\\
Verificatori:         & \verifiedBy\\
\bottomrule
\end{tabularx}

\end{center}

\newpage


% Storico delle modifiche
\section*{Storia delle modifiche}
\begin{tabularx}{\textwidth}{lXll}
\toprule
Versione & Descrzione intervento & Redattore & Data\\
\midrule % inserire qui il contenuto della tabella
0.11 & Approvazione documento. & Stefano Farronato & 2012-12-01\\
0.10 & Verifica documento. & Marco Schivo & 2012-12-01\\
0.9 & Stesura del punto ''Fattibilità del progetto''. Stesura del punto ''Confronto con gli altri capitolati'' & Andrea Rizzi & 2012-12-01\\
0.8 & Revisione dei punti stesi fino alla presente versione. Stesura del punto ''AJAX''. Stesura del punto ''Database relazionali''. Stesura del punto ''Conclusioni sul dominio'' & Andrea Rizzi & 2012-12-01\\
0.7 & Stesura del punto ''JQuery''. & Andrea Rizzi & 2012-11-30\\
0.6 & Stesura del punto ''CSS3''. & Riccardo Tresoldi & 2012-11-30\\
0.5 & Stesura del punto ''HTML5''. & Riccardo Tresoldi& 2012-11-30\\
0.4 & Stesura del punto ''WebRTC''. & Andrea Rizzi & 2012-11-30\\
0.3 & Stesura del punto ''Javascript'' e ''Protocolli e funzionalità di Google Chrome''.  & Riccardo Tresoldi & 2012-11-30\\
0.2 & Stesura del punto ''Java - webSoket''.  & Riccardo Tresoldi & 2012-11-29\\
0.1 & Stesura del punto ''Dominio tecnologico'' (introduzione). Stesura del punto ''Valutazione rischi''. Stesura del punto ''Vantaggi potenziali''. Stesura del punto ''Dominio applicativo''.  & Andrea Rizzi & 2012-11-29\\
0.0 & Creazione del documento e definizione dei punti chiave del documento. Stesura della ''Descrizione sommaria del capitolato''. & Andrea Rizzi & 2012-11-29\\
\bottomrule
\end{tabularx}
\newpage

% inclusione dell'indice
% shared/toc.tex
%
% Questo file contiene le istruzioni che generano l'indice o gli indici del
% documento (utile nel caso in cui decidessimo di avere anche un indice delle
% tabelle e/o un indice delle figure).

% imposta lo stile di pagina per i titoli definito nel preambolo
\pagestyle{toc}
% imposta i numeri di pagina romani minuscoli
\pagenumbering{roman}

% genera automaticamente l'indice di LaTeX
\tableofcontents

% se è true \INDICETABELLE allora genera l'indice delle tabelle, altrimenti non fa nulla
\ifthenelse{\equal{\INDICETABELLE}{true}}{%
  \clearpage % l'indice delle tabelle, se c'è, deve andare a pagina nuova
  \listoftables
}{}

% se è true |INDICEFIGURE allora genera l'indice delle figure, altrimenti non fa nulla
\ifthenelse{\equal{\INDICEFIGURE}{true}}{%
  \clearpage % l'indice delle figure, se c'è, deve andare a pagina nuova
  \listoffigures
}{}

%in ogni caso occorre andare a pagina nuova dopo gli indici
\clearpage


% Alcuni aggiustamenti per le pagine
\pagenumbering{arabic}
\setcounter{page}{1}
\pagestyle{normal}

% Qui ha inizio il documento vero e proprio...

\begin{abstract}
Con il presente documento, il gruppo Software Synthesis, itende dimostrare la fattibilità della realizzazione del progetto MyTalk. Si cercherà di stabilire quali tecnologie sono necessarie al conseguimento dell'obbiettivo, e le problematiche insite nell'affrontarlo, sia sul piano dei requisiti che sul piano delle tecnologie.
\end{abstract}
\newpage

\section{Descrizione sommaria del capitolato scelto}
Il capitolato C1 (denominato MyTalk) proposto dall'azienda italiana Zucchetti, prevede la creazione di un software di comunicazione audio/video, basato sul progetto (attualmente ancora in fase di sviluppo) \underline{WebRTC}. Proprio a causa della recente introduzione, \underline{WebRTC} è facilmente soggetto a modifiche (in relazione alla data in cui è scritto il presente documento, l'ultima modifica risale a poche settimane orsono, il giorno 15 novembre). Di conseguenza il committente ha precisato i seguenti punti fondamentali:

\begin{itemize}
	\item l'architettura software deve basarsi su un modello elastico e facilmente scalabile in seguito a modifiche del pacchetto \underline{WebRTC};
	\item i requisiti opzionali sono modificabili/eliminabili/aggiungibili in corso d'opera a causa della difficoltà a priori di valutare chiaramente la loro fattibilità.
\end{itemize}

Senza prendere in considerazione la completezza dei requisiti obbligatori (stilati nel documento \textit{analisi\_dei\_requisiti.1.0.pd}), si riportano di seguito alcuni punti fondamentali da cui è possibile trarre spunti per la determinazione di problemi tecnici e progettuali:

\begin{itemize}
	\item l'applicativo non dovrà richiedere installazione ne di \underline{plugin} ne di componenti aggiuntivi. Sfrutterà semplicemente il \underline{browser} Google \underline{Chrome};
	\item di base, vi è un \underline{server} scritto in \underline{java} con l'implementazione di webSocket, al quale i client dovranno connettersi, ma che non dovrà partecipare alla comunicazione tra gli utenti;
	\item comunicazione audio;
	\item comunicazione video.
\end{itemize}

\section{Studio del dominio}

\subsection{Dominio tecnologico}
La realizzazione del progetto MyTalk richiede la conoscenza di alcuni strumenti tecnologici obbligatori, senza i quali risulterebbe impossibile rispondere alle esigenze del committente. Di seguito riportiamo un elenco delle conoscenze richieste, arricchite da:

\begin{itemize}
	\item uno o più riferimenti alle parti che potrebbero (sempre in termini di analisi preliminare) richiedere una conoscenza più o meno elevata di tale tecnica/tecnologia (estratte dal capitolato o da considerazioni interne al gruppo);
	\item i vantaggi che si otterrebbero dall'uso di tale tecnologia;
	\item una valutazione delle conoscenze di tale dominio da parte dei membri del team di progetto, quantificate su una scala di autovalutazione che va da 1 (nessuna conoscenza) a 5 (conoscenza totale della tecnologia).
\end{itemize}
Per una descrizione più accurata delle varie tecnologie, rimandiamo al glossario allegato.

\subsubsection{\underline{WebRTC}}
\begin{description}
	\item{\scshape\bfseries Riferimenti:} ''Il progetto deve essere basato sulla tecnologia \underline{WebRTC}, parte delle proposte di evoluzione dell'\underline{HTML5}.'' - estratto dal capitolato C1.

	\item{\scshape\bfseries Vantaggi d'implementazione:} l'applicativo renderebbe quasi nullo il TCO. Esso funzionerebbe solo con l'apposità installazione del brower Google \underline{Chrome}, e non richiedere l'installazione di componenti aggiuntivi o \underline{plugin}.
	
	\item{\scshape\bfseries Conoscenza attuale:} ogni componente del gruppo si trova ad affontare per la prima volta tale tecnologia, definendo un livello di conoscenza pari a 1. Sara quindi necessario formare il personale sulle caratteristiche della tecnologia stessa.
\end{description}

\subsubsection{\underline{java} - webSoket}
\begin{description}
	\item{\scshape\bfseries Riferimenti:} dal capitolato: ''La parte \underline{server}, necessaria solo nella fase di inizializzazione della chiamata, dovrà essere realizzata in \underline{java} e utilizzare il protocollo di comunicazione WebSocket.'' - estratto dal capitolato C1.
	
	\item{\scshape\bfseries Vantaggi d'implementazione:} le webSoket permettono ai \underline{browser} e al \underline{server} di ''parlare'' in maniera asincrona e senza bisogno dell'interazione dell'utente.

	\item{\scshape\bfseries Conoscenza attuale:} in merito a \underline{java}, ogni componente del gruppo ha già un livello di formazione pari a 4. Per quanto conprende la libreria  WebSocket, il gruppo ha una conoscenza basilare (livello 2).  
\end{description}

\subsubsection{\underline{HTML5}}
\begin{description}
	\item{\scshape\bfseries Riferimenti:} ''Il progetto deve essere basato sulla tecnologia \underline{WebRTC}, parte delle proposte di evoluzione dell'\underline{HTML5}.'' - estratto dal capitolato C1.
	
	\item{\scshape\bfseries Vantaggi d'implementazione:} tra le caratteristiche più interessanti, e che offrono spunti per l'implementazione di nuovi requisiti facoltativi, \underline{HTML5} supporta Canvas, che permette di utilizzare \underline{JavaScript} per creare animazioni e grafica bitmap. Un esempio di tale implementazione è la possibilià di condividere una lavagna grafica.
	
	\item{\scshape\bfseries Conoscenza attuale:} tutti i componenti del gruppo presentano un livello medio di conoscenza pari a 5, sull'utilizzo base di \underline{HTML}. Per quanto riguarda l'utilizzo di \underline{HTML5}, in particolar modo delle componenti grafiche offerte (Canvas), solo tre componenti del gruppo hanno un livello pari a 3. Per i restanti cinque sarà necessario organizzare una sessione di formazione.
\end{description}

\subsubsection{CSS3}
\begin{description}
	\item{\scshape\bfseries Riferimenti:} non pervenuti nel capitolato. Inteso in relazione ad una considerazione del gruppo, risulta interessante il loro utilizzoo al fine di gestire l'apparato grafico dell'applicativo in modo sistematico e performante.
	
\item{\scshape\bfseries Vantaggi d'implementazione:} permette una maggiore manuntenibilità del sorgente \underline{HTML}. Le istruzioni di formattazione sono accentrate in un solo punto, sono richiamabili in più punti della stessa pagina (si pensi alle classi di elementi \underline{HTML}) e possono essere condivise fra più pagine se si ricollega ad esse lo stesso \underline{CSS}\@. Ne deriva inoltre un certo incremento prestazionale dal momento che le pagine web genereranno una minore dimensione in byte, neccessitando meno tempo per il loro download. \\Il \underline{CSS} può inoltre risiedere nalla cache del \underline{browser} ed avere pertanto tempi di accesso rapidi e senza comportare ulteriori richieste di trasmissione di dati dal \underline{server}.
	
	\item{\scshape\bfseries Conoscenza attuale:} cinque componenti del gruppo presentano un livello di formazione medio pari a 3. Per gli altri due componenti si dovrà imporre uno studio autodidattico della tecnologia, al più affiancatato dalla collaborazione dei colleghi più esperti.
\end{description}

\subsubsection{\underline{JavaScript}}
\begin{description} 
	\item{\scshape\bfseries Riferimenti:}
  ''il programma che sarà realizzato non deve essere inteso come una pagina Web ma come un software che per l'occasione utilizzi il linguaggio \underline{JavaScript} e le librerie contenute nel \underline{browser}'' - estratto dal capitolato C1.

	\item{\scshape\bfseries Vantaggi d'implementazione:} \underline{JavaScript} è alla base di altre tecnologie d'interesse, come \underline{AJAX} e \underline{JQuery}. La sua conoscenza risulta pertanto un obbligo fondamentale.

	\item{\scshape\bfseries Conoscenza attuale:} quattro componenti del gruppo hanno già un livello di formazione pari a 3. Per gli altri tre sarà necessario organizzare una sessione di formazione.
\end{description}

\subsubsection{\underline{JQuery}}
\begin{description}
	\item{\scshape\bfseries Riferimenti:} non pervenuti nel capitolato. Inteso in relazione ad una considerazione del gruppo, risulta interessante il loro utilizzo al fine di riutilizzare funzionalità \underline{JavaScript} già presenti.
	
	\item{\scshape\bfseries Vantaggi d'implementazione:} permette di semplificare l'attraversamento del codice \underline{HTML}, la gestione degli eventi, le animazioni e le interazioni \underline{AJAX} (intese come chiamate asincrone). Il framework rende il codice più sintetico e limita al minimo l’estensione degli oggetti globali per ottenere la massima compatibilità con altre librerie. Da questo principio è nata una libreria in grado di offrire un'ampia gamma di funzionalità, che vanno dalla manipolazione degli stili \underline{CSS} e degli elementi \underline{HTML}, agli effetti grafici, a comodi metodi per chiamate \underline{AJAX} cross-browser. Il tutto viene effettuato senza modificare nessuno degli oggetti nativi \underline{JavaScript}.
	
	\item{\scshape\bfseries Conoscenza attuale:} due componenti del gruppo hanno già un livello di formazione pari a 3. Per gli altri cinque, si prevede che sarà sufficiente diffondere del materiale per uno studio autodidattico.
\end{description}

\subsubsection{AJAX}
\begin{description}
	\item{\scshape\bfseries Riferimenti:} ''L'intero sistema deve essere contenuto in un unica pagina Web.'' - estratto dal capitolato C1.

	\item{\scshape\bfseries Vantaggi d'implementazione:} tale tecnologia consente l'aggiornamento dinamico di una pagina web senza esplicito comando d'aggiornamento da parte dell'utente. Si osservi che \underline{AJAX} è asincrono nel senso che i dati extra sono richiesti al \underline{server} e caricati in background senza interferire con il comportamento della pagina esistente.
	
	\item{\scshape\bfseries Conoscenza attuale:} tutti i componenti del gruppo si trovano ad affrontare tale tecnologia per la prima volta (livello 1). Si suggerisce quindi uno studio collettivo, e la ricerca di script o funzioni che già implementino le funzionalità desiderate.
\end{description}

\subsubsection{Protocolli e funzionalità di Google \underline{Chrome}}
\begin{description}
	\item{\scshape\bfseries Riferimenti:}
  ''L'estensione dell'\underline{HTML5} \underline{WebRTC} presente nel \underline{browser} \underline{Chrome} si propone di rendere semplice la realizzazione di questi programmi e di far sì che le componenti necessarie siano installate praticamente in ogni computer.'' - estratto dal capitolato C1.
  
 	\item{\scshape\bfseries Vantaggi d'implementazione:} Il supporto di \underline{Chrome} a \underline{WebRTC} è realizzato attraverso una serie di \underline{API} accessibili ai programmi \underline{JavaScript}. In particolare citiamo: PeerConnection, MediaStream e DataChannel.\\Inoltre si sottolinea che la possibilità di registrare gli stream trasmessi e di condividere lo schermo sono fra le funzionalità che sono in programma di integrare nel prossimo futuro, permettendoci di prendere in considerazione la possibilità di sviluppare alcuni interessanti requisiti facoltativi.
	
	\item{\scshape\bfseries Conoscenza attuale:} tutti i componenti del gruppo si trovano ad affrontare tale tecnologia per la prima volta (livello 1), si suggerisce quindi uno studio collettivo. Inoltre è consigliabile che il gruppo si tenga aggiornato sul rilasio di ulteriori funzionalità.
\end{description}

\subsubsection{\underline{Database} relazionali}
\begin{description}
	\item{\scshape\bfseries Riferimenti:} non pervenuti nel capitolato. In relazione ad una considerazione del gruppo, risulta interessante il loro utilizzo al fine di creare un sistema per memorizzare gli utenti registrati e le loro impostazioni dell'applicazione, cosi da dare la possibilità di impostare le proprie configurazioni (per esempio linguistiche) semplicemente eseguendo un login. Ciò comporterà un'ulteriore riduzione del \underline{TCO}.
  
 	\item{\scshape\bfseries Vantaggi d'implementazione:} la realizzazione di un DB relazionale è necessaria per la gestione della lista utenti. Per la creazione di tale DB sotto tecnologia \underline{MySQL}, il team conta di potersi appoggiare allo spazio web appositamente creato per il progetto stesso.
	
	\item{\scshape\bfseries Conoscenza attuale:} tutti i componenti del gruppo hanno già una buona conoscenza dell'argomento. In particolare in merito alla tecnologia \underline{MySQL} e SQLite, il livello di formazione medio è 4.
\end{description}

\subsection{Dominio applicativo}
Essenziale, per analizzare la fattibilità del progetto e stabilire se esiste o meno una prova documentata della realizzazione di un sistema similare.\\ Sotto tale tema, si riporta che il primo progetto di comunicazione audio/video tramite \underline{WebRTC} è stato creato da \textit{Dubango Telecom}, e prende il nome di \textit{sipML5}.\\ \textit{SipML5} è il primo \underline{client} SIP che si basa su \underline{WebRTC} scritto totalmente in \underline{JavaScript} e completamente Open Source. La notizia, evidenziata dal sito: \url{http://www.html5today.it/link/sipml5-primo-\underline{client}-sip-scritto-interamente-html5} dimostra come il progetto sia fattibile quantomeno nella realizzazione dei requisiti obbligatori. Si ribadisce che il progetto sipML5 è open source, inoltre gli sviluppatori ne permettono l'accesso in lettura ai membri "non partecipanti". Per accedere ai sorgenti sarà sufficiente seguire le indicazioni riportate alla pagina: \url{http://code.google.com/p/sipml5/source/checkout}.\\In merito ad altri applicativi software di voip, è possibile trarre spunti per l'implementazione di requisiti opzionali, analizzando programmi come Skype, perseguendo la filosofia che ci impone di ''imparare dai migliori'' sull'attuale mercato.

\subsection{Conclusioni sul dominio}
Per quanto riguarda il dominio tecnologico:
\begin{itemize}
	\item[•] Dall'analisi del capitolato è emerso l'uso obbligatorio di 5 tecnologie: \underline{WebRTC}, \underline{HTML5}, Protocolli di Google Crome, WebSoket, \underline{JavaScript}. Alcune di tali tecnologie risultano sconosciute per alcuni componenti del team. Tuttavia la ricerca di informazioni riguardanti il loro utilizzo è facilitata dalla mole di riferimenti a tutorial reperibili in rete o in testi dedicati. Molti di questi linguaggi sono inoltre fonte di studio per il corso di studi ''Tecnologie Web'', ne consegue che il gruppo avrà modo di approfondire il loro utilizzo durante il secondo trimestre dell'anno accademico 2012-2013, affiancando ad uno studio autodidattico, le conoscenze apprese.
	\item[•] Alle tecnologie obbligatorie se ne affiancano altre facoltative, il cui vantanggio è garantire un riuso sostanziale di funzionalità rese già disponibili (vedi \underline{JQuery}).
	\item[•] Il dominio applicativo dimostra che un progetto simile essite già. Il team intende studiare i sorgenti resi disponibili dal gruppo \textit{Dubango Telecom}, al fine di trarre spunti sull'utilizzo del \underline{WebRTC} e, se si riscontrasse la possibilità, riutilizzare parte delle funzionalità proposte dal progeto \textit{SipML5}.
	\item[•] L'ormai consolidato utilizzo di programmi di comunicazione RT, permette ai membri del gruppo di avere una conoscenza basilare sui requisiti utente, desiderabili per un end user.
\end{itemize}

\section{Valutazione del capitolato}

\subsection{Valutazione dei rischi}
\begin{itemize}
	\item Le tecnologie richieste per la creazione dell'applicativo, sono attualmente in via di definizione. La tecnologia \underline{WebRTC} non è ancora uno standard, e risulta quasi certo le sue \underline{API} vengano modificate in corso d'opera del progetto. Si pensi che nella data attuale, l'ultima modifica apportata all'architettura risale al 15 novembre. Tuttavia risulta sensato supporre che le modifiche non intaccheranno in termini distruttivi le basi già istanziate. Questa risulta ovviamente essere una supposizione che non rappresenta necessariamente la realtà dei fatti, tuttavia come si evidenzierà nel paragrafo successivo, ciò può comportare anche un punto a favore nello sviluppo progettuale.
	\item Malgrado siano state rilevate diverse tecnologie d'implementazione, il livello medio di formazione, dei componenti del gruppo, è relativamente basso. Sarà quindi necessario puntare molto sulla formazione del team. Ciò può influenzare negativamente le tempstiche di sviluppo, e inoltre può comportare errori di valutazione sulla fattibilità di alcuni requisiti.
\end{itemize}

\subsection{Vantaggi potenziali}
\begin{itemize}
	\item Estratto dal capitolato: ''In corso d'opera non sarà possibile variare/modificare i requisiti minimi (obbligatori per accettare il prodotto). Sarà invece possibile variare i requisiti opzionali, in quanto saranno i gruppi vincitori dell'appalto a modificarli/eliminarli/aggiungerli''. La conseguenza di tale informazione rende meno traumatico l'utilizzo del \underline{WebRTC}, in quanto se si evidenzia l'impossibilità di soddisfare alcuni requisiti obbligatori sarà possibilie segnalarlo al committente, discuterne, ed eventualmente abbandonarne lo sviluppo. Questa pratica è ovviamente sconsigliata, al contrario risulta più ragionevole partire con meno requisiti opzionali, er eventualmente procedere in seguito con l'aggiunta di tali requisiti se se ne riconosce l'effettiva soddisfacibilità.
	\item Il software da sviluppare si poggia completamente sul \underline{browser} \underline{Chrome}. Di conseguenza su avrà un completo svincolo dalla gestione delle dipendenze dei sistemi operativi sottostanti.
	\item L'applicativo software poggia le sue funzionalità su un \textit{sito web} che dovrà fungere da \underline{server}. Per la creazione di tale sito si potrà sfruttare lo spazio web del gruppo, sia per sperimentare il corretto funzionamento delle parti (in fase di sviluppo), sia per permettere al committente di accedere all'ambiente, al fine di verificare lo stato del prodotto.
	\item Riprendendo quanto citato nel paragrafo 3.1 "Valutazione rischi", la possibilità che alcune strutture logiche varino durante lo sviluppo del progetto, imporrà al team di creare un architettura sensata e ben curata. Dando quindi un occhio di riguardo al riuso e alla "scalabilità" della struttura, saranno incentivati l'utilizzo di pattern appropriati e ricercati. Tali considerazioni dovranno incoraggiare i membri del gruppo alla ricerca di un architettura il più performante possibile.
\end{itemize}

\section{Fattibilità del progetto}
Il progetto MyTalk è stato preso in considerazione da Software Synthesis per l'interesse generale sulle tecnologie d'implementazione. Malgrado il livello di formazione su tali tecnologie non sia attualmente sufficiente alla realizzazione del progetto, il team è sicuro di poter apprendere quanto necessario in una periodo che permetta il corretto avvio della fase di progettazione. Per tale motivo, e per quanto già riportato nel paragrafo 2.3 ''conclusioni sul dominio'' e nel capitolo 3 ''valutazione del capitolato'', Il gruppo Software Synthesis ha ritenuto il progetto C1 fattibile, ed è pertanto intenzionato a realizzato nei tempi e nei costi previsti.

\section{Confronto con gli altri capitolati}

\subsection{Capitolato C2}
Nel valutare il capitolato C2, il team si è trovato d'accordo nel ritenere il progetto fattibile. Tuttavia i rischi individuati hanno portato il team a scartare tale progetto in favore del capitolato C1

\subsubsection{Individuazione rischi}
\begin{itemize}
	\item Conoscenze totalmente assenti dei formati JSON, 3DS, OBJ \& MTL e delle tecnologie grafiche in generale.
	\item Perplessità riguardanti la creazione di opportuni algoritmi di conversione e ottimizzazione dei formati 3D.
\end{itemize}

\subsection{Capitolato C3}
Nel valutare il capitolato C3, il gruppo si è trovato d'accordo nel ritenere il progetto fattibile. Tuttavia sono sorte alcune ambiguità nell'analisi del capitolato. Alla richiesta pervenuta presso il committente, di organizzare una riunione per chiarire i dubbi, il gruppo si è trovato in difficoltà nel osservare le date proposte per l'incontro. Ritenendo che le tempistiche non fossero favorevoli allo sviluppo preliminare del capitolato, il gruppo Software Synthesis ha deciso di scartare il progetto C3.

\subsubsection{Individuazione rischi}
\begin{itemize}
	\item Dubbia comprensione di alcuni passaggi essenziali, in relazione all'imposibilità di chiarirli in breve tempo, hanno fatto del capitolato C3, un progetto a rischio. Inoltre è stato decisivo l'insorgere del timore di non realizzare entro tempi congrui eventuali difficoltà d'implementazione in fase di scelta del capitolato, e dell'impossibilità temporale di considerare un altro progetto in sostituzione ad esso.
	\item Scarse conoscenze da parte dei componenti del team in merito al dominio tecnologico.
\end{itemize}

\subsection{Capitolato C4}
Il capitolato C4 è stato valutato, in seguito ad un'analisi preliminare, fattibile ma con un forte rischio di sforare nelle tempistiche di consegna.

\subsubsection{Individuazione rischi}
\begin{itemize}
	\item Eccessiva complessità di alcuni requisiti obbligatori, in particolare legati alla richiesta di fornire un applicativo funzionante sia su dispositivi mobile che desktop.
	\item Scarse conoscenze da parte dei componenti del team in merito al dominio tecnologico (programmazione per dispositivi mobili).
\end{itemize}



\end{document}
