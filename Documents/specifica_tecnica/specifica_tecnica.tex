% analisi_dei_requisiti/analisi_dei_requisiti.tex

%TODO:
% sistemare date e autori sulla base del PdP!

% **************************************************
% Macro specifiche per il documento corrente
% **************************************************
% Nome
\newcommand{\docName}{Specifica tecnica}
% Nome file
\newcommand{\docFileName}{specifica\_tecnica.1.0.pdf}
% Versione
\newcommand{\docVers}{0.1}
% Data creazione
\newcommand{\creationDate}{2013-01-16}
% Data ultima modifica
\newcommand{\modificationDate}{2013-01-16}
% Stato in {Approvato, Non approvato}
\newcommand{\docState}{Non approvato}
% Uso in {Interno, Esterno}
\newcommand{\docUsage}{Interno}
% Destinatari da specificare come nome1\\ &nome2\\ ecc.
\newcommand{\docDistributionList}{Progettisti del team}
% Redattori da specificare come nome1\\ &nome2\\ ecc.
\newcommand{\docAuthors}{}
% Approvato da
\newcommand{\approvedBy}{}
% Verificatori
\newcommand{\verifiedBy}{}
% Perscorso (relativo o assoluto) che punta alla directory contenente shared/
% come sua sottodirectory (per comodità chiamiamola 'doc root').
\newcommand{\docRoot}{..}
% definire se si vuole l'indice delle tabelle
\def\INDICETABELLE{false}
% definire se si vuole l'indice delle figure
\def\INDICEFIGURE{false}


% importa il preambolo condiviso da tutti i documenti
% shared/preamble.tex
%
% Questo documento contiene la parte del preambolo condivisa e viene pertanto
% richiamato nel 'master' di tutti i documenti di progetto.  Al suo interno
% contiene le inclusioni (e le configurazioni) di tutti i package richiesti per
% la compilazione dei documenti, le macro di carattere generale e la definizione
% degli stili di pagina.

\documentclass[a4paper,10pt]{article}

% **************************************************
% Macro generiche
% **************************************************
\newcommand{\team}{Software Synthesis}                    % chi siamo
\newcommand{\email}{info@softwaresynthesis.org}           % e-mail
\newcommand{\caName}{MyTalk}                              % titolo capitolato
\newcommand{\manager}{SynthesisRequirementManager}        % nome del sistema di tracciamento
\newcommand{\memberdata}[1]{%
  \texttt{\textcolor{RedOrange}{#1}}}                     % attributi di una classe
\newcommand{\method}[1]{\texttt{\textcolor{Emerald}{#1}}} % metodi di una classe
\newcommand{\exception}[1]{%
  \texttt{\textcolor{RedViolet}{#1}}}                     % eccezione
% \newcommand{\handler}[1]{\texttt{\textcolor{Maroon}{#1}}} % per gli event handler
\newcommand{\inglese}[1]{%
  \foreignlanguage{english}{\textit{#1}}}                 % per i testi in lingua inglese
\newcommand{\purpose}{%                                     scopo del prodotto
Con il progetto ``\caName'' si intende un sistema software di comunicazione tra utenti mediante \underline{browser} senza la necessit{\`a} di installazione di \underline{plugin} e/o software esterni. L'utilizzatore avr{\`a} la possibilit{\`a} di interagire con un altro utente tramite una comunicazione audio - audio/video - testuale e, inoltre, ottenere delle statistiche sull'attivit{\`a} in tempo reale.%
}
\newcommand{\glossaryIntro}{%                               introduzione al glossario
Al fine di evitare incomprensioni dovute all'uso di termini tecnici nei documenti, viene redatto e allegato il documento \textit{glossario.4.0.pdf} dove vengono definiti e descritti tutti i termini marcati con una sottolineatura.%
}


% **************************************************
% Codifica e lingua dei documenti
% **************************************************
\usepackage[utf8x]{inputenc}                              % codifica caratteri dei documenti sorgenti
\usepackage[english,italian]{babel}                       % localizzazione ai fini di sillabazione e cross-references
\usepackage[T1]{fontenc}                                  % codifica font di output

% **************************************************
% Definizione geometria della pagina
% **************************************************
\usepackage[a4paper,head=4cm,top=4.5cm,bottom=3cm,left=3cm,right=3cm,bindingoffset=5mm]{geometry}

% *************************************************
% Intestazioni e piè di pagina personalizzati
% *************************************************
\usepackage{fancyhdr}
% stile normale
\fancypagestyle{normal}{
\fancyhead{}                                              % intestazione
\fancyhead[RE,RO]{
\begin{picture}(0,0)
  \put(-410,0){\includegraphics[width=1.02\textwidth]{header_logo}}
  \put(-410,10){\sffamily\large\leftmark}
\end{picture}
\vspace{-4pt}
}
\renewcommand{\headrulewidth}{0pt}                       % riga sotto l'intestazione
\cfoot{}                                                  % piè di pagina
\fancyfoot[RO,LE]{\sffamily
  pag.~\thepage{} di \pageref{LastPage}}                  % a dx nelle pag. dispari e a sx in quelle pari
\fancyfoot[RE,LO]{\sffamily\docFileName{}}
\renewcommand{\footrulewidth}{.4pt}                       % riga sopra il piè di pagina
}
% stile per gli indici
\fancypagestyle{toc}{
\fancyhead{}                                              % intestazione
\fancyhead[RE,RO]{
\begin{picture}(0,0)
  \put(-410,0){\includegraphics[width=1.02\textwidth]{header_logo}}
\end{picture}
}
\renewcommand{\headrule}{}                                % nessuna riga sotto l'intestazione
\cfoot{}                                                  % piè di pagina
\fancyfoot[RO,LE]{\sffamily\thepage{}}                    % a dx nelle pag. dispari e a sx in quelle pari
\fancyfoot[RE,LO]{\sffamily\docFileName{} -- v.\docVers}
\renewcommand{\footrulewidth}{.4pt}                       % riga sopra il piè di pagina
}

\pagestyle{fancy}                                         % premetto: non so usare bene le marche:
\renewcommand{\sectionmark}[1]{\markboth{#1}{#1}}         % se qualcuno ha idee migliori si faccia avanti!

% **************************************************
% Tabelle
% **************************************************
\usepackage{tabularx}                                     % tabelle di larghezza fissa con una o più colonne variabili
\usepackage{multirow}                                     % colonne con colonne che si estendono per più righe
\usepackage{booktabs}                                     % per inserire l'ambiente table e le righe orizz. nelle tabelle
\usepackage{longtable}			                              % tabelle oltre i limiti di pagina

% **************************************************
% Cross-references e collegamenti ipertestuali
% **************************************************
\usepackage[hidelinks]{hyperref}
\hypersetup{%
  colorlinks=false, linktocpage=false, pdfborder={0,0,0}, pdfstartpage=1, pdfstartview=FitV,%
  urlcolor=Cyan, linkcolor=Cyan, citecolor=Black, %pagecolor=Black,%
  pdftitle={\docName}, pdfauthor={\team}, pdfsubject={}, pdfkeywords={},%
  pdfcreator={pdflatex}, pdfproducer={pdflatex with hyperref package}%
}

% **************************************************
% Immagini e grafica
% **************************************************
\usepackage{graphicx}                                     % supporto ad aspetti avanzati delle immagini
\usepackage[table,usenames,dvipsnames]{xcolor}            % tabelle con righe colorate e alternate
\graphicspath{{\docRoot/pics/}}                           % percorso contenente tutti i file immagini
\usepackage{float}                                        % per rendere non flottanti gli ambienti flottanti
\usepackage[italian]{varioref}                            % testo completo riferimenti in italiano

% **************************************************
% Definizioni di colori
% **************************************************
\definecolor{myBlue}{RGB}{1,167,236}
\definecolor{lightblue}{RGB}{213,243,253}%{119,218,247}
\definecolor{llightblue}{RGB}{229,255,255}

% **************************************************
% Altri pacchetti opzionali
% **************************************************     
\usepackage{lastpage}                                     % per sapere il numero totale di pagine
\usepackage{eurosym}                                      % per il simbolo dell'euro usare \EUR{x} dove x è l'importo
\usepackage{ifthen}                                       % permette la scelta di rami condizionali nella compilazione
\usepackage{enumitem}                                     % permette di configurare gli elenchi puntati e numerati


% Fine del preambolo e inizio del documento
\begin{document}

% Inclusione della prima pagina
% shared/firstpage.tex
%
% Questo documento definisce il contenuto della prima pagina, che si suppone
% essere uguale in tutti i documenti.  Oltre al logo e al titolo, la prima
% pagina contiene i metadati relativi al documento in cui viene inclusa.


% rimuove intestazioni e piè di pagina
\pagestyle{empty}

\begin{center}

% logo del gruppo
\includegraphics[width=1.5\textwidth]{logo}

\vspace{1in}

% titolo del documento
{\Huge\bfseries \docName}

\vspace{1in}

% tabella riepilogativa
\begin{tabularx}{.7\textwidth}{>{\bfseries\sffamily}l>{\sffamily}l}
\toprule
\multicolumn{2}{>{\sffamily}c}{Informazioni sul documento}\\
\midrule
Nome file:            & \docFileName\\
Versione:             & \docVers\\
Data creazione:       & \creationDate\\
Data ultima modifica: & \modificationDate\\
Stato:                & \docState\\
Uso:                  & \docUsage\\
Redattori:            & \docAuthors\\
Approvato da:         & \approvedBy\\
Verificatori:         & \verifiedBy\\
\bottomrule
\end{tabularx}

\end{center}

\newpage


% Storico delle modifiche
\section*{Storia delle modifiche}
\begin{center}
\begin{longtable}{lp{.32\textwidth}lll}
\toprule
Versione & Descrizione intervento & Membro & Ruolo & Data\\
\midrule % inserire qui il contenuto della tabella
0.1 & Creazione del documento e stesura della sezione ``Introduzione''. & Andrea Rizzi & Non lo so & 2013-01-16\\
\bottomrule
\end{longtable}
\end{center}
\newpage

% inclusione dell'indice
% shared/toc.tex
%
% Questo file contiene le istruzioni che generano l'indice o gli indici del
% documento (utile nel caso in cui decidessimo di avere anche un indice delle
% tabelle e/o un indice delle figure).

% imposta lo stile di pagina per i titoli definito nel preambolo
\pagestyle{toc}
% imposta i numeri di pagina romani minuscoli
\pagenumbering{roman}

% genera automaticamente l'indice di LaTeX
\tableofcontents

% se è true \INDICETABELLE allora genera l'indice delle tabelle, altrimenti non fa nulla
\ifthenelse{\equal{\INDICETABELLE}{true}}{%
  \clearpage % l'indice delle tabelle, se c'è, deve andare a pagina nuova
  \listoftables
}{}

% se è true |INDICEFIGURE allora genera l'indice delle figure, altrimenti non fa nulla
\ifthenelse{\equal{\INDICEFIGURE}{true}}{%
  \clearpage % l'indice delle figure, se c'è, deve andare a pagina nuova
  \listoffigures
}{}

%in ogni caso occorre andare a pagina nuova dopo gli indici
\clearpage


% Alcuni aggiustamenti per le pagine
\pagenumbering{arabic}
\setcounter{page}{1}
\pagestyle{normal}

% Qui ha inizio il documento vero e proprio...

\newpage

\section{Introduzione}
\subsection{Scopo del prodotto}
\purpose

\subsection{Scopo del documento}
Il presente documento è stato redatto al fine di produrre le specifiche sulla progettazione ad alto livello, del prodotto MyTalk. A tal fine il documento presenterà:

\begin{itemize}
	\item Un elenco con le specifiche dei design pattern utilizzati. Per ogni design pattern sarà fornita: una breve descrizione riguardante la struttura generale, il motivo per cui è stato scelto e di conseguenza usato, e l'elenco dei contesti dove è stato utilizzato il design patterns.
	\item Una descrizione dettagliata dei componenti rilevati in fase di progettazione indicando il tipo,
la funzione e l'obbiettivo.
	\item L'architettura d'alto livello del sistema.
	\item I diagrammi UML per definire i flussi principali di controllo dell'applicativo.
	\item Il tracciamento dei requisiti e delle componenti, negli schemi: requisiti-componeneti e componenti-requisiti.
\end{itemize}

\subsection{Glossario}
\glossaryIntro

\clearpage
\section{Riferimenti}

\subsection{Normativi}
\begin{itemize}
\item[] \textit{piano\_di\_qualifica.2.0.pdf} allegato.
\item[] \textit{norme\_di\_progetto.2.0.pdf} allegato.
\item[] \textit{analisi\_dei\_requisiti.2.0.pdf} allegato
\end{itemize}

\subsection{Informativi}
\begin{itemize}
\item[] Capitolato d'appalto: \caName{}, v1.0, redatto e rilasciato dal proponente Zucchetti s.r.l. reperibile all'indirizzo \url{http://www.math.unipd.it/~tullio/IS-1/2012/Progetto/C1.pdf};
\item[] testo di consultazione: \textit{Software Engineering (8th edition) Ian Sommerville, Pearson Education | Addison Wesley};
\item[] manuale all'utilizzo dei design pattens: \textit{Design Patterns, Elementi per il riuso di software a oggetti - (1/Ed. italiana) Eric Gamma, Richard Helm, Ralph Johnson, John Vlissides, Pearson Education};
\item[] \textit{glossario.1.0.pdf} allegato.
\end{itemize}

\clearpage
\section{Design Pattern}
\subsection{Template descrizione design pattern}
\subsubsection{Descrizione}
\subsubsection{Schema in UML}
\subsubsection{Motivazione}
\subsubsection{Configuration Item che lo implementano}
%mettere in evidenza le CI che lo usano, elencandole per schema architetturale: CI del Server, CI per ClientUniversale, CI per Client
\clearpage
\section{Introduzione all'architettura di sistema}
\clearpage
\section{Architettura MyTalk-Server}
\subsection{Componenti evidenziate}
\subsubsection{Template della componente X}
\begin{description}
	\item{\scshape\bfseries Descrizione:} 
	\item{\scshape\bfseries Diagramma del package:}
	\item{\scshape\bfseries Classi utilizzate:} 
\end{description}
\subsection{Classi utilizzate}
\subsubsection{Template classe X}
\begin{description}
	\item{\scshape\bfseries Descrizione:} 
	\item{\scshape\bfseries Diagramma della classe:}
	\item{\scshape\bfseries Componenti che ne fanno uso:} 
\end{description}
\subsection{Diagramma del package}
\subsection{Diagramma delle classi}
\clearpage
\section{Architettura MyTalk-ClientUniversale}
\subsection{Componenti evidenziate}
\subsubsection{Template della componente X}
\begin{description}
	\item{\scshape\bfseries Descrizione:} 
	\item{\scshape\bfseries Diagramma del package:}
	\item{\scshape\bfseries Classi utilizzate:} 
\end{description}
\subsection{Classi utilizzate}
\subsubsection{Template classe X}
\begin{description}
	\item{\scshape\bfseries Descrizione:} 
	\item{\scshape\bfseries Diagramma della classe:}
	\item{\scshape\bfseries Componenti che ne fanno uso:} 
\end{description}
\subsection{Diagramma del package}
\subsection{Diagramma delle classi}
\clearpage
\section{Architettura MyTalk-ClientSoftwareSynthesis}
\subsection{Componenti evidenziate}
\subsubsection{Template della componente X}
\begin{description}
	\item{\scshape\bfseries Descrizione:} 
	\item{\scshape\bfseries Diagramma del package:}
	\item{\scshape\bfseries Classi utilizzate:} 
\end{description}
\subsection{Classi utilizzate}
\subsubsection{Template classe X}
\begin{description}
	\item{\scshape\bfseries Descrizione:} 
	\item{\scshape\bfseries Diagramma della classe:}
	\item{\scshape\bfseries Componenti che ne fanno uso:} 
\end{description}
\subsection{Diagramma del package}
\subsection{Diagramma delle classi}
\clearpage
\section{Conclusioni sull'architettura}
\subsection{Diagrammi delle attività}
\subsection{Diagrammi di sequenza}
\clearpage
\section{Tracciamenti}
\subsection{Tracciamenti Requisiti-Componenti}
\subsection{Tracciamenti Componenti-Requisiti}
\subsection{Tracciamenti Componenti-DesignPattern}
\subsection{Tracciamenti DesignPattern-Componenti}
\subsection{Tracciamenti Componenti-Classi}
\subsection{Tracciamenti Classi-Componenti}

\end{document}