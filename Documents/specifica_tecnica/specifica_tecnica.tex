% **************************************************
% Macro specifiche per il documento corrente
% **************************************************
% Nome
\newcommand{\docName}{Specifica tecnica}
% Nome file
\newcommand{\docFileName}{specifica\_tecnica.1.0.pdf}
% Versione
\newcommand{\docVers}{0.1}
% Data creazione
\newcommand{\creationDate}{2013-01-16}
% Data ultima modifica
\newcommand{\modificationDate}{2013-01-17}
% Stato in {Approvato, Non approvato}
\newcommand{\docState}{Non approvato}
% Uso in {Interno, Esterno}
\newcommand{\docUsage}{Interno}
% Destinatari da specificare come nome1\\ &nome2\\ ecc.
\newcommand{\docDistributionList}{Team SoftwareSynthesis}
% Redattori da specificare come nome1\\ &nome2\\ ecc.
\newcommand{\docAuthors}{}
% Approvato da
\newcommand{\approvedBy}{}
% Verificatori
\newcommand{\verifiedBy}{}
% Perscorso (relativo o assoluto) che punta alla directory contenente shared/
% come sua sottodirectory (per comodità chiamiamola 'doc root').
\newcommand{\docRoot}{..}
% definire se si vuole l'indice delle tabelle
\def\INDICETABELLE{false}
% definire se si vuole l'indice delle figure
\def\INDICEFIGURE{false}

% importa il preambolo condiviso da tutti i documenti
% shared/preamble.tex
%
% Questo documento contiene la parte del preambolo condivisa e viene pertanto
% richiamato nel 'master' di tutti i documenti di progetto.  Al suo interno
% contiene le inclusioni (e le configurazioni) di tutti i package richiesti per
% la compilazione dei documenti, le macro di carattere generale e la definizione
% degli stili di pagina.

\documentclass[a4paper,10pt]{article}

% **************************************************
% Macro generiche
% **************************************************
\newcommand{\team}{Software Synthesis}                    % chi siamo
\newcommand{\email}{info@softwaresynthesis.org}           % e-mail
\newcommand{\caName}{MyTalk}                              % titolo capitolato
\newcommand{\manager}{SynthesisRequirementManager}        % nome del sistema di tracciamento
\newcommand{\memberdata}[1]{%
  \texttt{\textcolor{RedOrange}{#1}}}                     % attributi di una classe
\newcommand{\method}[1]{\texttt{\textcolor{Emerald}{#1}}} % metodi di una classe
\newcommand{\exception}[1]{%
  \texttt{\textcolor{RedViolet}{#1}}}                     % eccezione
% \newcommand{\handler}[1]{\texttt{\textcolor{Maroon}{#1}}} % per gli event handler
\newcommand{\inglese}[1]{%
  \foreignlanguage{english}{\textit{#1}}}                 % per i testi in lingua inglese
\newcommand{\purpose}{%                                     scopo del prodotto
Con il progetto ``\caName'' si intende un sistema software di comunicazione tra utenti mediante \underline{browser} senza la necessit{\`a} di installazione di \underline{plugin} e/o software esterni. L'utilizzatore avr{\`a} la possibilit{\`a} di interagire con un altro utente tramite una comunicazione audio - audio/video - testuale e, inoltre, ottenere delle statistiche sull'attivit{\`a} in tempo reale.%
}
\newcommand{\glossaryIntro}{%                               introduzione al glossario
Al fine di evitare incomprensioni dovute all'uso di termini tecnici nei documenti, viene redatto e allegato il documento \textit{glossario.4.0.pdf} dove vengono definiti e descritti tutti i termini marcati con una sottolineatura.%
}


% **************************************************
% Codifica e lingua dei documenti
% **************************************************
\usepackage[utf8x]{inputenc}                              % codifica caratteri dei documenti sorgenti
\usepackage[english,italian]{babel}                       % localizzazione ai fini di sillabazione e cross-references
\usepackage[T1]{fontenc}                                  % codifica font di output

% **************************************************
% Definizione geometria della pagina
% **************************************************
\usepackage[a4paper,head=4cm,top=4.5cm,bottom=3cm,left=3cm,right=3cm,bindingoffset=5mm]{geometry}

% *************************************************
% Intestazioni e piè di pagina personalizzati
% *************************************************
\usepackage{fancyhdr}
% stile normale
\fancypagestyle{normal}{
\fancyhead{}                                              % intestazione
\fancyhead[RE,RO]{
\begin{picture}(0,0)
  \put(-410,0){\includegraphics[width=1.02\textwidth]{header_logo}}
  \put(-410,10){\sffamily\large\leftmark}
\end{picture}
\vspace{-4pt}
}
\renewcommand{\headrulewidth}{0pt}                       % riga sotto l'intestazione
\cfoot{}                                                  % piè di pagina
\fancyfoot[RO,LE]{\sffamily
  pag.~\thepage{} di \pageref{LastPage}}                  % a dx nelle pag. dispari e a sx in quelle pari
\fancyfoot[RE,LO]{\sffamily\docFileName{}}
\renewcommand{\footrulewidth}{.4pt}                       % riga sopra il piè di pagina
}
% stile per gli indici
\fancypagestyle{toc}{
\fancyhead{}                                              % intestazione
\fancyhead[RE,RO]{
\begin{picture}(0,0)
  \put(-410,0){\includegraphics[width=1.02\textwidth]{header_logo}}
\end{picture}
}
\renewcommand{\headrule}{}                                % nessuna riga sotto l'intestazione
\cfoot{}                                                  % piè di pagina
\fancyfoot[RO,LE]{\sffamily\thepage{}}                    % a dx nelle pag. dispari e a sx in quelle pari
\fancyfoot[RE,LO]{\sffamily\docFileName{} -- v.\docVers}
\renewcommand{\footrulewidth}{.4pt}                       % riga sopra il piè di pagina
}

\pagestyle{fancy}                                         % premetto: non so usare bene le marche:
\renewcommand{\sectionmark}[1]{\markboth{#1}{#1}}         % se qualcuno ha idee migliori si faccia avanti!

% **************************************************
% Tabelle
% **************************************************
\usepackage{tabularx}                                     % tabelle di larghezza fissa con una o più colonne variabili
\usepackage{multirow}                                     % colonne con colonne che si estendono per più righe
\usepackage{booktabs}                                     % per inserire l'ambiente table e le righe orizz. nelle tabelle
\usepackage{longtable}			                              % tabelle oltre i limiti di pagina

% **************************************************
% Cross-references e collegamenti ipertestuali
% **************************************************
\usepackage[hidelinks]{hyperref}
\hypersetup{%
  colorlinks=false, linktocpage=false, pdfborder={0,0,0}, pdfstartpage=1, pdfstartview=FitV,%
  urlcolor=Cyan, linkcolor=Cyan, citecolor=Black, %pagecolor=Black,%
  pdftitle={\docName}, pdfauthor={\team}, pdfsubject={}, pdfkeywords={},%
  pdfcreator={pdflatex}, pdfproducer={pdflatex with hyperref package}%
}

% **************************************************
% Immagini e grafica
% **************************************************
\usepackage{graphicx}                                     % supporto ad aspetti avanzati delle immagini
\usepackage[table,usenames,dvipsnames]{xcolor}            % tabelle con righe colorate e alternate
\graphicspath{{\docRoot/pics/}}                           % percorso contenente tutti i file immagini
\usepackage{float}                                        % per rendere non flottanti gli ambienti flottanti
\usepackage[italian]{varioref}                            % testo completo riferimenti in italiano

% **************************************************
% Definizioni di colori
% **************************************************
\definecolor{myBlue}{RGB}{1,167,236}
\definecolor{lightblue}{RGB}{213,243,253}%{119,218,247}
\definecolor{llightblue}{RGB}{229,255,255}

% **************************************************
% Altri pacchetti opzionali
% **************************************************     
\usepackage{lastpage}                                     % per sapere il numero totale di pagine
\usepackage{eurosym}                                      % per il simbolo dell'euro usare \EUR{x} dove x è l'importo
\usepackage{ifthen}                                       % permette la scelta di rami condizionali nella compilazione
\usepackage{enumitem}                                     % permette di configurare gli elenchi puntati e numerati


% Fine del preambolo e inizio del documento
\begin{document}

% Inclusione della prima pagina
% shared/firstpage.tex
%
% Questo documento definisce il contenuto della prima pagina, che si suppone
% essere uguale in tutti i documenti.  Oltre al logo e al titolo, la prima
% pagina contiene i metadati relativi al documento in cui viene inclusa.


% rimuove intestazioni e piè di pagina
\pagestyle{empty}

\begin{center}

% logo del gruppo
\includegraphics[width=1.5\textwidth]{logo}

\vspace{1in}

% titolo del documento
{\Huge\bfseries \docName}

\vspace{1in}

% tabella riepilogativa
\begin{tabularx}{.7\textwidth}{>{\bfseries\sffamily}l>{\sffamily}l}
\toprule
\multicolumn{2}{>{\sffamily}c}{Informazioni sul documento}\\
\midrule
Nome file:            & \docFileName\\
Versione:             & \docVers\\
Data creazione:       & \creationDate\\
Data ultima modifica: & \modificationDate\\
Stato:                & \docState\\
Uso:                  & \docUsage\\
Redattori:            & \docAuthors\\
Approvato da:         & \approvedBy\\
Verificatori:         & \verifiedBy\\
\bottomrule
\end{tabularx}

\end{center}

\newpage


%---------------------------RUOLI----------------------------
%FASE 1:
%Progettisti: TRES, STEFANO, SCHIVO;
%FASE 2:
%Progettisti: DIEGO, ELENA, RIZZI

%Verificatore: Andrea Meneghinello
%Responsabile finale TRES
%------------------------------------------------------------

% Storico delle modifiche
\section*{Storia delle modifiche}
\begin{center}
\begin{longtable}{lp{.32\textwidth}lll}
\toprule
Versione & Descrizione intervento & Membro & Ruolo & Data\\
\midrule % inserire qui il contenuto della tabella
0.2 & Stesura dell'introduzione ai design pattern. Stesura dell'introduzione ai tracciamenti. & Stefano Farronato & Progettista & 2013-01-17\\
0.1 & Creazione del documento e stesura della sezione ``Introduzione''. & Riccardo Tresoldi & Progettista & 2013-01-16\\
\bottomrule
\end{longtable}
\end{center}
\newpage

% inclusione dell'indice
% shared/toc.tex
%
% Questo file contiene le istruzioni che generano l'indice o gli indici del
% documento (utile nel caso in cui decidessimo di avere anche un indice delle
% tabelle e/o un indice delle figure).

% imposta lo stile di pagina per i titoli definito nel preambolo
\pagestyle{toc}
% imposta i numeri di pagina romani minuscoli
\pagenumbering{roman}

% genera automaticamente l'indice di LaTeX
\tableofcontents

% se è true \INDICETABELLE allora genera l'indice delle tabelle, altrimenti non fa nulla
\ifthenelse{\equal{\INDICETABELLE}{true}}{%
  \clearpage % l'indice delle tabelle, se c'è, deve andare a pagina nuova
  \listoftables
}{}

% se è true |INDICEFIGURE allora genera l'indice delle figure, altrimenti non fa nulla
\ifthenelse{\equal{\INDICEFIGURE}{true}}{%
  \clearpage % l'indice delle figure, se c'è, deve andare a pagina nuova
  \listoffigures
}{}

%in ogni caso occorre andare a pagina nuova dopo gli indici
\clearpage


% Alcuni aggiustamenti per le pagine
\pagenumbering{arabic}
\setcounter{page}{1}
\pagestyle{normal}

% Qui ha inizio il documento vero e proprio...

\newpage

\section{Introduzione}
\subsection{Scopo del prodotto}
\purpose

\subsection{Scopo del documento}
Il presente documento è stato redatto al fine di produrre le specifiche sulla progettazione ad alto livello, del prodotto MyTalk. A tal fine il documento presenterà:

\begin{itemize}
	\item Un elenco con le specifiche dei design pattern utilizzati.
	\item Una descrizione dettagliata dei componenti rilevati in fase di progettazione indicando il tipo,
la funzione e l'obbiettivo.
	\item L'architettura d'alto livello del sistema.
	\item I diagrammi UML per definire i flussi principali di controllo dell'applicativo.
	\item Il tracciamento dei requisiti e delle componenti, negli schemi: requisiti-componenti e componenti-requisiti.
\end{itemize}

\subsection{Glossario}
\glossaryIntro

\clearpage
\section{Riferimenti}

\subsection{Normativi}
\begin{itemize}
\item[] \textit{piano\_di\_qualifica.2.0.pdf} allegato.
\item[] \textit{norme\_di\_progetto.2.0.pdf} allegato.
\item[] \textit{analisi\_dei\_requisiti.2.0.pdf} allegato
\end{itemize}

\subsection{Informativi}
\begin{itemize}
\item[] Capitolato d'appalto: \caName{}, v1.0, redatto e rilasciato dal proponente Zucchetti s.r.l. reperibile all'indirizzo \url{http://www.math.unipd.it/~tullio/IS-1/2012/Progetto/C1.pdf};
\item[] testo di consultazione: \textit{Software Engineering (8th edition) Ian Sommerville, Pearson Education | Addison Wesley};
\item[] manuale all'utilizzo dei design pattens: \textit{Design Patterns, Elementi per il riuso di software a oggetti - (1/Ed. italiana) Eric Gamma, Richard Helm, Ralph Johnson, John Vlissides, Pearson Education};
\item[] \textit{glossario.1.0.pdf} allegato.
\end{itemize}

\section{Strumenti utilizzati}
\subsection{Java}

\subsection{Hibernate}

%TODO: da completare

\clearpage
\section{Design Pattern}
In questa sezione discuteremo i design pattern utilizzati nella progettazione delle componenti. Ogni design pattern sarà proposto con la seguente forma:

\begin{itemize}
	\item \textbf{Scopo}: verrà proposto lo scopo generico del pattern, al fine di evidenziare subito la sua utilità.
	\item \textbf{Diagramma esemplificativo}: si riporterà lo schema UML, rappresentante un implementazione generica del design pattern in esame.
	\item \textbf{Vantaggi derivanti}: si darà un elenco dei vantaggi apportati dall'utilizzo del pattern, in particolare sotto il profilo della manutenzione e del riuso del codice.
	\item \textbf{Componenti che lo implementano}: infine verranno elencati i componenti dell'architettura di sistema, che implementano il pattern descritto.
\end{itemize}

Per una visione d'insieme dei delle componenti utilizzate da un pattern, e dei pattern utilizzati da un componente, rimandiamo alle sottosezioni ``Tracciamenti Componenti-Design Pattern'' e ``Tracciamenti Design Pattern-Componenti'' della sezione ``Tracciamenti''.


%\subsection{Adapter}
%\subsubsection{Scopo}
%Convertire l'interfaccia di una classe in un altra interfaccia richiesta dal \underline{client} e consente a classi diverse di operare insieme quando ciò non sarebbe altrimenti possibile a causa di interfacce incompatibili.
%\subsubsection{Diagramma esemplificativo}
%\begin{figure}[h]
%\centering
%\includegraphics[width=.8\textwidth]{adapter}
%\caption{Diagramma ad alto livello del pattern Adapter.}\label{fig:adapter}
%\end{figure}
%\subsubsection{Vantaggi derivanti}
%\begin{itemize}
%\item consente di adattare una classe esistente senza doverla ridefinire;
%\item un unico oggetto può adattare più classi.
%\end{itemize}
%\subsubsection{Componenti che lo implementano}

\subsection{Composite}
\subsubsection{Scopo}
Il pattern Composite ha lo scopo di comporre oggetti in strutture ad albero al fine di rappresentare gerarchie parte-tutto e consentire ai \underline{client} di trattare oggetti singoli e composizioni in modo uniforme. Permette inoltre di gestire strutture dati gerarchicizzate con elementi ``foglie'' ed elementi ``contenitori'', l'ideale per la struttura ``gruppo'' e ``utente''.
\subsubsection{Diagramma esemplificativo}
\begin{figure}[h]
\centering
\includegraphics[width=.8\textwidth]{composite}
\caption{Diagramma ad alto livello del pattern Composite.}\label{fig:composite}
\end{figure}
\subsubsection{Componenti che lo implementano}
\begin{description}
\item{\bfseries\scshape Gestione della rubrica (lato server)}\\
Composite permette di trattare in maniera omogenea singoli oggetti e oggetti composti, come gli utenti e gruppi di utenti della rubrica. Inoltre, visto che rende più semplice l'aggiunta di componenti, permetterebbe in futuro l'integrazione di nuove tipologie di utenti senza la necessità di modificare la struttura preesistente.

Lo svantaggio principale che comporta l'uso di Composite è la mancanza di limiti nell'aggiunta di nuove tipologie di componenti. Per far fronte a questo rischio si è introdotta la classe \texttt{org.softwaresynthesis.mytalk.server.abook.AddressBook} che controlla l'accesso alla struttura dati corrispondente alla rubrica.
\end{description}

\subsection{Data Access Object (DAO)}
\subsubsection{Scopo}
Il pattern DAO ha lo scopo di disaccoppiare la logica di business dalla logica di accesso ai dati. Questo si ottiene spostando la logica di accesso ai dati dai componenti di business ad una classe DAO rendendo i componenti che implementano la logica di business indipendenti dalla natura del dispositivo di persistenza. Questo approccio garantisce che un eventuale cambiamento del dispositivo di persistenza non comporti modifiche sui componenti di business.

\subsubsection{Diagramma esemplificativo}
\begin{figure}[h]
\centering
\includegraphics[width=.8\textwidth]{dao}
\caption{Diagramma ad alto livello del pattern DataAccessObject.}\label{fig:dao}
\end{figure}

\subsubsection{Componenti che lo implementano}
\begin{description}
\item{\scshape\bfseries Gestione Database}\\
Le classi DAO consentono di isolare l'accesso alle tabelle del database dalla parte di business logic facendo corrispondere alle invocazioni di metodo le opportune operazioni sui record del database. L'utilizzo di tale pattern crea inoltre un maggiore livello di astrazione e mantiene una rigida separazione tra i sottosistemi corrispondenti a model e presenter.
\end{description}

\subsection{Façade}
\subsubsection{Scopo}
Fornire un interfaccia unificata per un insieme di interfacce presenti in un sottosistema. Façade definisce un interfaccia di livello più alto che rende il sottosistema più semplice da utilizzare.

\subsubsection{Diagramma esemplificativo}
\begin{figure}[h]
\centering
\includegraphics[width=.8\textwidth]{facade}
\caption{Diagramma ad alto livello del pattern Facade.}\label{fig:facade}
\end{figure}

\subsubsection{Componenti che lo implementano}
\begin{description}
  \item{\scshape \bfseries Façade del server}\\
  L'uso di Façade permette di esporre verso i client una sorta di interfaccia semplificata nascondendo i componenti del sottosistema \texttt{server}, fornendo un punto di accesso centralizzato e riducendo il numero di dipendenze funzionali fra le classi del server e i componenti appartenenti a sottosistemi esterni.
  \item{\scshape \bfseries Façade del presenter}\\
Tramite questo design pattern si introduce un livello di indirettezza fra il sottosistema \texttt{clientpresenter} e \texttt{clientview} con il vantaggio di rendere i due sottosistemi indipendenti.
\end{description}

\subsection{Factory Method}
\subsubsection{Scopo}
Definisce un'interfaccia per la creazione di un oggetto, lasciando alle sottoclassi la decisione sulla classe che deve essere istanziata e consente di deferire l'istanziazione di una classe alle sottoclassi.

\subsubsection{Diagramma esemplificativo}
\begin{figure}[h]
\centering
\includegraphics[width=.8\textwidth]{factory_method}
\caption{Diagramma ad alto livello del pattern Factory Method.}\label{fig:factory_method}
\end{figure}

\subsubsection{Componenti che lo implementano}
\begin{description}
  \item{\scshape\ttfamily Facade del server}\\
Factory Method permette ai client di ottenere con facilità degli oggetti proxy che specializzano le interfacce \texttt{server.dao.IAudioMessage}, \texttt{server.dao.IAudioVideoMessage} e \texttt{server.dao.IUserData}. Questo permette di ridurre il traffico di rete in quanto oggetti potenzialmente di grandi dimensioni rimangono sul server e vengono scaricati solo quando se ne presenta l'effettiva necessità.
%TODO: Un altro oggetto specializzato che si ottiene mediante il ServerFacade è l'Adapter per la connessione di rete (da definire!)
\end{description}

\subsection{Model-View-Presenter}
\subsubsection{Scopo}
Il pattern architetturale \foreignlanguage{english}{Model-View-Presenter} similmente a quanto accade per \foreignlanguage{english}{Model-View-Controller} (MVC), ha lo scopo di mantenere separata la \textit{business logic}, cioè la gestione dei dati secondo le regole di un determinato dominio e la loro memorizzazione in forma persistente, dalla presentazione e manipolazione mediante interfaccia utente.
\subsubsection{Diagramma esemplificativo}
\begin{figure}[h]
\centering
\includegraphics[width=.8\textwidth]{mvpHLdiagram}
\caption{Diagramma ad alto livello del pattern MVP.}\label{fig:mvpHL}
\end{figure}

\subsubsection{Componenti che lo implementano}
MVP viene utilizzato come il pattern più ad alto livello del nostro sistema. La distinzione fra \textit{model}, \textit{presenter} e \textit{view} è infatti rispecchiata dalla suddivisione dell'architettura nei tre sottosistemi \texttt{server}, \texttt{clientpresenter} e \texttt{clientview}.

In generale, l'utilizzo di MVP riduce l'accoppiamento tra i sottosistemi minimizzando le modifiche richieste a ognuno di essi come conseguenza di cambiamenti all'interno degli altri.

Inoltre, le componenti di questo sottosistema non sono vincolate a utilizzare la rete per accedere alle informazioni che sono memorizzate sul server quando queste sono già disponibili (e possono essere elaborate) sul client, migliorando l'esperienza utente.

In particolare, le parti del sistema che utilizzano questo pattern corrispondono ai sottosistemi:
\begin{description}
  \item{\ttfamily server}
  \item{\ttfamily clientpresenter}
  \item{\ttfamily clientview} 
\end{description}

\subsection{Observer}
\subsubsection{Scopo}
Definire una dipendenza uno a molti fra oggetti, in modo tale che se un oggetto cambia il suo stato tutti gli oggetti dipendenti da questo siano notificati e aggiornati automaticamente.

\subsubsection{Diagramma esemplificativo}
\begin{figure}[h]
\centering
\includegraphics[width=.7\textwidth]{observer}
\caption{Diagramma ad alto livello del pattern Observer.}\label{fig:observer}
\end{figure}

\subsubsection{Componenti che lo implementano}
\begin{description}
  \item{\scshape\ttfamily Gestione stato}\\
Il pattern Observer è utile in quanto permette agli utenti di osservare lo stato degli altri, ricevendo in modo automatico e trasparente una notifica nel caso in cui quest'ultimo subisse variazioni, essendo ogni utente sia osservato che osservatore. Al momento della connessione, infatti, ogni utente si registra come osservatore sui suoi contatti che sono online e, al contempo, li aggiunge tra i propri osservatori. In tal modo gli utenti notificano in broadcast le loro variazioni di stato e sono sempre aggiornati sullo stato dei contatti della loro rubrica.
\end{description}

\subsection{Proxy}
\subsubsection{Scopo}
Fornisce un placeholder per un altro oggetto in modo da controllarne l'accesso e consentire un uso ottimizzato della memoria.

\subsubsection{Diagramma esemplificativo}
\begin{figure}
\includegraphics[width=.8\textwidth]{proxy}
\caption{Diagramma ad alto livello del pattern Proxy.}\label{fig:proxy}
\end{figure}

\subsubsection{Componenti che lo implementano}
\begin{description}
  \item{\bfseries\scshape Gestione rubrica}\\
L'utilizzo di un proxy al posto di un utente permette di raggiungere una maggiore efficienza limitando l'utilizzo della rete evitando all'utente di percepire una eccessiva lentezza che comprometterebbe la sua esperienza. Inoltre, tramite un proxy è possibile controllare l'accesso ai dati (che risiedono nel server) garantendo un migliore livello di protezione.
  \item{\bfseries\scshape Gestione segreteria}\\
Poiché i messaggi audio e, soprattutto, i messaggi audio/video possono essere di grandi dimensioni, i proxy permettono di ottimizzare il consumo della memoria e di evitare l'attesa da parte dell'utente se non strettamente necessario.
\end{description}

\subsection{Singleton}
\subsubsection{Scopo}
Il pattern creazionale Singleton, garantisce che una determinata classe possa essere istanziata una sola volta, e di fornirne un punto di accesso globale. Questo pattern va utilizzato negli ambiti in cui si ha la necessità che l'accesso ad una determinata entità sia unico, in modo da permettere la gestione ottimale della risorsa stessa.

\subsubsection{Diagramma esemplificativo}
\begin{figure}[h]
\centering
\includegraphics[width=.8\textwidth]{singleton}
\caption{Diagramma ad alto livello del pattern Singleton.}\label{fig:singleton}
\end{figure}

\subsubsection{Componenti che lo implementano}
\begin{description}
  \item{\scshape\bfseries Façade del server}\\
Il pattern Singleton pone un limite superiore stretto al numero di istanze che possono esistere di una determinata classe e perciò è utile utilizzarlo per poter controllare il numero di oggetti \texttt{server.StandardServerFacade} che in questo caso è pari a uno. L'unicità dell'oggetto façade garantisce la presenza di un solo punto di accesso alle funzionalità del sottosistema \texttt{server}.
  \item{\scshape\bfseries Façade del presenter}
Il pattern Singleton è altresì utile nella componente \texttt{client.StandardPresenterFacade} per gli stessi motivi evidenziati nella componente precedente, ossia l'unicità del punto di accesso alle funzionalità del sottosistema \texttt{clientpresenter}.
\end{description}

\subsection{State}
\subsubsection{Scopo}
Permette ad un oggetto di cambiare il suo comportamento al variare del suo stato interno, quindi a run-time. L'oggetto si comporterà come se avesse cambiato la sua classe.

\subsubsection{Diagramma esemplificativo}
\begin{figure}[h]
\centering
\includegraphics[width=.7\textwidth]{state}
\caption{Diagramma ad alto livello del pattern State.}\label{fig:state}
\end{figure}

\subsubsection{Componenti che lo implementano}
\begin{description}
\item{Gestione dello Stato}\\
Il pattern State permette di gestire gli utenti del sistema determinando un comportamento diverso per questi ultimi a seconda del loro stato. È stata definita un'apposita gerarchia di stati che permette quindi di specializzare nella maniera più adatta alle necessità del sistema le operazioni sugli utenti senza bisogno di condizionali annidati.
\end{description}
\clearpage

\section{Introduzione all'architettura di sistema}
\clearpage

\section{Architettura MyTalk-Server}
%TODO: da completare
Infine, si fa notare che i nomi di tutte le classi riportate nella sezione sono implicitamente parte del package \texttt{org.softwaresynthesis.mytalk.server} pertanto tale prefisso sarà omesso nella loro denominazione.

\subsection{Componenti evidenziate}

\subsubsection{Gestione Database}
\begin{description}
\item{\scshape\bfseries Descrizione:}\\
Gestione Database è la componente che si occupa di rappresentare la struttura del database relazionale su cui poggia l'applicativo. Le singole classi in esso definite rappresentano quindi le tabelle del database. In termini tecnici Gestione Database implementa il design pattern DAO\@.

Tramite questa componente, il sistema potrà quindi effettuare operazione di lettura e scrittura di entità all'interno del database. Le classi che costituiscono la componente dovranno quindi essere dotate di:

\begin{itemize}
	\item metodi getter per restituire i singoli attributi dell'istanza;
	\item metodi setter per garantire un corretto inserimento dei dati prima di registrare l'istanza nel database.
\end{itemize}

Si informa inoltre che le classi di tale componente dovranno interagire con il \underline{framework} Hibernate, al fine di ottenere lo scopo precisato.

	\item{\scshape\bfseries Diagramma delle classi:}
	\item{\scshape\bfseries Classi utilizzate:}
\begin{itemize}
  \item \texttt{dao.AudioMessage}
  \item \texttt{dao.AudioVideoMessage}
  \item \texttt{dao.IAudioMessage}
  \item \texttt{dao.IAudioVideoMessage}
  \item \texttt{dao.IGroup}
  \item \texttt{dao.IUserData}
  \item \texttt{dao.StandardGroup}
  \item \texttt{dao.StandardUserData}
\end{itemize}
\end{description}

\subsubsection{Gestione connessione}
\begin{description}
	\item{\scshape\bfseries Descrizione:}\\
Tale componente ingloba le classi destinate a stabilire le routine di connessione. Il server ha la consapevolezza degli oggetti rappresentati una connessione tra client. Si ricorda infatti che: ``ogni cliente per comunicare con altri, deve prima connettersi al server e richiedere a questo un oggetto rappresentante la linea di comunicazione con il destinatario''.

Le specifiche di tale oggetto sono descritte dall'interfaccia \texttt{connection.IChannelInfo}, di cui si fornisce un'implementazione standard \texttt{connection.WebRTCInfo}.

%Al fine di rendere l'architettura di tale componente il più manutenibile possibile, si è deciso di usare il pattern Adapter. In pratica esternamente il sistema ha esclusivamente la conoscenza dell'interfaccia \texttt{connection.ISocket}. Quindi l'esterno è svincolato dal conoscere il tipo preciso di protocollo di connessione (che ricordiamo nel nostro caso essere \texttt{da\_definire}). Poiché la classe che implementa il protocollo di comunicazione esiste già, nasce l'esigenza di ``adattare'' tale classe alla nostra interfaccia (\texttt{connection.ISocket}). Ciò giustifica l'esistenza della classe \texttt{connection.WebSocketAdapter}.
	\item{\scshape\bfseries Diagramma delle classi:}
	\item{\scshape\bfseries Classi utilizzate:}
\begin{itemize}
  \item \texttt{connection.IChannelInfo}
  \item \texttt{connection.WebRTCInfo}
\end{itemize}
\end{description}

\subsubsection{Gestione rubrica}
\begin{description}
	\item{\scshape\bfseries Descrizione:}\\
La rubrica è organizzata in gruppi e sono previste due categorie di default: la \textit{blacklist} e la \textit{whitelist}. L'utente può aggiungere ulteriori gruppi in base alle sue esigenze ma esclusivamente all'interno della \textit{whitelist}. Per trattare in maniera omogenea i gruppi di contatti e i singoli contatti si è utilizzato il design pattern Composite.

In particolare, \texttt{abook.IContact} rappresenta l'interfaccia principale comune a ogni tipologia di contatto e viene estesa dalle due interfacce \texttt{dao.IUserData} e \texttt{dao.IGroup}. La componente comprende anche l'interfaccia \texttt{abook.IAddressBook} e la relativa implementazione \texttt{abook.AddressBook}. Nell'implementazione specificata \texttt{abook.Addressbook} vincola il sistema a garantire che ogni utente abbia i due gruppi di default sopra descritti, come richiesto dai requisiti.

Infine, la classe \texttt{abook.UserDataProxy} viene utilizzata come proxy per lo scambio di dati fra il sottosistema \texttt{server} e il sottosistema \texttt{clientpresenter}.
	\item{\scshape\bfseries Diagramma delle classi:}
	\item{\scshape\bfseries Classi utilizzate:}\\
\begin{itemize}
  \item \texttt{abook.AddressBook}
  \item \texttt{abook.IAddressBook}
  \item \texttt{abook.IContact}
  \item \texttt{abook.UserDataProxy}
  \item \texttt{dao.IGroup}
  \item \texttt{dao.IUserData}
\end{itemize}
\end{description}

\subsubsection{Gestione stato}
\begin{description}
	\item{\scshape\bfseries Descrizione:}\\
Le classi di tale componente sono utilizzate per gestire lo stato degli utenti, permettendo un comportamento diverso delle istanze di \texttt{dao.StandardUserData} a seconda dello stato in cui si trova l'utente corrispondente. Gli stati possibili sono ``online'' e ``offline'' che sono rappresentati dalle classi \texttt{state.StateOnline} e \texttt{state.StateOffline} rispettivamente.
	
Gli utenti che si trovano nello stato online possono trovarsi in due situazioni: ``occupato'' o ``disponibile'', rappresentati a loro volta dalle classi \texttt{state.StateOccupied} e \texttt{state.StateAvailable}. L'utente si troverà nello stato ``occupato'' solo se è impegnato in una conversazione. Lo stato può essere controllato solo dal sistema in risposta agli eventi di connessione e di chiamata ma non è direttamente accessibile dall'utente.
	
Ad esempio, la chiamata viene trattata in modo differente a seconda che l'utente si trovi nello stato ``disponibile'' o ``occupato''/``offline'', dal momento che nel primo caso la chiamata va a buon fine mentre nel secondo verrà attivato il meccanismo di segreteria telefonica.
	
Inoltre, i cambiamenti di stato vengono notificati a tutti gli utenti presenti in rubrica tali che si trovano nello stato online.
	\item{\scshape\bfseries Diagramma delle classi:}
	\item{\scshape\bfseries Classi utilizzate:}\\ 
	\begin{itemize}
          \item \texttt{dao.StandardUserData}
          \item \texttt{state.IState}
          \item \texttt{state.StateAvailable}
          \item \texttt{state.StateOccupied}
          \item \texttt{state.StateOffline}
          \item \texttt{state.StateOnline}
	\end{itemize}
\end{description}

\subsubsection{Gestione segreteria}
\begin{description}
	\item{\scshape\bfseries Descrizione:}\\
Il sistema segreteria telefonica corrisponde all'interfaccia \texttt{message.IMessageBox} e alla relativa implementazione \texttt{message.StandardMessageBox} che permettono un accesso centralizzato all'insieme di messaggi che un determinato utente ha ricevuto.

I messaggi audio e audio/video sul server sono rappresentati dalle classi \texttt{dao.IAudioMessage} (implementata da \texttt{dao.AudioMessage}) e \texttt{dao.IAudioVideoMessage} (implementata da \texttt{dao.AudioVideoMessage}) rispettivamente. L'onere di caricare in memoria e gestire l'interno contenuto del messaggio è posticipato al momento di effettiva necessità mediante l'utilizzo dei \textit{virtual proxy} corrispondenti alle classi \texttt{message.AudioMessageProxy} e \texttt{message.AudioVideoMessageProxy}.
	\item{\scshape\bfseries Diagramma delle classi:}
	\item{\scshape\bfseries Classi utilizzate:}
\begin{itemize}
  \item \texttt{dao.AudioMessage}
  \item \texttt{dao.AudioVideoMessage}
  \item \texttt{dao.IAudioMessage}
  \item \texttt{dao.IAudioVideoMessage}
  \item \texttt{message.AudioMessageProxy}
  \item \texttt{message.AudioVideoMessageProxy}
\end{itemize}
\end{description}

\subsubsection{Façade del server}
\begin{description}
	\item{\scshape\bfseries Descrizione:}\\
L'interfaccia \texttt{IServerFacade} e la relativa implementazione \texttt{StandardServerFacade}, nella quale si è scelto di applicare il design pattern Singleton, forniscono una sorta di interfaccia alle funzionalità offerte dal sottosistema server alle componenti che risiedono nel client. Le funzionalità esposte consentono di gestire i messaggi presenti in segreteria, le richieste di comunicazione con altri utenti il login/registrazione degli utenti.
	\item{\scshape\bfseries Diagramma delle classi:}
	\item{\scshape\bfseries Classi utilizzate:}\\
\begin{itemize}
  \item \texttt{IServerFacade}
  \item \texttt{StandardServerFacade}
\end{itemize}
\end{description}

\subsection{Diagramma del package}

\subsection{Diagramma delle classi}
\clearpage

\section{Architettura MyTalk-client Universale}

\subsection{Componenti evidenziate}
\subsubsection{Gestione comunicazione}
\begin{description}
	\item{\scshape\bfseries Descrizione:} 
	\item{\scshape\bfseries Diagramma delle classi:}
	\item{\scshape\bfseries Classi utilizzate:} 
\end{description}

\subsubsection{Façade del presenter}
\begin{description}
	\item{\scshape\bfseries Descrizione:} 
	\item{\scshape\bfseries Diagramma delle classi:}
	\item{\scshape\bfseries Classi utilizzate:} 
\end{description}

\subsection{Diagramma del package}

\subsection{Diagramma delle classi}
\clearpage

\section{Architettura MyTalk-clientSoftwareSynthesis}

\subsection{Componenti evidenziate}

\subsubsection{Template della componente X}
\begin{description}
	\item{\scshape\bfseries Descrizione:} 
	\item{\scshape\bfseries Diagramma del package:}
	\item{\scshape\bfseries Classi utilizzate:} 
\end{description}

\subsection{Classi utilizzate}

%\subsubsection{Template classe X}
%\begin{description}
%	\item{\scshape\bfseries Descrizione:} 
%	\item{\scshape\bfseries Diagramma della classe:}
%	\item{\scshape\bfseries Componenti che ne fanno uso:} 
%\end{description}

\subsection{Diagramma del package}

\subsection{Diagramma delle classi}
\clearpage

\section{Descrizione delle classi}

\subsection{Package org.softwaresynthesis.mytalk.server.dao}
\subsubsection{IAudioMessage}
\begin{description}
	\item{\scshape\bfseries Descrizione:}\\
Interfaccia per i messaggi audio della segreteria telefonica che viene implementata dalle classi \texttt{AudioMessage} e dal suo proxy \texttt{message.AudioMessageProxy}.
% contiene operazione astratta play() per riprodurre il messaggio (se presente) o scaricarlo e avviarne la riproduzione
	\item{\scshape\bfseries Componenti che ne fanno uso:}\\
	  \begin{itemize}
	    \item[-] Gestione segreteria
	  \end{itemize}
\end{description}

\subsubsection{IAudioVideoMessage}
\begin{description}
	\item{\scshape\bfseries Descrizione:}\\
Interfaccia per i messaggi audio/video della segreteria telefonica, a sua volta implementata dalle classi \texttt{AudioVideoMessage} e dal relativo proxy \texttt{message.AudioVideoMessage}.
% contiene operazione astratta play() per riprodurre il messaggio (se presente) o scaricarlo e avviarne la riproduzione
	\item{\scshape\bfseries Componenti che ne fanno uso:} 
	  \begin{itemize}
	    \item[-] Gestione segreteria
	  \end{itemize}
\end{description}

\subsubsection{AudioMessage}
\begin{description}
	\item{\scshape\bfseries Descrizione:}\\
Classe che rappresenta un messaggio audio nella segreteria telefonica di un utente, implementa l'interfaccia \texttt{IAudioMessage}.
	\item{\scshape\bfseries Componenti che ne fanno uso:}
	  \begin{itemize}
	    \item[-] Gestione segreteria
	  \end{itemize}
\end{description}

\subsubsection{AudioVideoMessage}
\begin{description}
	\item{\scshape\bfseries Descrizione:}\\
Classe che rappresenta un messaggio audio/video nella segreteria telefonica di un utente, implementa l'interfaccia \texttt{IAudioVideoMessage}.
	\item{\scshape\bfseries Componenti che ne fanno uso:}
	  \begin{itemize}
	    \item[-] Gestione segreteria
	  \end{itemize}
\end{description}

\subsubsection{IGroup}
\begin{description}
	\item{\scshape\bfseries Descrizione:}\\
Interfaccia per i gruppi interni alla rubrica, prevede un'operazione \texttt{add(IUserData)} per l'aggiunta di un nuovo contatto al gruppo e un'operazione \texttt{remove(IUserData)} per la sua rimozione.
	\item{\scshape\bfseries Componenti che ne fanno uso:} 
	  \begin{itemize}
	    \item Gestione rubrica
	  \end{itemize}
\end{description}

\subsubsection{StandardGroup}
\begin{description}
	\item{\scshape\bfseries Descrizione:}\\
Implementazione dell'interfaccia \texttt{IGroup}.
	\item{\scshape\bfseries Componenti che ne fanno uso:} 
\end{description}

\subsubsection{IUserData}
\begin{description}
	\item{\scshape\bfseries Descrizione:} 
	\item{\scshape\bfseries Componenti che ne fanno uso:} 
\end{description}

\subsubsection{StandardUserData}
\begin{description}
	\item{\scshape\bfseries Descrizione:} 
	\item{\scshape\bfseries Componenti che ne fanno uso:} 
\end{description}

\subsection{Package org.softwaresynthesis.mytalk.server.connection}
\subsubsection{ISocket}
\begin{description}
	\item{\scshape\bfseries Descrizione:} 
	\item{\scshape\bfseries Componenti che ne fanno uso:} 
\end{description}

\subsubsection{WebSocketAdapter}
\begin{description}
	\item{\scshape\bfseries Descrizione:} 
	\item{\scshape\bfseries Componenti che ne fanno uso:} 
\end{description}

\subsection{Package org.softwaresynthesis.mytalk.server.abook}
\subsubsection{IContact}
\begin{description}
	\item{\scshape\bfseries Descrizione:} 
	\item{\scshape\bfseries Componenti che ne fanno uso:} 
\end{description}

\subsubsection{IAddressBook}
\begin{description}
	\item{\scshape\bfseries Descrizione:} 
	\item{\scshape\bfseries Componenti che ne fanno uso:} 
\end{description}

\subsubsection{AddressBook}
\begin{description}
	\item{\scshape\bfseries Descrizione:} 
	\item{\scshape\bfseries Componenti che ne fanno uso:} 
\end{description}

\subsection{Package org.softwaresynthesis.mytalk.server.state}
\subsubsection{IState}
\begin{description}
	\item{\scshape\bfseries Descrizione:} 
	\item{\scshape\bfseries Componenti che ne fanno uso:} 
\end{description}

\subsubsection{StateOnline}
\begin{description}
	\item{\scshape\bfseries Descrizione:} 
	\item{\scshape\bfseries Componenti che ne fanno uso:} 
\end{description}

\subsubsection{StateOffline}
\begin{description}
	\item{\scshape\bfseries Descrizione:} 
	\item{\scshape\bfseries Componenti che ne fanno uso:} 
\end{description}

\subsubsection{StateAvailable}
\begin{description}
	\item{\scshape\bfseries Descrizione:} 
	\item{\scshape\bfseries Componenti che ne fanno uso:} 
\end{description}

\subsubsection{StateOccupied}
\begin{description}
	\item{\scshape\bfseries Descrizione:} 
	\item{\scshape\bfseries Componenti che ne fanno uso:} 
\end{description}

\subsection{Package org.softwaresynthesis.mytalk.server.message}
\subsubsection{IMessageBox}
\begin{description}
	\item{\scshape\bfseries Descrizione:} 
	\item{\scshape\bfseries Componenti che ne fanno uso:} 
\end{description}

\subsubsection{StandardMessageBox}
\begin{description}
	\item{\scshape\bfseries Descrizione:} 
	\item{\scshape\bfseries Componenti che ne fanno uso:} 
\end{description}

\subsubsection{AudioMessageProxy}
\begin{description}
	\item{\scshape\bfseries Descrizione:} 
	\item{\scshape\bfseries Componenti che ne fanno uso:} 
\end{description}

\subsubsection{AudioVideoMessageProxy}
\begin{description}
	\item{\scshape\bfseries Descrizione:} 
	\item{\scshape\bfseries Componenti che ne fanno uso:} 
\end{description}

\subsection{Package org.softwaresynthesis.mytalk.server}
\subsubsection{IServerFacade}
\begin{description}
	\item{\scshape\bfseries Descrizione:} 
	\item{\scshape\bfseries Componenti che ne fanno uso:} 
\end{description}

\subsubsection{StandardServerFacade}
\begin{description}
	\item{\scshape\bfseries Descrizione:} 
	\item{\scshape\bfseries Componenti che ne fanno uso:} 
\end{description}

\subsection{Package org.softwaresynthesis.mytalk.client}
\subsubsection{IClient}
\begin{description}
	\item{\scshape\bfseries Descrizione:} 
	\item{\scshape\bfseries Componenti che ne fanno uso:} 
\end{description}

\subsubsection{StandardClient}
\begin{description}
	\item{\scshape\bfseries Descrizione:} 
	\item{\scshape\bfseries Componenti che ne fanno uso:} 
\end{description}

\subsubsection{IPresenterFacade}
\begin{description}
	\item{\scshape\bfseries Descrizione:} 
	\item{\scshape\bfseries Componenti che ne fanno uso:} 
\end{description}

\subsubsection{mytalk.client.StandardPresenterFacade}
\begin{description}
	\item{\scshape\bfseries Descrizione:} 
	\item{\scshape\bfseries Componenti che ne fanno uso:} 
\end{description}
\clearpage

\section{Conclusioni sull'architettura}

\subsection{Diagrammi delle attività}

\subsection{Diagrammi di sequenza}
\clearpage

\section{Tracciamenti}
Nella seguente sezione vengono proposti tutti i tracciamenti eseguiti mediante il sistema Synthsis Requirment Manager. I tracciamenti proposti sono giustificati dalle seguenti due motivazioni:

\begin{itemize}
	\item Dimostrare il soddisfacimento per necessarietà e sufficienza della corrispondenza tra gli elementi tracciati (e.g. una componente deve rispondere necessariamente alle esigenze di uno o più requisiti, tali insomma che ne giustifichino l'esistenza. D'altro canto è richiesto che ogni requisito definito in fase d'analisi sia soddisfatto e risolto da almeno una componente).
	\item dare una lettura generale delle varie: componenti, requisiti, design pattern e classi.
\end{itemize}

\subsection{Tracciamenti Requisiti-Componenti}

\subsection{Tracciamenti Componenti-Requisiti}

\subsection{Tracciamenti Componenti-DesignPattern}

\subsection{Tracciamenti DesignPattern-Componenti}

\subsection{Tracciamenti Componenti-Classi}

\subsection{Tracciamenti Classi-Componenti}

\end{document}