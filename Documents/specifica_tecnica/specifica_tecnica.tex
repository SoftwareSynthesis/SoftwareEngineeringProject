% **************************************************
% Macro specifiche per il documento corrente
% **************************************************
% Nome
\newcommand{\docName}{Specifica tecnica}
% Nome file
\newcommand{\docFileName}{specifica\_tecnica.1.0.pdf}
% Versione
\newcommand{\docVers}{0.1}
% Data creazione
\newcommand{\creationDate}{2013-01-16}
% Data ultima modifica
\newcommand{\modificationDate}{2013-01-17}
% Stato in {Approvato, Non approvato}
\newcommand{\docState}{Non approvato}
% Uso in {Interno, Esterno}
\newcommand{\docUsage}{Interno}
% Destinatari da specificare come nome1\\ &nome2\\ ecc.
\newcommand{\docDistributionList}{Team SoftwareSynthesis}
% Redattori da specificare come nome1\\ &nome2\\ ecc.
\newcommand{\docAuthors}{}
% Approvato da
\newcommand{\approvedBy}{}
% Verificatori
\newcommand{\verifiedBy}{}
% Perscorso (relativo o assoluto) che punta alla directory contenente shared/
% come sua sottodirectory (per comodità chiamiamola 'doc root').
\newcommand{\docRoot}{..}
% definire se si vuole l'indice delle tabelle
\def\INDICETABELLE{false}
% definire se si vuole l'indice delle figure
\def\INDICEFIGURE{false}

% importa il preambolo condiviso da tutti i documenti
% shared/preamble.tex
%
% Questo documento contiene la parte del preambolo condivisa e viene pertanto
% richiamato nel 'master' di tutti i documenti di progetto.  Al suo interno
% contiene le inclusioni (e le configurazioni) di tutti i package richiesti per
% la compilazione dei documenti, le macro di carattere generale e la definizione
% degli stili di pagina.

\documentclass[a4paper,10pt]{article}

% **************************************************
% Macro generiche
% **************************************************
\newcommand{\team}{Software Synthesis}                    % chi siamo
\newcommand{\email}{info@softwaresynthesis.org}           % e-mail
\newcommand{\caName}{MyTalk}                              % titolo capitolato
\newcommand{\manager}{SynthesisRequirementManager}        % nome del sistema di tracciamento
\newcommand{\memberdata}[1]{%
  \texttt{\textcolor{RedOrange}{#1}}}                     % attributi di una classe
\newcommand{\method}[1]{\texttt{\textcolor{Emerald}{#1}}} % metodi di una classe
\newcommand{\exception}[1]{%
  \texttt{\textcolor{RedViolet}{#1}}}                     % eccezione
% \newcommand{\handler}[1]{\texttt{\textcolor{Maroon}{#1}}} % per gli event handler
\newcommand{\inglese}[1]{%
  \foreignlanguage{english}{\textit{#1}}}                 % per i testi in lingua inglese
\newcommand{\purpose}{%                                     scopo del prodotto
Con il progetto ``\caName'' si intende un sistema software di comunicazione tra utenti mediante \underline{browser} senza la necessit{\`a} di installazione di \underline{plugin} e/o software esterni. L'utilizzatore avr{\`a} la possibilit{\`a} di interagire con un altro utente tramite una comunicazione audio - audio/video - testuale e, inoltre, ottenere delle statistiche sull'attivit{\`a} in tempo reale.%
}
\newcommand{\glossaryIntro}{%                               introduzione al glossario
Al fine di evitare incomprensioni dovute all'uso di termini tecnici nei documenti, viene redatto e allegato il documento \textit{glossario.4.0.pdf} dove vengono definiti e descritti tutti i termini marcati con una sottolineatura.%
}


% **************************************************
% Codifica e lingua dei documenti
% **************************************************
\usepackage[utf8x]{inputenc}                              % codifica caratteri dei documenti sorgenti
\usepackage[english,italian]{babel}                       % localizzazione ai fini di sillabazione e cross-references
\usepackage[T1]{fontenc}                                  % codifica font di output

% **************************************************
% Definizione geometria della pagina
% **************************************************
\usepackage[a4paper,head=4cm,top=4.5cm,bottom=3cm,left=3cm,right=3cm,bindingoffset=5mm]{geometry}

% *************************************************
% Intestazioni e piè di pagina personalizzati
% *************************************************
\usepackage{fancyhdr}
% stile normale
\fancypagestyle{normal}{
\fancyhead{}                                              % intestazione
\fancyhead[RE,RO]{
\begin{picture}(0,0)
  \put(-410,0){\includegraphics[width=1.02\textwidth]{header_logo}}
  \put(-410,10){\sffamily\large\leftmark}
\end{picture}
\vspace{-4pt}
}
\renewcommand{\headrulewidth}{0pt}                       % riga sotto l'intestazione
\cfoot{}                                                  % piè di pagina
\fancyfoot[RO,LE]{\sffamily
  pag.~\thepage{} di \pageref{LastPage}}                  % a dx nelle pag. dispari e a sx in quelle pari
\fancyfoot[RE,LO]{\sffamily\docFileName{}}
\renewcommand{\footrulewidth}{.4pt}                       % riga sopra il piè di pagina
}
% stile per gli indici
\fancypagestyle{toc}{
\fancyhead{}                                              % intestazione
\fancyhead[RE,RO]{
\begin{picture}(0,0)
  \put(-410,0){\includegraphics[width=1.02\textwidth]{header_logo}}
\end{picture}
}
\renewcommand{\headrule}{}                                % nessuna riga sotto l'intestazione
\cfoot{}                                                  % piè di pagina
\fancyfoot[RO,LE]{\sffamily\thepage{}}                    % a dx nelle pag. dispari e a sx in quelle pari
\fancyfoot[RE,LO]{\sffamily\docFileName{} -- v.\docVers}
\renewcommand{\footrulewidth}{.4pt}                       % riga sopra il piè di pagina
}

\pagestyle{fancy}                                         % premetto: non so usare bene le marche:
\renewcommand{\sectionmark}[1]{\markboth{#1}{#1}}         % se qualcuno ha idee migliori si faccia avanti!

% **************************************************
% Tabelle
% **************************************************
\usepackage{tabularx}                                     % tabelle di larghezza fissa con una o più colonne variabili
\usepackage{multirow}                                     % colonne con colonne che si estendono per più righe
\usepackage{booktabs}                                     % per inserire l'ambiente table e le righe orizz. nelle tabelle
\usepackage{longtable}			                              % tabelle oltre i limiti di pagina

% **************************************************
% Cross-references e collegamenti ipertestuali
% **************************************************
\usepackage[hidelinks]{hyperref}
\hypersetup{%
  colorlinks=false, linktocpage=false, pdfborder={0,0,0}, pdfstartpage=1, pdfstartview=FitV,%
  urlcolor=Cyan, linkcolor=Cyan, citecolor=Black, %pagecolor=Black,%
  pdftitle={\docName}, pdfauthor={\team}, pdfsubject={}, pdfkeywords={},%
  pdfcreator={pdflatex}, pdfproducer={pdflatex with hyperref package}%
}

% **************************************************
% Immagini e grafica
% **************************************************
\usepackage{graphicx}                                     % supporto ad aspetti avanzati delle immagini
\usepackage[table,usenames,dvipsnames]{xcolor}            % tabelle con righe colorate e alternate
\graphicspath{{\docRoot/pics/}}                           % percorso contenente tutti i file immagini
\usepackage{float}                                        % per rendere non flottanti gli ambienti flottanti
\usepackage[italian]{varioref}                            % testo completo riferimenti in italiano

% **************************************************
% Definizioni di colori
% **************************************************
\definecolor{myBlue}{RGB}{1,167,236}
\definecolor{lightblue}{RGB}{213,243,253}%{119,218,247}
\definecolor{llightblue}{RGB}{229,255,255}

% **************************************************
% Altri pacchetti opzionali
% **************************************************     
\usepackage{lastpage}                                     % per sapere il numero totale di pagine
\usepackage{eurosym}                                      % per il simbolo dell'euro usare \EUR{x} dove x è l'importo
\usepackage{ifthen}                                       % permette la scelta di rami condizionali nella compilazione
\usepackage{enumitem}                                     % permette di configurare gli elenchi puntati e numerati


% Fine del preambolo e inizio del documento
\begin{document}

% Inclusione della prima pagina
% shared/firstpage.tex
%
% Questo documento definisce il contenuto della prima pagina, che si suppone
% essere uguale in tutti i documenti.  Oltre al logo e al titolo, la prima
% pagina contiene i metadati relativi al documento in cui viene inclusa.


% rimuove intestazioni e piè di pagina
\pagestyle{empty}

\begin{center}

% logo del gruppo
\includegraphics[width=1.5\textwidth]{logo}

\vspace{1in}

% titolo del documento
{\Huge\bfseries \docName}

\vspace{1in}

% tabella riepilogativa
\begin{tabularx}{.7\textwidth}{>{\bfseries\sffamily}l>{\sffamily}l}
\toprule
\multicolumn{2}{>{\sffamily}c}{Informazioni sul documento}\\
\midrule
Nome file:            & \docFileName\\
Versione:             & \docVers\\
Data creazione:       & \creationDate\\
Data ultima modifica: & \modificationDate\\
Stato:                & \docState\\
Uso:                  & \docUsage\\
Redattori:            & \docAuthors\\
Approvato da:         & \approvedBy\\
Verificatori:         & \verifiedBy\\
\bottomrule
\end{tabularx}

\end{center}

\newpage


%---------------------------RUOLI----------------------------
%FASE 1:
%Progettisti: TRES, STEFANO, SCHIVO;
%FASE 2:
%Progettisti: DIEGO, ELENA, RIZZI

%Verificatore: Andrea Meneghinello
%Responsabile finale TRES
%------------------------------------------------------------

% Storico delle modifiche
\section*{Storia delle modifiche}
\begin{center}
\begin{longtable}{lp{.32\textwidth}lll}
\toprule
Versione & Descrizione intervento & Membro & Ruolo & Data\\
\midrule % inserire qui il contenuto della tabella
0.2 & Stesura dell'introduzione ai design pattern. Stesura dell'introduzione ai tracciamenti. & Stefano Farronato & Progettista & 2013-01-17\\
0.1 & Creazione del documento e stesura della sezione ``Introduzione''. & Riccardo Tresoldi & Progettista & 2013-01-16\\
\bottomrule
\end{longtable}
\end{center}
\newpage

% inclusione dell'indice
% shared/toc.tex
%
% Questo file contiene le istruzioni che generano l'indice o gli indici del
% documento (utile nel caso in cui decidessimo di avere anche un indice delle
% tabelle e/o un indice delle figure).

% imposta lo stile di pagina per i titoli definito nel preambolo
\pagestyle{toc}
% imposta i numeri di pagina romani minuscoli
\pagenumbering{roman}

% genera automaticamente l'indice di LaTeX
\tableofcontents

% se è true \INDICETABELLE allora genera l'indice delle tabelle, altrimenti non fa nulla
\ifthenelse{\equal{\INDICETABELLE}{true}}{%
  \clearpage % l'indice delle tabelle, se c'è, deve andare a pagina nuova
  \listoftables
}{}

% se è true |INDICEFIGURE allora genera l'indice delle figure, altrimenti non fa nulla
\ifthenelse{\equal{\INDICEFIGURE}{true}}{%
  \clearpage % l'indice delle figure, se c'è, deve andare a pagina nuova
  \listoffigures
}{}

%in ogni caso occorre andare a pagina nuova dopo gli indici
\clearpage


% Alcuni aggiustamenti per le pagine
\pagenumbering{arabic}
\setcounter{page}{1}
\pagestyle{normal}

% Qui ha inizio il documento vero e proprio...

\newpage

\section{Introduzione}
\subsection{Scopo del prodotto}
\purpose

\subsection{Scopo del documento}
Il presente documento è stato redatto al fine di produrre le specifiche sulla progettazione ad alto livello, del prodotto MyTalk. A tal fine il documento presenterà:

\begin{itemize}
	\item Un elenco con le specifiche dei design pattern utilizzati.
	\item Una descrizione dettagliata dei componenti rilevati in fase di progettazione indicando il tipo,
la funzione e l'obbiettivo.
	\item L'architettura d'alto livello del sistema.
	\item I diagrammi UML per definire i flussi principali di controllo dell'applicativo.
	\item Il tracciamento dei requisiti e delle componenti, negli schemi: requisiti-componeneti e componenti-requisiti.
\end{itemize}

\subsection{Glossario}
\glossaryIntro

\clearpage
\section{Riferimenti}

\subsection{Normativi}
\begin{itemize}
\item[] \textit{piano\_di\_qualifica.2.0.pdf} allegato.
\item[] \textit{norme\_di\_progetto.2.0.pdf} allegato.
\item[] \textit{analisi\_dei\_requisiti.2.0.pdf} allegato
\end{itemize}

\subsection{Informativi}
\begin{itemize}
\item[] Capitolato d'appalto: \caName{}, v1.0, redatto e rilasciato dal proponente Zucchetti s.r.l. reperibile all'indirizzo \url{http://www.math.unipd.it/~tullio/IS-1/2012/Progetto/C1.pdf};
\item[] testo di consultazione: \textit{Software Engineering (8th edition) Ian Sommerville, Pearson Education | Addison Wesley};
\item[] manuale all'utilizzo dei design pattens: \textit{Design Patterns, Elementi per il riuso di software a oggetti - (1/Ed. italiana) Eric Gamma, Richard Helm, Ralph Johnson, John Vlissides, Pearson Education};
\item[] \textit{glossario.1.0.pdf} allegato.
\end{itemize}

\clearpage
\section{Design Pattern}
In questa sezione discuteremo i design pattern utilizzati nella progettazione delle componenti. Ogni desing pattern sarà proposto con la seguente forma:

\begin{itemize}
	\item \textbf{Scopo}: verrà proposto lo scopo generico del pattern, al fine di evidenziare subito la sua utilità.
	\item \textbf{Motivazione}: verrà definito perché l'applicativo necessita di tale pattern, in relazione alle componenti che lo dovranno utilizzare.
	\item \textbf{Diagramma esemplificativo}: si riporterà lo schema UML, rappresentante un implementazione generica del design pattern in esame.
	\item \textbf{Vantaggi derivanti}: si darà un elenco dei vantaggi apportati dall'utilizzo del pattern, in particolare sotto il profilo della manutenzione e del riuso del codice.
	\item \textbf{Componenti che lo implementano}: infine verranno elencati i componenti dell'architettura di sistema, che implementano il pattern descritto.
\end{itemize}

Per una visione d'insieme dei delle componenti utilizzate da un pattern, e dei pattern utilizzati da un componente, rimandiamo alle sottosezioni ``Tracciamenti Componenti-DesignPattern'' e ``Tracciamenti DesignPattern-Componenti'' della sezione ``Tracciamenti''.

%\subsection{Template descrizione design pattern}
%\subsubsection{Scopo}
%\subsubsection{Diagramma esemplificativo}
%\subsubsection{Vantaggi derivanti}
%\begin{itemize}
%\item
%\item
%\end{itemize}
%\subsubsection{Componenti che lo implementano}

\subsection{Model-View-Presenter}

\subsubsection{Scopo}
Il pattern architetturale \foreignlanguage{english}{Model-View-Presenter} (d'ora in avanti MVP), similmente a quanto accade per \foreignlanguage{english}{Model-View-Controller} (MVC), ha lo scopo di mantenere separata la \textit{business logic}, cioè la gestione dei dati secondo le regole di un determinato dominio e la loro memorizzazione in forma persistente, dalla presentazione e manipolazione mediante interfaccia utente.

Mantenere questi due aspetti separati ha il vantaggio di ridurre l'accoppiamento fra le componenti del sistema permettendo, ad esempio, di modificare la grafica (View) senza dover per questo preoccuparsi delle operazioni di aggiornamento dei dati.

\subsubsection{Motivazione}
MVP, concepito alla fine degli anni '70, è ritornato in auge con la diffusione di JavaScript come linguaggio di programmazione per la computazione lato \underline{client} allo scopo di superare le difficoltà che emergevano applicando MVC in un contesto distribuito in cui non tutte le componenti risiedono sullo stesso host.

Un'applicazione rigorosa di MVC porterebbe infatti a far risiedere tanto la \textit{business logic} racchiusa nel Model quanto la \textit{application logic} di cui si occupa il Controller sul \underline{server}, lasciando la sola visualizzazione grafica dei dati e la ricezione dell'input utente tramite interfaccia grafica al \underline{client}, che non ha dunque alcuna esigenza di eseguire computazione localmente (\textit{thin client}).

Il problema principale di una simile configurazione sta nel massiccio impiego dell'infrastruttura di rete per gestire le comunicazioni fra la View (isolata sul \underline{client}) e le altre componenti, che implica potenzialmente a un decadimento prestazionale (e quindi una peggiore esperienza utente) nel caso in cui la rete non sia sufficientemente affidabile e/o veloce.

MVP mantiene intatto il principio mutuato da MVC secondo cui i dati e la \textit{business logic} devono risiedere sul \underline{server} e la View sul \underline{client} ma introduce una nuova componente chiamata Presenter, ancora sul \underline{client}, che interagisce con la View modificandola e incapsulando parte della \textit{application logic} da un lato ed è responsabile di effettuare le operazioni sul Model dall'altro.

Il grande vantaggio nell'utilizzo di MVP rispetto a MVC è che per tutte le operazioni che non coinvolgono direttamente il modello non è necessario appoggiarsi alla connessione di rete in quanto tutti i dati e la logica necessaria sono già disponibili sul \underline{client}.

A differenza di MVC, però, in MVP l'accoppiamento fra View e Presenter è più stretto e, in genere, in genere vi è una corrispondenza biunivoca fra l'una e l'altro (mentre in MVC uno stesso Controller può occuparsi di gestire più viste al contempo).

\subsubsection{Diagramma esemplificativo}
\begin{figure}[h]
\centering
\includegraphics[width=.7\textwidth]{mvpHLdiagram}
\caption{Diagramma ad alto livello del pattern MVP.}\label{fig:mvpHL}
\end{figure}


\subsubsection{Vantaggi derivanti}

\subsubsection{Componenti che lo implementano}

\subsection{Singleton}

\subsubsection{Scopo}
Il pattern creazionale Singleton, garantisce che una determinata classe possa essere istanziata una sola volta, e di fornirne un punto di accesso globale. Questo pattern va utilizzato negli ambiti in cui si ha la necessità che l'accesso ad una determinata entità sia unico, in modo da permettere la gestione ottimale della risorsa stessa.

\subsubsection{Diagramma esemplificativo}
\begin{figure}[h]
\centering
\includegraphics[width=.7\textwidth]{singletonHLdiagram}
\caption{Diagramma ad alto livello del pattern Singleton.}\label{fig:singletonHL}
\end{figure}

\subsubsection{Vantaggi derivanti}
\begin{itemize}
\item accesso controllato a un'unica istanza;
\item riduzione dello spazio dei nomi in quanto riduce l'uso di variabili globali;
\item permette di gestire un numero variabile di istanze.
\end{itemize}

\subsubsection{Componenti che lo implementano}

\subsection{State}
\subsubsection{Scopo}
Permette ad un oggetto di cambiare il suo comportamento al variare del suo stato interno, quindi a run-time. L'oggetto si comporterà come se avesse cambiato la sua classe.
\subsubsection{Diagramma esemplificativo}
\begin{figure}[h]
\centering
\includegraphics[width=.8\textwidth]{DesignPatternState}
\caption{Diagramma ad alto livello del pattern State.}\label{fig:state}
\end{figure}
\subsubsection{Vantaggi derivanti}
\begin{itemize}
\item specializza il comportamento associato ad uno stato;
\item rende esplicita la transazione si stato (tale condizione è espressa esplicitamente);
\item condivisione di oggetti di stato.
\end{itemize}

\subsubsection{Componenti che lo implementano}

\subsection{Composite}
\subsubsection{Scopo}
Il pattern Composite ha lo scopo di comporre oggetti in strutture ad albero al fine di rappresentare gerarchie parte-tutto e consentire ai client di trattare oggetti singoli e composizioni in modo uniforme. Permette inoltre di gestire strutture dati gerarchicizzate con elementi "foglie" ed elementi "contenitori", l'ideale per la struttura "gruppo" e "utente".
\subsubsection{Diagramma esemplificativo}
\subsubsection{Vantaggi derivanti}
\begin{itemize}
\item definisce gerarchie di classi costituite da oggetti primitivi e composti;
\item semplifica il client in quanto posso trattare oggetti singoli e strutture astratte in modo uniforme;
\item rende più semplice l'aggiunta di nuovi componenti.
\end{itemize}
\subsubsection{Componenti che lo implementano}

\subsection{Data Access Object (DAO)}
\subsubsection{Scopo}
Il pattern DAO ha lo scopo di disaccoppiare la logica di business dalla logica di accesso ai dati. Questo si ottiene spostando la logica di accesso ai dati dai componenti di business ad una classe DAO rendendo i componenti che implementano la logica di business indipendenti dalla natura del dispositivo di persistenza. Questo approccio garantisce che un eventuale cambiamento del dispositivo di persistenza non comporti modifiche sui componenti di business.
\subsubsection{Diagramma esemplificativo}
\subsubsection{Vantaggi derivanti}
\begin{itemize}
\item stratifica e isola l'accesso ad una tabella dalla parte di business logic;
\item crea un maggiore livello di astrazione;
\item mantiene una rigida separazione tra le componenti di un'applicazione (Model - Controller).
\end{itemize}
\subsubsection{Componenti che lo implementano}

\subsection{Factory Method}
\subsubsection{Scopo}
Definisce un'interfaccia per la creazione di un oggetto, lasciando alle sottoclassi la decisione sulla classe che deve essere istanziata e consente di deferire l'istanziazione di una classe alle sottoclassi.
\subsubsection{Diagramma esemplificativo}
\subsubsection{Vantaggi derivanti}
\begin{itemize}
\item fornisce un punto di aggancio per le sottoclassi per la produzione di una versione specializzata di un oggetto;
\item connette gerarchie di classi parallele;
\end{itemize}
\subsubsection{Componenti che lo implementano}

\subsection{Adapter}
\subsubsection{Scopo}
Convertire l'interfaccia di una classe in un altra interfaccia richiesta dal client e consente a classi diverse di operare insieme quando ciò non sarebbe altrimenti possibile a causa di interfacce incompatibili.
\subsubsection{Diagramma esemplificativo}
\subsubsection{Vantaggi derivanti}
\begin{itemize}
\item consente di adattare una classe esistente senza doverla ridefinire;
\item un unico oggetto può adattare più classi.
\end{itemize}
\subsubsection{Componenti che lo implementano}

\subsection{Façade}
\subsubsection{Scopo}
Fornire un interfaccia unificata per un insieme di interfacce presenti in un sottosistema. Façade definisce un interfaccia di livello più alto che rende il sottosistema più semplice da utilizzare.
\subsubsection{Diagramma esemplificativo}
\subsubsection{Vantaggi derivanti}
\begin{itemize}
\item nasconde i componenti del sottosistema al client in modo da rendere il suo utilizzo più semplice;
\item promuove l'accopiamento lasco fra sistema e client, in modo da non coinvolgere il client in caso di modifiche al sistema stesso.
\end{itemize}
\subsubsection{Componenti che lo implementano}

\clearpage
\section{Introduzione all'architettura di sistema}

\clearpage
\section{Architettura MyTalk-Server}

\subsection{Componenti evidenziate}

\subsubsection{Template della componente X}
\begin{description}
	\item{\scshape\bfseries Descrizione:} 
	\item{\scshape\bfseries Diagramma del package:}
	\item{\scshape\bfseries Classi utilizzate:} 
\end{description}

\subsection{Classi utilizzate}

\subsubsection{Template classe X}
\begin{description}
	\item{\scshape\bfseries Descrizione:} 
	\item{\scshape\bfseries Diagramma della classe:}
	\item{\scshape\bfseries Componenti che ne fanno uso:} 
\end{description}

\subsection{Diagramma del package}

\subsection{Diagramma delle classi}
\clearpage

\section{Architettura MyTalk-ClientUniversale}

\subsection{Componenti evidenziate}

\subsubsection{Template della componente X}
\begin{description}
	\item{\scshape\bfseries Descrizione:} 
	\item{\scshape\bfseries Diagramma del package:}
	\item{\scshape\bfseries Classi utilizzate:} 
\end{description}

\subsection{Classi utilizzate}

\subsubsection{Template classe X}
\begin{description}
	\item{\scshape\bfseries Descrizione:} 
	\item{\scshape\bfseries Diagramma della classe:}
	\item{\scshape\bfseries Componenti che ne fanno uso:} 
\end{description}

\subsection{Diagramma del package}

\subsection{Diagramma delle classi}
\clearpage

\section{Architettura MyTalk-ClientSoftwareSynthesis}

\subsection{Componenti evidenziate}

\subsubsection{Template della componente X}
\begin{description}
	\item{\scshape\bfseries Descrizione:} 
	\item{\scshape\bfseries Diagramma del package:}
	\item{\scshape\bfseries Classi utilizzate:} 
\end{description}

\subsection{Classi utilizzate}

\subsubsection{Template classe X}
\begin{description}
	\item{\scshape\bfseries Descrizione:} 
	\item{\scshape\bfseries Diagramma della classe:}
	\item{\scshape\bfseries Componenti che ne fanno uso:} 
\end{description}

\subsection{Diagramma del package}

\subsection{Diagramma delle classi}
\clearpage

\section{Conclusioni sull'architettura}

\subsection{Diagrammi delle attività}

\subsection{Diagrammi di sequenza}
\clearpage

\section{Tracciamenti}
Nella seguente sezione vengono proposti tutti i tracciamenti eseguiti mediante il sistema Synthsis Requirment Manager. I tracciamenti proposti sono giustificati dalle seguenti due motivazioni:

\begin{itemize}
	\item Dimostrare il soddisfacimento per necessarietà e sufficienza della corrispondenza tra gli elementi tracciati (e.g. una componente deve rispondere necessariamente alle esigenze di uno o più requisiti, tali insomma che ne giustifichino l'esistenza. D'altro canto è richiesto che ogni requisito definito in fase d'analisi sia soddisfatto e risolto da almeno una componente).
	\item dare una lettura generale delle varie: componenti, requisiti, design pattern e classi.
\end{itemize}

\subsection{Tracciamenti Requisiti-Componenti}

\subsection{Tracciamenti Componenti-Requisiti}

\subsection{Tracciamenti Componenti-DesignPattern}

\subsection{Tracciamenti DesignPattern-Componenti}

\subsection{Tracciamenti Componenti-Classi}

\subsection{Tracciamenti Classi-Componenti}

\end{document}