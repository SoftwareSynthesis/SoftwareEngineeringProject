% **************************************************
% Macro specifiche per il documento corrente
% **************************************************
% Nome
\newcommand{\docName}{Manuale di installazione}
% Nome file
\newcommand{\docFileName}{manuale\_di\_installazione}
% Versione
\newcommand{\docVers}{1.0}
% Data creazione
\newcommand{\creationDate}{2013-03-24}
% Data ultima modifica
\newcommand{\modificationDate}{2013-03-25}
% Stato in {Approvato, Non approvato}
\newcommand{\docState}{Approvato}
% Uso in {Interno, Esterno}
\newcommand{\docUsage}{Esterno}
% Destinatari da specificare come nome1\\ &nome2\\ ecc.
\newcommand{\docDistributionList}{Installatore del sistema}
% Redattori da specificare come nome1\\ &nome2\\ ecc.
\newcommand{\docAuthors}{Stefano Farronato}
% Approvato da
\newcommand{\approvedBy}{Andrea Rizzi}
% Verificatori
\newcommand{\verifiedBy}{Marco Schivo}
% Perscorso (relativo o assoluto) che punta alla directory contenente shared/
% come sua sottodirectory (per comodità chiamiamola 'doc root').
\newcommand{\docRoot}{..}
% definire se si vuole l'indice delle tabelle
\def\INDICETABELLE{false}
% definire se si vuole l'indice delle figure
\def\INDICEFIGURE{false}

% importa il preambolo condiviso da tutti i documenti
% shared/preamble.tex
%
% Questo documento contiene la parte del preambolo condivisa e viene pertanto
% richiamato nel 'master' di tutti i documenti di progetto.  Al suo interno
% contiene le inclusioni (e le configurazioni) di tutti i package richiesti per
% la compilazione dei documenti, le macro di carattere generale e la definizione
% degli stili di pagina.

\documentclass[a4paper,10pt]{article}

% **************************************************
% Macro generiche
% **************************************************
\newcommand{\team}{Software Synthesis}                    % chi siamo
\newcommand{\email}{info@softwaresynthesis.org}           % e-mail
\newcommand{\caName}{MyTalk}                              % titolo capitolato
\newcommand{\manager}{SynthesisRequirementManager}        % nome del sistema di tracciamento
\newcommand{\memberdata}[1]{%
  \texttt{\textcolor{RedOrange}{#1}}}                     % attributi di una classe
\newcommand{\method}[1]{\texttt{\textcolor{Emerald}{#1}}} % metodi di una classe
\newcommand{\exception}[1]{%
  \texttt{\textcolor{RedViolet}{#1}}}                     % eccezione
% \newcommand{\handler}[1]{\texttt{\textcolor{Maroon}{#1}}} % per gli event handler
\newcommand{\inglese}[1]{%
  \foreignlanguage{english}{\textit{#1}}}                 % per i testi in lingua inglese
\newcommand{\purpose}{%                                     scopo del prodotto
Con il progetto ``\caName'' si intende un sistema software di comunicazione tra utenti mediante \underline{browser} senza la necessit{\`a} di installazione di \underline{plugin} e/o software esterni. L'utilizzatore avr{\`a} la possibilit{\`a} di interagire con un altro utente tramite una comunicazione audio - audio/video - testuale e, inoltre, ottenere delle statistiche sull'attivit{\`a} in tempo reale.%
}
\newcommand{\glossaryIntro}{%                               introduzione al glossario
Al fine di evitare incomprensioni dovute all'uso di termini tecnici nei documenti, viene redatto e allegato il documento \textit{glossario.4.0.pdf} dove vengono definiti e descritti tutti i termini marcati con una sottolineatura.%
}


% **************************************************
% Codifica e lingua dei documenti
% **************************************************
\usepackage[utf8x]{inputenc}                              % codifica caratteri dei documenti sorgenti
\usepackage[english,italian]{babel}                       % localizzazione ai fini di sillabazione e cross-references
\usepackage[T1]{fontenc}                                  % codifica font di output

% **************************************************
% Definizione geometria della pagina
% **************************************************
\usepackage[a4paper,head=4cm,top=4.5cm,bottom=3cm,left=3cm,right=3cm,bindingoffset=5mm]{geometry}

% *************************************************
% Intestazioni e piè di pagina personalizzati
% *************************************************
\usepackage{fancyhdr}
% stile normale
\fancypagestyle{normal}{
\fancyhead{}                                              % intestazione
\fancyhead[RE,RO]{
\begin{picture}(0,0)
  \put(-410,0){\includegraphics[width=1.02\textwidth]{header_logo}}
  \put(-410,10){\sffamily\large\leftmark}
\end{picture}
\vspace{-4pt}
}
\renewcommand{\headrulewidth}{0pt}                       % riga sotto l'intestazione
\cfoot{}                                                  % piè di pagina
\fancyfoot[RO,LE]{\sffamily
  pag.~\thepage{} di \pageref{LastPage}}                  % a dx nelle pag. dispari e a sx in quelle pari
\fancyfoot[RE,LO]{\sffamily\docFileName{}}
\renewcommand{\footrulewidth}{.4pt}                       % riga sopra il piè di pagina
}
% stile per gli indici
\fancypagestyle{toc}{
\fancyhead{}                                              % intestazione
\fancyhead[RE,RO]{
\begin{picture}(0,0)
  \put(-410,0){\includegraphics[width=1.02\textwidth]{header_logo}}
\end{picture}
}
\renewcommand{\headrule}{}                                % nessuna riga sotto l'intestazione
\cfoot{}                                                  % piè di pagina
\fancyfoot[RO,LE]{\sffamily\thepage{}}                    % a dx nelle pag. dispari e a sx in quelle pari
\fancyfoot[RE,LO]{\sffamily\docFileName{} -- v.\docVers}
\renewcommand{\footrulewidth}{.4pt}                       % riga sopra il piè di pagina
}

\pagestyle{fancy}                                         % premetto: non so usare bene le marche:
\renewcommand{\sectionmark}[1]{\markboth{#1}{#1}}         % se qualcuno ha idee migliori si faccia avanti!

% **************************************************
% Tabelle
% **************************************************
\usepackage{tabularx}                                     % tabelle di larghezza fissa con una o più colonne variabili
\usepackage{multirow}                                     % colonne con colonne che si estendono per più righe
\usepackage{booktabs}                                     % per inserire l'ambiente table e le righe orizz. nelle tabelle
\usepackage{longtable}			                              % tabelle oltre i limiti di pagina

% **************************************************
% Cross-references e collegamenti ipertestuali
% **************************************************
\usepackage[hidelinks]{hyperref}
\hypersetup{%
  colorlinks=false, linktocpage=false, pdfborder={0,0,0}, pdfstartpage=1, pdfstartview=FitV,%
  urlcolor=Cyan, linkcolor=Cyan, citecolor=Black, %pagecolor=Black,%
  pdftitle={\docName}, pdfauthor={\team}, pdfsubject={}, pdfkeywords={},%
  pdfcreator={pdflatex}, pdfproducer={pdflatex with hyperref package}%
}

% **************************************************
% Immagini e grafica
% **************************************************
\usepackage{graphicx}                                     % supporto ad aspetti avanzati delle immagini
\usepackage[table,usenames,dvipsnames]{xcolor}            % tabelle con righe colorate e alternate
\graphicspath{{\docRoot/pics/}}                           % percorso contenente tutti i file immagini
\usepackage{float}                                        % per rendere non flottanti gli ambienti flottanti
\usepackage[italian]{varioref}                            % testo completo riferimenti in italiano

% **************************************************
% Definizioni di colori
% **************************************************
\definecolor{myBlue}{RGB}{1,167,236}
\definecolor{lightblue}{RGB}{213,243,253}%{119,218,247}
\definecolor{llightblue}{RGB}{229,255,255}

% **************************************************
% Altri pacchetti opzionali
% **************************************************     
\usepackage{lastpage}                                     % per sapere il numero totale di pagine
\usepackage{eurosym}                                      % per il simbolo dell'euro usare \EUR{x} dove x è l'importo
\usepackage{ifthen}                                       % permette la scelta di rami condizionali nella compilazione
\usepackage{enumitem}                                     % permette di configurare gli elenchi puntati e numerati


% Fine del preambolo e inizio del documento
\begin{document}

% Inclusione della prima pagina
% shared/firstpage.tex
%
% Questo documento definisce il contenuto della prima pagina, che si suppone
% essere uguale in tutti i documenti.  Oltre al logo e al titolo, la prima
% pagina contiene i metadati relativi al documento in cui viene inclusa.


% rimuove intestazioni e piè di pagina
\pagestyle{empty}

\begin{center}

% logo del gruppo
\includegraphics[width=1.5\textwidth]{logo}

\vspace{1in}

% titolo del documento
{\Huge\bfseries \docName}

\vspace{1in}

% tabella riepilogativa
\begin{tabularx}{.7\textwidth}{>{\bfseries\sffamily}l>{\sffamily}l}
\toprule
\multicolumn{2}{>{\sffamily}c}{Informazioni sul documento}\\
\midrule
Nome file:            & \docFileName\\
Versione:             & \docVers\\
Data creazione:       & \creationDate\\
Data ultima modifica: & \modificationDate\\
Stato:                & \docState\\
Uso:                  & \docUsage\\
Redattori:            & \docAuthors\\
Approvato da:         & \approvedBy\\
Verificatori:         & \verifiedBy\\
\bottomrule
\end{tabularx}

\end{center}

\newpage


% Storico delle modifiche
\section*{Storia delle modifiche}

\begin{tabularx}{\textwidth}{lXlll}
\toprule
Versione & Descrizione intervento & Redattore & Ruolo & Data\\
\midrule % inserire qui il contenuto della tabella
1.0 & Approvazione documento& Andrea Rizzi & Responsabile & 2013-03-25\\
0.3 & Verifica del documento & Marco Schivo & Verificatore & 2013-03-25\\
0.3 & Inserite immagini guida& Stefano Farronato & Verificatore & 2013-03-25\\
0.2 & Documentate sezioni  di prerequisiti, materiale fornito e installazione& Stefano Farronato & Verificatore & 2013-03-24\\
0.1 & Prima stesura documento & Stefano Farronato & Verificatore & 2013-03-24\\
\bottomrule
\end{tabularx}
\newpage

% inclusione dell'indice
% shared/toc.tex
%
% Questo file contiene le istruzioni che generano l'indice o gli indici del
% documento (utile nel caso in cui decidessimo di avere anche un indice delle
% tabelle e/o un indice delle figure).

% imposta lo stile di pagina per i titoli definito nel preambolo
\pagestyle{toc}
% imposta i numeri di pagina romani minuscoli
\pagenumbering{roman}

% genera automaticamente l'indice di LaTeX
\tableofcontents

% se è true \INDICETABELLE allora genera l'indice delle tabelle, altrimenti non fa nulla
\ifthenelse{\equal{\INDICETABELLE}{true}}{%
  \clearpage % l'indice delle tabelle, se c'è, deve andare a pagina nuova
  \listoftables
}{}

% se è true |INDICEFIGURE allora genera l'indice delle figure, altrimenti non fa nulla
\ifthenelse{\equal{\INDICEFIGURE}{true}}{%
  \clearpage % l'indice delle figure, se c'è, deve andare a pagina nuova
  \listoffigures
}{}

%in ogni caso occorre andare a pagina nuova dopo gli indici
\clearpage


% Alcuni aggiustamenti per le pagine
\pagenumbering{arabic}
\setcounter{page}{1}
\pagestyle{normal}

\section{Introduzione}
\subsection{Scopo del prodotto}
Il prodotto ``MyTalk'' è un sistema software di comunicazione tra utenti mediante \underline{browser} internet, senza la necessità di installazione di \underline{plugin} e/o software esterni. Gli utenti avranno la possibilità di interagire mediante una comunicazione audio - audio/video - testuale ottenendo le statistiche sull'attività in tempo reale.

\subsection{Scopo del Manuale}
Lo scopo del presente manuale è quello di fornire una guida per l'installazione dell'applicazione. Tale documento contiene la descrizione delle operazioni da eseguire per portare a termine correttamente la fase di installazione e rendere quindi l'applicativo funzionante.

\subsection{Destinatario del manuale}
Questo manuale è rivolto agli installatori di sistema.

\section{Prerequisiti}
L'installatore deve avere a disposizione:
\begin{itemize}
\item connessione alla rete internet;
\item un server con installato \textit{Tomcat 7} (un contenitore servlet open source);
\item il DBMS \textit{MySQL}, per l'installazione seguire la procedura standard descritta nel sito ufficiale \url{www.mysql.com};
\item console \textit{PhpMyAdmin} per l'amministrazione della base di dati;
\item permessi per accedere alla sezione \textit{manager} di \textit{Tomcat};
\item permessi per creare una variabile di sistema.
\end{itemize} 

\section{Materiale fornito}
\begin{itemize}
\item directory \caName{} contenente il codice compilato in formato \textit{.war};
\item file \textit{.sql} per la creazione del database;
\item files \textit{html} contenenti la documentazione delle classi.
\end{itemize} 

\section{Installazione}
\begin{enumerate}
\item Nella cartella di installazione di PhpMyAdmin aprire il file \texttt{config.inc.php} e verficare le seguenti variabili di sistema:
\begin{itemize}
\item \verb+$cfg['Servers'][\$i]['auth_type']+ sia impostata con il valore \texttt{cookie} (in modo da consentire l'accesso tramite \textit{username} e \textit{password});
\end{itemize}
\item Nella cartella di installazione di PhpMyAdmin aprire il file \texttt{my.ini} e verficare le seguenti variabili di sistema:
\begin{itemize}
\item \texttt{lower\_case\_table\_names} sia impostata a 0 (in modo che MySQL sia \inglese{case-sensitive})
\end{itemize}
\item Configurare una variabile di ambiente, per l'utente che ha accesso al server, denominata \textit{MyTalkConfiguration} che punta alla cartella di installazione di \textit{TomCat 7};

\item aprire il pannello di controllo di PhpMyAdmin, importare il file \textit{.sql} e confermare il caricamento

\begin{figure}[H]
  \includegraphics[width=\textwidth]{ManInst_InstallazioneDB}
\caption{Installazione Database}\label{fig:InstallazioneDB}
\end{figure}
una volta eseguita l'operazione, si sarà creato il database e la relativa struttura;

\begin{figure}[H]
  \includegraphics[width=\textwidth]{ManInst_EsitoInstDB}
\caption{Esito installazione Database}\label{fig:EsitoInstDB}
\end{figure}


\item Fare il deploy del file \textit{MyTalk.war} nella sezione manager di \textit{TomCat}

\begin{figure}[H]
  \includegraphics[width=\textwidth]{ManInst_Deploy}
\caption{Deploy blabla}\label{fig:Deploy}
\end{figure}

al termine della procedura la schermata principale di \textit{TomCat} dovrà essere equivalente alla seguente;

\begin{figure}[H]
  \includegraphics[width=\textwidth]{ManInst_EsitoDeploy}
\caption{Esito deploy blabla}\label{fig:EsitoDeploy}
\end{figure}


\item verificare infine che il proprio server cosi configurato sia visibile dalla rete internet.
\end{enumerate}

\begin{figure}[H]
  \includegraphics[width=\textwidth]{ManInst_InstallazioneDB}
\caption{Installazione Database}\label{fig:InstallazioneDB}
\end{figure}


\end{document}
