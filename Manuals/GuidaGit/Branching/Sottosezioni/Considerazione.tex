\subsection{Considerazioni}
Concludiamo ora la sezione con alcune considerazioni sul branching in Git. Dato che in Git un ramo è semplicemente un file che contiene i 40 caratteri di checksum SHA-1 del commit al quale punta, i rami possono essere creati e distrutti facilmente. Creare un nuovo ramo è semplice e veloce quanto scrivere 41 byte (40 caratteri ed il fine riga) all'interno di un file.

Questo è in netto contrasto con molti altri VCS, che funzionano copiando tutti i file di un progetto in una seconda directory. Questa operazione può richiedere diversi secondi o minuti a seconda della dimensione del progetto, mentre in Git è istantaneo.

Inoltre, dato che vengono registrati i genitori dei commit, trovare una base di unione per il merging è generalmente molto semplice da portare a termine. La ramificazione sviluppata in quest'ottica aiuta ed incoraggia gli sviluppatori a creare e fare uso dei rami di sviluppo.
