\subsection{Spostarsi da un ramo all'altro}
Il comando da utilizzare per spostarsi da un ramo all'altro è:

\begin{center}
\texttt{git checkout NomeRamo}
\end{center}

dove ''NomeRamo'' va sostituito con il nome del ramo in cui ci vogliamo spostare.

Continuiamo con l'esempio precedente per capire meglio cosa accade. Dopo aver lavorato un pò sui tre file precedenti vogliamo modificare il file ''Index.php'' e ''License.txt'' poichè vogliamo creare una versione a parte, che quindi avrà una licenza diversa di distribuzione. Dopo aver creato un nuovo ramo per ospitare questa modifica ci spostiamo su di esso con il comando appena illustrato. A questo punto il nostro progetto possiede due rami ma in sostanza è ancora quello di partenza. Ora eseguimo la nostra modifica, nei due file, ed eseguiamo l'operazione di staging e successivamente eseguimo il commit per questi ultimi.

Dopo aver eseguito il commit il ramo appena creato, si è spostato in avanti contenendo queste modifiche ma il ramo master invece, punta ancora allo stato che avevano i tre file prima della creazione del nuovo ramo. Se torniano al ramo principale osserviamo che avvengono due cose sostanzialmente:

\begin{itemize}
\item il puntatore \textit{HEAD} è tornato indietro per puntare all'ultimo commit del ramo master;
\item i file presenti nella cartella di lavoro sono tornati allo stato rappresentato dall'ultimo commit nel ramo master.
\end {itemize}

A questo punto il nostro progetto possiede due versioni separate ed indipendenti, ognuna con una sua storia.
