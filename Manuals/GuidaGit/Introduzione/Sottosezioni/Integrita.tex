\subsection{Integrità}
Git è un sistema integro perchè ogni cosa è controllata, tramite checksum, prima di essere salvata ed è referenziata da un checksum. Ciò significa che non è possibile modificare il contenuto di una qualsiasi file o directory senza che il sistema non se ne accorga. 

Il meccanismo utilizzato da Git per il checksum è un hash, denominato SHA-1. Si tratta di una stringa di 40 caratteri, composta da caratteri esadecimali, calcolata in base al contenuto del file o della struttura di directory in Git. Il sistema immagazzina ogni cosa nel proprio database non per nome di file ma tramite codice hash SHA-1.