\subsection{Aggiornare la sorgente remota}
Quando vogliamo che i nostri collaboratori abbiano in locale le nostre modifiche dobbiamo aggiornare prima il repository remoto che essi useranno per aggiornare il loro locale. Per aggiornare usiamo il comando:

\begin{center}
\texttt{git push URL/Alias NomeRamo}
\end{center}

Dove con URL/Alias si intende l'indirizzo del repository remoto o l'alias ad esso associato, mentre NomeRamo è il ramo del repository remoto che vogliamo aggiornare, la maggior parte delle volte sarà il ramo principale, ossia il ramo \textit{master}.

Se il comando dovesse essere rifiutato, la causa molto probabilmente sarà che contemporaneamente a noi qualcun altro ha fatto la stessa operazione, quindi noi prima dobbiamo scaricarci in locale i suoi aggiornamenti dopo di che possiamo ripetere la procedura di aggiornamento della sorgente remota.