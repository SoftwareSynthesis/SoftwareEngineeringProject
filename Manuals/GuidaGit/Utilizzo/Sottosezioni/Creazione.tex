\subsection{Creazione del repository locale}
Quando iniziamo a lavorare ad un progetto ci si presentano due situazioni mutualmente esclusive:

\begin{itemize}
\item il progetto è creato da noi, quindi un nuovo repository deve essere inizializzato;
\item il progetto è già esistente, quindi dobbiamo clonare in locale un repository esistente.
\end{itemize}

Se ci troviamo nella prima situazione quello che dobbiamo fare per inizializzare un nuovo repository è quella di creare una cartella che ospiti la cartella di Git e l'area di lavoro, successivamente ci spostiamo al suo interno con una finestra di terminale e lanciamo il seguente comando:

\begin{center}
\texttt{git init}
\end{center}

al termine avremo una cartella nascosta denominata ''.git'', questa è la cartella di Git.

Se invece siamo nel secondo caso, in particolare significa che entriamo come collaboratori in un progetto, dobbiamo spostarci nella cartella che conterrà il pacchetto, ossia la cartella di Git e l'ambiente di lavoro, e lanciare il seguente comando:

\begin{center}
\texttt{git clone URL NomeDirectory}
\end{center}

dove ''URL'' va sostituita con l'URL del repository remoto. L'opzione ''NomeDirectory'' specifica il nome della directory locale che conterrà il repository, se omessa il nome sarà quello del server remoto.
